\section*{HW6 Due to December 4 11:59 PM}
\question{1}{Problem 36.3}\\
\begin{itemize}
    \item [(a)] Prove the \textit{Fierz identities}
    \begin{align}
        (\chi^\dagger_1 \overline{\sigma}^\mu \chi_2)  (\chi^\dagger_3 \overline{\sigma}_\mu \chi_4) &= -2 (\chi^\dagger_1 \chi_3^\dagger) (\chi_2 \chi_4) \tag{36.58}\\
        (\chi^\dagger_1 \overline{\sigma^\mu} \chi_2)  (\chi^\dagger_3 \overline{\sigma_\mu} \chi_4) &=  (\chi^\dagger_1 \overline{\sigma}^\mu \chi_4)  (\chi^\dagger_3 \overline{\sigma}_\mu \chi_2)\tag{36.59}
    \end{align}
    \item [(b)] Define the Dirac fields
    \begin{align}
        \Psi_1 \equiv \begin{pmatrix}
            \chi_i \\ \xi^\dagger_i
        \end{pmatrix}, \quad \Psi_i^C \equiv \begin{pmatrix}
            \xi_i \\ \chi^\dagger_i
        \end{pmatrix}\tag{36.60}
    \end{align}
    Use eqs.~(36.58) and (36.59) to prove the Dirac form of the Fierz identities,
    \begin{align}
        (\overline{\Psi}_1 \gamma^\mu P_L \Psi_2) (\overline{\Psi}_3 \gamma_\mu P_L \Psi_4) &= - 2 (\overline{\Psi}_1 P_R \Psi_3^C) (\overline{\Psi}_4^C P_L \Psi_2) \tag{36.61}\\
        (\overline{\Psi}_1 \gamma^\mu P_L \Psi_2) (\overline{\Psi}_3 \gamma_\mu P_L \Psi_4) &= (\overline{\Psi}_1 \gamma^\mu P_L \Psi_4) (\overline{\Psi}_3 \gamma_\mu P_L \Psi_2)\tag{36.62}
    \end{align}
    \item [(c)] By writing both sides out in terms of Weyl fields, show that
    \begin{align}
        \overline{\Psi}_1 \gamma^\mu P_R \Psi_2 &= -\overline{\Psi}_2^C \gamma^\mu P_L \Psi_1^C\tag{36.63}\\
        \overline{\Psi}_1  P_L \Psi_2 &= \overline{\Psi}_2^C P_L \Psi_1^C\tag{36.64}\\
        \overline{\Psi}_1  P_R \Psi_2 &= \overline{\Psi}_2^C P_R \Psi_1^C\tag{36.65}.
    \end{align}
    Combining equations~(36.63–36.65) with equations~(36.61–36.62) yields more useful forms of the Fierz identities.
\end{itemize}
\answer{}
\begin{itemize}
    \item [(a)]
\end{itemize}
We start from the left-hand side of equation~(36.58):
\begin{align}
    (\chi^\dagger_1 \overline{\sigma}^\mu \chi_2)  (\chi^\dagger_3 \overline{\sigma}_\mu \chi_4) &= (\chi^\dagger_1)_{\dot{a}} (\overline{\sigma}^\mu)^{\dot{a}b} (\chi_2)_b (\chi^\dagger_3)_{\dot{c}} (\overline{\sigma}_\mu)^{\dot{c}d} (\chi_4)_d \\
    &= (\chi^\dagger_1)_{\dot{a}}  (\chi_2)_b (\chi^\dagger_3)_{\dot{c}}  (\chi_4)_d (\overline{\sigma}^\mu)^{\dot{a}b}  (\overline{\sigma}_\mu)^{\dot{c}d}
\end{align}
Using the identity in equations~(35.4),(35.19)
\begin{align}
    (\sigma^{\mu})_{a\dot{a}}  (\sigma_{\mu})_{b\dot{b}} = -2 \epsilon_{ab} \epsilon_{\dot{a}\dot{b}}\tag{35.4}\\
    (\overline{\sigma}^\mu)^{\dot{a}a} \equiv \epsilon^{ab} \epsilon^{\dot{a}\dot{b}}  (\sigma_\mu)_{b\dot{b}}\tag{35.19}
\end{align}
we have
\begin{align}
    (\overline{\sigma}^\mu)^{\dot{a}b}  (\overline{\sigma}_\mu)^{\dot{c}d} &= \epsilon^{be} \epsilon^{\dot{a}\dot{f}}  (\sigma^\mu)_{e\dot{f}} \epsilon^{dg} \epsilon^{\dot{c}\dot{h}}  (\sigma_\mu)_{g\dot{h}} \\
    &= \epsilon^{be} \epsilon^{dg} \epsilon^{\dot{a}\dot{f}} \epsilon^{\dot{c}\dot{h}}  (\sigma^\mu)_{e\dot{f}}  (\sigma_\mu)_{g\dot{h}} \\
    &= -2 \epsilon^{be} \epsilon^{dg} \epsilon^{\dot{a}\dot{f}} \epsilon^{\dot{c}\dot{h}} \epsilon_{eg} \epsilon_{\dot{f}\dot{h}} \\
    &= -2 \epsilon^{be} \delta^d_e \epsilon^{\dot{a}\dot{f}} \delta^{\dot{c}}_{\dot{f}} \\
    &= -2 \epsilon^{bd} \epsilon^{\dot{a}\dot{c}}
\end{align}
Substituting this back, we get
\begin{align}
    (\chi^\dagger_1 \overline{\sigma}^\mu \chi_2)  (\chi^\dagger_3 \overline{\sigma}_\mu \chi_4) &= -2 (\chi^\dagger_1)_{\dot{a}}  (\chi_2)_b (\chi^\dagger_3)_{\dot{c}}  (\chi_4)_d \epsilon^{bd} \epsilon^{\dot{a}\dot{c}} \\
    &= -2 (\chi^\dagger_1)_{\dot{a}}  (\chi^\dagger_3)_{\dot{c}} \epsilon^{\dot{a}\dot{c}}  (\chi_2)_b  (\chi_4)_d \epsilon^{bd} \\
    &=-2 (\chi^\dagger_1)^{\dot{c}}  (\chi^\dagger_3)_{\dot{c}}  (\chi_2)_d  (\chi_4)^d \\
    &=-2 (\chi^\dagger_1 \chi_3^\dagger) (\chi_2 \chi_4)\times (-1)(-1)\\
    &=-2 (\chi^\dagger_1 \chi_3^\dagger) (\chi_2 \chi_4)
\end{align}
This proves equation~(36.58). Similarly, we can prove equation~(36.59):
\begin{align}
    (\chi^\dagger_1 \overline{\sigma}^\mu \chi_2)  (\chi^\dagger_3 \overline{\sigma}_\mu \chi_4) &=-2 (\chi^\dagger_1 \chi_3^\dagger) (\chi_2 \chi_4)\\
    &= -2 (\chi^\dagger_1 \chi_3^\dagger) (\chi_4 \chi_2)\\
    &= (\chi^\dagger_1 \overline{\sigma}^\mu \chi_4)  (\chi^\dagger_3 \overline{\sigma}_\mu \chi_2)
\end{align}
This proves equation~(36.59).
\begin{itemize}
    \item [(b)]
\end{itemize}
First $\Psi=\begin{pmatrix}
    \chi_a \\ (\xi^\dagger)^{\dot{a}}
\end{pmatrix}$, so $\overline{\Psi} = (\xi^a, (\chi^\dagger)_{\dot{a}})$. Also, $P_L \Psi = \begin{pmatrix}
    \chi_a \\ 0
\end{pmatrix}$ and $P_R \Psi = \begin{pmatrix}
    0 \\ (\xi^\dagger)^{\dot{a}}
\end{pmatrix}$. Thus,
\begin{align}
    \overline{\Psi}_1 \gamma^\mu P_L \Psi_2 &= (\xi^a_1, (\chi^\dagger_1)_{\dot{a}}) \begin{pmatrix}
        0 & \sigma^\mu_{a\dot{b}} \\ \overline{\sigma}^{\mu \dot{a} b} & 0
    \end{pmatrix} \begin{pmatrix}
        (\chi_2)_b \\ 0
    \end{pmatrix} \\
    &= \xi^a_1 \sigma^\mu_{a\dot{b}} (\chi_2)_b \\
    &= (\chi^\dagger_1 \overline{\sigma}^\mu \chi_2)
\end{align}
Similarly,
\begin{align}
    \overline{\Psi}_3 \gamma_\mu P_L \Psi_4 &= (\chi^\dagger_3 \overline{\sigma}_\mu \chi_4)
\end{align}
Therefore,
\begin{align}
    (\overline{\Psi}_1 \gamma^\mu P_L \Psi_2) (\overline{\Psi}_3 \gamma_\mu P_L \Psi_4) &= (\chi^\dagger_1 \overline{\sigma}^\mu \chi_2)  (\chi^\dagger_3 \overline{\sigma}_\mu \chi_4) \\
    &= -2 (\chi^\dagger_1 \chi_3^\dagger) (\chi_2 \chi_4) \tag{from (a)}
\end{align}
Now, for the right-hand side of equation~(36.61):
\begin{align}
    \overline{\Psi}_1 P_R \Psi_3^C &= (\xi^a_1, (\chi^\dagger_1)_{\dot{a}}) \begin{pmatrix}
        0 & 0 \\ 0 & 1
    \end{pmatrix} \begin{pmatrix}
        (\xi_3)_b \\ (\chi^\dagger_3)^{\dot{b}}
    \end{pmatrix} \\
    &= (\chi^\dagger_1)_{\dot{a}}  (\chi^\dagger_3)^{\dot{a}} \\
    &= (\chi^\dagger_1 \chi_3^\dagger)
\end{align}
Similarly,
\begin{align}
    \overline{\Psi}_4^C P_L \Psi_2 &= ( (\xi^a_4, (\chi^\dagger_4)_{\dot{a}}) \begin{pmatrix}
        1 & 0 \\ 0 & 0
    \end{pmatrix} \begin{pmatrix}
        (\chi_2)_b \\ (\xi^\dagger_2)^{\dot{b}}
    \end{pmatrix} \\
    &= (\xi^a_4)  (\chi_2)_a \\
    &= (\chi_4 \chi_2)= (\chi_2 \chi_4)
\end{align}
Thus,
\begin{align}
    -2 (\overline{\Psi}_1 P_R \Psi_3^C) (\overline{\Psi}_4^C P_L \Psi_2) &= -2 (\chi^\dagger_1 \chi_3^\dagger) (\chi_2 \chi_4)
\end{align}
This proves equation~(36.61). Similarly, we can prove equation~(36.62):
\begin{align}
    (\overline{\Psi}_1 \gamma^\mu P_L \Psi_2) (\overline{\Psi}_3 \gamma_\mu P_L \Psi_4) &= (\chi^\dagger_1 \overline{\sigma}^\mu \chi_2)  (\chi^\dagger_3 \overline{\sigma}_\mu \chi_4) \\
    &= (\chi^\dagger_1 \overline{\sigma}^\mu \chi_4)  (\chi^\dagger_3 \overline{\sigma}_\mu \chi_2) \tag{from (a)}\\
    &= (\overline{\Psi}_1 \gamma^\mu P_L \Psi_4) (\overline{\Psi}_3 \gamma_\mu P_L \Psi_2)
\end{align}
This proves equation~(36.62).
\begin{itemize}
    \item [(c)]
\end{itemize}
First, we compute the left-hand side of equation~(36.63):
\begin{align}
    \overline{\Psi}_1 \gamma^\mu P_R \Psi_2 &= (\xi^a_1, (\chi^\dagger_1)_{\dot{a}}) \begin{pmatrix}
        0 & \sigma^\mu_{a\dot{b}} \\ \overline{\sigma}^{\mu \dot{a} b} & 0
    \end{pmatrix} \begin{pmatrix}
        0 \\ (\xi^\dagger_2)^{\dot{b}}
    \end{pmatrix} \\
    &= \xi^a_1 \sigma^\mu_{a\dot{b}} (\xi^\dagger_2)^{\dot{b}} 
\end{align}
Next, we compute the right-hand side of equation~(36.63):
\begin{align}
    \overline{\Psi}_2^C \gamma^\mu P_L \Psi_1^C &= (\chi^a_2, (\xi^\dagger_2)_{\dot{a}}) \begin{pmatrix}
        0 & \sigma^\mu_{a\dot{b}} \\ \overline{\sigma}^{\mu \dot{a} b} & 0
    \end{pmatrix} \begin{pmatrix}
        (\xi_1)_b \\ 0
    \end{pmatrix} \\
    &= (\xi^\dagger_2)_{\dot{a}} \overline{\sigma}^{\mu \dot{a} b} (\xi_1)_b 
\end{align}
Using the identity 
\begin{align}
    (\overline{\sigma}^\mu)^{\dot{a}a} \equiv \epsilon^{ab} \epsilon^{\dot{a}\dot{b}}  (\sigma_\mu)_{b\dot{b}}\tag{35.19}
\end{align}
we have
\begin{align}
    (\xi^\dagger_2)_{\dot{a}} \overline{\sigma}^{\mu \dot{a} b} (\xi_1)_b &= (\xi^\dagger_2)_{\dot{a}} \epsilon^{bc} \epsilon^{\dot{a}\dot{b}}  (\sigma_\mu)_{c\dot{b}} (\xi_1)_b \\
    &= - \xi_1^c \sigma^\mu_{c\dot{b}} (\xi^\dagger_2)^{\dot{b}} \\
    &= - \xi^a_1 \sigma^\mu_{a\dot{b}} (\xi^\dagger_2)^{\dot{b}}\\
    &= \overline{\Psi}_2^C \gamma^\mu P_L \Psi_1^C= - \overline{\Psi}_1 \gamma^\mu P_R \Psi_2
\end{align}
This proves equation~(36.63). Similarly, we can prove equations~(36.64) and (36.65):
\begin{align}
    LHS=\overline{\Psi}_1  P_L \Psi_2 &= (\xi^a_1, (\chi^\dagger_1)_{\dot{a}}) \begin{pmatrix}
        1 & 0 \\ 0 & 0
    \end{pmatrix} \begin{pmatrix}
        (\chi_2)_a \\ (\xi^\dagger_2)^{\dot{a}}
    \end{pmatrix} \\
    &= \xi^a_1 (\chi_2)_a = (\xi_1 \chi_2)
\end{align}
\begin{align}
    RHS =\overline{\Psi}_2^C P_L \Psi_1^C &= (\chi^a_2, (\xi^\dagger_2)_{\dot{a}}) \begin{pmatrix}
        1 & 0 \\ 0 & 0
    \end{pmatrix} \begin{pmatrix}
        (\xi_1)_a \\ (\chi^\dagger_1)^{\dot{a}}
    \end{pmatrix} \\
    &=  (\chi_2)_a \xi^a_1 = (\chi_2 \xi_1) = (\xi_1 \chi_2)
\end{align}
This proves equation~(36.64).
\begin{align}
    LHS=\overline{\Psi}_1  P_R \Psi_2 &= (\xi^a_1, (\chi^\dagger_1)_{\dot{a}}) \begin{pmatrix}
        0 & 0 \\ 0 & 1
    \end{pmatrix} \begin{pmatrix}
        (\chi_2)_a \\ (\xi^\dagger_2)^{\dot{a}}
    \end{pmatrix} \\
    &= (\chi^\dagger_1)_{\dot{a}}  (\xi^\dagger_2)^{\dot{a}} =(\chi_1^\dagger \xi_2^\dagger)
\end{align}
\begin{align}
    RHS =\overline{\Psi}_2^C P_R \Psi_1^C &= (\chi^a_2, (\xi^\dagger_2)_{\dot{a}}) \begin{pmatrix}
        0 & 0 \\ 0 & 1
    \end{pmatrix} \begin{pmatrix}
        (\xi_1)_a \\ (\chi^\dagger_1)^{\dot{a}}
    \end{pmatrix} \\
    &= (\xi^\dagger_2)_{\dot{a}} (\chi^\dagger_1)^{\dot{a}}  =(\xi_2^\dagger\chi_1^\dagger )=(\chi_1^\dagger \xi_2^\dagger)
\end{align}
This proves equation~(36.65).\qed


\clearpage
\question{2}{38.1}\\
Use equation~(38.12) to compute $u_s(\mathbf{p})$ and $v_s(\mathbf{p})$ explicitly. Hint: Show that the matrix $2i\hat{\mathbf{p}}\cdot\mathbf{K}$ has eigenvalues $\pm 1$, and that, for any matrix $A$ with eigenvalues $\pm 1$, $\exp(cA) = \cosh(c) + A \sinh(c)$, where $c$ is an arbitrary complex number.\\
\textbf{Extra question:} What is the expression in the large energy limit $E_\mathbf{p} \gg m$? Please write down the result.\\
\begin{align}
    u_s(\mathbf{p})=\exp{(i\eta \hat{\mathbf{p}}\cdot\mathbf{K})}u_s(\mathbf{0}),\quad v_s(\mathbf{p})=\exp{(i\eta \hat{\mathbf{p}}\cdot\mathbf{K})}v_s(\mathbf{0})\tag{38.12}
\end{align}
\answer{}
We start by showing that the matrix $2i\hat{\mathbf{p}}\cdot\mathbf{K}$ has eigenvalues $\pm 1$. The boost generators $\mathbf{K}$ in the Dirac representation are given by
\begin{align}
    K^j =\frac{i}{2} \gamma^j \gamma^0 = \frac{i}{2} \begin{pmatrix}
        0 & \sigma^j \\ -\sigma^j & 0
    \end{pmatrix} \begin{pmatrix}
        0 & I \\ I & 0
    \end{pmatrix} = \frac{i}{2} \begin{pmatrix}
        \sigma^j & 0 \\ 0 & -\sigma^j
    \end{pmatrix} 
\end{align}
Thus,
\begin{align}
    2i\hat{\mathbf{p}}\cdot\mathbf{K} = 2i \sum_{j=1}^3 \hat{p}_j K^j = 2i \sum_{j=1}^3 \hat{p}_j \frac{i}{2} \begin{pmatrix}
        \sigma^j & 0 \\ 0 & -\sigma^j
    \end{pmatrix} = - \begin{pmatrix}
        \hat{\mathbf{p}}\cdot\boldsymbol{\sigma} & 0 \\ 0 & -\hat{\mathbf{p}}\cdot\boldsymbol{\sigma}
    \end{pmatrix},
\end{align}
where $\hat{\mathbf{p}}\cdot\boldsymbol{\sigma}=\begin{pmatrix}
    \hat{p}_3 & \hat{p}_1 - i\hat{p}_2 \\ \hat{p}_1 + i\hat{p}_2 & -\hat{p}_3
\end{pmatrix}$. Now we want to prove $\hat{\mathbf{p}}\cdot\boldsymbol{\sigma}$ has eigenvalues $\pm 1$, since $\hat{\mathbf{p}}$ is a unit vector. The characteristic polynomial of $\hat{\mathbf{p}}\cdot\boldsymbol{\sigma}$ is given by
\begin{align}
    \det(\hat{\mathbf{p}}\cdot\boldsymbol{\sigma} - \lambda I) = \det\begin{pmatrix}
        \hat{p}_3 - \lambda & \hat{p}_1 - i\hat{p}_2 \\ \hat{p}_1 + i\hat{p}_2 & -\hat{p}_3 - \lambda
    \end{pmatrix} = (\hat{p}_3 - \lambda)(-\hat{p}_3 - \lambda) - (\hat{p}_1 - i\hat{p}_2)(\hat{p}_1 + i\hat{p}_2)
\end{align}
Simplifying this, we get
\begin{align}
    \det(\hat{\mathbf{p}}\cdot\boldsymbol{\sigma} - \lambda I) = \lambda^2 - (\hat{p}_1^2 + \hat{p}_2^2 + \hat{p}_3^2) = \lambda^2 - 1
\end{align}
Setting the determinant to zero, we find the eigenvalues:
\begin{align}
    \lambda^2 - 1 = 0 \implies \lambda^2 = 1 \implies \lambda = \pm 1
\end{align}
Thus, $\hat{\mathbf{p}}\cdot\boldsymbol{\sigma}$ has eigenvalues $\pm 1$. Consequently, the matrix $2i\hat{\mathbf{p}}\cdot\mathbf{K}$ has eigenvalues $\pm 1$ as well. Now we can use the identity for any matrix $A$ with eigenvalues $\pm 1$:
\begin{align}
    \exp(cA) = \cosh(c) + A \sinh(c)
\end{align}
where $c$ is an arbitrary complex number. Applying this to our case with $A = 2i\hat{\mathbf{p}}\cdot\mathbf{K}$ and $c = \eta/2$, we have
\begin{align}
    &\exp(i\eta \hat{\mathbf{p}}\cdot\mathbf{K}) = \cosh\left(\frac{\eta}{2}\right) + (2i\hat{\mathbf{p}}\cdot\mathbf{K}) \sinh\left(\frac{\eta}{2}\right)\\
    =& \cosh\left(\frac{\eta}{2}\right) - \begin{pmatrix}
        \hat{\mathbf{p}}\cdot\boldsymbol{\sigma} & 0 \\ 0 & -\hat{\mathbf{p}}\cdot\boldsymbol{\sigma}
    \end{pmatrix} \sinh\left(\frac{\eta}{2}\right)\\
    =& \begin{pmatrix}
        \cosh\left(\frac{\eta}{2}\right) - \hat{\mathbf{p}}\cdot\boldsymbol{\sigma} \sinh\left(\frac{\eta}{2}\right) & 0 \\ 0 & \cosh\left(\frac{\eta}{2}\right) + \hat{\mathbf{p}}\cdot\boldsymbol{\sigma} \sinh\left(\frac{\eta}{2}\right)
    \end{pmatrix}
\end{align}
Now, we can compute $u_s(\mathbf{p})$, 
\begin{align}
    u_same(\mathbf{p})=&\exp{(i\eta \hat{\mathbf{p}}\cdot\mathbf{K})}u_s(\mathbf{0})\\
    =& \begin{pmatrix}
        \cosh\left(\frac{\eta}{2}\right) - \hat{\mathbf{p}}\cdot\boldsymbol{\sigma} \sinh\left(\frac{\eta}{2}\right) & 0 \\ 0 & \cosh\left(\frac{\eta}{2}\right) + \hat{\mathbf{p}}\cdot\boldsymbol{\sigma} \sinh\left(\frac{\eta}{2}\right)
    \end{pmatrix} \begin{pmatrix}
        \sqrt{m} \,\xi_s \\ \sqrt{m} \,\xi_s
    \end{pmatrix}\\
    =& \sqrt{m} \begin{pmatrix}
        \left(\cosh\left(\frac{\eta}{2}\right) - \hat{\mathbf{p}}\cdot\boldsymbol{\sigma} \sinh\left(\frac{\eta}{2}\right)\right) \xi_s \\ \left(\cosh\left(\frac{\eta}{2}\right) + \hat{\mathbf{p}}\cdot\boldsymbol{\sigma} \sinh\left(\frac{\eta}{2}\right)\right) \xi_s
    \end{pmatrix},
\end{align}
where $\xi_{+}=\begin{pmatrix}
    1 \\ 0
\end{pmatrix}$ and $\xi_{-}=\begin{pmatrix}
    0 \\ 1
\end{pmatrix}$. Hence, we have
\begin{align}
    u_{+}(\mathbf{p})=& \sqrt{m} \begin{pmatrix}
        \cosh\left(\frac{\eta}{2}\right) - \hat{p}_3 \sinh\left(\frac{\eta}{2}\right) \\ - (\hat{p}_1 + i\hat{p}_2) \sinh\left(\frac{\eta}{2}\right) \\ 
        \cosh\left(\frac{\eta}{2}\right) + \hat{p}_3 \sinh\left(\frac{\eta}{2}\right) \\ (\hat{p}_1 + i\hat{p}_2) \sinh\left(\frac{\eta}{2}\right)
    \end{pmatrix}\\
    u_{-}(\mathbf{p})=& \sqrt{m} \begin{pmatrix}
        - (\hat{p}_1 - i\hat{p}_2) \sinh\left(\frac{\eta}{2}\right) \\ \cosh\left(\frac{\eta}{2}\right) + \hat{p}_3 \sinh\left(\frac{\eta}{2}\right) \\ 
        (\hat{p}_1 - i\hat{p}_2) \sinh\left(\frac{\eta}{2}\right) \\ \cosh\left(\frac{\eta}{2}\right) - \hat{p}_3 \sinh\left(\frac{\eta}{2}\right)
    \end{pmatrix}
\end{align}
Similarly, we can compute $v_s(\mathbf{p})$,
\begin{align}
    v_s(\mathbf{p})=&\exp{(i\eta \hat{\mathbf{p}}\cdot\mathbf{K})}v_s(\mathbf{0})\\
    =& \begin{pmatrix}
        \cosh\left(\frac{\eta}{2}\right) - \hat{\mathbf{p}}\cdot\boldsymbol{\sigma} \sinh\left(\frac{\eta}{2}\right) & 0 \\ 0 & \cosh\left(\frac{\eta}{2}\right) + \hat{\mathbf{p}}\cdot\boldsymbol{\sigma} \sinh\left(\frac{\eta}{2}\right)
    \end{pmatrix} \begin{pmatrix}
        \sqrt{m} \,\xi_s \\ -\sqrt{m} \,\xi_s
    \end{pmatrix}\\
    =& \sqrt{m} \begin{pmatrix}
        \left(\cosh\left(\frac{\eta}{2}\right) - \hat{\mathbf{p}}\cdot\boldsymbol{\sigma} \sinh\left(\frac{\eta}{2}\right)\right) \xi_s \\ -\left(\cosh\left(\frac{\eta}{2}\right) + \hat{\mathbf{p}}\cdot\boldsymbol{\sigma} \sinh\left(\frac{\eta}{2}\right)\right) \xi_s
    \end{pmatrix},
\end{align}
where $\xi_{+}=\begin{pmatrix}
    0 \\ 1
\end{pmatrix}$ and $\xi_{-}=\begin{pmatrix}
    -1 \\ 0
\end{pmatrix}$. Hence, we have
\begin{align}
    v_{+}(\mathbf{p})=& \sqrt{m} \begin{pmatrix}
        - (\hat{p}_1 - i\hat{p}_2) \sinh\left(\frac{\eta}{2}\right) \\ \cosh\left(\frac{\eta}{2}\right) + \hat{p}_3 \sinh\left(\frac{\eta}{2}\right) \\ 
        (\hat{p}_1 - i\hat{p}_2) \sinh\left(\frac{\eta}{2}\right) \\ -\left(\cosh\left(\frac{\eta}{2}\right) - \hat{p}_3 \sinh\left(\frac{\eta}{2}\right)\right)
    \end{pmatrix}\\
    v_{-}(\mathbf{p})=& \sqrt{m} \begin{pmatrix}
        \cosh\left(\frac{\eta}{2}\right) - \hat{p}_3 \sinh\left(\frac{\eta}{2}\right) \\ - (\hat{p}_1 + i\hat{p}_2) \sinh\left(\frac{\eta}{2}\right) \\ 
        -\left(\cosh\left(\frac{\eta}{2}\right) + \hat{p}_3 \sinh\left(\frac{\eta}{2}\right)\right) \\ -(\hat{p}_1 + i\hat{p}_2) \sinh\left(\frac{\eta}{2}\right)
    \end{pmatrix}
\end{align}
Now, we can express $\cosh\left(\frac{\eta}{2}\right)$ and $\sinh\left(\frac{\eta}{2}\right)$ in terms of energy $E_\mathbf{p}$ and mass $m$. We know that
\begin{align}
    \cosh(\eta) = \frac{E_\mathbf{p}}{m}, \quad \sinh(\eta) = \frac{|\mathbf{p}|}{m}
\end{align}
Using the half-angle formulas for hyperbolic functions, we have
\begin{align}
    \cosh\left(\frac{\eta}{2}\right) = \sqrt{\frac{E_\mathbf{p} + m}{2m}}, \quad \sinh\left(\frac{\eta}{2}\right) = \sqrt{\frac{E_\mathbf{p} - m}{2m}}
\end{align}
Substituting these back into the expressions for $u_s(\mathbf{p})$ and $v_s(\mathbf{p})$, we get
\begin{align}
    u_{+}(\mathbf{p})=& \begin{pmatrix}
        \sqrt{\frac{E_\mathbf{p} + m}{2}} - \hat{p}_3 \sqrt{\frac{E_\mathbf{p} - m}{2}} \\ - (\hat{p}_1 + i\hat{p}_2) \sqrt{\frac{E_\mathbf{p} - m}{2}} \\ 
        \sqrt{\frac{E_\mathbf{p} + m}{2}} + \hat{p}_3 \sqrt{\frac{E_\mathbf{p} - m}{2}} \\ (\hat{p}_1 + i\hat{p}_2) \sqrt{\frac{E_\mathbf{p} - m}{2}}
    \end{pmatrix}\\
    u_{-}(\mathbf{p})=& \begin{pmatrix}
        - (\hat{p}_1 - i\hat{p}_2) \sqrt{\frac{E_\mathbf{p} - m}{2}} \\ \sqrt{\frac{E_\mathbf{p} + m}{2}} + \hat{p}_3 \sqrt{\frac{E_\mathbf{p} - m}{2}} \\ 
        (\hat{p}_1 - i\hat{p}_2) \sqrt{\frac{E_\mathbf{p} - m}{2}} \\ \sqrt{\frac{E_\mathbf{p} + m}{2}} - \hat{p}_3 \sqrt{\frac{E_\mathbf{p} - m}{2}}
    \end{pmatrix}\\
    v_{+}(\mathbf{p})=& \begin{pmatrix}
        - (\hat{p}_1 - i\hat{p}_2) \sqrt{\frac{E_\mathbf{p} - m}{2}} \\ \sqrt{\frac{E_\mathbf{p} + m}{2}} + \hat{p}_3 \sqrt{\frac{E_\mathbf{p} - m}{2}} \\ 
        (\hat{p}_1 - i\hat{p}_2) \sqrt{\frac{E_\mathbf{p} - m}{2}} \\ -\left(\sqrt{\frac{E_\mathbf{p} + m}{2}} - \hat{p}_3 \sqrt{\frac{E_\mathbf{p} - m}{2}}\right)
    \end{pmatrix}\\
    v_{-}(\mathbf{p})=& \begin{pmatrix}
        \sqrt{\frac{E_\mathbf{p} + m}{2}} - \hat{p}_3 \sqrt{\frac{E_\mathbf{p} - m}{2}} \\ - (\hat{p}_1 + i\hat{p}_2) \sqrt{\frac{E_\mathbf{p} - m}{2}} \\ 
        -\left(\sqrt{\frac{E_\mathbf{p} + m}{2}} + \hat{p}_3 \sqrt{\frac{E_\mathbf{p} - m}{2}}\right) \\ -(\hat{p}_1 + i\hat{p}_2) \sqrt{\frac{E_\mathbf{p} - m}{2}}
    \end{pmatrix}
\end{align}
In the large energy limit $E_\mathbf{p} \gg m$, we have
\begin{align}
    \sqrt{\frac{E_\mathbf{p} + m}{2}} \approx \sqrt{\frac{E_\mathbf{p}}{2}}, \quad \sqrt{\frac{E_\mathbf{p} - m}{2}} \approx \sqrt{\frac{E_\mathbf{p}}{2}}
\end{align}
Thus, the expressions for $u_s(\mathbf{p})$ and $v_s(\mathbf{p})$ simplify to
\begin{align}
    u_{+}(\mathbf{p}) \approx& \sqrt{\frac{E_\mathbf{p}}{2}} \begin{pmatrix}
        1 - \hat{p}_3 \\ - (\hat{p}_1 + i\hat{p}_2) \\ 
        1 + \hat{p}_3 \\ (\hat{p}_1 + i\hat{p}_2)
    \end{pmatrix}\\
    u_{-}(\mathbf{p}) \approx& \sqrt{\frac{E_\mathbf{p}}{2}} \begin{pmatrix}
        - (\hat{p}_1 - i\hat{p}_2) \\ 1 + \hat{p}_3 \\ 
        (\hat{p}_1 - i\hat{p}_2) \\ 1 - \hat{p}_3
    \end{pmatrix}\\
    v_{+}(\mathbf{p}) \approx& \sqrt{\frac{E_\mathbf{p}}{2}} \begin{pmatrix}
        - (\hat{p}_1 - i\hat{p}_2) \\ 1 + \hat{p}_3 \\ 
        (\hat{p}_1 - i\hat{p}_2) \\ - (1 - \hat{p}_3)
    \end{pmatrix}\\
    v_{-}(\mathbf{p}) \approx& \sqrt{\frac{E_\mathbf{p}}{2}} \begin{pmatrix}
        1 - \hat{p}_3 \\ - (\hat{p}_1 + i\hat{p}_2) \\ 
        - (1 + \hat{p}_3) \\ -(\hat{p}_1 + i\hat{p}_2)
    \end{pmatrix}
\end{align}\qed


\clearpage
\question{3}{45.2}\\
Use the Feynman rules to write down (at tree level) $i\mathcal{T}$ for the processes: $e^+e^+\to e^+e^+$ and $\varphi\varphi\to e^+e^-\varphi$. \\
\textbf{Remark:} Do not write $\varphi\varphi\to e^+e^-$. Also, please draw Feynman diagrams when doing this problem. Remember the Lagrangian is
\begin{align}
    \mathcal{L}= -\frac{1}{2} \partial_\mu \varphi \partial^\mu \varphi - \frac{1}{2} m^2_\varphi \varphi^2 + \overline{\Psi}(i  \slashed{\partial} - m)\Psi + g\varphi \overline{\Psi}\Psi.
\end{align}
\answer{}
\begin{figure}[!h]
    \centering
    % t-channel
    \begin{subfigure}{0.48\textwidth} % 佔據文本寬度的 48%
        \centering
        \begin{tikzpicture}
        \begin{feynman}
            % 外部粒子
            \vertex (i1) at (-2,1) {\(p_1\)};
            \vertex (i2) at (-2,-1) {\(p_2\)};
            \vertex (f1) at ( 2,1) {\(k_1\)};
            \vertex (f2) at ( 2,-1) {\(k_2\)};

            % 頂點
            \vertex (v1) at (0,+0.5);
            \vertex (v2) at (0,-0.5);
            
            % 繪製傳播子 (t-channel: 粒子直線傳播)
            \diagram*{
                (f1) -- [fermion] (v1) -- [fermion] (i1), % p1 -> k1
                (f2) -- [fermion] (v2) -- [fermion] (i2), % p2 -> k2
                (v1) -- [scalar,momentum={\(q_t=p_{1}-k_{1}\)}] (v2), % 中間傳播子
            };
            
            \node[above=3pt of v1] {\(ig\)};
            \node[below=3pt of v2] {\(ig\)};
        \end{feynman}
        \end{tikzpicture}
        \caption{t-channel diagram.}
        \label{HW6:pairs-ee-ee-t-channel}
    \end{subfigure}
    %\hfill % 在兩個子圖形之間插入最大的水平空間
    % u-channel
    \begin{subfigure}{0.48\textwidth}
        \centering
        \begin{tikzpicture}
        \begin{feynman}
            % 外部粒子 (External Particles)
            \vertex (i1) at (-2,1) {\(p_1\)};
            \vertex (i2) at (-2,-1) {\(p_2\)};
            \vertex (f1) at ( 2.5,1) {\(k_1\)};
            \vertex (f2) at ( 2.5,-1) {\(k_2\)};

            % 頂點 (Vertices)
            \vertex (v1) at (0,+0.5);
            \vertex (v2) at (0,-0.5);
            
            % 繪製傳播子 (u-channel: 粒子交叉傳播)
            \diagram*{
                (f1) -- [fermion] (v1) -- [fermion] (i2), % p1 -> k2
                (f2) -- [fermion] (v2) -- [fermion] (i1), % p2 -> k1
                (v2) -- [scalar, momentum'={\(q_u=p_{1}-k_{2}\)}] (v1), % 中間傳播子
            };
            
            % 標記耦合常數
            \node[above=3pt of v1] {\(ig\)};
            \node[below=3pt of v2] {\(ig\)};
        \end{feynman}
        \end{tikzpicture}
        \caption{u-channel diagram.}
        \label{HW6:pairs-ee-ee-u-channel}
    \end{subfigure}
    \caption{Feynman diagrams for $e^+e^+\to e^+e^+$ at tree level (t and u channels).}
    \label{HW6:fig:e^+e^+_to_e^+e^+_channels}
\end{figure}

\begin{figure}[!h]
    \centering
    % 第一個子圖
    \begin{subfigure}{0.32\textwidth} % 佔據文本寬度的 48%
        \centering
        \begin{tikzpicture}
        \begin{feynman}
            % 外部粒子
            \vertex (i1) at (-2,1) {\(p_1\)};
            \vertex (i2) at (-2,-1) {\(p_2\)};
            \vertex (f1) at ( 2,0.5) {\(k_1\)};
            \vertex (f2) at ( 2,-1) {\(k_2\)};
            \vertex (phi3) at ( 2,1.5) {\(k_3\)};

            % 頂點
            \vertex (v1) at (0,1);
            \vertex (v2) at (0,-1);
            \vertex (v3) at (1,1);
            
            % 繪製線
            \diagram*{
                (i1) -- [scalar] (v1) ,
                (f2)-- [fermion] (v2)--[fermion,momentum= {\(q_2=p_{2}-k_{2}\)}] (v1)--[fermion,momentum= {\(q_1=k_1+k_3\)}] (v3)--[fermion] (f1),
                (i2) -- [scalar] (v2) ,
                (v3) -- [scalar] (phi3),
                };
        
            \node[above=3pt of v1] ;
            \node[below=3pt of v2] ;
            \node[above=3pt of v3] ;
        \end{feynman}
        \end{tikzpicture}
        \caption{Diagram 1 for $\varphi\varphi\to e^+e^-\varphi$.}
        \label{HW6:phiphi-to-pair-e-phi-diagram1}
    \end{subfigure}
    \hfill % 在兩個子圖形之間插入最大的水平空間
    % 第二個子圖
    \begin{subfigure}{0.32\textwidth}
        \centering
        \begin{tikzpicture}
        \begin{feynman}
            % 外部粒子
            \vertex (i1) at (-2,1) {\(p_1\)};
            \vertex (i2) at (-2,-1) {\(p_2\)};
            \vertex (f1) at ( 2,0.5) {\(k_2\)};
            \vertex (f2) at ( 2,-1) {\(k_1\)};
            \vertex (phi3) at ( 2,1.5) {\(k_3\)};

            % 頂點
            \vertex (v1) at (0,1);
            \vertex (v2) at (0,-1);
            \vertex (v3) at (1,1);
            
            % 繪製線
            \diagram*{
                (i1) -- [scalar] (v1) ,
                (f1) -- [fermion] (v3)--[fermion, momentum'={\(q_2=-k_2-k_3\)}] (v1)--[fermion, momentum'={\(q_1=k_1-p_2\)}] (v2)-- [fermion] (f2),
                (i2) -- [scalar] (v2) ,
                (v3) -- [scalar] (phi3),
                };
        
            \node[above=3pt of v1] ;
            \node[below=3pt of v2] ;
            \node[above=3pt of v3] ;
        \end{feynman}
        \end{tikzpicture}
        \caption{Diagram 2 for $\varphi\varphi\to e^+e^-\varphi$.}
        \label{HW6:phiphi-to-pair-e-phi-diagram2}
    \end{subfigure}
    \hfill % 在兩個子圖形之間插入最大的水平空間
\begin{subfigure}{0.32\textwidth}
        \centering
        \begin{tikzpicture}
        \begin{feynman}
            % 外部粒子
            \vertex (i1) at (-2,1) {\(p_1\)};
            \vertex (i2) at (-2,-1) {\(p_2\)};
            \vertex (f1) at ( 2,1) {\(k_2\)};
            \vertex (f2) at ( 2,-1) {\(k_1\)};
            \vertex (phi3) at ( 2,0) {\(k_3\)};

            % 頂點
            \vertex (v1) at (0,1);
            \vertex (v2) at (0,-1);
            \vertex (v3) at (0,0);
            
            % 繪製線
            \diagram*{
                (i1) -- [scalar] (v1) ,
                (f1) -- [fermion] (v1)--[fermion,momentum'={\(q_2=p_1-k_2\)}] (v3)--[fermion,momentum'={\(q_1=k_1-p_2\)}] (v2)-- [fermion] (f2),
                (i2) -- [scalar] (v2) ,
                (v3) -- [scalar] (phi3),
                };
        
            \node[above=3pt of v1] ;
            \node[below=3pt of v2] ;
            \node[left=3pt of v3] ;
        \end{feynman}
        \end{tikzpicture}
        \caption{Diagram 3 for $\varphi\varphi\to e^+e^-\varphi$.}
        \label{HW6:phiphi-to-pair-e-phi-diagram3}
    \end{subfigure}\\
    \begin{subfigure}{0.32\textwidth} % 佔據文本寬度的 48%
        \centering
        \begin{tikzpicture}
        \begin{feynman}
            % 外部粒子
            \vertex (i1) at (-2,1) {\(p_2\)};
            \vertex (i2) at (-2,-1) {\(p_1\)};
            \vertex (f1) at ( 2,0.5) {\(k_1\)};
            \vertex (f2) at ( 2,-1) {\(k_2\)};
            \vertex (phi3) at ( 2,1.5) {\(k_3\)};

            % 頂點
            \vertex (v1) at (0,1);
            \vertex (v2) at (0,-1);
            \vertex (v3) at (1,1);
            
            % 繪製線
            \diagram*{
                (i1) -- [scalar] (v1) ,
                (f2)-- [fermion] (v2)--[fermion,momentum= {\(q_2=p_{1}-k_{2}\)}] (v1)--[fermion,momentum= {\(q_1=k_1+k_3\)}] (v3)--[fermion] (f1),
                (i2) -- [scalar] (v2) ,
                (v3) -- [scalar] (phi3),
                };
        
            \node[above=3pt of v1] ;
            \node[below=3pt of v2] ;
            \node[above=3pt of v3] ;
        \end{feynman}
        \end{tikzpicture}
        \caption{Diagram 1 for $\varphi\varphi\to e^+e^-\varphi$.}
        \label{HW6:phiphi-to-pair-e-phi-diagram4}
    \end{subfigure}
    \hfill % 在兩個子圖形之間插入最大的水平空間
    % 第二個子圖
    \begin{subfigure}{0.32\textwidth}
        \centering
        \begin{tikzpicture}
        \begin{feynman}
            % 外部粒子
            \vertex (i1) at (-2,1) {\(p_2\)};
            \vertex (i2) at (-2,-1) {\(p_1\)};
            \vertex (f1) at ( 2,0.5) {\(k_2\)};
            \vertex (f2) at ( 2,-1) {\(k_1\)};
            \vertex (phi3) at ( 2,1.5) {\(k_3\)};   
            % 頂點
            \vertex (v1) at (0,1);
            \vertex (v2) at (0,-1);
            \vertex (v3) at (1,1);  
            % 繪製線
            \diagram*{
                (i1) -- [scalar] (v1) ,
                (f1) -- [fermion] (v3)--[fermion, momentum'={\(q_2=-k_2-k_3\)}] (v1)--[fermion, momentum'={\(q_1=k_1-p_1\)}] (v2)-- [fermion] (f2),
                (i2) -- [scalar] (v2) ,
                (v3) -- [scalar] (phi3),
                };

            \node[above=3pt of v1] ;
            \node[below=3pt of v2] ;
            \node[above=3pt of v3] ;
        \end{feynman}
        \end{tikzpicture}
        \caption{Diagram 2 for $\varphi\varphi\to e^+e^-\varphi$.}
        \label{HW6:phiphi-to-pair-e-phi-diagram5}
    \end{subfigure}
    \hfill % 在兩個子圖形之間插入最大的水平空間
\begin{subfigure}{0.32\textwidth}
        \centering
        \begin{tikzpicture}
        \begin{feynman}
            % 外部粒子
            \vertex (i1) at (-2,1) {\(p_2\)};
            \vertex (i2) at (-2,-1) {\(p_1\)};
            \vertex (f1) at ( 2,1) {\(k_2\)};
            \vertex (f2) at ( 2,-1) {\(k_1\)};
            \vertex (phi3) at ( 2,0) {\(k_3\)};
            % 頂點
            \vertex (v1) at (0,1);
            \vertex (v2) at (0,-1);
            \vertex (v3) at (0,0);
            % 繪製線
            \diagram*{
                (i1) -- [scalar] (v1) ,
                (f1) -- [fermion] (v1)--[fermion,momentum'={\(q_2=p_2-k_2\)}] (v3)--[fermion,momentum'={\(q_1=k_1-p_1\)}] (v2)-- [fermion] (f2),
                (i2) -- [scalar] (v2) ,
                (v3) -- [scalar] (phi3),
                };
            
            \node[above=3pt of v1] ;
            \node[below=3pt of v2] ;
            \node[left=3pt of v3] ;
        \end{feynman}
        \end{tikzpicture}
        \caption{Diagram 3 for $\varphi\varphi\to e^+e^-\varphi$.}
        \label{HW6:phiphi-to-pair-e-phi-diagram6}
    \end{subfigure}
    \caption{Two Feynman diagrams for $\varphi\varphi\to e^+e^-\varphi$ at tree level.}
    \label{HW6:phiphi-to-pair-e-phi}
\end{figure}
\begin{itemize}
    \item  \textbf{For the process $ e^+ e^+ \to e^+ e^+ $:}
\end{itemize}
The tree-level amplitude for the process $ e^+ e^+ \to e^+ e^+ $ consists of two diagrams: the t-channel and u-channel exchanges of a scalar particle. The total amplitude is given by the sum of the contributions from both channels. The amplitude for the t-channel diagram (Figure~\ref{HW6:pairs-ee-ee-t-channel}) is 
\begin{align}
    i\mathcal{T}_t = (ig)^2 \left[\overline{v}(p_2)  v(k_2)\right]\frac{-i}{(p_1 - k_1)^2 + m_\varphi^2 - i\epsilon} \left[\overline{v}(p_1) v(k_1)\right].
\end{align}
Similarly, the amplitude for the u-channel diagram (Figure~\ref{HW6:pairs-ee-ee-u-channel}) is
\begin{align}
    i\mathcal{T}_u = (ig)^2 \left[\overline{v}(p_1)  v(k_2)\right]\frac{-i}{(p_1 - k_2)^2 + m_\varphi^2 - i\epsilon} \left[\overline{v}(p_2) v(k_1)\right]. 
\end{align}
Thus, the total amplitude for the process \( e^+ e^+ \to e^+ e^+ \) is
\begin{align}
    &i\mathcal{T} = i\mathcal{T}_t - i\mathcal{T}_u\\
    =&(ig^2) \left[\overline{v}(p_2)  v(k_2)\right]\frac{-i}{(p_1 - k_1)^2 + m_\varphi^2 - i\epsilon} \left[\overline{v}(p_1) v(k_1)\right] \\
    - &(ig^2) \left[\overline{v}(p_1)  v(k_2)\right]\frac{-i}{(p_1 - k_2)^2 + m_\varphi^2 - i\epsilon} \left[\overline{v}(p_2) v(k_1)\right].
\end{align}
The minus sign arises due to the antisymmetry of the fermionic wavefunctions under exchange.
\begin{itemize}
    \item  \textbf{For the process $\varphi \varphi \to e^+ e^- \varphi$:}
\end{itemize}
The tree-level amplitude for the process $\varphi \varphi \to e^+ e^- \varphi$ consists of three diagrams, as shown in Figure~\ref{HW6:phiphi-to-pair-e-phi}. The total amplitude is given by the sum of the contributions from all three diagrams. The amplitude for Diagram 1 (Figure~\ref{HW6:phiphi-to-pair-e-phi-diagram1}) is
\begin{align}
    i\mathcal{T}_1 = (ig)^3 \left[\overline{u}(k_1) \frac{-i(-\slashed{q}_1 + m)}{q_1^2 + m^2 - i\epsilon} \frac{-i(-\slashed{q}_2 + m)}{q_2^2 + m^2 - i\epsilon} v(k_2)\right],
\end{align}
where $q_1=k_1+k_3$ and $q_2=p_2 - k_2$. The amplitude for Diagram 2 (Figure~\ref{HW6:phiphi-to-pair-e-phi-diagram2}) is
\begin{align}
    i\mathcal{T}_2 = (ig)^3 \left[\overline{u}(k_1) \frac{-i(-\slashed{q}_1 + m)}{q_1^2 + m^2 - i\epsilon} \frac{-i(-\slashed{q}_2 + m)}{q_2^2 + m^2 - i\epsilon} v(k_2)\right],
\end{align}
where $q_1=k_1 - p_2$ and $q_2=-k_2 - k_3$. The amplitude for Diagram 3 (Figure~\ref{HW6:phiphi-to-pair-e-phi-diagram3}) is
\begin{align}
    i\mathcal{T}_3 = (ig)^3 \left[\overline{u}(k_1) \frac{-i(-\slashed{q}_1 + m)}{q_1^2 + m^2 - i\epsilon} \frac{-i(-\slashed{q}_2 + m)}{q_2^2 + m^2 - i\epsilon} v(k_2)\right],
\end{align}
where $q_1=k_1 - p_2$ and $q_2=p_1 - k_2$. The amplitude for Diagram 4 (Figure~\ref{HW6:phiphi-to-pair-e-phi-diagram4}) is
\begin{align}
    i\mathcal{T}_4 = (ig)^3 \left[\overline{u}(k_1) \frac{-i(-\slashed{q}_1 + m)}{q_1^2 + m^2 - i\epsilon} \frac{-i(-\slashed{q}_2 + m)}{q_2^2 + m^2 - i\epsilon} v(k_2)\right],
\end{align}
where $q_1=k_1+k_3$ and $q_2=p_1 - k_2$. The amplitude for Diagram 5 (Figure~\ref{HW6:phiphi-to-pair-e-phi-diagram5}) is
\begin{align}
    i\mathcal{T}_5 = (ig)^3 \left[\overline{u}(k_1) \frac{-i(-\slashed{q}_1 + m)}{q_1^2 + m^2 - i\epsilon} \frac{-i(-\slashed{q}_2 + m)}{q_2^2 + m^2 - i\epsilon} v(k_2)\right],
\end{align}
where $q_1=k_1 - p_1$ and $q_2=-k_2 - k_3$. The amplitude for Diagram 6 (Figure~\ref{HW6:phiphi-to-pair-e-phi-diagram6}) is
\begin{align}
    i\mathcal{T}_6 = (ig)^3 \left[\overline{u}(k_1) \frac{-i(-\slashed{q}_1 + m)}{q_1^2 + m^2 - i\epsilon} \frac{-i(-\slashed{q}_2 + m)}{q_2^2 + m^2 - i\epsilon} v(k_2)\right],
\end{align}
where $q_1=k_1 - p_1$ and $q_2=p_2 - k_2$. Thus, the total amplitude for the process $\varphi \varphi \to e^+ e^- \varphi$ is
\begin{align}
    &i\mathcal{T} = i\mathcal{T}_1 + i\mathcal{T}_2 + i\mathcal{T}_3+ i\mathcal{T}_4 + i\mathcal{T}_5 + i\mathcal{T}_6 \\
    =& (ig)^3 \left[\overline{u}(k_1) \frac{-i(-\slashed{k}_1 - \slashed{k}_3 + m)}{(k_1 + k_3)^2 + m^2 - i\epsilon} \frac{-i(-\slashed{p}_2 + \slashed{k}_2 + m)}{(p_2 - k_2)^2 + m^2 - i\epsilon} v(k_2)\right] \\
    +& (ig)^3 \left[\overline{u}(k_1) \frac{-i(-\slashed{k}_1 + \slashed{p}_2 + m)}{(k_1 - p_2)^2 + m^2 - i\epsilon} \frac{-i(\slashed{k}_2 + \slashed{k}_3 + m)}{(-k_2 - k_3)^2 + m^2 - i\epsilon} v(k_2)\right] \\
    +& (ig)^3 \left[\overline{u}(k_1) \frac{-i(-\slashed{k}_1 + \slashed{p}_2 + m)}{(k_1 - p_2)^2 + m^2 - i\epsilon} \frac{-i(-\slashed{p}_1 + \slashed{k}_2 + m)}{(p_1 - k_2)^2 + m^2 - i\epsilon} v(k_2)\right]\\
    +& (ig)^3 \left[\overline{u}(k_1) \frac{-i(-\slashed{k}_1 - \slashed{k}_3 + m)}{(k_1 + k_3)^2 + m^2 - i\epsilon} \frac{-i(-\slashed{p}_1 + \slashed{k}_2 + m)}{(p_1 - k_2)^2 + m^2 - i\epsilon} v(k_2)\right] \\
    +& (ig)^3 \left[\overline{u}(k_1) \frac{-i(-\slashed{k}_1 + \slashed{p}_1 + m)}{(k_1 - p_1)^2 + m^2 - i\epsilon} \frac{-i(\slashed{k}_2 + \slashed{k}_3 + m)}{(-k_2 - k_3)^2 + m^2 - i\epsilon} v(k_2)\right] \\
    +& (ig)^3 \left[\overline{u}(k_1) \frac{-i(-\slashed{k}_1 + \slashed{p}_1 + m)}{(k_1 - p_1)^2 + m^2 - i\epsilon} \frac{-i(-\slashed{p}_2 + \slashed{k}_2 + m)}{(p_2 - k_2)^2 + m^2 - i\epsilon} v(k_2)\right].
\end{align}
\qed



\clearpage
\question{4}{48.2}\\
Compute $\langle|\mathcal{T}|^2\rangle$ for $e^+e^-\to \varphi\varphi$. You should find that your result is the same as that for $e^-\varphi\to e^-\varphi$, but with $s\leftrightarrow  t$, and an extra overall minus sign. This relationship is known as \textit{crossing symmetry}. There is an overall minus sign for each fermion that is moved from the initial to the final state.\\
\textbf{Remark:} Please compute for $e^-\varphi\to e^-\varphi$, do not compute for $e^+e^-\to \varphi\varphi$. Please also draw Feynman diagrams. Remember the Lagrangian is
\begin{align}
    \mathcal{L}= -\frac{1}{2} \partial_\mu \varphi \partial^\mu \varphi - \frac{1}{2} m^2_\varphi \varphi^2 + \overline{\Psi}(i  \slashed{\partial} - m)\Psi + g\varphi \overline{\Psi}\Psi.
\end{align}
\answer{}
\begin{figure}[!h]
    \centering
    % t-channel
    \begin{subfigure}{0.48\textwidth} % 佔據文本寬度的 48%
        \centering
        \begin{tikzpicture}
        \begin{feynman}
            % 外部粒子
            \vertex (i1) at (-2,1) {\(p_1\)};
            \vertex (i2) at (-2,-1) {\(k_1\)};
            \vertex (f1) at ( 2,1) {\(p_2\)};
            \vertex (f2) at ( 2,-1) {\(k_2\)};

            % 頂點
            \vertex (v1) at (-1,0);
            \vertex (v2) at (1,0);
            
            % 繪製傳播子 (t-channel: 粒子直線傳播)
            \diagram*{
                (i1)--[fermion](v1), 
                (i2)--[scalar](v1),
                (v2)--[fermion](f1),
                (v2)--[scalar](f2), 
                (v1) -- [fermion,momentum={\(q_s=p_{1}+k_{1}\)}] (v2), % 中間傳播子
            };
            
            \node[left=5pt of v1] {\(ig\)};
            \node[right=5pt of v2] {\(ig\)};
        \end{feynman}
        \end{tikzpicture}
        \caption{s-channel diagram.}
        \label{HW6:ephi-ephi-s-channel}
    \end{subfigure}
    %\hfill % 在兩個子圖形之間插入最大的水平空間
    % u-channel
    \begin{subfigure}{0.48\textwidth}
        \centering
        \begin{tikzpicture}
        \begin{feynman}
            % 外部粒子 (External Particles)     
            \vertex (i1) at (-2,1) {\(p_1\)};
            \vertex (i2) at (-2,-1) {\(k_1\)};
            \vertex (f1) at ( 2,1) {\(p_2\)};
            \vertex (f2) at ( 2,-1) {\(k_2\)};
            % 頂點 (Vertices)
            \vertex (v1) at (0,0.5);
            \vertex (v2) at (0,-0.5);
            % 繪製傳播子 (u-channel: 粒子交叉傳播)
            \diagram*{
                (i1)--[fermion](v1), 
                (i2)--[scalar](v2),
                (v2)--[fermion](f1),
                (v1)--[scalar](f2), 
                (v1) -- [fermion, momentum'={\(q_u=p_{1}-k_{2}\)}] (v2), % 中間傳播子
            };

            % 標記耦合常數
            \node[above=5pt of v1] {\(ig\)};
            \node[below=5pt of v2] {\(ig\)};
        \end{feynman}
        \end{tikzpicture}
        \caption{u-channel diagram.}
        \label{HW6:ephi-ephi-u-channel}
    \end{subfigure}
    \caption{Feynman diagrams for $e^-\varphi\to e^-\varphi$ at tree level (s and u channels).}
    \label{HW6:ephi-ephi}
\end{figure}
The tree-level amplitude for the process $ e^- \varphi \to e^- \varphi $ consists of two diagrams: the s-channel and u-channel exchanges of a fermion. The total amplitude is given by the sum of the contributions from both channels. The amplitude for the s-channel diagram (Figure~\ref{HW6:ephi-ephi-s-channel}) is
\begin{align}
    i\mathcal{T}_s = (ig)^2 \left[\overline{u}(p_2) \frac{-i(-\slashed{q}_s + m)}{q_s^2 + m^2 - i\epsilon} u(p_1)\right],
\end{align}
where $q_s = p_1 + k_1$. Similarly, the amplitude for the u-channel diagram (Figure~\ref{HW6:ephi-ephi-u-channel}) is
\begin{align}
    i\mathcal{T}_u = (ig)^2 \left[\overline{u}(p_2) \frac{-i(-\slashed{q}_u + m)}{q_u^2 + m^2 - i\epsilon} u(p_1)\right],
\end{align}
where $q_u = p_1 - k_2$. Thus, the total amplitude for the process $e^- \varphi \to e^- \varphi$ is 
\begin{align}
    &i\mathcal{T} = i\mathcal{T}_s + i\mathcal{T}_u\\
    =&(ig^2) \left[\overline{u}(p_2) \frac{-i(-\slashed{q}_s + m)}{q_s^2 + m^2 - i\epsilon} u(p_1)\right] \\
    + &(ig^2) \left[\overline{u}(p_2) \frac{-i(-\slashed{q}_u + m)}{q_u^2 + m^2 - i\epsilon} u(p_1)\right].
\end{align}
To compute $\langle|\mathcal{T}|^2\rangle$, we average over initial spins and sum over final spins:
\begin{align}
    \langle|\mathcal{T}|^2\rangle = \frac{1}{2} \sum_{\text{spins}} |\mathcal{T}|^2.
\end{align}
We define $\mathcal{A}$ as
\begin{align}
    \mathcal{A} &=  \left[\overline{u}_{s_2}(p_2) \frac{(-\slashed{q}_s + m)}{q_s^2 + m^2} u_{s_1}(p_1)\right]+  \left[\overline{u}_{s_2}(p_2) \frac{(-\slashed{q}_u + m)}{q_u^2 + m^2} u_{s_1}(p_1)\right]\\
    &=\left[ \overline{u}_{s_2}(p_2) \left(\frac{-\slashed{q}_s + m}{q_s^2 + m^2} + \frac{-\slashed{q}_u + m}{q_u^2 + m^2}\right) u_{s_1}(p_1)\right]\\
    &=\left[ \overline{u}_{s_2}(p_2) \left(\frac{-(\slashed{p}_1 + \slashed{k}_1) + m}{(p_1 + k_1)^2 + m^2} + \frac{-(\slashed{p}_1 - \slashed{k}_2) + m}{(p_1 - k_2)^2 + m^2}\right) u_{s_1}(p_1)\right]\\
    &=\left[ \overline{u}_{s_2}(p_2) \left(\frac{-\slashed{p}_1 - \slashed{k}_1 + m}{-s + m^2} + \frac{-\slashed{p}_1 + \slashed{k}_2 + m}{-u + m^2}\right) u_{s_1}(p_1)\right]\\
    &=\left[ \overline{u}_{s_2}(p_2) \left(\frac{ - \slashed{k}_1 + 2m}{-s + m^2} + \frac{ + \slashed{k}_2 + 2m}{-u + m^2}\right) u_{s_1}(p_1)\right]
\end{align}
and its Hermitian conjugate $\mathcal{A}^\dagger$ as
\begin{align}
    \mathcal{A}^\dagger &=  \left[\overline{u}_{s_1}(p_1) \frac{-\slashed{k}_1 + 2m}{-s + m^2} + \frac{ + \slashed{k}_2 + 2m}{-u + m^2} u_{s_2}(p_2)\right]
\end{align}
where $s_1$ and $s_2$ are the spin indices for the initial and final electrons, respectively, and we have used the Mandelstam variables:
\begin{align}
    s = -(p_1 + k_1)^2, \quad u = -(p_1 - k_2)^2.
\end{align} 
Also, we apply the identity:
\begin{align}
    (\slashed{p} + m) u_s(p) = 0 \implies -\slashed{p} u_s(p) = +m u_s(p).
\end{align}
Then, we have
\begin{align}
    \langle|\mathcal{T}|^2\rangle = \frac{g^4}{2} \sum_{s_1,s_2} A A^\dagger.
\end{align}
Hence,
\begin{align}
    \sum_{s_1,s_2} \mathcal{A} \mathcal{A}^\dagger =& \sum_{s_1,s_2} \left[ \overline{u}_{s_2}(p_2) \left(\frac{ - \slashed{k}_1 + 2m}{-s + m^2} + \frac{ + \slashed{k}_2 + 2m}{-u + m^2}\right) u_{s_1}(p_1)\right] \\
    &\times \left[ \overline{u}_{s_1}(p_1) \left(\frac{- \slashed{k}_1 + 2m}{-s + m^2} + \frac{ + \slashed{k}_2 + 2m}{-u + m^2}\right) u_{s_2}(p_2)\right]\\
    =& \sum_{s_1,s_2} \Bigg[ \overline{u}_{s_2}(p_2) \left(\frac{ - \slashed{k}_1 + 2m}{-s + m^2} + \frac{ + \slashed{k}_2 + 2m}{-u + m^2}\right) u_{s_1}(p_1) \overline{u}_{s_1}(p_1) \\
    &\times \left(\frac{- \slashed{k}_1 + 2m}{-s + m^2} + \frac{ + \slashed{k}_2 + 2m}{-u + m^2}\right) u_{s_2}(p_2)\Bigg]\\
    =& \text{Tr}\Bigg[  \left(\frac{ - \slashed{k}_1 + 2m}{-s + m^2} + \frac{ + \slashed{k}_2 + 2m}{-u + m^2}\right) u_{s_1}(p_1) \overline{u}_{s_1}(p_1) \\
    &\times \left(\frac{- \slashed{k}_1 + 2m}{-s + m^2} + \frac{ + \slashed{k}_2 + 2m}{-u + m^2}\right) u_{s_2}(p_2)\overline{u}_{s_2}(p_2)\Bigg]\\
    =& \text{Tr}\Bigg[  \left(\frac{ - \slashed{k}_1 + 2m}{-s + m^2} + \frac{ + \slashed{k}_2 + 2m}{-u + m^2}\right) (-\slashed{p}_1 + m) \\
    &\times \left(\frac{- \slashed{k}_1 + 2m}{-s + m^2} + \frac{ + \slashed{k}_2 + 2m}{-u + m^2}\right) (-\slashed{p}_2 + m)\Bigg]
\end{align}
where we have used the completeness relation for spinors:
\begin{align}
    \sum_s u_s(p) \overline{u}_s(p) = -\slashed{p} + m.
\end{align}
Since only even numbers of gamma matrices contribute to the trace, we find
\begin{align}
    \sum_{s_1,s_2} \mathcal{A} \mathcal{A}^\dagger &=\frac{1}{(-s+m^2)^2} \text{Tr}\Big[ (- \slashed{k}_1 + 2m)(-\slashed{p}_1 + m)(- \slashed{k}_1 + 2m)(-\slashed{p}_2 + m)\Big] \\
    &+ \frac{1}{(-u+m^2)^2} \text{Tr}\Big[ ( + \slashed{k}_2 + 2m)(-\slashed{p}_1 + m)( + \slashed{k}_2 + 2m)(-\slashed{p}_2 + m)\Big] \\
    &+ \frac{1}{(-s+m^2)(-u+m^2)} \text{Tr}\Big[ (- \slashed{k}_1 + 2m)(-\slashed{p}_1 + m)( + \slashed{k}_2 + 2m)(-\slashed{p}_2 + m)\Big]\\
    &+\frac{1}{(-u+m^2)(-s+m^2)} \text{Tr}\Big[ ( + \slashed{k}_2 + 2m)(-\slashed{p}_1 + m)( - \slashed{k}_1 + 2m)(-\slashed{p}_2 + m)\Big]\\
    &=\frac{1}{(s - m^2)^2} \text{Tr}\Big[ (\slashed{k}_1 - 2m)(\slashed{p}_1 - m)(\slashed{k}_1 - 2m)(\slashed{p}_2 - m)\Big] \\
    &+ \frac{1}{(u - m^2)^2} \text{Tr}\Big[ ( \slashed{k}_2 + 2m)(\slashed{p}_1 - m)(  \slashed{k}_2 + 2m)(\slashed{p}_2 - m)\Big] \\
    &+ \frac{-1}{(s - m^2)(u - m^2)} \text{Tr}\Big[ (\slashed{k}_1 - 2m)(\slashed{p}_1 - m)( + \slashed{k}_2 + 2m)(\slashed{p}_2 - m)\Big]\\
    &+ \frac{-1}{(u - m^2)(s - m^2)} \text{Tr}\Big[ ( \slashed{k}_2 + 2m)(\slashed{p}_1 - m)( \slashed{k}_1 - 2m)(\slashed{p}_2 - m)\Big].
\end{align}
Now, we compute each trace term separately:
\begin{align}
    \text{First term: }&\text{Tr}\Big[ (\slashed{k}_1 - 2m)(\slashed{p}_1 - m)(\slashed{k}_1 - 2m)(\slashed{p}_2 - m)\Big] \\
    =& \text{Tr}\Big[ \slashed{k}_1 \slashed{p}_1 \slashed{k}_1 \slashed{p}_2 +4m^2 \slashed{p}_1 \slashed{p}_2 +m^2 \slashed{k}_1 \slashed{k}_1 +4m^2 \slashed{k}_1 \slashed{p}_2+4m^2 \slashed{p}_1 \slashed{k}_1 +4m^4 \Big]\\
    \text{Second term: }&\text{Tr}\Big[ ( \slashed{k}_2 + 2m)(\slashed{p}_1 - m)(  \slashed{k}_2 + 2m)(\slashed{p}_2 - m)\Big] \\
    =& \text{Tr}\Big[ \slashed{k}_2 \slashed{p}_1 \slashed{k}_2 \slashed{p}_2 +4m^2 \slashed{p}_1 \slashed{p}_2 +m^2 \slashed{k}_2 \slashed{k}_2 -4m^2 \slashed{k}_2 \slashed{p}_2 -4m^2 \slashed{p}_1 \slashed{k}_2 +4m^4 \Big]\\
    \text{Third term: }&\text{Tr}\Big[ (\slashed{k}_1 - 2m)(\slashed{p}_1 - m)(  \slashed{k}_2 + 2m)(\slashed{p}_2 - m)\Big] \\
    =& \text{Tr}\Big[ \slashed{k}_1 \slashed{p}_1 \slashed{k}_2 \slashed{p}_2 +m^2 \slashed{k}_1 \slashed{k}_2 -2m^2\slashed{k}_1 (\slashed{p}_1+\slashed{p}_2) +2m^2 \slashed{k}_2 (\slashed{p}_1+\slashed{p}_2) -4m^2 \slashed{p}_1 \slashed{p}_2 -4m^4 \Big]\\
    \text{Fourth term: }&\text{Tr}\Big[ ( \slashed{k}_2 + 2m)(\slashed{p}_1 - m)( \slashed{k}_1 - 2m)(\slashed{p}_2 - m)\Big] \\
    =& \text{Tr}\Big[ \slashed{k}_2 \slashed{p}_1 \slashed{k}_1 \slashed{p}_2 +m^2 \slashed{k}_2 \slashed{k}_1 -2m^2\slashed{k}_2 (\slashed{p}_1+\slashed{p}_2) +2m^2 \slashed{k}_1 (\slashed{p}_1+\slashed{p}_2) -4m^2 \slashed{p}_1 \slashed{p}_2 -4m^4 \Big].
\end{align}
Using the trace identities: 
\begin{align}
    \text{Tr}[\slashed{a}\slashed{b}\slashed{c}\slashed{d}] = 4[(ad)(bc)-(ac)(bd)+(ab)(cd)], \quad \text{Tr}[\slashed{a}\slashed{b}] = -4(ab), \quad \text{Tr}[\mathcal{I}]=4,
\end{align}
and the on-shell conditions $p_1^2 = p_2^2 = -m^2$ and $k_1^2=k_2^2=-m_\varphi^2$, we find
\begin{align}
    \text{First term: }&\text{Tr}\Big[ \slashed{k}_1 \slashed{p}_1 \slashed{k}_1 \slashed{p}_2 +4m^2 \slashed{p}_1 \slashed{p}_2 +m^2 \slashed{k}_1 \slashed{k}_1 +4m^2 \slashed{k}_1 \slashed{p}_2+4m^2 \slashed{p}_1 \slashed{k}_1 +4m^4 \Big]\\
    =& 4\Big[\textcolor{blue}{(k_1p_2)(p_1k_1)}-\textcolor{red}{(k_1k_1)(p_1p_2)}+\textcolor{blue}{(k_1p_1)(k_1p_2)}-\textcolor{red}{4m^2(p_1p_2)}\\
    -&m^2(k_1k_1)-4m^2(k_1p_2)-4m^2(p_1k_1)+4m^2 \Big]  \\
    =& 4\Big[2(k_1 p_2)(p_1 k_1) + (m_\varphi^2-4m^2) (p_1 p_2) +m^2 m_\varphi^2 -4m^2 (k_1 p_2)-4m^2 (p_1 k_1) +4m^4 \Big]\\
    \text{Second term: }&\text{Tr}\Big[ \slashed{k}_2 \slashed{p}_1 \slashed{k}_2 \slashed{p}_2 +4m^2 \slashed{p}_1 \slashed{p}_2 +m^2 \slashed{k}_2 \slashed{k}_2 -4m^2 \slashed{k}_2 \slashed{p}_2 -4m^2 \slashed{p}_1 \slashed{k}_2 +4m^4 \Big]\\
    =& 4\Big[\textcolor{blue}{(k_2p_2)(p_1k_2)}-\textcolor{red}{(k_2k_2)(p_1p_2)}+\textcolor{blue}{(k_2p_1)(k_2p_2)}-\textcolor{red}{4m^2(p_1p_2)}\\
    -&m^2(k_2k_2)+4m^2(k_2p_2)+4m^2(p_1k_2)+4m^2 \Big]  \\
    =& 4\Big[2(k_2 p_2)(p_1 k_2) + (m_\varphi^2-4m^2) (p_1 p_2) +m^2 m_\varphi^2 +4m^2 (k_2 p_2)+4m^2 (p_1 k_2) +4m^4 \Big]\\
    \text{Third term: }&\text{Tr}\Big[ \slashed{k}_1 \slashed{p}_1 \slashed{k}_2 \slashed{p}_2 +m^2 \slashed{k}_1 \slashed{k}_2 -2m^2\slashed{k}_1 (\slashed{p}_1+\slashed{p}_2) +2m^2 \slashed{k}_2 (\slashed{p}_1+\slashed{p}_2) -4m^2 \slashed{p}_1 \slashed{p}_2 -4m^4 \Big] \\
    =& 4\Big[(k_1p_2)(p_1k_2) -(k_1k_2)(p_1p_2)+ (k_1p_1)(k_2p_2) -m^2 (k_1k_2) +2m^2 (k_1(p_1+p_2)) \\
    &-2m^2 (k_2(p_1+p_2)) +4m^2 (p_1p_2) -4m^4 \Big]\\
    =&4\Big[ (k_1p_2)(p_1k_2) -(k_1k_2)(p_1p_2)+ (k_1p_1)(k_2p_2)-m^2 (k_1k_2) \\
    +&2m^2 (k_1p_1+k_1p_2-k_2p_1-k_2p_2) -4m^2 (p_1p_2) -4m^4\Big]\\
    \text{Fourth term: }&\text{Tr}\Big[ \slashed{k}_2 \slashed{p}_1 \slashed{k}_1 \slashed{p}_2 +m^2 \slashed{k}_2 \slashed{k}_1 -2m^2\slashed{k}_2 (\slashed{p}_1+\slashed{p}_2) +2m^2 \slashed{k}_1 (\slashed{p}_1+\slashed{p}_2) +4m^2 \slashed{p}_1 \slashed{p}_2 -4m^4 \Big] \\
    =& 4\Big[(k_2p_2)(p_1k_1) -(k_2k_1)(p_1p_2)+ (k_2p_1)(k_1p_2) -m^2 (k_2k_1) +2m^2 (k_2(p_1+p_2)) \\
    &-2m^2 (k_1(p_1+p_2)) +4m^2 (p_1p_2) -4m^4 \Big ]\\
    =&4\Big[ (k_2p_2)(p_1k_1) -(k_2k_1)(p_1p_2)+ (k_2p_1)(k_1p_2)-m^2 (k_2k_1) \\
    +&2m^2 (k_2p_1+k_2p_2-k_1p_1-k_1p_2) +4m^2 (p_1p_2) -4m^4\Big].
\end{align}
We can express the dot products in terms of the Mandelstam variables:
\begin{align}
    s = -(p_1 + k_1)^2&=-(p_2 + k_2)^2= m^2 + m_\varphi^2 - 2(p_1 k_1)= m^2 + m_\varphi^2 - 2(p_2 k_2), \\
    u = -(p_1 - k_2)^2&=-(p_2 - k_1)^2= m^2 + m_\varphi^2 + 2(k_2 p_1)= m^2 + m_\varphi^2 + 2(k_1 p_2), \\
    t = -(k_1 - k_2)^2&= -(p_1 - p_2)^2= 2m^2 + 2(p_1 p_2)= 2m_\varphi^2 + 2(k_1 k_2).
\end{align}
Thus, by $s+t+u = 2m^2 + 2m_\varphi^2$, we have
\begin{align}
    (p_1 k_1) &= \frac{m^2 + m_\varphi^2 - s}{2}, \\
    (p_2 k_2) &= \frac{m^2 + m_\varphi^2 - s}{2}, \\
    (k_2 p_1) &= \frac{u - m^2 - m_\varphi^2}{2}, \\
    (k_1 p_2) &= \frac{u - m^2 - m_\varphi^2}{2}, \\
    (p_1 p_2) &= \frac{t - 2m^2}{2}= \frac{-(s+u) + 2m_\varphi^2}{2}, \\
    (k_1 k_2) &= \frac{t - 2m_\varphi^2}{2}= \frac{-(s+u) + 2m^2}{2}.
\end{align}
Substituting these expressions back into the sum, we find (by Mathematica):
\begin{align}
    \sum_{s_1,s_2} \mathcal{A} \mathcal{A}^\dagger = 2\times&\Bigg[\frac{7 m^4+m^2 \left(-8 m_{\varphi}^2+9 s+u\right)+m_{\varphi}^4-s u}{\left(m^2-s\right)^2} \\
    &+ \frac{7 m^4+m^2 \left(-8 m_{\varphi}^2+s+9 u\right)+m_{\varphi}^4-s u}{\left(m^2-u\right)^2} \\
    &+\frac{2 \left(9 m^4+m^2 \left(3 (s+u)-8 m_{\varphi}^2\right)-m_{\varphi}^4+s u\right)}{\left(m^2-s\right) \left(m^2-u\right)}\Bigg].
\end{align}
Therefore, the final result for $\langle|\mathcal{T}|^2\rangle$ is
\begin{align}
    \langle|\mathcal{T}|^2\rangle =& \frac{g^4}{2} \sum_{s_1,s_2} \mathcal{A} \mathcal{A}^\dagger \\
    =& g^4 \Bigg[\frac{7 m^4+m^2 \left(-8 m_{\varphi}^2+9 s+u\right)+m_{\varphi}^4-s u}{\left(m^2-s\right)^2} \\
    &+ \frac{7 m^4+m^2 \left(-8 m_{\varphi}^2+s+9 u\right)+m_{\varphi}^4-s u}{\left(m^2-u\right)^2} \\
    &+\frac{2 \left(9 m^4+m^2 \left(3 (s+u)-8 m_{\varphi}^2\right)-m_{\varphi}^4+s u\right)}{\left(m^2-s\right) \left(m^2-u\right)}\Bigg].
\end{align}
\qed

