\section*{HW6 Due to December 4 11:59 PM}
\question{1}{Problem 36.3}\\
\begin{itemize}
    \item [(a)] Prove the \textit{Fierz identities}
    \begin{align}
        (\chi^\dagger_1 \overline{\sigma}^\mu \chi_2)  (\chi^\dagger_3 \overline{\sigma}_\mu \chi_4) &= -2 (\chi^\dagger_1 \chi_3^\dagger) (\chi_2 \chi_4) \tag{36.58}\\
        (\chi^\dagger_1 \overline{\sigma^\mu} \chi_2)  (\chi^\dagger_3 \overline{\sigma_\mu} \chi_4) &=  (\chi^\dagger_1 \overline{\sigma}^\mu \chi_4)  (\chi^\dagger_3 \overline{\sigma}_\mu \chi_2)\tag{36.59}
    \end{align}
    \item [(b)] Define the Dirac fields
    \begin{align}
        \Psi_1 \equiv \begin{pmatrix}
            \chi_i \\ \xi^\dagger_i
        \end{pmatrix}, \quad \Psi_i^C \equiv \begin{pmatrix}
            \xi_i \\ \chi^\dagger_i
        \end{pmatrix}\tag{36.60}
    \end{align}
    Use eqs.~(36.58) and (36.59) to prove the Dirac form of the Fierz identities,
    \begin{align}
        (\overline{\Psi}_1 \gamma^\mu P_L \Psi_2) (\overline{\Psi}_3 \gamma_\mu P_L \Psi_4) &= - 2 (\overline{\Psi}_1 P_R \Psi_3^C) (\overline{\Psi}_4^C P_L \Psi_2) \tag{36.61}\\
        (\overline{\Psi}_1 \gamma^\mu P_L \Psi_2) (\overline{\Psi}_3 \gamma_\mu P_L \Psi_4) &= (\overline{\Psi}_1 \gamma^\mu P_L \Psi_4) (\overline{\Psi}_3 \gamma_\mu P_L \Psi_2)\tag{36.62}
    \end{align}
    \item [(c)] By writing both sides out in terms of Weyl fields, show that
    \begin{align}
        \overline{\Psi}_1 \gamma^\mu P_R \Psi_2 &= -\overline{\Psi}_2^C \gamma^\mu P_L \Psi_1^C\tag{36.63}\\
        \overline{\Psi}_1  P_L \Psi_2 &= \overline{\Psi}_2^C P_L \Psi_1^C\tag{36.64}\\
        \overline{\Psi}_1  P_R \Psi_2 &= \overline{\Psi}_2^C P_R \Psi_1^C\tag{36.65}.
    \end{align}
    Combining equations~(36.63–36.65) with equations~(36.61–36.62) yields more useful forms of the Fierz identities.
\end{itemize}
\answer{}
\begin{itemize}
    \item [(a)]
\end{itemize}
We start from the left-hand side of equation~(36.58):
\begin{align}
    (\chi^\dagger_1 \overline{\sigma}^\mu \chi_2)  (\chi^\dagger_3 \overline{\sigma}_\mu \chi_4) &= (\chi^\dagger_1)_{\dot{a}} (\overline{\sigma}^\mu)^{\dot{a}b} (\chi_2)_b (\chi^\dagger_3)_{\dot{c}} (\overline{\sigma}_\mu)^{\dot{c}d} (\chi_4)_d \\
    &= (\chi^\dagger_1)_{\dot{a}}  (\chi_2)_b (\chi^\dagger_3)_{\dot{c}}  (\chi_4)_d (\overline{\sigma}^\mu)^{\dot{a}b}  (\overline{\sigma}_\mu)^{\dot{c}d}
\end{align}
Using the identity in equations~(35.4),(35.19)
\begin{align}
    (\sigma^{\mu})_{a\dot{a}}  (\sigma_{\mu})_{b\dot{b}} = -2 \epsilon_{ab} \epsilon_{\dot{a}\dot{b}}\tag{35.4}\\
    (\overline{\sigma}^\mu)^{\dot{a}a} \equiv \epsilon^{ab} \epsilon^{\dot{a}\dot{b}}  (\sigma_\mu)_{b\dot{b}}\tag{35.19}
\end{align}
we have
\begin{align}
    (\overline{\sigma}^\mu)^{\dot{a}b}  (\overline{\sigma}_\mu)^{\dot{c}d} &= \epsilon^{be} \epsilon^{\dot{a}\dot{f}}  (\sigma^\mu)_{e\dot{f}} \epsilon^{dg} \epsilon^{\dot{c}\dot{h}}  (\sigma_\mu)_{g\dot{h}} \\
    &= \epsilon^{be} \epsilon^{dg} \epsilon^{\dot{a}\dot{f}} \epsilon^{\dot{c}\dot{h}}  (\sigma^\mu)_{e\dot{f}}  (\sigma_\mu)_{g\dot{h}} \\
    &= -2 \epsilon^{be} \epsilon^{dg} \epsilon^{\dot{a}\dot{f}} \epsilon^{\dot{c}\dot{h}} \epsilon_{eg} \epsilon_{\dot{f}\dot{h}} \\
    &= -2 \epsilon^{be} \delta^d_e \epsilon^{\dot{a}\dot{f}} \delta^{\dot{c}}_{\dot{f}} \\
    &= -2 \epsilon^{bd} \epsilon^{\dot{a}\dot{c}}
\end{align}
Substituting this back, we get
\begin{align}
    (\chi^\dagger_1 \overline{\sigma}^\mu \chi_2)  (\chi^\dagger_3 \overline{\sigma}_\mu \chi_4) &= -2 (\chi^\dagger_1)_{\dot{a}}  (\chi_2)_b (\chi^\dagger_3)_{\dot{c}}  (\chi_4)_d \epsilon^{bd} \epsilon^{\dot{a}\dot{c}} \\
    &= -2 (\chi^\dagger_1)_{\dot{a}}  (\chi^\dagger_3)_{\dot{c}} \epsilon^{\dot{a}\dot{c}}  (\chi_2)_b  (\chi_4)_d \epsilon^{bd} \\
    &=-2 (\chi^\dagger_1)^{\dot{c}}  (\chi^\dagger_3)_{\dot{c}}  (\chi_2)_d  (\chi_4)^d \\
    &=-2 (\chi^\dagger_1 \chi_3^\dagger) (\chi_2 \chi_4)\times (-1)(-1)\\
    &=-2 (\chi^\dagger_1 \chi_3^\dagger) (\chi_2 \chi_4)
\end{align}
This proves equation~(36.58). Similarly, we can prove equation~(36.59):
\begin{align}
    (\chi^\dagger_1 \overline{\sigma}^\mu \chi_2)  (\chi^\dagger_3 \overline{\sigma}_\mu \chi_4) &=-2 (\chi^\dagger_1 \chi_3^\dagger) (\chi_2 \chi_4)\\
    &= -2 (\chi^\dagger_1 \chi_3^\dagger) (\chi_4 \chi_2)\\
    &= (\chi^\dagger_1 \overline{\sigma}^\mu \chi_4)  (\chi^\dagger_3 \overline{\sigma}_\mu \chi_2)
\end{align}
This proves equation~(36.59).
\begin{itemize}
    \item [(b)]
\end{itemize}
First $\Psi=\begin{pmatrix}
    \chi_a \\ (\xi^\dagger)^{\dot{a}}
\end{pmatrix}$, so $\overline{\Psi} = (\xi^a, (\chi^\dagger)_{\dot{a}})$. Also, $P_L \Psi = \begin{pmatrix}
    \chi_a \\ 0
\end{pmatrix}$ and $P_R \Psi = \begin{pmatrix}
    0 \\ (\xi^\dagger)^{\dot{a}}
\end{pmatrix}$. Thus,
\begin{align}
    \overline{\Psi}_1 \gamma^\mu P_L \Psi_2 &= (\xi^a_1, (\chi^\dagger_1)_{\dot{a}}) \begin{pmatrix}
        0 & \sigma^\mu_{a\dot{b}} \\ \overline{\sigma}^{\mu \dot{a} b} & 0
    \end{pmatrix} \begin{pmatrix}
        (\chi_2)_b \\ 0
    \end{pmatrix} \\
    &= \xi^a_1 \sigma^\mu_{a\dot{b}} (\chi_2)_b \\
    &= (\chi^\dagger_1 \overline{\sigma}^\mu \chi_2)
\end{align}
Similarly,
\begin{align}
    \overline{\Psi}_3 \gamma_\mu P_L \Psi_4 &= (\chi^\dagger_3 \overline{\sigma}_\mu \chi_4)
\end{align}
Therefore,
\begin{align}
    (\overline{\Psi}_1 \gamma^\mu P_L \Psi_2) (\overline{\Psi}_3 \gamma_\mu P_L \Psi_4) &= (\chi^\dagger_1 \overline{\sigma}^\mu \chi_2)  (\chi^\dagger_3 \overline{\sigma}_\mu \chi_4) \\
    &= -2 (\chi^\dagger_1 \chi_3^\dagger) (\chi_2 \chi_4) \tag{from (a)}
\end{align}
Now, for the right-hand side of equation~(36.61):
\begin{align}
    \overline{\Psi}_1 P_R \Psi_3^C &= (\xi^a_1, (\chi^\dagger_1)_{\dot{a}}) \begin{pmatrix}
        0 & 0 \\ 0 & 1
    \end{pmatrix} \begin{pmatrix}
        (\xi_3)_b \\ (\chi^\dagger_3)^{\dot{b}}
    \end{pmatrix} \\
    &= (\chi^\dagger_1)_{\dot{a}}  (\chi^\dagger_3)^{\dot{a}} \\
    &= (\chi^\dagger_1 \chi_3^\dagger)
\end{align}
Similarly,
\begin{align}
    \overline{\Psi}_4^C P_L \Psi_2 &= ( (\xi^a_4, (\chi^\dagger_4)_{\dot{a}}) \begin{pmatrix}
        1 & 0 \\ 0 & 0
    \end{pmatrix} \begin{pmatrix}
        (\chi_2)_b \\ (\xi^\dagger_2)^{\dot{b}}
    \end{pmatrix} \\
    &= (\xi^a_4)  (\chi_2)_a \\
    &= (\chi_4 \chi_2)= (\chi_2 \chi_4)
\end{align}
Thus,
\begin{align}
    -2 (\overline{\Psi}_1 P_R \Psi_3^C) (\overline{\Psi}_4^C P_L \Psi_2) &= -2 (\chi^\dagger_1 \chi_3^\dagger) (\chi_2 \chi_4)
\end{align}
This proves equation~(36.61). Similarly, we can prove equation~(36.62):
\begin{align}
    (\overline{\Psi}_1 \gamma^\mu P_L \Psi_2) (\overline{\Psi}_3 \gamma_\mu P_L \Psi_4) &= (\chi^\dagger_1 \overline{\sigma}^\mu \chi_2)  (\chi^\dagger_3 \overline{\sigma}_\mu \chi_4) \\
    &= (\chi^\dagger_1 \overline{\sigma}^\mu \chi_4)  (\chi^\dagger_3 \overline{\sigma}_\mu \chi_2) \tag{from (a)}\\
    &= (\overline{\Psi}_1 \gamma^\mu P_L \Psi_4) (\overline{\Psi}_3 \gamma_\mu P_L \Psi_2)
\end{align}
This proves equation~(36.62).
\begin{itemize}
    \item [(c)]
\end{itemize}
First, we compute the left-hand side of equation~(36.63):
\begin{align}
    \overline{\Psi}_1 \gamma^\mu P_R \Psi_2 &= (\xi^a_1, (\chi^\dagger_1)_{\dot{a}}) \begin{pmatrix}
        0 & \sigma^\mu_{a\dot{b}} \\ \overline{\sigma}^{\mu \dot{a} b} & 0
    \end{pmatrix} \begin{pmatrix}
        0 \\ (\xi^\dagger_2)^{\dot{b}}
    \end{pmatrix} \\
    &= \xi^a_1 \sigma^\mu_{a\dot{b}} (\xi^\dagger_2)^{\dot{b}} 
\end{align}
Next, we compute the right-hand side of equation~(36.63):
\begin{align}
    \overline{\Psi}_2^C \gamma^\mu P_L \Psi_1^C &= (\chi^a_2, (\xi^\dagger_2)_{\dot{a}}) \begin{pmatrix}
        0 & \sigma^\mu_{a\dot{b}} \\ \overline{\sigma}^{\mu \dot{a} b} & 0
    \end{pmatrix} \begin{pmatrix}
        (\xi_1)_b \\ 0
    \end{pmatrix} \\
    &= (\xi^\dagger_2)_{\dot{a}} \overline{\sigma}^{\mu \dot{a} b} (\xi_1)_b 
\end{align}
Using the identity 
\begin{align}
    (\overline{\sigma}^\mu)^{\dot{a}a} \equiv \epsilon^{ab} \epsilon^{\dot{a}\dot{b}}  (\sigma_\mu)_{b\dot{b}}\tag{35.19}
\end{align}
we have
\begin{align}
    (\xi^\dagger_2)_{\dot{a}} \overline{\sigma}^{\mu \dot{a} b} (\xi_1)_b &= (\xi^\dagger_2)_{\dot{a}} \epsilon^{bc} \epsilon^{\dot{a}\dot{b}}  (\sigma_\mu)_{c\dot{b}} (\xi_1)_b \\
    &= - \xi_1^c \sigma^\mu_{c\dot{b}} (\xi^\dagger_2)^{\dot{b}} \\
    &= - \xi^a_1 \sigma^\mu_{a\dot{b}} (\xi^\dagger_2)^{\dot{b}}\\
    &= \overline{\Psi}_2^C \gamma^\mu P_L \Psi_1^C= - \overline{\Psi}_1 \gamma^\mu P_R \Psi_2
\end{align}
This proves equation~(36.63). Similarly, we can prove equations~(36.64) and (36.65):
\begin{align}
    LHS=\overline{\Psi}_1  P_L \Psi_2 &= (\xi^a_1, (\chi^\dagger_1)_{\dot{a}}) \begin{pmatrix}
        1 & 0 \\ 0 & 0
    \end{pmatrix} \begin{pmatrix}
        (\chi_2)_a \\ (\xi^\dagger_2)^{\dot{a}}
    \end{pmatrix} \\
    &= \xi^a_1 (\chi_2)_a = (\xi_1 \chi_2)
\end{align}
\begin{align}
    RHS =\overline{\Psi}_2^C P_L \Psi_1^C &= (\chi^a_2, (\xi^\dagger_2)_{\dot{a}}) \begin{pmatrix}
        1 & 0 \\ 0 & 0
    \end{pmatrix} \begin{pmatrix}
        (\xi_1)_a \\ (\chi^\dagger_1)^{\dot{a}}
    \end{pmatrix} \\
    &=  (\chi_2)_a \xi^a_1 = (\chi_2 \xi_1) = (\xi_1 \chi_2)
\end{align}
This proves equation~(36.64).
\begin{align}
    LHS=\overline{\Psi}_1  P_R \Psi_2 &= (\xi^a_1, (\chi^\dagger_1)_{\dot{a}}) \begin{pmatrix}
        0 & 0 \\ 0 & 1
    \end{pmatrix} \begin{pmatrix}
        (\chi_2)_a \\ (\xi^\dagger_2)^{\dot{a}}
    \end{pmatrix} \\
    &= (\chi^\dagger_1)_{\dot{a}}  (\xi^\dagger_2)^{\dot{a}} =(\chi_1^\dagger \xi_2^\dagger)
\end{align}
\begin{align}
    RHS =\overline{\Psi}_2^C P_R \Psi_1^C &= (\chi^a_2, (\xi^\dagger_2)_{\dot{a}}) \begin{pmatrix}
        0 & 0 \\ 0 & 1
    \end{pmatrix} \begin{pmatrix}
        (\xi_1)_a \\ (\chi^\dagger_1)^{\dot{a}}
    \end{pmatrix} \\
    &= (\xi^\dagger_2)_{\dot{a}} (\chi^\dagger_1)^{\dot{a}}  =(\xi_2^\dagger\chi_1^\dagger )=(\chi_1^\dagger \xi_2^\dagger)
\end{align}
This proves equation~(36.65).\qed


\clearpage
\question{2}{38.1}\\
Use equation~(38.12) to compute $u_s(\mathbf{p})$ and $v_s(\mathbf{p})$ explicitly. Hint: Show that the matrix $2i\hat{\mathbf{p}}\cdot\mathbf{K}$ has eigenvalues $\pm 1$, and that, for any matrix $A$ with eigenvalues $\pm 1$, $\exp(cA) = \cosh(c) + A \sinh(c)$, where $c$ is an arbitrary complex number.\\
\textbf{Extra question:} What is the expression in the large energy limit $E_\mathbf{p} \gg m$? Please write down the result.\\
\begin{align}
    u_s(\mathbf{p})=\exp{(i\eta \hat{\mathbf{p}}\cdot\mathbf{K})}u_s(\mathbf{0}),\quad v_s(\mathbf{p})=\exp{(i\eta \hat{\mathbf{p}}\cdot\mathbf{K})}v_s(\mathbf{0})\tag{38.12}
\end{align}
\answer{}
We start by showing that the matrix $2i\hat{\mathbf{p}}\cdot\mathbf{K}$ has eigenvalues $\pm 1$. The boost generators $\mathbf{K}$ in the Dirac representation are given by
\begin{align}
    K^j =\frac{i}{2} \gamma^j \gamma^0 = \frac{i}{2} \begin{pmatrix}
        0 & \sigma^j \\ -\sigma^j & 0
    \end{pmatrix} \begin{pmatrix}
        0 & I \\ I & 0
    \end{pmatrix} = \frac{i}{2} \begin{pmatrix}
        \sigma^j & 0 \\ 0 & -\sigma^j
    \end{pmatrix} 
\end{align}
Thus,
\begin{align}
    2i\hat{\mathbf{p}}\cdot\mathbf{K} = 2i \sum_{j=1}^3 \hat{p}_j K^j = 2i \sum_{j=1}^3 \hat{p}_j \frac{i}{2} \begin{pmatrix}
        \sigma^j & 0 \\ 0 & -\sigma^j
    \end{pmatrix} = - \begin{pmatrix}
        \hat{\mathbf{p}}\cdot\boldsymbol{\sigma} & 0 \\ 0 & -\hat{\mathbf{p}}\cdot\boldsymbol{\sigma}
    \end{pmatrix}
\end{align}
Now we want to prove $\hat{\mathbf{p}}\cdot\boldsymbol{\sigma}$ has eigenvalues $\pm 1$, since $\hat{\mathbf{p}}$ is a unit vector. The characteristic polynomial of $\hat{\mathbf{p}}\cdot\boldsymbol{\sigma}$ is given by
\begin{align}
    \det(\hat{\mathbf{p}}\cdot\boldsymbol{\sigma} - \lambda I) = \det\begin{pmatrix}
        \hat{p}_3 - \lambda & \hat{p}_1 - i\hat{p}_2 \\ \hat{p}_1 + i\hat{p}_2 & -\hat{p}_3 - \lambda
    \end{pmatrix} = (\hat{p}_3 - \lambda)(-\hat{p}_3 - \lambda) - (\hat{p}_1 - i\hat{p}_2)(\hat{p}_1 + i\hat{p}_2)
\end{align}
Simplifying this, we get
\begin{align}
    \det(\hat{\mathbf{p}}\cdot\boldsymbol{\sigma} - \lambda I) = \lambda^2 - (\hat{p}_1^2 + \hat{p}_2^2 + \hat{p}_3^2) = \lambda^2 - 1
\end{align}
Setting the determinant to zero, we find the eigenvalues:
\begin{align}
    \lambda^2 - 1 = 0 \implies \lambda^2 = 1 \implies \lambda = \pm 1
\end{align}
Thus, $\hat{\mathbf{p}}\cdot\boldsymbol{\sigma}$ has eigenvalues $\pm 1$. Consequently, the matrix $2i\hat{\mathbf{p}}\cdot\mathbf{K}$ has eigenvalues $\pm 1$ as well. Now we can use the identity for any matrix $A$ with eigenvalues $\pm 1$:
\begin{align}
    \exp(cA) = \cosh(c) + A \sinh(c)
\end{align}
where $c$ is an arbitrary complex number. Applying this to our case with $A = 2i\hat{\mathbf{p}}\cdot\mathbf{K}$ and $c = i\eta/2$, we have
\begin{align}
    \exp(i\eta \hat{\mathbf{p}}\cdot\mathbf{K}) =& \cosh\left(\frac{i\eta}{2}\right) + (2i\hat{\mathbf{p}}\cdot\mathbf{K}) \sinh\left(\frac{i\eta}{2}\right)\\
    =& \cos\left(\frac{\eta}{2}\right) + i(2i\hat{\mathbf{p}}\cdot\mathbf{K}) \sin\left(\frac{\eta}{2}\right)\\
    =& \cos\left(\frac{\eta}{2}\right) - 2(\hat{\mathbf{p}}\cdot\mathbf{K}) \sin\left(\frac{\eta}{2}\right)
\end{align}




\clearpage
\question{3}{45.2}\\
Use the Feynman rules to write down (at tree level) $i\mathcal{T}$ for the processes: $e^+e^+\to e^+e^+$ and $\varphi\varphi\to e^+e^-\varphi$. \\
\textbf{Remark:} Do not write $\varphi\varphi\to e^+e^-$. Also, please draw Feynman diagrams when doing this problem. Remember the Lagrangian is
\begin{align}
    \mathcal{L}= -\frac{1}{2} \partial_\mu \varphi \partial^\mu \varphi - \frac{1}{2} m^2_\varphi \varphi^2 + \overline{\Psi}(i  \slashed{\partial} - m)\Psi + g\varphi \overline{\Psi}\Psi.
\end{align}
\answer{}

\clearpage
\question{4}{48.2}\\
Compute $\langle|\mathcal{T}|^2\rangle$ for $e^+e^-\to \varphi\varphi$. You should find that your result is the same as that for $e^-\varphi\to e^-\varphi$, but with $s\leftrightarrow  t$, and an extra overall minus sign. This relationship is known as \textit{crossing symmetry}. There is an overall minus sign for each fermion that is moved from the initial to the final state.\\
\textbf{Remark:} Please compute for $e^-\varphi\to e^-\varphi$, do not compute for $e^+e^-\to \varphi\varphi$. Please also draw Feynman diagrams. Remember the Lagrangian is
\begin{align}
    \mathcal{L}= -\frac{1}{2} \partial_\mu \varphi \partial^\mu \varphi - \frac{1}{2} m^2_\varphi \varphi^2 + \overline{\Psi}(i  \slashed{\partial} - m)\Psi + g\varphi \overline{\Psi}\Psi.
\end{align}
\answer{}