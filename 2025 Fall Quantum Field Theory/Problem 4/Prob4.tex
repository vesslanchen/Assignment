\section*{HW4 Due to November 4 11:59 PM}

\question{1}{Problem 14.1}\\
Derive a generalization of Feynman's formula, 
\begin{align}
    \frac{1}{A_1^{\alpha_1} A_2^{\alpha_2} \cdots A_n^{\alpha_n}}=\frac{\Gamma(\sum_i \alpha _i)}{\prod_i \Gamma(\alpha_i)} \frac{1}{(n-1)!}\int dF_n \frac{\prod_i x_i^{\alpha_i-1}}{\left(\sum_i x_i A_i\right)^{\sum_i \alpha_i}}.
\end{align}
\begin{align}
    \int dF_n =(n-1)!\int_0^1 d x_1 \int_0^1 d x_2 \cdots \int_0^1 d x_n \delta\left(\sum_{i=1}^n x_i-1\right).
\end{align}
Hint: start with 
\begin{align}
    \frac{\Gamma(\alpha)}{A^\alpha}=\int_0^\infty d t \, t^{\alpha -1} e^{-t A},
\end{align}
which defines the gamma function. Put an index on $A$, $\alpha$ and $t$, and take the product. Then multiply on the right-hand side by 
\begin{align}
    1=\int_0^\infty ds \delta(s-\sum_i t_i).
\end{align}
Make the change of variables $t_i = s x_i$ and carry out the integral over $s$.
\answer{}
By definesion of the gamma function, we have
\begin{align}
    \frac{1}{A_i^{\alpha_i}}=\frac{1}{\Gamma(\alpha_i)}\int_0^\infty d t_i \, t_i^{\alpha_i -1} e^{-t_i A_i}.
\end{align}
Then we have 
\begin{align}
    &\frac{1}{A^{\alpha_1}_1 A^{\alpha_2}_2 \cdots A^{\alpha_n}_n}=\prod_{i=1}^n \frac{1}{\Gamma(\alpha_i)}\int_0^\infty d t_i \, t_i^{\alpha_i -1} e^{-t_i A_i}\\
    =& \frac{1}{\prod_{i=1}^n \Gamma(\alpha_i)}\int_0^\infty d t_1 \int_0^\infty d t_2 \cdots \int_0^\infty d t_n \prod_{i=1}^n t_i^{\alpha_i -1} e^{-t_i A_i}\\
    =& \frac{1}{\prod_{i=1}^n \Gamma(\alpha_i)}\int_0^\infty d t_1 \int_0^\infty d t_2 \cdots \int_0^\infty d t_n \prod_{i=1}^n \big(t_i^{\alpha_i -1} e^{-t_i A_i} \big)\int_0^\infty ds \delta(s-\sum_{i=1}^n t_i).
\end{align}
We make the change of variables $t_i = s x_i$, then we have
\begin{align}
    &\frac{1}{A^{\alpha_1}_1 A^{\alpha_2}_2 \cdots A^{\alpha_n}_n}=\frac{1}{\prod_{i=1}^n \Gamma(\alpha_i)}\int_0^\infty ds \int_0^\infty d x_1 \int_0^\infty d x_2 \cdots \int_0^\infty d x_n \prod_{i=1}^n \big((s x_i)^{\alpha_i -1} e^{-s x_i A_i} \big)\delta(s-s\sum_{i=1}^n x_i) s\\
    =& \frac{1}{\prod_{i=1}^n \Gamma(\alpha_i)}\int_0^\infty ds \, s^{\sum_{i=1}^n \alpha_i -1} e^{-s \sum_{i=1}^n x_i A_i} \int_0^\infty d x_1 \int_0^\infty d x_2 \cdots \int_0^\infty d x_n \prod_{i=1}^n x_i^{\alpha_i -1} \delta(1-\sum_{i=1}^n x_i)\\
    =& \frac{1}{\prod_{i=1}^n \Gamma(\alpha_i)}\int_0^\infty d x_1 \int_0^\infty d x_2 \cdots \int_0^\infty d x_n \prod_{i=1}^n x_i^{\alpha_i -1} \delta(1-\sum_{i=1}^n x_i) \int_0^\infty ds \, s^{\sum_{i=1}^n \alpha_i -1} e^{-s \sum_{i=1}^n x_i A_i}\\
    =& \frac{1}{\prod_{i=1}^n \Gamma(\alpha_i)}\int_0^\infty d x_1 \int_0^\infty d x_2 \cdots \int_0^\infty d x_n \prod_{i=1}^n x_i^{\alpha_i -1} \delta(1-\sum_{i=1}^n x_i) \frac{\Gamma(\sum_{i=1}^n \alpha_i)}{\left(\sum_{i=1}^n x_i A_i\right)^{\sum_{i=1}^n \alpha_i}}\\
    =& \frac{\Gamma(\sum_{i=1}^n \alpha_i)}{\prod_{i=1}^n \Gamma(\alpha_i)} \frac{1}{(n-1)!}\int dF_n \frac{\prod_{i=1}^n x_i^{\alpha_i-1}}{\left(\sum_{i=1}^n x_i A_i\right)^{\sum_{i=1}^n \alpha_i}}.
\end{align}
Hence proved the formula.
\qed

\clearpage
\question{2}{Problem 14.2}\\
Verify eq.~(14.23).
\begin{align}
    \Omega_d=\frac{2 \pi^{d/2}}{\Gamma(d/2)}.
\end{align}
\answer{}
We start with the Gaussian integral in $d$ dimensions,
\begin{align}
    I_d=\int d^d x \, e^{-\mathbf{x}^2}.
\end{align}
In catrtesian coordinates, we have
\begin{align}
    I_d=\left(\int_{-\infty}^\infty d x \, e^{-x^2}\right)^d=(\sqrt{\pi})^d=\pi^{d/2}.
\end{align}
In spherical coordinates, we have
\begin{align}
    I_d=\int_0^\infty d r \, r^{d-1} e^{-r^2} \int d\Omega_d=\Omega_d \int_0^\infty d r \, r^{d-1} e^{-r^2}.    
\end{align}
Make the change of variable $t=r^2$, then we have
\begin{align}
    I_d=\frac{\Omega_d}{2} \int_0^\infty d t \, t^{d/2 -1} e^{-t}=\frac{\Omega_d}{2} \Gamma(d/2),
\end{align}
where we have used the definition of the gamma function
\begin{align}
    \Gamma(\alpha)=\int_0^\infty d t \, t^{\alpha -1} e^{-t}.
\end{align}
Equating the two expressions for $I_d$, we have
\begin{align}
    \Omega_d=\frac{2 \pi^{d/2}}{\Gamma(d/2)}.
\end{align}
\qed

\clearpage

\question{3}{Problem 14.5}\\
Compute the $O(\lambda)$ correction ot the propagator in $\varphi^4$ theory (see problem 9.2) in $d=4-\epsilon$ spacetime dimensions, and compute the $O(\lambda)$ terms in $A$ and $B$.
\answer{}


\begin{figure}[!h]
    \centering
    \begin{tikzpicture}
    \begin{feynman}
      \vertex (inputP1) at (-2,  0) {\(p\)};
      \vertex (vertex) at (0,0) [label=below:{\(-i\lambda\)}];
      \vertex (outputP1) at (2,0) {\(p\)};
      \vertex (upper) at (0,1.5) [label=above:{\(l\)}];
        \diagram* {
            (inputP1) -- [scalar] (vertex) -- [scalar] (outputP1),
            (vertex) -- [half left, scalar, looseness=1.5, ] (upper),
            (vertex) -- [half right, scalar, looseness=1.5,] (upper),
        };
    \end{feynman}
\end{tikzpicture}
    \caption{The Feynman diagram with the $\phi^4$ propagator for $1$-loop correction at $O(\lambda)$.}
    \label{HW4-fig-phi^4-propagator-correction}
\end{figure}

\begin{figure}[!h]
    \centering
    \begin{tikzpicture}
    \begin{feynman}
        % 外部粒子
        \vertex (i1) at (-2,0) {\(p\)};
        \vertex (f1) at ( 2,0) {\(p\)};
        % 中間 counterterm 點
        \vertex (v1) at (0,0);
        
        % 畫 propagator 線
        \diagram*{
            (i1) -- [scalar] (v1) -- [scalar] (f1),
        };
        
        % 加上 counterterm 標記
        \draw[fill=black, thick] (v1) circle (3pt);
        \node[below=3pt of v1] {\(-i(Ap^2 + Bm^2)\)};
    \end{feynman}
    \end{tikzpicture}
    \caption{The Feynman diagram with the $\phi^4$ propagator for $1$-loop counter term at $O(\lambda)$.}
    \label{HW4:fig:phi^4-propagator-counterterm}
\end{figure}
First, we write down the Lagrangian for the $\varphi^4$ theory,
\begin{align}
    &\mathcal{L}=\mathcal{L_0}+\mathcal{L}_I,\\
    &\mathcal{L}_0=-\frac{1}{2}\partial_\mu \varphi\partial^\mu\varphi -\frac{1}{2} m^2 \varphi^2,\\
    &\mathcal{L}_I=-\frac{Z_\lambda}{4!}\lambda \varphi^4 +\mathcal{L}_{ct},\\
    &\mathcal{L}_{ct}=-\frac{1}{2}(Z_\varphi -1)\partial_\mu \varphi\partial^\mu\varphi -\frac{1}{2}(Z_m -1)m^2 \varphi^2.
\end{align}
For the $O(\lambda)$ correction to the propagator, the Feynman diagram is shown in Figure~\ref{HW4-fig-phi^4-propagator-correction}. The corresponding amplitude is given by
\begin{align}
    i \Sigma(p)=\frac{1}{2}(-i \lambda) \frac{1}{i}\int \frac{d^4 l}{(2\pi)^4} \frac{1}{l^2 +m^2 -i \epsilon}- i(A p^2 + B m^2) + O(\lambda^2),
\end{align}
where the factor $\frac{1}{2}$ is the symmetry factor for this diagram. Consider the wick rotation to Euclidean space ($d^4 l \rightarrow i d^4 l_E$ and $l^2 \rightarrow  l_E^2$), we have
\begin{align}
    \Sigma(p)=\frac{-\lambda}{2}\int \frac{d^d l_E}{(2\pi)^d} \frac{1}{l_E^2 +m^2} - (A p^2 + B m^2) + O(\lambda^2),
\end{align}
where the $m=m-i\epsilon$ prescription is implied. Using the formula derived in Problem 14.1, we have
\begin{align}
    \int \frac{d^d l_E}{(2\pi)^d} \frac{1}{l_E^2 +m^2}=& \int \frac{d^d l_E}{(2\pi)^d} \int_0^\infty d x \, e^{-x(l_E^2 +m^2)}\\
    =& \int_0^\infty d x \, e^{-x m^2} \int \frac{d^d l_E}{(2\pi)^d} e^{-x l_E^2}\\
    =& \int_0^\infty d x \, e^{-x m^2} \frac{1}{(2\pi)^d} \left(\frac{\pi}{x}\right)^{d/2}\\
    =& \frac{1}{(4\pi)^{d/2}} \int_0^\infty d x \, x^{-d/2} e^{-x m^2}\\
    =& \frac{1}{(4\pi)^{d/2}} (m^2)^{d/2 -1} \Gamma(1 - d/2).
\end{align}
Substituting $d=4-\epsilon$, we have
\begin{align}
    \int \frac{d^d l_E}{(2\pi)^d} \frac{1}{l_E^2 +m^2}=& \frac{1}{(4\pi)^{2-\epsilon/2}} (m^2)^{1 -\epsilon/2} \Gamma(-1+\epsilon/2 )\\
    =& \frac{m^2}{16 \pi^2} \left(\frac{4\pi}{m^2}\right)^{\epsilon/2} \Gamma(-1+\epsilon/2 ).
\end{align}
Also, when we consider dimension regularization, we need to introduce a mass scale $\mu$ to keep the coupling constant $\lambda$ dimensionless. Thus, we have
\begin{align}
    \lambda\to \lambda \widetilde{\mu}^\epsilon.
\end{align}
Therefore, we have
\begin{align}
    \Sigma(p)=&\frac{-\lambda \widetilde{\mu}^\epsilon}{2} \frac{m^2}{16 \pi^2} \left(\frac{4\pi}{m^2}\right)^{\epsilon/2} \Gamma(-1+\epsilon/2 ) - (A p^2 + B m^2) + O(\lambda^2)\\
    =&\frac{-\lambda m^2}{32 \pi^2} \left(\frac{4\pi \widetilde{\mu}^2}{m^2}\right)^{\epsilon/2} \Gamma(-1+\epsilon/2 ) - (A p^2 + B m^2) + O(\lambda^2).
\end{align}
Using the expansion of the gamma function around the pole at $-1$ and the $4\pi\widetilde{\mu}^2/m^2$ around $\epsilon=0$,
\begin{align}
    \Gamma(-1+\epsilon/2)&=-\frac{2}{\epsilon} -1 +\gamma + O(\epsilon),\\
    \left(\frac{4\pi \widetilde{\mu}^2}{m^2}\right)^{\epsilon/2}&=1+\frac{\epsilon}{2} \ln\left(\frac{4\pi \widetilde{\mu}^2}{m^2}\right)+O(\epsilon^2),
\end{align}
we have
\begin{align}
    \Sigma(p)=&\frac{-\lambda m^2}{32 \pi^2} \left[-\frac{2}{\epsilon} -1 +\gamma \right] \left[1+\frac{\epsilon}{2} \ln\left(\frac{4\pi \widetilde{\mu}^2}{m^2}\right)\right] - (A p^2 + B m^2) + O(\lambda^2)\\
    =&\frac{\lambda m^2}{32 \pi^2} \left[ \frac{2}{\epsilon} -\gamma +1 +\ln\left(\frac{4\pi \widetilde{\mu}^2}{m^2}\right)\right] - (A p^2 + B m^2) + O(\lambda^2)\\
    =&\frac{\lambda m^2}{32 \pi^2} \left[ \frac{2}{\epsilon}  +1 +\ln\left(\frac{4\pi \widetilde{\mu}^2}{e^\gamma m^2}\right)\right] - (A p^2 + B m^2) + O(\lambda^2)
\end{align}
To satisfy the renormalization conditions,
\begin{align}
    \left.\frac{d}{d p^2} \Sigma(p)\right|_{p^2=-m^2}&=0,\\
    \Sigma(p^2=-m^2)&=0,
\end{align}
we have
\begin{align}
    A&=0 + O(\lambda^2),\\
    B&=\frac{\lambda}{32 \pi^2} \left[ \frac{2}{\epsilon}  +1 +\ln\left(\frac{4\pi \widetilde{\mu}^2}{e^\gamma m^2}\right)\right] + O(\lambda^2).
\end{align}
In summary, we have
\begin{align}
    \Sigma(p)=&\frac{\lambda m^2}{32 \pi^2} \left[ \frac{2}{\epsilon}  +1 +\ln\left(\frac{4\pi \widetilde{\mu}^2}{e^\gamma m^2}\right)\right] - (A p^2 + B m^2) + O(\lambda^2)=0 + O(\lambda^2),\\
    A&=0 + O(\lambda^2),\\
    B&=\frac{\lambda}{32 \pi^2} \left[ \frac{2}{\epsilon}  +1 +\ln\left(\frac{4\pi \widetilde{\mu}^2}{e^\gamma m^2}\right)\right] + O(\lambda^2).
\end{align}
\qed
\clearpage
\question{4}{Problem 16.1}\\
Compute the $O(\lambda^2)$ correction in $\mathbf{V}_4$ in $\varphi^4$ theory in $d=4-\epsilon$ spacetime dimensions. Take $\mathbf{V}_4=\lambda$ when all four external momenta are on shell, and $s=4m^2$. What is the $O(\lambda)$ contribution to $C$? 
\answer{}


\begin{figure}[!h]
    \centering
    \begin{tikzpicture}
    \begin{feynman}
      \vertex (inputP1) at (-2,  2) {\(p_1\)};
      \vertex (inputP2) at (-2,  -2) {\(p_2\)};
      \vertex (vertex0) at (0,0) [label=left:{\(-iZ_\lambda\lambda\)}];
      \vertex (outputP1) at (2,2) {\(p_3\)};
      \vertex (outputP2) at (2,-2) {\(p_4\)};
        \diagram* {
            (inputP1) -- [scalar] (vertex) -- [scalar] (inputP2),
            (outputP1) -- [scalar] (vertex) -- [scalar] (outputP2),
        };
    \end{feynman}
\end{tikzpicture}
    \caption{The Feynman diagram with the $\phi^4$ vertex for tree level at $O(\lambda)$.}
    \label{HW4-fig-phi^4-vertex-tree-level}
\end{figure}

\begin{figure}[!h]
    \centering
    \begin{tikzpicture}
    \begin{feynman}
      \vertex (inputP1) at (-2,  2) {\(p_1\)};
      \vertex (inputP2) at (-2,  -2) {\(p_2\)};
      \vertex (vertex1) at (-1,0) [label=left:{\(-i\lambda\)}];
      \vertex (vertex2) at (1,0) [label=right:{\(-i\lambda\)}];
      \vertex (outputP1) at (2,2) {\(p_3\)};
      \vertex (outputP2) at (2,-2) {\(p_4\)};
        \diagram* {
            (inputP1) -- [scalar] (vertex1) -- [scalar] (inputP2),
            (outputP1) -- [scalar] (vertex2) -- [scalar] (outputP2),
            vertex1 -- [half left, scalar, looseness=1.5,edge label=\(k\)]  vertex2,
            vertex1 -- [half right, scalar, looseness=1.5,edge label'=\(k-p_1-p_2\)] vertex2,
        };
    \end{feynman}
\end{tikzpicture}
    \caption{The Feynman diagram with the $\phi^4$ vertex for $1$-loop correction at $O(\lambda^2)$ in the s-channel.}
    \label{HW4-fig-phi^4-vertex-s-channel}
\end{figure}
\begin{figure}[!h]
    \centering
    \begin{tikzpicture}
    \begin{feynman}
      \vertex (inputP1) at (-2,  2) {\(p_1\)};
      \vertex (inputP2) at (-2,  -2) {\(p_3\)};
      \vertex (vertex1) at (-1,0) [label=left:{\(-i\lambda\)}];
      \vertex (vertex2) at (1,0) [label=right:{\(-i\lambda\)}];
      \vertex (outputP1) at (2,2) {\(p_2\)};
      \vertex (outputP2) at (2,-2) {\(p_4\)};
        \diagram* {
            (inputP1) -- [scalar] (vertex1) -- [scalar] (inputP2),
            (outputP1) -- [scalar] (vertex2) -- [scalar] (outputP2),
            vertex1 -- [half left, scalar, looseness=1.5,edge label=\(k\)]  vertex2,
            vertex1 -- [half right, scalar, looseness=1.5,edge label'=\(k-p_1-p_3\)] vertex2,
        };
    \end{feynman}
\end{tikzpicture}
    \caption{The Feynman diagram with the $\phi^4$ vertex for $1$-loop correction at $O(\lambda^2)$ in the t-channel.}
    \label{HW4-fig-phi^4-vertex-t-channel}
\end{figure}


\begin{figure}[!h]
    \centering
    \begin{tikzpicture}
    \begin{feynman}
      \vertex (inputP1) at (-2,  2) {\(p_1\)};
      \vertex (inputP2) at (-2,  -2) {\(p_4\)};
      \vertex (vertex1) at (-1,0) [label=left:{\(-i\lambda\lambda\)}];
      \vertex (vertex2) at (1,0) [label=right:{\(-i\lambda\lambda\)}];
      \vertex (outputP1) at (2,2) {\(p_2\)};
      \vertex (outputP2) at (2,-2) {\(p_3\)};
        \diagram* {
            (inputP1) -- [scalar] (vertex1) -- [scalar] (inputP2),
            (outputP1) -- [scalar] (vertex2) -- [scalar] (outputP2),
            vertex1 -- [half left, scalar, looseness=1.5,edge label=\(k\)]  vertex2,
            vertex1 -- [half right, scalar, looseness=1.5,edge label'=\(k-p_1-p_4\)] vertex2,
        };
    \end{feynman}
\end{tikzpicture}
    \caption{The Feynman diagram with the $\phi^4$ vertex for $1$-loop correction at $O(\lambda^2)$ in the u-channel.}
    \label{HW4-fig-phi^4-vertex-u-channel}
\end{figure}

At $O(\lambda^2)$, there are three Feynman diagrams contributing to the $1$-loop correction to the $\phi^4$ vertex, as shown in Figure~\ref{HW4-fig-phi^4-vertex-s-channel},~\ref{HW4-fig-phi^4-vertex-t-channel} and~\ref{HW4-fig-phi^4-vertex-u-channel}. The tree-level diagram is shown in Figure~\ref{HW4-fig-phi^4-vertex-tree-level}. The corresponding amplitude is given by
\begin{align}
    i \mathbf{V}_4=&iV_{tree}+ iV_4^{(s)} + iV_4^{(t)} + iV_4^{(u)}\\
    =& -i Z_\lambda\lambda + \left(\frac{1}{2}\right)(-i \lambda)^2 \Big(\frac{1}{i}\Big)^2\int \frac{d^4 k}{(2\pi)^4} \frac{1}{k^2 +m^2} \frac{1}{(k - p_1 - p_2)^2 +m^2}\\
    &+ \left(\frac{1}{2}\right)(-i \lambda)^2 \Big(\frac{1}{i}\Big)^2\int \frac{d^4 k}{(2\pi)^4} \frac{1}{k^2 +m^2} \frac{1}{(k - p_1 - p_3)^2 +m^2}\\
    &+ \left(\frac{1}{2}\right)(-i \lambda)^2 \Big(\frac{1}{i}\Big)^2\int \frac{d^4 k}{(2\pi)^4} \frac{1}{k^2 +m^2} \frac{1}{(k - p_1 - p_4)^2 +m^2}.
\end{align}
Using Feynman parameterization, we have
\begin{align}
    &\int \frac{d^4 k}{(2\pi)^4} \frac{1}{k^2 +m^2} \frac{1}{(k - p_i - p_j)^2 +m^2}\\
    =&\int_0^1 d x \int \frac{d^4 k}{(2\pi)^4} \frac{1}{\left[k^2 -2 x k \cdot (p_i + p_j) + x (p_i + p_j)^2 + m^2\right]^2}\\
    =&\int_0^1 d x \int \frac{d^4 k}{(2\pi)^4} \frac{1}{\left[(k - x (p_i + p_j))^2 + x(1-x)(p_i + p_j)^2 + m^2\right]^2}\\
    =&\int_0^1 d x \int \frac{d^4 q}{(2\pi)^4} \frac{1}{\left[q^2 + D_{ij}\right]^2}\\
    =&i\int_0^1 d x \int \frac{d^4 q_E}{(2\pi)^4} \frac{1}{\left[q_E^2 + D_{ij}\right]^2},
\end{align}
where $D_{ij} = x(1-x)(p_i + p_j)^2 + m^2$ and we have performed the wick rotation to Euclidean space ($d^4 k \rightarrow i d^4 k_E$ and $k^2 \rightarrow  k_E^2$). Using the eq.~(14.27) in textbook, we have
\begin{align}
    \int \frac{d^d \bar{q}}{(2\pi)^d} \frac{(\bar{q}^2)^a}{\left[\bar{q}^2 + D_{ij}\right]^b}= \frac{\Gamma(b - a - d/2) \Gamma(a + d/2)}{(4\pi)^{d/2}\Gamma(b) \Gamma(d/2)} D_{ij}^{-(b - a - d/2)}.
\end{align}
Thus, with $a=0$, $b=2$ and $d=4-\epsilon$, we have
\begin{align}
    \int \frac{d^4 q_E}{(2\pi)^4} \frac{1}{\left[q_E^2 + D_{ij}\right]^2}=& \frac{\Gamma(2 - 0 - (4-\epsilon)/2) \Gamma(0 + (4-\epsilon)/2)}{(4\pi)^{(4-\epsilon)/2}\Gamma(2) \Gamma((4-\epsilon)/2)} D_{ij}^{-(2 - 0 - (4-\epsilon)/2)}\\
    =& \frac{\Gamma(\epsilon/2) \Gamma(2 -\epsilon/2)}{(4\pi)^{2-\epsilon/2} \cdot 1 \cdot \Gamma(2 -\epsilon/2)} D_{ij}^{-\epsilon/2}\\
    =& \frac{1}{(4\pi)^{2-\epsilon/2}} \Gamma(\epsilon/2) D_{ij}^{-\epsilon/2}\\
    =&\frac{1}{(4\pi)^{2}} \left(\frac{4\pi}{D_{ij}}\right)^{\epsilon/2} \Gamma(\epsilon/2).
\end{align}
Besides, when we consider dimension regularization, we need to introduce a mass scale $\mu$ to keep the coupling constant $\lambda$ dimensionless. Thus, we have
\begin{align}
    \lambda\to \lambda \widetilde{\mu}^\epsilon.
\end{align}
Therefore, we have
\begin{align}
    i \mathbf{V}_4=& -i Z_\lambda \lambda + \left(\frac{1}{2}\right)(-i \widetilde{\mu}^\epsilon \lambda)^2 \Big(\frac{1}{i}\Big)^2 i \int_0^1 d x \frac{1}{(4\pi)^{2}} \left(\frac{4\pi}{D_{12}}\right)^{\epsilon/2} \Gamma(\epsilon/2)\\
    &+ \left(\frac{1}{2}\right)(-i \widetilde{\mu}^\epsilon \lambda)^2 \Big(\frac{1}{i}\Big)^2 i \int_0^1 d x \frac{1}{(4\pi)^{2}} \left(\frac{4\pi}{D_{13}}\right)^{\epsilon/2} \Gamma(\epsilon/2)\\
    &+ \left(\frac{1}{2}\right)(-i \widetilde{\mu}^\epsilon \lambda)^2 \Big(\frac{1}{i}\Big)^2 i \int_0^1 d x \frac{1}{(4\pi)^{2}} \left(\frac{4\pi}{D_{14}}\right)^{\epsilon/2} \Gamma(\epsilon/2)\\
    =& -i Z_\lambda \lambda - i (\widetilde{\mu}^\epsilon \lambda)^2 \frac{1}{2(4\pi)^{2}} \Gamma(\epsilon/2) \int_0^1 d x \left[\left(\frac{4\pi}{D_{12}}\right)^{\epsilon/2} + \left(\frac{4\pi}{D_{13}}\right)^{\epsilon/2} + \left(\frac{4\pi}{D_{14}}\right)^{\epsilon/2}\right].
\end{align}
In other words, we have
\begin{align}
    \mathbf{V}_4=& - Z_\lambda \lambda - (\widetilde{\mu}^\epsilon \lambda)^2 \frac{1}{2(4\pi)^{2}} \Gamma(\epsilon/2) \int_0^1 d x \left[\left(\frac{4\pi}{D_{12}}\right)^{\epsilon/2} + \left(\frac{4\pi}{D_{13}}\right)^{\epsilon/2} + \left(\frac{4\pi}{D_{14}}\right)^{\epsilon/2}\right].
\end{align}
To satisfy the renormalization condition,
\begin{align}
    \mathbf{V}_4 = -\lambda \quad \text{at} \quad s=4m^2,
\end{align}
we have
\begin{align}
    Z_\lambda=& 1 + \lambda \frac{1}{2(4\pi)^{2}} \Gamma(\epsilon/2) \int_0^1 d x \left[\left(\frac{4\pi \widetilde{\mu}^2}{D_{12}}\right)^{\epsilon/2} + \left(\frac{4\pi \widetilde{\mu}^2}{D_{13}}\right)^{\epsilon/2} + \left(\frac{4\pi \widetilde{\mu}^2}{D_{14}}\right)^{\epsilon/2}\right] + O(\lambda^2).
\end{align}
For $s=4m^2$ and $t=u=0$ (by assuming all $p_i=(m, \mathbf{0})$), we have
\begin{align}
    D_{12}=& x(1-x)(-s) + m^2 = (1 - 4 x(1-x)) m^2=(1-2x)^2 m^2,\\
    D_{13}=& x(1-x)(-t) + m^2 = m^2,\\
    D_{14}=& x(1-x)(-u) + m^2 = m^2.
\end{align}
Thus, we have
\begin{align}
    Z_\lambda=& 1 + \lambda \frac{1}{2(4\pi)^{2}} \Gamma(\epsilon/2) \int_0^1 d x \left[\left(\frac{4\pi \widetilde{\mu}^2}{(1-2x)^2 m^2}\right)^{\epsilon/2} + 2\left(\frac{4\pi \widetilde{\mu}^2}{m^2}\right)^{\epsilon/2}\right] + O(\lambda^2)\\
    =& 1 + \lambda \frac{1}{2(4\pi)^{2}} \Gamma(\epsilon/2) \left[\int_0^1 d x \left(\frac{4\pi \widetilde{\mu}^2}{(1-2x)^2 m^2}\right)^{\epsilon/2} + 2\left(\frac{4\pi \widetilde{\mu}^2}{m^2}\right)^{\epsilon/2}\right] + O(\lambda^2).
\end{align}
For small $\epsilon$, we have
\begin{align}
    \Gamma(\epsilon/2)&=\frac{2}{\epsilon} -\gamma + O(\epsilon),\\
    \left(\frac{4\pi \widetilde{\mu}^2}{(1-2x)^2 m^2}\right)^{\epsilon/2}&=1 + \frac{\epsilon}{2} \ln\left(\frac{4\pi \widetilde{\mu}^2}{(1-2x)^2 m^2}\right) + O(\epsilon^2),\\
    \left(\frac{4\pi \widetilde{\mu}^2}{m^2}\right)^{\epsilon/2}&=1 + \frac{\epsilon}{2} \ln\left(\frac{4\pi \widetilde{\mu}^2}{m^2}\right) + O(\epsilon^2).
\end{align}
Substituting these expansions into the expression of $Z_\lambda$, we have
\begin{align}
    Z_\lambda=& 1 + \lambda \frac{1}{2(4\pi)^{2}} \left(\frac{2}{\epsilon} -\gamma\right) \left[\int_0^1 d x \left(1 + \frac{\epsilon}{2} \ln\left(\frac{4\pi \widetilde{\mu}^2}{(1-2x)^2 m^2}\right)\right) + 2\left(1 + \frac{\epsilon}{2} \ln\left(\frac{4\pi \widetilde{\mu}^2}{m^2}\right)\right)\right] + O(\lambda^2)\\
    =& 1 + \lambda\frac{1}{2(4\pi)^{2}} \Bigg[\int_0^1 d x \left(\frac{2}{\epsilon} -\gamma +\ln\left(\frac{4\pi \widetilde{\mu}^2}{(1-2x)^2 m^2}\right)\right) + 2\left(\frac{2}{\epsilon} -\gamma +\ln\left(\frac{4\pi \widetilde{\mu}^2}{m^2}\right)\right)\Bigg] + O(\lambda^2)\\
    =& 1 + \lambda\frac{1}{2(4\pi)^{2}} \Bigg[\int_0^1 d x \left(\frac{2}{\epsilon} +\ln\left(\frac{4\pi \widetilde{\mu}^2/e^\gamma}{ (1-2x)^2 m^2}\right)\right) + 2\left(\frac{2}{\epsilon} +\ln\left(\frac{4\pi \widetilde{\mu}^2/e^\gamma}{ m^2}\right)\right)\Bigg] + O(\lambda^2)\\
    =&1 + \lambda\frac{1}{2(4\pi)^{2}} \Bigg[\int_0^1 d x \left(\frac{2}{\epsilon} +\ln\left(\frac{\mu^2}{ (1-2x)^2 m^2}\right)\right) + 2\left(\frac{2}{\epsilon} +\ln\left(\frac{\mu^2}{ m^2}\right)\right)\Bigg] + O(\lambda^2)\\
    =&1 + \lambda\frac{1}{2(4\pi)^{2}} \Bigg[\int_0^1 d x \left(\frac{2}{\epsilon} +\ln\left(\frac{\mu^2/m^2}{ (1-2x)^2 }\right)\right) + 2\left(\frac{2}{\epsilon} +2\ln\left(\frac{\mu}{ m}\right)\right)\Bigg] + O(\lambda^2)\\
    =&1 + \lambda\frac{1}{2(4\pi)^{2}} \Bigg[ \left(\frac{2}{\epsilon}+2+2\ln(\mu/m) \right) + 2\left(\frac{2}{\epsilon} +2\ln\left(\frac{\mu}{ m}\right)\right)\Bigg] + O(\lambda^2)\\
    =&1 + \lambda\frac{1}{2(4\pi)^{2}} \left(\frac{6}{\epsilon} +2 +6\ln(\mu/m) \right) + O(\lambda^2)\\
    =&1 + \frac{3\lambda}{(4\pi)^{2}} \left(\frac{1}{\epsilon} +\frac{1}{3} +\ln(\mu/m) \right) + O(\lambda^2).
\end{align}
In summary, we have
\begin{align}
    Z_\lambda-1=& \frac{3\lambda}{(4\pi)^{2}} \left(\frac{1}{\epsilon} +\frac{1}{3} +\ln(\mu/m) \right) + O(\lambda^2),\\
    i\mathbf{V}_4=& -i\lambda + i(Z_\lambda -1) \lambda + O(\lambda^3)\\
    =& -i\lambda + \frac{3i\lambda^2}{(4\pi)^{2}} \left(\frac{1}{\epsilon} +\frac{1}{3} +\ln(\mu/m) \right) + O(\lambda^3).
\end{align}
\qed
