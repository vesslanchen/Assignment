\section*{HW4 Due to November 4 11:59 PM}

\question{1}{Problem 14.1}\\
Derive a generalization of Feynman's formula, 
\begin{align}
    \frac{1}{A_1^{\alpha_1} A_2^{\alpha_2} \cdots A_n^{\alpha_n}}=\frac{\Gamma(\sum_i \alpha _i)}{\prod_i \Gamma(\alpha_i)} \frac{1}{(n-1)!}\int dF_n \frac{\prod_i x_i^{\alpha_i-1}}{\left(\sum_i x_i A_i\right)^{\sum_i \alpha_i}}.
\end{align}
\begin{align}
    \int dF_n =(n-1)!\int_0^1 d x_1 \int_0^1 d x_2 \cdots \int_0^1 d x_n \delta\left(\sum_{i=1}^n x_i-1\right).
\end{align}
Hint: start with 
\begin{align}
    \frac{\Gamma(\alpha)}{A^\alpha}=\int_0^\infty d t \, t^{\alpha -1} e^{-t A},
\end{align}
which defines the gamma function. Put an index on $A$, $\alpha$ and $t$, and take the product. Then multiply on the right-hand side by 
\begin{align}
    1=\int_0^\infty ds \delta(s-\sum_i t_i).
\end{align}
Make the change of variables $t_i = s x_i$ and carry out the integral over $s$.
\answer{}
By definesion of the gamma function, we have
\begin{align}
    \frac{1}{A_i^{\alpha_i}}=\frac{1}{\Gamma(\alpha_i)}\int_0^\infty d t_i \, t_i^{\alpha_i -1} e^{-t_i A_i}.
\end{align}
Then we have 
\begin{align}
    &\frac{1}{A^{\alpha_1}_1 A^{\alpha_2}_2 \cdots A^{\alpha_n}_n}=\prod_{i=1}^n \frac{1}{\Gamma(\alpha_i)}\int_0^\infty d t_i \, t_i^{\alpha_i -1} e^{-t_i A_i}\\
    =& \frac{1}{\prod_{i=1}^n \Gamma(\alpha_i)}\int_0^\infty d t_1 \int_0^\infty d t_2 \cdots \int_0^\infty d t_n \prod_{i=1}^n t_i^{\alpha_i -1} e^{-t_i A_i}\\
    =& \frac{1}{\prod_{i=1}^n \Gamma(\alpha_i)}\int_0^\infty d t_1 \int_0^\infty d t_2 \cdots \int_0^\infty d t_n \prod_{i=1}^n \big(t_i^{\alpha_i -1} e^{-t_i A_i} \big)\int_0^\infty ds \delta(s-\sum_{i=1}^n t_i).
\end{align}
We make the change of variables $t_i = s x_i$, then we have
\begin{align}
    &\frac{1}{A^{\alpha_1}_1 A^{\alpha_2}_2 \cdots A^{\alpha_n}_n}=\frac{1}{\prod_{i=1}^n \Gamma(\alpha_i)}\int_0^\infty ds \int_0^\infty d x_1 \int_0^\infty d x_2 \cdots \int_0^\infty d x_n \prod_{i=1}^n \big((s x_i)^{\alpha_i -1} e^{-s x_i A_i} \big)\delta(s-s\sum_{i=1}^n x_i) s\\
    =& \frac{1}{\prod_{i=1}^n \Gamma(\alpha_i)}\int_0^\infty ds \, s^{\sum_{i=1}^n \alpha_i -1} e^{-s \sum_{i=1}^n x_i A_i} \int_0^\infty d x_1 \int_0^\infty d x_2 \cdots \int_0^\infty d x_n \prod_{i=1}^n x_i^{\alpha_i -1} \delta(1-\sum_{i=1}^n x_i)\\
    =& \frac{1}{\prod_{i=1}^n \Gamma(\alpha_i)}\int_0^\infty d x_1 \int_0^\infty d x_2 \cdots \int_0^\infty d x_n \prod_{i=1}^n x_i^{\alpha_i -1} \delta(1-\sum_{i=1}^n x_i) \int_0^\infty ds \, s^{\sum_{i=1}^n \alpha_i -1} e^{-s \sum_{i=1}^n x_i A_i}\\
    =& \frac{1}{\prod_{i=1}^n \Gamma(\alpha_i)}\int_0^\infty d x_1 \int_0^\infty d x_2 \cdots \int_0^\infty d x_n \prod_{i=1}^n x_i^{\alpha_i -1} \delta(1-\sum_{i=1}^n x_i) \frac{\Gamma(\sum_{i=1}^n \alpha_i)}{\left(\sum_{i=1}^n x_i A_i\right)^{\sum_{i=1}^n \alpha_i}}\\
    =& \frac{\Gamma(\sum_{i=1}^n \alpha_i)}{\prod_{i=1}^n \Gamma(\alpha_i)} \frac{1}{(n-1)!}\int dF_n \frac{\prod_{i=1}^n x_i^{\alpha_i-1}}{\left(\sum_{i=1}^n x_i A_i\right)^{\sum_{i=1}^n \alpha_i}}.
\end{align}
Hence proved the formula.
\qed

\clearpage
\question{2}{Problem 14.2}\\
Verify eq.~(14.23).
\begin{align}
    \Omega_d=\frac{2 \pi^{d/2}}{\Gamma(d/2)}.
\end{align}
\answer{}
We start with the Gaussian integral in $d$ dimensions,
\begin{align}
    I_d=\int d^d x \, e^{-\mathbf{x}^2}.
\end{align}
In catrtesian coordinates, we have
\begin{align}
    I_d=\left(\int_{-\infty}^\infty d x \, e^{-x^2}\right)^d=(\sqrt{\pi})^d=\pi^{d/2}.
\end{align}
In spherical coordinates, we have
\begin{align}
    I_d=\int_0^\infty d r \, r^{d-1} e^{-r^2} \int d\Omega_d=\Omega_d \int_0^\infty d r \, r^{d-1} e^{-r^2}.    
\end{align}
Make the change of variable $t=r^2$, then we have
\begin{align}
    I_d=\frac{\Omega_d}{2} \int_0^\infty d t \, t^{d/2 -1} e^{-t}=\frac{\Omega_d}{2} \Gamma(d/2),
\end{align}
where we have used the definition of the gamma function
\begin{align}
    \Gamma(\alpha)=\int_0^\infty d t \, t^{\alpha -1} e^{-t}.
\end{align}
Equating the two expressions for $I_d$, we have
\begin{align}
    \Omega_d=\frac{2 \pi^{d/2}}{\Gamma(d/2)}.
\end{align}
\qed

\clearpage

\question{3}{Problem 14.5}\\
Compute the $O(\lambda)$ correction ot the propagator in $\varphi^4$ theory (see problem 9.2) in $d=4-\epsilon$ spacetime dimensions, and compute the $O(\lambda)$ terms in $A$ and $B$.
\answer{}


\begin{figure}[!h]
    \centering
    \begin{tikzpicture}
    \begin{feynman}
      \vertex (inputP1) at (-2,  0) {\(p\)};
      \vertex (vertex) at (0,0) [label=below:{\(i\lambda\)}];
      \vertex (outputP1) at (2,0) {\(p\)};
      \vertex (upper) at (0,1.5) [label=above:{\(l\)}];
        \diagram* {
            (inputP1) -- [scalar] (vertex) -- [scalar] (outputP1),
            (vertex) -- [half left, scalar, looseness=1.5, ] (upper),
            (vertex) -- [half right, scalar, looseness=1.5,] (upper),
        };
    \end{feynman}
\end{tikzpicture}
    \caption{The Feynman diagram with the $\phi^4$ propagator for $1$-loop correction at $O(\lambda)$.}
    \label{HW4-fig-phi^4-propagator-correction}
\end{figure}

\begin{figure}[!h]
    \centering
    \begin{tikzpicture}
    \begin{feynman}
        % 外部粒子
        \vertex (i1) at (-2,0) {\(p\)};
        \vertex (f1) at ( 2,0) {\(p\)};
        % 中間 counterterm 點
        \vertex (v1) at (0,0);
        
        % 畫 propagator 線
        \diagram*{
            (i1) -- [scalar] (v1) -- [scalar] (f1),
        };
        
        % 加上 counterterm 標記
        \draw[fill=black, thick] (v1) circle (3pt);
        \node[below=3pt of v1] {\(-i(Ap^2 + Bm^2)\)};
    \end{feynman}
    \end{tikzpicture}
    \caption{The Feynman diagram with the $\phi^4$ propagator for $1$-loop counter term at $O(\lambda)$.}
    \label{HW4:fig:phi^4-propagator-counterterm}
\end{figure}
First, we write down the Lagrangian for the $\varphi^4$ theory,
\begin{align}
    &\mathcal{L}=\mathcal{L_0}+\mathcal{L}_I,\\
    &\mathcal{L}_0=-\frac{1}{2}\partial_\mu \varphi\partial^\mu\varphi -\frac{1}{2} m^2 \varphi^2,\\
    &\mathcal{L}_I=-\frac{Z_\lambda}{4!}\lambda \varphi^4 +\mathcal{L}_{ct},\\
    &\mathcal{L}_{ct}=-\frac{1}{2}(Z_\varphi -1)\partial_\mu \varphi\partial^\mu\varphi -\frac{1}{2}(Z_m -1)m^2 \varphi^2.
\end{align}
For the $O(\lambda)$ correction to the propagator, the Feynman diagram is shown in Figure~\ref{HW4-fig-phi^4-propagator-correction}. The corresponding amplitude is given by
\begin{align}
    i \Sigma(p)=\frac{1}{2}(i \lambda) \frac{1}{i}\int \frac{d^4 l}{(2\pi)^4} \frac{1}{l^2 +m^2 -i \epsilon}- i(A p^2 + B m^2) + O(\lambda^2),
\end{align}
where the factor $\frac{1}{2}$ is the symmetry factor for this diagram. Consider the wick rotation to Euclidean space ($d^4 l \rightarrow i d^4 l_E$ and $l^2 \rightarrow  l_E^2$), we have
\begin{align}
    \Sigma(p)=\frac{\lambda}{2}\int \frac{d^d l_E}{(2\pi)^d} \frac{1}{l_E^2 +m^2} - (A p^2 + B m^2) + O(\lambda^2),
\end{align}
where the $m=m-i\epsilon$ prescription is implied. Using the formula derived in Problem 14.1, we have
\begin{align}
    \int \frac{d^d l_E}{(2\pi)^d} \frac{1}{l_E^2 +m^2}=& \int \frac{d^d l_E}{(2\pi)^d} \int_0^\infty d x \, e^{-x(l_E^2 +m^2)}\\
    =& \int_0^\infty d x \, e^{-x m^2} \int \frac{d^d l_E}{(2\pi)^d} e^{-x l_E^2}\\
    =& \int_0^\infty d x \, e^{-x m^2} \frac{1}{(2\pi)^d} \left(\frac{\pi}{x}\right)^{d/2}\\
    =& \frac{1}{(4\pi)^{d/2}} \int_0^\infty d x \, x^{-d/2} e^{-x m^2}\\
    =& \frac{1}{(4\pi)^{d/2}} (m^2)^{d/2 -1} \Gamma(1 - d/2).
\end{align}
Substituting $d=4-\epsilon$, we have
\begin{align}
    \int \frac{d^d l_E}{(2\pi)^d} \frac{1}{l_E^2 +m^2}=& \frac{1}{(4\pi)^{2-\epsilon/2}} (m^2)^{1 -\epsilon/2} \Gamma(-1+\epsilon/2 )\\
    =& \frac{m^2}{16 \pi^2} \left(\frac{4\pi}{m^2}\right)^{\epsilon/2} \Gamma(-1+\epsilon/2 ).
\end{align}
Also, when we consider dimension regularization, we need to introduce a mass scale $\mu$ to keep the coupling constant $\lambda$ dimensionless. Thus, we have
\begin{align}
    \lambda\to \lambda \widetilde{\mu}^\epsilon.
\end{align}


\clearpage
\question{4}{Problem 16.1}\\
Compute the $O(\lambda^2)$ correction in $\mathbf{V}_4$ in $\varphi^4$ theory in $d=4-\epsilon$ spacetime dimensions. Take $\mathbf{V}_4=\lambda$ when all four external momenta are on shell, and $s=4m^2$. What is the $O(\lambda)$ contribution to $C$? 
\answer{}


