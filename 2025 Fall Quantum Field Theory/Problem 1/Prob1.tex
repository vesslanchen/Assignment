
\section*{HW1 Due to September 23 11:59 PM}

\question{1}{Problem 1.2}

With the Hamiltonian of eq.~(1.32), show that the state defined in eq.~(1.33) obeys the abstract Schrodinger equation, eq.~(1.1), if and only if the wave function obeys eq.~(1.30). Your demonstration should apply both to the case of bosons, where the particle creation and annihilation operators obey the commutation relations of eq.~(1.31), and to fermions, where the particle creation and annihilation operators obey the anti-commutation relations of eq.~(1.38).
\begin{equation}
\begin{aligned}
    i \hbar \frac{\partial}{\partial t}|\psi, t\rangle=H|\psi, t\rangle
\end{aligned}
\label{eq:1.1}
\tag{1.1}
\end{equation}


\begin{equation}
\begin{aligned}
i \hbar \frac{\partial}{\partial t} \psi=\left[\sum_{j=1}^n\left(-\frac{\hbar^2}{2 m} \nabla_j^2+U\left(\mathbf{x}_j\right)\right)+\sum_{j=1}^n \sum_{k=1}^{j-1} V\left(\mathbf{x}_j-\mathbf{x}_k\right)\right] \psi
\end{aligned}
\tag{1.30}
\label{eq:1.30}
\end{equation}

\begin{equation}
\begin{aligned}
{\left[a(\mathbf{x}), a\left(\mathbf{x}^{\prime}\right)\right] } & =0 \\
{\left[a^{\dagger}(\mathbf{x}), a^{\dagger}\left(\mathbf{x}^{\prime}\right)\right] } & =0 \\
{\left[a(\mathbf{x}), a^{\dagger}\left(\mathbf{x}^{\prime}\right)\right] } & =\delta^3\left(\mathbf{x}-\mathbf{x}^{\prime}\right)
\end{aligned}
\tag{1.31}
\label{eq:1.31}
\end{equation}

\begin{equation}
\begin{aligned}
H= & \int d^3 x a^{\dagger}(\mathbf{x})\left(-\frac{\hbar^2}{2 m} \nabla^2+U(\mathbf{x})\right) a(\mathbf{x}) \\
& +\frac{1}{2} \int d^3 x d^3 y V(\mathbf{x}-\mathbf{y}) a^{\dagger}(\mathbf{x}) a^{\dagger}(\mathbf{y}) a(\mathbf{y}) a(\mathbf{x})
\end{aligned}
\tag{1.32}
\label{eq:1.32}
\end{equation}

\begin{equation}
\begin{aligned}
|\psi, t\rangle=\int d^3 x_1 \ldots d^3 x_n \psi\left(\mathbf{x}_1, \ldots, \mathbf{x}_n ; t\right) a^{\dagger}\left(\mathbf{x}_1\right) \ldots a^{\dagger}\left(\mathbf{x}_n\right)|0\rangle
\end{aligned}
\tag{1.33}
\label{eq:1.33}
\end{equation}

\answer{}
We first consider boson case, and then we have 
\begin{align}
    &LHS = i\hbar\frac{\partial}{\partial t}|\psi, t\rangle\\
    =&i\hbar \frac{\partial}{\partial t}\int d^3 x_1 \ldots d^3 x_n \psi\left(\mathbf{x}_1, \ldots, \mathbf{x}_n ; t\right) a^{\dagger}\left(\mathbf{x}_1\right) \ldots a^{\dagger}\left(\mathbf{x}_n\right)|0\rangle\\
    =& \int d^3 x_1 \ldots d^3 x_n i\hbar\frac{\partial}{\partial t}\psi\left(\mathbf{x}_1, \ldots, \mathbf{x}_n ; t\right) a^{\dagger}\left(\mathbf{x}_1\right) \ldots a^{\dagger}\left(\mathbf{x}_n\right)|0\rangle\\
    =&\int d^3 x_1 \ldots d^3 x_n \left[\sum_{j=1}^n\left(-\frac{\hbar^2}{2 m} \nabla_j^2+U\left(\mathbf{x}_j\right)\right)+\sum_{j=1}^n \sum_{k=1}^{j-1} V\left(\mathbf{x}_j-\mathbf{x}_k\right)\right] \psi a^{\dagger}\left(\mathbf{x}_1\right) \ldots a^{\dagger}\left(\mathbf{x}_n\right)|0\rangle
\end{align}

\begin{align}
    &RHS=H|\psi, t\rangle\\
    =&\left[\int d^3 x a^{\dagger}(\mathbf{x})\left(-\frac{\hbar^2}{2 m} \nabla^2+U(\mathbf{x})\right) a(\mathbf{x}) 
 +\frac{1}{2} \int d^3 x d^3 y V(\mathbf{x}-\mathbf{y}) a^{\dagger}(\mathbf{x}) a^{\dagger}(\mathbf{y}) a(\mathbf{y}) a(\mathbf{x})\right]|\psi, t\rangle\\
 =&\int d^3 x a^{\dagger}(\mathbf{x})\left(-\frac{\hbar^2}{2 m} \nabla^2+U(\mathbf{x})\right) a(\mathbf{x}) \int d^3 x_1 \ldots d^3 x_n \psi\left(\mathbf{x}_1, \ldots, \mathbf{x}_n ; t\right) a^{\dagger}\left(\mathbf{x}_1\right) \ldots a^{\dagger}\left(\mathbf{x}_n\right)|0\rangle\label{eq:self}
 \\ 
+&\frac{1}{2} \int d^3 x d^3 y V(\mathbf{x}-\mathbf{y}) a^{\dagger}(\mathbf{x}) a^{\dagger}(\mathbf{y}) a(\mathbf{y}) a(\mathbf{x})\int d^3 x_1 \ldots d^3 x_n \psi\left(\mathbf{x}_1, \ldots, \mathbf{x}_n ; t\right) a^{\dagger}\left(\mathbf{x}_1\right) \ldots a^{\dagger}\left(\mathbf{x}_n\right)|0\rangle\label{eq:inter}
\end{align}
For the term in Equation~\ref{eq:self}, by considering $[a(\mathbf{x}),a^\dagger\mathbf{(x')}]=\delta^{3}(\mathbf{x}-\mathbf{x}')$, we have 
\begin{align}
    &a(\mathbf{x})a^{\dagger}\left(\mathbf{x}_1\right) \ldots a^{\dagger}\left(\mathbf{x}_n\right)\\
    =& [a^{\dagger}(\mathbf{x}_1)a(\mathbf{x})+\delta^{3}(\mathbf{x}-\mathbf{x}_1)]a^{\dagger}\left(\mathbf{x}_2\right) \ldots a^{\dagger}\left(\mathbf{x}_n\right)\\
    =& a^{\dagger}(\mathbf{x}_1)a(\mathbf{x})a^{\dagger}\left(\mathbf{x}_2\right) \ldots a^{\dagger}\left(\mathbf{x}_n\right)+\delta^{3}(\mathbf{x}-\mathbf{x}_1)a^{\dagger}\left(\mathbf{x}_2\right) \ldots a^{\dagger}\left(\mathbf{x}_n\right)\\
    =&a^{\dagger}\left(\mathbf{x}_1\right) \ldots a^{\dagger}\left(\mathbf{x}_n\right)a(\mathbf{x})+
    \sum_{j=1}^n \delta^{3}(\mathbf{x}-\mathbf{x}_j) a^{\dagger}(\mathbf{x}_1)a^\dagger(\mathbf{x}_2)\ldots a^\dagger(\mathbf{x}_{j-1})a^\dagger(\mathbf{x}_{j+1})\ldots a^\dagger(\mathbf{x}_n)\label{eq:result-1}\\
    =&0+\sum_{j=1}^n \delta^{3}(\mathbf{x}-\mathbf{x}_j)  \mathcal{O}_j, \quad \mathcal{O}_j= a^{\dagger}(\mathbf{x}_1)a^\dagger(\mathbf{x}_2)\ldots a^\dagger(\mathbf{x}_{j-1})a^\dagger(\mathbf{x}_{j+1})\ldots a^\dagger(\mathbf{x}_n)
\end{align}
we can drop the first term in Equation~\ref{eq:result-1} since this term will act on the $|0\rangle$, giving $0$. Hence, we have
\begin{align}
    &\int d^3 x a^{\dagger}(\mathbf{x})\left(-\frac{\hbar^2}{2 m} \nabla^2+U(\mathbf{x})\right) a(\mathbf{x}) \int d^3 x_1 \ldots d^3 x_n \psi\left(\mathbf{x}_1, \ldots, \mathbf{x}_n ; t\right) a^{\dagger}\left(\mathbf{x}_1\right) \ldots a^{\dagger}\left(\mathbf{x}_n\right)|0\rangle\\
    =&\sum_{j=1}^n\int d^3 x a^{\dagger}(\mathbf{x})\left(-\frac{\hbar^2}{2 m} \nabla^2+U(\mathbf{x})\right)
    \int d^3 x_1 \ldots d^3 x_n   \psi\left(\mathbf{x}_1, \ldots, \mathbf{x}_n ; t\right) \delta^{3}(\mathbf{x}-\mathbf{x}_j)  \mathcal{O}_j |0\rangle\\
    =&\sum_{j=1}^n\int d^3 x \int d^3 x_1 \ldots d^3 x_n  a^{\dagger}(\mathbf{x})\left(-\frac{\hbar^2}{2 m} \nabla^2+U(\mathbf{x})\right)
      \psi\left(\mathbf{x}_1, \ldots, \mathbf{x}_n ; t\right)  \delta^{3}(\mathbf{x}-\mathbf{x}_j) \mathcal{O}_j |0\rangle\\
    =&\sum_{j=1}^n \int d^3 x_1 \ldots d^3 x_n  a^{\dagger}(\mathbf{x}_j)\left(-\frac{\hbar^2}{2 m} \nabla^2_j+U(\mathbf{x}_j)\right)
      \psi\left(\mathbf{x}_1, \ldots, \mathbf{x}_n ; t\right)  \mathcal{O}_j |0\rangle\ \\
    =&\sum_{j=1}^n \int d^3 x_1 \ldots d^3 x_n  a^{\dagger}(\mathbf{x}_j)\left(-\frac{\hbar^2}{2 m} \nabla^2_j+U(\mathbf{x}_j)\right)
      \psi\left(\mathbf{x}_1, \ldots, \mathbf{x}_n ; t\right)  \mathcal{O}_j |0\rangle \\
    =&\sum_{j=1}^n \int d^3 x_1 \ldots d^3 x_n  \left(-\frac{\hbar^2}{2 m} \nabla^2_j+U(\mathbf{x}_j)\right)
      \psi\left(\mathbf{x}_1, \ldots, \mathbf{x}_n ; t\right)  a^{\dagger}(\mathbf{x}_j)\mathcal{O}_j |0\rangle\\
    =& \int d^3 x_1 \ldots d^3 x_n \sum_{j=1}^n \left(-\frac{\hbar^2}{2 m} \nabla^2_j+U(\mathbf{x}_j)\right)
      \psi\left(\mathbf{x}_1, \ldots, \mathbf{x}_n ; t\right)  a^{\dagger}\left(\mathbf{x}_1\right) \ldots a^{\dagger}\left(\mathbf{x}_n\right)|0\rangle,
\end{align}
where $a^{\dagger}\left(\mathbf{x}_1\right) \ldots a^{\dagger}\left(\mathbf{x}_n\right)=a^{\dagger}(\mathbf{x}_j)\mathcal{O}_j$ since they (boson fields) commute. Now, we do the same thing for the term in Equation~\ref{eq:inter}, we have 
\begin{align}
    &a(\mathbf{y})a(\mathbf{x})a^{\dagger}\left(\mathbf{x}_1\right) \ldots a^{\dagger}\left(\mathbf{x}_n\right)\\
    =&a(\mathbf{y})\sum_{j=1}^n \delta^{3}(\mathbf{x}-\mathbf{x}_j)  \mathcal{O}_j, \quad \mathcal{O}_j= a^{\dagger}(\mathbf{x}_1)a^\dagger(\mathbf{x}_2)\ldots a^\dagger(\mathbf{x}_{j-1})a^\dagger(\mathbf{x}_{j+1})\ldots a^\dagger(\mathbf{x}_n)\\
    =&\sum_{i\neq j}^n\sum_{j=1}^n \delta^{3}(\mathbf{y}-\mathbf{x}_i) \delta^{3}(\mathbf{x}-\mathbf{x}_j) \mathcal{T}_{ij},\quad \mathcal{T}_{ij}=\prod_{k\neq i , j}^na^\dagger(\mathbf{x}_k).
\end{align}
Hence, we have 

\begin{align}
    &\frac{1}{2} \int d^3 x d^3 y V(\mathbf{x}-\mathbf{y}) a^{\dagger}(\mathbf{x}) a^{\dagger}(\mathbf{y}) a(\mathbf{y}) a(\mathbf{x})\int d^3 x_1 \ldots d^3 x_n \psi\left(\mathbf{x}_1, \ldots, \mathbf{x}_n ; t\right) a^{\dagger}\left(\mathbf{x}_1\right) \ldots a^{\dagger}\left(\mathbf{x}_n\right)|0\rangle\\
    =&\frac{1}{2} \int d^3 x d^3 y \int d^3 x_1 \ldots d^3 x_n V(\mathbf{x}-\mathbf{y}) a^{\dagger}(\mathbf{x}) a^{\dagger}(\mathbf{y}) \psi\left(\mathbf{x}_1, \ldots, \mathbf{x}_n ; t\right) a(\mathbf{y}) a(\mathbf{x}) a^{\dagger}\left(\mathbf{x}_1\right) \ldots a^{\dagger}\left(\mathbf{x}_n\right)|0\rangle\\
    =&\frac{1}{2} \int d^3 x d^3 y \int d^3 x_1 \ldots d^3 x_n V(\mathbf{x}-\mathbf{y}) a^{\dagger}(\mathbf{x}) a^{\dagger}(\mathbf{y}) \psi\left(\mathbf{x}_1, \ldots, \mathbf{x}_n ; t\right) \sum_{i\neq j}^n\sum_{j=1}^n \delta^{3}(\mathbf{y}-\mathbf{x}_i) \delta^{3}(\mathbf{x}-\mathbf{x}_j) \mathcal{T}_{ij}|0\rangle\\
    =&\sum_{i\neq j}^n\sum_{j=1}^n\frac{1}{2} \int d^3 x d^3 y \int d^3 x_1 \ldots d^3 x_n V(\mathbf{x}-\mathbf{y}) \psi\left(\mathbf{x}_1, \ldots, \mathbf{x}_n ; t\right)  \delta^{3}(\mathbf{y}-\mathbf{x}_i) \delta^{3}(\mathbf{x}-\mathbf{x}_j) a^{\dagger}(\mathbf{x}) a^{\dagger}(\mathbf{y}) \mathcal{T}_{ij}|0\rangle\\
    =&\sum_{i\neq j}^n\sum_{j=1}^n\frac{1}{2} \int d^3 x_1 \ldots d^3 x_n V(\mathbf{x}_j-\mathbf{x}_i) \psi\left(\mathbf{x}_1, \ldots, \mathbf{x}_n ; t\right)   a^{\dagger}(\mathbf{x}_j) a^{\dagger}(\mathbf{x}_i) \mathcal{T}_{ij}|0\rangle\\
    =&\int d^3 x_1 \ldots d^3 x_n \sum_{i\neq j}^n\sum_{j=1}^n\frac{1}{2}  V(\mathbf{x}_j-\mathbf{x}_i) \psi\left(\mathbf{x}_1, \ldots, \mathbf{x}_n ; t\right)  a^{\dagger}\left(\mathbf{x}_1\right) \ldots a^{\dagger}\left(\mathbf{x}_n\right)|0\rangle\\
    =&\int d^3 x_1 \ldots d^3 x_n \sum_{j=1}^n \sum_{k=1}^{j-1} V\left(\mathbf{x}_j-\mathbf{x}_k\right) \psi\left(\mathbf{x}_1, \ldots, \mathbf{x}_n ; t\right)  a^{\dagger}\left(\mathbf{x}_1\right) \ldots a^{\dagger}\left(\mathbf{x}_n\right)|0\rangle, 
\end{align}
where $a^{\dagger}(\mathbf{x}_j) a^{\dagger}(\mathbf{x}_i) \mathcal{T}_{ij}=a^{\dagger}\left(\mathbf{x}_1\right) \ldots a^{\dagger}\left(\mathbf{x}_n\right)$ for the same reason. Hence, we have proved the $LHS=RHS$ and Equation~\ref{eq:1.1} for the boson field case. 

For fermion fields, the only difference is the anti-commutation relation. We start from Equation~\ref{eq:self} again, by considering$\{a(\mathbf{x}),a^\dagger(\mathbf{x})\}=\delta^3(\mathbf{x}-\mathbf{x}')$, and we have 
\begin{align}
    &a(\mathbf{x})a^{\dagger}\left(\mathbf{x}_1\right) \ldots a^{\dagger}\left(\mathbf{x}_n\right)\\
    =&[-a^{\dagger}(\mathbf{x}_1)a(\mathbf{x})+\delta^{3}(\mathbf{x}-\mathbf{x}_1)]a^{\dagger}\left(\mathbf{x}_2\right) \ldots a^{\dagger}\left(\mathbf{x}_n\right)\\
    =&(-1)^na^{\dagger}\left(\mathbf{x}_1\right) \ldots a^{\dagger}\left(\mathbf{x}_n\right)a(\mathbf{x})+
    \sum_{j=1}^n (-1)^{j-1} \delta^{3}(\mathbf{x}-\mathbf{x}_j) a^{\dagger}(\mathbf{x}_1)a^\dagger(\mathbf{x}_2)\ldots a^\dagger(\mathbf{x}_{j-1})a^\dagger(\mathbf{x}_{j+1})\ldots a^\dagger(\mathbf{x}_n)\label{eq:result-2}\\
    =&0+\sum_{j=1}^n (-1)^{j-1} \delta^{3}(\mathbf{x}-\mathbf{x}_j) \mathcal{O}_{j},\quad \mathcal{O}_j= a^{\dagger}(\mathbf{x}_1)a^\dagger(\mathbf{x}_2)\ldots a^\dagger(\mathbf{x}_{j-1})a^\dagger(\mathbf{x}_{j+1})\ldots a^\dagger(\mathbf{x}_n),
\end{align}
where the $0$ term comes from the same reason. Then the term in Equation~\ref{eq:self} is given by 
\begin{align}
     &\int d^3 x a^{\dagger}(\mathbf{x})\left(-\frac{\hbar^2}{2 m} \nabla^2+U(\mathbf{x})\right) a(\mathbf{x}) \int d^3 x_1 \ldots d^3 x_n \psi\left(\mathbf{x}_1, \ldots, \mathbf{x}_n ; t\right) a^{\dagger}\left(\mathbf{x}_1\right) \ldots a^{\dagger}\left(\mathbf{x}_n\right)|0\rangle\\
     =&\int d^3 x \int d^3 x_1 \ldots d^3 x_n a^{\dagger}(\mathbf{x})\left(-\frac{\hbar^2}{2 m} \nabla^2+U(\mathbf{x})\right)  \psi\left(\mathbf{x}_1, \ldots, \mathbf{x}_n ; t\right) a(\mathbf{x}) a^{\dagger}\left(\mathbf{x}_1\right) \ldots a^{\dagger}\left(\mathbf{x}_n\right)|0\rangle\\
     =&\int d^3 x \int d^3 x_1 \ldots d^3 x_n a^{\dagger}(\mathbf{x})\left(-\frac{\hbar^2}{2 m} \nabla^2+U(\mathbf{x})\right)  \psi\left(\mathbf{x}_1, \ldots, \mathbf{x}_n ; t\right) \sum_{j=1}^n (-1)^j \delta^{3}(\mathbf{x}-\mathbf{x}_j) \mathcal{O}_{j}|0\rangle\\
     =&\sum_{j=1}^n \int d^3 x \int d^3 x_1 \ldots d^3 x_n a^{\dagger}(\mathbf{x})\left(-\frac{\hbar^2}{2 m} \nabla^2+U(\mathbf{x})\right)  \psi\left(\mathbf{x}_1, \ldots, \mathbf{x}_n ; t\right) (-1)^j \delta^{3}(\mathbf{x}-\mathbf{x}_j) \mathcal{O}_{j}|0\rangle\\
     =&\sum_{j=1}^n  \int d^3 x_1 \ldots d^3 x_n a^{\dagger}(\mathbf{x}_j)\left(-\frac{\hbar^2}{2 m} \nabla^2_j+U(\mathbf{x}_j)\right)  \psi\left(\mathbf{x}_1, \ldots, \mathbf{x}_n ; t\right) (-1)^j \mathcal{O}_{j}|0\rangle\\
     =&\sum_{j=1}^n  \int d^3 x_1 \ldots d^3 x_n \left(-\frac{\hbar^2}{2 m} \nabla^2_j+U(\mathbf{x}_j)\right)  \psi\left(\mathbf{x}_1, \ldots, \mathbf{x}_n ; t\right) (-1)^j a^{\dagger}(\mathbf{x}_j)\mathcal{O}_{j}|0\rangle\\
     =&\int d^3 x_1 \ldots d^3 x_n \sum_{j=1}^n \left(-\frac{\hbar^2}{2 m} \nabla^2_j+U(\mathbf{x}_j)\right)
      \psi\left(\mathbf{x}_1, \ldots, \mathbf{x}_n ; t\right)  a^{\dagger}\left(\mathbf{x}_1\right) \ldots a^{\dagger}\left(\mathbf{x}_n\right)|0\rangle,
\end{align}
where $a^{\dagger}\left(\mathbf{x}_1\right) \ldots a^{\dagger}\left(\mathbf{x}_n\right)=(-1)^{j-1}a^{\dagger}(\mathbf{x}_j)\mathcal{O}_j$ by anti-commutation relation of fermion fields. Next, given the term in Equation~\ref{eq:inter}, we have
\begin{align}
    &a(\mathbf{y})a(\mathbf{x})a^{\dagger}\left(\mathbf{x}_1\right) \ldots a^{\dagger}\left(\mathbf{x}_n\right)\\
    =&a(\mathbf{y})\sum_{j=1}^n (-1)^{j-1}\delta^{3}(\mathbf{x}-\mathbf{x}_j)  \mathcal{O}_j, \quad \mathcal{O}_j= a^{\dagger}(\mathbf{x}_1)a^\dagger(\mathbf{x}_2)\ldots a^\dagger(\mathbf{x}_{j-1})a^\dagger(\mathbf{x}_{j+1})\ldots a^\dagger(\mathbf{x}_n)\\
    =&\sum_{i< j}^n\sum_{j=1}^n(-1)^{i-1} (-1)^{j-1}\delta^{3}(\mathbf{y}-\mathbf{x}_i) \delta^{3}(\mathbf{x}-\mathbf{x}_j) \mathcal{T}_{ij}+\sum_{i>j}^n\sum_{j=1}^n(-1)^{i-2} (-1)^{j-1}\delta^{3}(\mathbf{y}-\mathbf{x}_i) \delta^{3}(\mathbf{x}-\mathbf{x}_j) \mathcal{T}_{ij},
\end{align}
where $\mathcal{T}_{ij}=\prod_{k\neq i , j}^na^\dagger(\mathbf{x}_k)$. Now, we can simplify Equation~\ref{eq:inter} and it is given by 
\begin{align}
    &\frac{1}{2} \int d^3 x d^3 y V(\mathbf{x}-\mathbf{y}) a^{\dagger}(\mathbf{x}) a^{\dagger}(\mathbf{y}) a(\mathbf{y}) a(\mathbf{x})\int d^3 x_1 \ldots d^3 x_n \psi\left(\mathbf{x}_1, \ldots, \mathbf{x}_n ; t\right) a^{\dagger}\left(\mathbf{x}_1\right) \ldots a^{\dagger}\left(\mathbf{x}_n\right)|0\rangle\\
    =&\frac{1}{2} \int d^3 x d^3 y \int d^3 x_1 \ldots d^3 x_n V(\mathbf{x}-\mathbf{y}) a^{\dagger}(\mathbf{x}) a^{\dagger}(\mathbf{y}) \psi a(\mathbf{y}) a(\mathbf{x}) a^{\dagger}\left(\mathbf{x}_1\right) \ldots a^{\dagger}\left(\mathbf{x}_n\right)|0\rangle\\
    =&\frac{1}{2} \int d^3 x d^3 y \int d^3 x_1 \ldots d^3 x_n V(\mathbf{x}-\mathbf{y}) a^{\dagger}(\mathbf{x}) a^{\dagger}(\mathbf{y}) \psi \sum_{i< j}^n\sum_{j=1}^n(-1)^{i-1} (-1)^{j-1}\delta^{3}(\mathbf{y}-\mathbf{x}_i) \delta^{3}(\mathbf{x}-\mathbf{x}_j) \mathcal{T}_{ij}|0\rangle \notag\\
    +&\frac{1}{2} \int d^3 x d^3 y \int d^3 x_1 \ldots d^3 x_n V(\mathbf{x}-\mathbf{y}) a^{\dagger}(\mathbf{x}) a^{\dagger}(\mathbf{y}) \psi \sum_{i>j}^n\sum_{j=1}^n(-1)^{i-2} (-1)^{j-1}\delta^{3}(\mathbf{y}-\mathbf{x}_i) \delta^{3}(\mathbf{x}-\mathbf{x}_j) \mathcal{T}_{ij}|0\rangle\\
    =& \frac{1}{2} \int d^3 x_1 \ldots d^3 x_n \sum_{i< j}^n\sum_{j=1}^n(-1)^{i-1} (-1)^{j-1} V(\mathbf{x}_j-\mathbf{x}_i) a^{\dagger}(\mathbf{x}_j) a^{\dagger}(\mathbf{x}_i) \psi  \mathcal{T}_{ij}|0\rangle \notag
    \\
    +&\frac{1}{2} \int d^3 x_1 \ldots d^3 x_n \sum_{i> j}^n\sum_{j=1}^n(-1)^{i-2} (-1)^{j-1} V(\mathbf{x}_j-\mathbf{x}_i) a^{\dagger}(\mathbf{x}_j) a^{\dagger}(\mathbf{x}_i) \psi  \mathcal{T}_{ij}|0\rangle\\
    =&\frac{1}{2} \int d^3 x_1 \ldots d^3 x_n \sum_{i\neq j}^n\sum_{j=1}^n(-1)^{i-1} (-1)^{j-1} V(\mathbf{x}_j-\mathbf{x}_i) a^{\dagger}(\mathbf{x}_j) a^{\dagger}(\mathbf{x}_i) \psi  \mathcal{T}_{ij}|0\rangle\\
    =&\frac{1}{2} \int d^3 x_1 \ldots d^3 x_n \sum_{i\neq j}^n\sum_{j=1}^n V(\mathbf{x}_j-\mathbf{x}_i)  \psi   a^{\dagger}\left(\mathbf{x}_1\right) \ldots a^{\dagger}\left(\mathbf{x}_n\right)|0\rangle\\
    =&\int d^3 x_1 \ldots d^3 x_n \sum_{j=1}^n \sum_{k=1}^{j-1} V\left(\mathbf{x}_j-\mathbf{x}_k\right) \psi a^{\dagger}\left(\mathbf{x}_1\right) \ldots a^{\dagger}\left(\mathbf{x}_n\right)|0\rangle,
\end{align}
where $(-1)^{i-1}(-1)^{j-1} a^{\dagger}(\mathbf{x}_j) a^{\dagger}(\mathbf{x}_i)  \mathcal{T}_{ij} = a^{\dagger}\left(\mathbf{x}_1\right) \ldots a^{\dagger}\left(\mathbf{x}_n\right)$ by anti-commutation relation. In summary, we have proved both cases for boson fields and fermion fields. \qed
\clearpage


\question{2}{Problem 2.3}

Verify that eq. (2.16) follows from eq. (2.14).


\begin{align}
    U^{-1}(\Lambda)M^{\mu\nu}U(\Lambda)&={\Lambda^\mu}_\rho{\Lambda^\nu}_\sigma M^{\rho\sigma} \tag{2.14}\\
    \left[M^{\mu \nu}, M^{\rho \sigma}\right]&=i \hbar\left(g^{\mu \rho} M^{\nu \sigma}-(\mu \leftrightarrow \nu)\right)-(\rho \leftrightarrow \sigma)\tag{2.16}\\
    &=i \hbar\left(g^{\mu \rho} M^{\nu \sigma}-g^{\nu \rho} M^{\mu \sigma} - g^{\mu \sigma} M^{\nu \rho}+g^{\nu \sigma} M^{\mu \rho}
    \right)\tag{2.16}
\end{align}

\answer{}
Considering an infinitesimal transformation in $U(\Lambda)=1+\frac{i}{2\hbar}\delta\omega_{\alpha\beta}M^{\alpha\beta}$ and ${\Lambda^\mu}_\nu={\delta^\mu}_\nu+\delta{\omega^\mu}_\nu$, now we get 
\begin{align}
    LHS=&\left(1-\frac{i}{2\hbar}\delta\omega_{\alpha\beta}M^{\alpha\beta} \right) M^{\mu\nu}\left(1+\frac{i}{2\hbar}\delta\omega_{\alpha\beta}M^{\alpha\beta} \right)\\
    \Rightarrow&\delta\omega_{\alpha\beta}\frac{i}{2\hbar}[M^{\mu\nu},M^{\alpha\beta}]=\delta\omega_{\rho\sigma}\frac{i}{2\hbar}[M^{\mu\nu},M^{\rho\sigma}]\\
    RHS=&({\delta^\mu}_\rho+{\delta\omega^\mu}_\rho)({\delta^\nu}_\sigma+{\delta\omega^\nu}_\sigma)M^{\rho\sigma}\\
    \to&{\delta^\mu}_\rho{\delta\omega^\nu}_\sigma M^{\rho\sigma}+
    {\delta^\nu}_\sigma{\delta\omega^\mu}_\rho M^{\rho\sigma}={\delta\omega^\nu}_\sigma M^{\mu\sigma}+{\delta\omega^\mu}_\rho M^{\rho\nu}\\
  =& g^{\nu\rho} {\delta\omega}_{\rho\sigma} M^{\mu\sigma}+g^{\sigma\mu}{\delta\omega}_{\sigma\rho} M^{\rho\nu}=
  {\delta\omega}_{\rho\sigma}\left( g^{\nu\rho}  M^{\mu\sigma} - g^{\sigma\mu} M^{\rho\nu}\right),
\end{align}
we only consider the linear term $\delta\omega$. We can further simplify it to 
\begin{align}
    [M^{\mu\nu},M^{\rho\sigma}] &= \frac{2\hbar}{i}\left( g^{\nu\rho}  M^{\mu\sigma} - g^{\sigma\mu} M^{\rho\nu}\right)=2i\hbar\left(g^{\sigma\mu} M^{\rho\nu}- g^{\nu\rho}  M^{\mu\sigma}  \right)=2i\hbar\left(-g^{\mu\sigma} M^{\nu\rho}- g^{\nu\rho}  M^{\mu\sigma}  \right)\\
    &=-  [M^{\nu\mu},M^{\rho\sigma}]= 2i\hbar\left(g^{\nu\sigma} M^{\mu\rho}+ g^{\mu\rho}  M^{\nu\sigma}  \right)
\end{align}
Finally, we have 
\begin{align}
    &[M^{\mu\nu},M^{\rho\sigma}]\\
    =&\frac{1}{2} \left([M^{\mu\nu},M^{\rho\sigma}]-  [M^{\nu\mu},M^{\rho\sigma}]\right)\\
    =&i\hbar \left(g^{\nu\sigma} M^{\mu\rho}+ g^{\mu\rho}  M^{\nu\sigma} -g^{\mu\sigma} M^{\nu\rho}- g^{\nu\rho}  M^{\mu\sigma} \right).
\end{align}
\qed
\clearpage


\question{3}{Problem 2.8}


\begin{itemize}
    \item [(a)] Let $\Lambda=1+\delta\omega$ in eq.(2.26), and show that 
    \begin{align*}
        [\varphi(x),M^{\mu\nu}]=\mathcal{L}^{\mu\nu}\varphi(x),
    \end{align*}
    where 
    \begin{align*}
        \mathcal{L}^{\mu\nu} \equiv \frac{\hbar}{i}(x^\mu\partial^\nu - x^\nu\partial^\mu).
    \end{align*}
    \item [(b)]Show that $[[\varphi(x),M^{\mu\nu}],M^{\rho\sigma}]=\mathcal{L}^{\mu\nu}\mathcal{L}^{\rho\sigma}\varphi(x)$.
    \item [(c)] Prove the \textit{Jacobi identity}, $[[A,B],C]+[[B,C],A]+[[C,A],B]=0$. Hint: write out all the commutations.
    \item [(d)] Use your results from parts (b) and (c) to show that 
    \begin{align*}
        [\varphi(x),[M^{\mu\nu},M^{\rho\sigma}]]=(\mathcal{L}^{\mu\nu}\mathcal{L}^{\rho\sigma}-\mathcal{L}^{\rho\sigma}\mathcal{L}^{\mu\nu})\varphi(x).
        \tag{2.31}
    \end{align*}
    \item [(e)] Simplify the right-hand side of eq.~(2.31) as much as possible. 
    \item [(f)] Use your results from part (e) to verify eq.~(2.16), up to the possibility of a term on the right-hand side that commutes with $\varphi(x)$ and its derivatives. (Such a term, called a \textit{central charge}, in fact does not arise for the Lorentz algebra.)
\end{itemize}
\begin{align}
    U^{-1}(\Lambda) \varphi(x)U(\Lambda)=\varphi(\Lambda^{-1}x)
    \tag{2.26}
\end{align}
\answer{}
\begin{itemize}
    \item [(a)]
\end{itemize}
\begin{align}
    LHS=&U^{-1}(\Lambda) \varphi(x)U(\Lambda)\\
    =&\left(1-\frac{i}{2\hbar}\delta\omega_{\mu\nu}M^{\mu\nu} \right)  \varphi(x)\left(1+\frac{i}{2\hbar}\delta\omega_{\mu\nu}M^{\mu\nu} \right)\\
    \to&\delta\omega_{\mu\nu}\frac{i}{2\hbar}[\varphi(x),M^{\mu\nu}]\\
    RHS=&\varphi(\Lambda^{-1}x)=\varphi(({\delta^\mu}_\nu-{\delta\omega^\mu}_\nu)x^\nu)=\phi(x)-{\delta\omega^\mu}_\nu x^\nu\partial_\mu \phi(x)\\
    \to&-{\delta\omega^\mu}_\nu x^\nu\partial_\mu \varphi(x)=-{\delta\omega}_{\mu\nu} x^\nu\partial^\mu \varphi(x)={\delta\omega}_{\mu\nu}\frac{1}{2}\left(x^\mu\partial^\nu -x^\nu\partial^\mu
    \right)\varphi(x),
\end{align}
we only focus on the linear term $\delta\omega$. Now we have 
\begin{align}
    [\varphi(x),M^{\mu\nu}]=\frac{\hbar}{i}(x^\mu\partial^\nu -x^\nu\partial^\mu)\varphi(x)=\mathcal{L}^{\mu\nu}\varphi(x).
\end{align}

\begin{itemize}
    \item [(b)]
\end{itemize}
\begin{align}
    &[[\varphi(x),M^{\mu\nu}],M^{\rho\sigma}]=[\mathcal{L}^{\mu\nu}\varphi(x),M^{\rho\sigma}]=\mathcal{L}^{\mu\nu}\varphi(x)M^{\rho\sigma}-M^{\rho\sigma}\mathcal{L}^{\mu\nu}\varphi(x)\\
    =&\mathcal{L}^{\mu\nu}[\varphi(x),M^{\rho\sigma}]=\mathcal{L}^{\mu\nu}\mathcal{L}^{\rho\sigma}\varphi(x).
\end{align}

\begin{itemize}
    \item [(c)]
\end{itemize}
\begin{align}
    &[[A,B],C]+[[B,C],A]+[[C,A],B]\\
    =&({\color{blue}CAB}-{\color{red}CBA}-{\color{brown}ABC}+{\color{teal}BAC})+({\color{brown}ABC}-{\color{orange}ACB}-BCA+{\color{red}CBA})+(BCA-{\color{teal}BAC}-{\color{blue}CAB}+{\color{orange}ACB})\\
    =&0.
\end{align}

\begin{itemize}
    \item [(d)]
\end{itemize}
By the Jacobi identity, we have 
\begin{align}
    &0=[[\varphi(x),M^{\mu\nu}],M^{\rho\sigma}]+[[M^{\mu\nu},M^{\rho\sigma}],\varphi(x)]+[[M^{\rho\sigma},\varphi(x)],M^{\mu\nu}]\\
    \to&[\varphi(x),[M^{\mu\nu},M^{\rho\sigma}]]=[[\varphi(x),M^{\mu\nu}],M^{\rho\sigma}]+[[M^{\rho\sigma},\varphi(x)],M^{\mu\nu}]\\
    =&[[\varphi(x),M^{\mu\nu}],M^{\rho\sigma}]-[[\varphi(x),M^{\rho\sigma}],M^{\mu\nu}]\\
    =&\mathcal{L}^{\mu\nu}\mathcal{L}^{\rho\sigma}\varphi(x)-\mathcal{L}^{\rho\sigma}\mathcal{L}^{\mu\nu}\varphi(x)\\
    =&(\mathcal{L}^{\mu\nu}\mathcal{L}^{\rho\sigma}-\mathcal{L}^{\rho\sigma}\mathcal{L}^{\mu\nu})\varphi(x).\label{eq:result-3d}
\end{align}
\begin{itemize}
    \item [(e)]
\end{itemize}
For the result in Equation~\ref{eq:result-3d}, considering the relation $\partial^\mu x^\nu \varphi(x)=(g^{\mu\nu}+x^\nu\partial^\mu)\varphi(x)$, we can have
\begin{align}
    &\mathcal{L}^{\mu\nu}\mathcal{L}^{\rho\sigma}\varphi(x)=\frac{\hbar}{i}\frac{\hbar}{i}(x^\mu\partial^\nu -x^\nu\partial^\mu)(x^\rho\partial^\sigma -x^\sigma\partial^\rho)\varphi(x)\\
    =&\left(\frac{\hbar}{i}\right)^2 \left[
    x^\mu(g^{\nu\rho}+x^\rho\partial^\nu)\partial^\sigma-
    x^\nu(g^{\mu\rho}+x^\rho\partial^\mu)\partial^\sigma-x^\mu(g^{\nu\sigma}+x^\sigma\partial^\nu)\partial^\rho+x^\nu(g^{\mu\sigma}+x^\sigma\partial^\mu)\partial^\rho
    \right]\varphi(x)\\
    =&\left(\frac{\hbar}{i}\right)^2\left[
    g^{\nu\rho}x^\mu\partial^\sigma-g^{\mu\rho}x^{\nu}\partial^\sigma-g^{\nu\sigma}x^\mu\partial^\rho+g^{\mu\sigma}x^\nu\partial^\rho+
    x^\mu x^\rho\partial^\nu\partial^\sigma-x^\nu x^\rho\partial^\mu\partial^\sigma-x^\mu x^\sigma\partial^\nu\partial^\rho+x^\nu x^\sigma\partial^\mu\partial^\sigma
    \right]\varphi(x).
\end{align}
\begin{align}
    &\mathcal{L}^{\rho\sigma}\mathcal{L}^{\mu\nu}\varphi(x)\\
    =&\left(\frac{\hbar}{i}\right)^2\left[g^{\sigma\mu}\,x^\rho\partial^\nu
- g^{\rho\mu}\,x^\sigma\partial^\nu
- g^{\sigma\nu}\,x^\rho\partial^\mu
+ g^{\rho\nu}\,x^\sigma\partial^\mu 
+\,x^\rho x^\mu\,\partial^\sigma\partial^\nu
- x^\rho x^\nu\,\partial^\sigma\partial^\mu
- x^\sigma x^\mu\,\partial^\rho\partial^\nu
+ x^\sigma x^\nu\,\partial^\rho\partial^\mu
    \right]\varphi(x).
\end{align}
Using simpler forms to express:
\begin{align}
\mathcal{L}^{\mu\nu}\mathcal{L}^{\rho\sigma}\varphi(x)
=&\left(\frac{\hbar}{i}\right)^2\Big[
g^{\nu\rho}\,x^\mu\partial^\sigma
- g^{\mu\rho}\,x^\nu\partial^\sigma
- g^{\nu\sigma}\,x^\mu\partial^\rho
+ g^{\mu\sigma}\,x^\nu\partial^\rho \nonumber\\
&\qquad
+\,x^\mu x^\rho\,\partial^\nu\partial^\sigma
- x^\mu x^\sigma\,\partial^\nu\partial^\rho
- x^\nu x^\rho\,\partial^\mu\partial^\sigma
+ x^\nu x^\sigma\,\partial^\mu\partial^\rho
\Big]\varphi(x), \\
\mathcal{L}^{\rho\sigma}\mathcal{L}^{\mu\nu}\varphi(x)
=&\left(\frac{\hbar}{i}\right)^2\Big[
g^{\sigma\mu}\,x^\rho\partial^\nu
- g^{\rho\mu}\,x^\sigma\partial^\nu
- g^{\sigma\nu}\,x^\rho\partial^\mu
+ g^{\rho\nu}\,x^\sigma\partial^\mu \nonumber\\
&\qquad
+\,x^\rho x^\mu\,\partial^\sigma\partial^\nu
- x^\sigma x^\mu\,\partial^\rho\partial^\nu
- x^\rho x^\nu\,\partial^\sigma\partial^\mu
+ x^\sigma x^\nu\,\partial^\rho\partial^\mu
\Big]\varphi(x).
\end{align}
Combining those two results together, it gives 
\begin{align}
(\mathcal{L}^{\mu\nu}\mathcal{L}^{\rho\sigma}-\mathcal{L}^{\rho\sigma}\mathcal{L}^{\mu\nu})\varphi(x)
=&\left(\frac{\hbar}{i}\right)^2\Big[
g^{\nu\rho}(x^\mu\partial^\sigma-x^\sigma\partial^\mu)
- g^{\mu\rho}(x^\nu\partial^\sigma-x^\sigma\partial^\nu) \nonumber\\
&\qquad
- g^{\nu\sigma}(x^\mu\partial^\rho-x^\rho\partial^\mu)
+ g^{\mu\sigma}(x^\nu\partial^\rho-x^\rho\partial^\nu)
\Big]\varphi(x) \\[1em]
=&\;\frac{\hbar}{i}\Big(
g^{\nu\rho}\mathcal{L}^{\mu\sigma}
- g^{\mu\rho}\mathcal{L}^{\nu\sigma}
- g^{\nu\sigma}\mathcal{L}^{\mu\rho}
+ g^{\mu\sigma}\mathcal{L}^{\nu\rho}
\Big)\varphi(x)\\
=&i\hbar\Big(
 g^{\mu\rho}\mathcal{L}^{\nu\sigma}
+ g^{\nu\sigma}\mathcal{L}^{\mu\rho}
- g^{\nu\rho}\mathcal{L}^{\mu\sigma}
- g^{\mu\sigma}\mathcal{L}^{\nu\rho}
\Big)\varphi(x).
\end{align}
Actually, it looks similar to 
\begin{align}
    &[M^{\mu\nu},M^{\rho\sigma}]\notag\\
    =&i\hbar \left(g^{\nu\sigma} M^{\mu\rho}+ g^{\mu\rho}  M^{\nu\sigma} -g^{\mu\sigma} M^{\nu\rho}- g^{\nu\rho}  M^{\mu\sigma} \right)\tag{2.16}
\end{align}
\begin{itemize}
    \item [(f)]
\end{itemize}
Now we assume there is a non trivial term $\mathcal{C}$ on $[M^{\mu\nu},M^{\rho\sigma}]$, giving that 
\begin{align}
    [M^{\mu\nu},M^{\rho\sigma}]= i\hbar \left(g^{\nu\sigma} M^{\mu\rho}+ g^{\mu\rho}  M^{\nu\sigma} -g^{\mu\sigma} M^{\nu\rho}- g^{\nu\rho}  M^{\mu\sigma} +\mathcal{C} \right),
\end{align}
where $C$ can commutes with $\varphi$ and its derivatives. Now, we have 
\begin{align}
    &[\varphi(x),[M^{\mu\nu},M^{\rho\sigma}]]\tag{2.31}\\
    =&i\hbar [\varphi(x),\left(g^{\nu\sigma} M^{\mu\rho}+ g^{\mu\rho}  M^{\nu\sigma} -g^{\mu\sigma} M^{\nu\rho}- g^{\nu\rho}  M^{\mu\sigma} +\mathcal{C} \right)]\\
    =&i\hbar \bigg( g^{\nu\sigma}[\varphi(x),M^{\mu\rho}]+g^{\mu\rho}[\varphi(x),M^{\nu\sigma}]-g^{\mu\sigma}[\varphi,M^{\nu\rho}]-g^{\nu\rho}[\varphi(x),M^{\mu\sigma}]+[\varphi(x),\mathcal{C}]
    \bigg)\\
    =&i\hbar\big(
    g^{\mu\rho}\mathcal{L}^{\nu\sigma}
    + g^{\nu\sigma}\mathcal{L}^{\mu\rho}
    - g^{\nu\rho}\mathcal{L}^{\mu\sigma}
    - g^{\mu\sigma}\mathcal{L}^{\nu\rho}
    \big)\varphi(x)=(\mathcal{L}^{\mu\nu}\mathcal{L}^{\rho\sigma}-\mathcal{L}^{\rho\sigma}\mathcal{L}^{\mu\nu})\varphi(x)\\
    =&[\mathcal{L}^{\mu\nu},\mathcal{L}^{\rho\sigma}]\varphi(x).
\end{align}
Hence, with the central charge $\mathcal{C}$, the relation still holds. \qed
\clearpage
\question{4}{Problem 2.9}


Let us write 
\begin{align*}
    {\Lambda^\rho}_\tau={\delta^\rho}_\tau+\frac{i}{2\hbar}\delta\omega_{\mu\nu}{(S^{\mu\nu}_V)^\rho}_\tau,\tag{2.32}
\end{align*}
where 
\begin{align*}
    {(S^{\mu\nu}_V)^\rho}_\tau\equiv\frac{\hbar}{i}(g^{\mu\rho}{\delta^\nu}_\tau-g^{\nu\rho}{\delta^\mu}_\tau) \tag{2.33}
\end{align*}
are matrices which constitute the \textit{vector representation} of the Lorentz generators.
\begin{itemize}
    \item [(a)] Let $\Lambda=1+\delta\omega$ in eq.~(2.27), and show that 
    \begin{align*}
        [\partial^\rho \varphi(x),M^{\mu\nu}]=\mathcal{L}^{\mu\nu}\partial^\rho\varphi(x)+  {(S^{\mu\nu}_V)^\rho}_\tau\partial^{\tau}\varphi(x) \tag{2.34}
    \end{align*}
    \item [(b)] Show that the matrices $(S^{\mu\nu}_V)$ must have the same commutation relations as the operators $M^{\mu\nu}$. Hint: see the previous problem.
    \item [(c)]For a rotation by an angle $\theta$ about the $z$ axis, we have 
    \begin{align*}
        {\Lambda^\mu}_\nu=
        \begin{pmatrix}
            1 &0 &0&0\\
            0&\cos{\theta}&-\sin{\theta}&0\\
            0&\sin{\theta}&\cos{\theta}&0\\
            0&0&0&1
        \end{pmatrix}.
        \tag{2.35}
    \end{align*}
    Show that 
    \begin{align*}
        \Lambda=\exp{(-i\theta S^{12}_V/\hbar)}.  \tag{2.36}  
    \end{align*}
    \item [(d)] For a boost by \textit{rapidity} $\eta$ in the $z$ direction, we have 
    \begin{align*}
        {\Lambda^\mu}_\nu=
        \begin{pmatrix}
            \cosh{\eta} &0 &0&\sinh{\eta}\\
            0&1&0&0\\
            0&0&1&0\\
            \sinh{\eta}&0&0&\cosh{\eta}
        \end{pmatrix}.    
        \tag{2.37}
    \end{align*}
    Show that 
    \begin{align*}
        \Lambda=\exp{(+i\eta S^{30}_V/\hbar)}.     \tag{2.38}
    \end{align*}
\end{itemize}
\begin{align}
    U^{-1}(\Lambda)\partial^\rho\varphi(x)U(\Lambda)={\Lambda^{\rho}}_\mu \overline{\partial}^\mu \varphi(\Lambda^{-1}x),\quad \overline{x}^\mu={(\Lambda^{-1})^\mu}_\nu x^\nu, \quad\overline{\partial}^\mu={(\Lambda^{-1})^\mu}_\nu\partial^\nu\tag{2.27}
\end{align}
\answer{}
\begin{itemize}
    \item [(a)]
\end{itemize}
\begin{align}
    &LHS=\left(1-\frac{i}{2\hbar}\delta\omega_{\mu\nu}M^{\mu\nu} \right) \partial^\rho\varphi(x)\left(1+\frac{i}{2\hbar}\delta\omega_{\mu\nu}M^{\mu\nu} \right)\\
    =&\partial^\rho \varphi(x)+\delta\omega_{\mu\nu}\frac{i}{2\hbar}[\partial^\rho\varphi(x),M^{\mu\nu}]\\
    &RHS={\Lambda^{\rho}}_\mu \overline{\partial}^\mu \varphi(\Lambda^{-1}x)\\
    =&({\delta^\rho}_\mu+{\delta\omega^\rho}_\mu)\overline{\partial}^\mu \varphi(({\delta^\mu}_\nu-{\delta\omega^\mu}_\nu)x^\nu)\\
    =&({\delta^\rho}_\mu+{\delta\omega^\rho}_\mu)({\delta^\mu}_\nu-{\delta\omega^\mu}_\nu)\partial^\nu(\varphi(x)-{\delta\omega^\alpha}_\beta x^\beta\partial_\alpha \varphi(x))\\
    =&{\delta^{\rho}}_\nu (\partial^\nu\varphi(x)-{\delta\omega^\alpha}_\beta g^{\nu\beta}\partial_\alpha \varphi(x)-{\delta\omega^\alpha}_\beta x^{\beta}\partial^{\nu}\partial_\alpha \varphi(x))\\
    =&\partial^\rho\varphi(x)-{\delta\omega^\alpha}_\beta g^{\rho\beta}\partial_\alpha \varphi(x)-{\delta\omega^\alpha}_\beta x^{\beta}\partial^{\rho}\partial_\alpha \varphi(x)\\
    =& \partial^\rho\varphi(x)-{\delta\omega}_{\alpha\beta} g^{\rho\beta}\partial^\alpha \varphi(x)-{\delta\omega}_{\alpha\beta} x^{\beta}\partial^{\rho}\partial^\alpha \varphi(x)\\
    =& \partial^\rho\varphi(x)-{\delta\omega}_{\mu\nu} g^{\rho\nu}\partial^\mu \varphi(x)-{\delta\omega}_{\mu\nu} x^{\nu}\partial^{\rho}\partial^\mu \varphi(x)\\
    =& \partial^\rho\varphi(x)+{\delta\omega}_{\mu\nu}\left(-g^{\rho\nu}\partial^\mu -x^{\nu}\partial^{\rho}\partial^\mu \right)\varphi(x)\\
    =& \partial^\rho\varphi(x)+{\delta\omega}_{\mu\nu}\frac{1}{2}\left(-g^{\rho\nu}\partial^\mu -x^{\nu}\partial^{\rho}\partial^\mu +g^{\rho\mu}\partial^\nu +x^{\mu}\partial^{\rho}\partial^\nu \right)\varphi(x)
\end{align}
Hence, we have 
\begin{align}
&[\partial^\rho\varphi(x),M^{\mu\nu}]=\frac{\hbar}{i}\left(-g^{\rho\nu}\partial^\mu -x^{\nu}\partial^{\rho}\partial^\mu +g^{\rho\mu}\partial^\nu +x^{\mu}\partial^{\rho}\partial^\nu \right)\varphi(x)\\
=& \frac{\hbar}{i}(x^\mu\partial^\nu -x^\nu\partial^\mu)\partial^\rho\varphi(x)+\frac{\hbar}{i}(g^{\rho\mu}{\delta^\nu}_\tau \partial^\tau -g^{\rho\nu}{\delta^{\mu}}_\tau \partial^\tau)\varphi(x)\\
=&\frac{\hbar}{i}(x^\mu\partial^\nu -x^\nu\partial^\mu)\partial^\rho\varphi(x)+\frac{\hbar}{i}(g^{\mu\rho}{\delta^\nu}_\tau  -g^{\nu\rho}{\delta^{\mu}}_\tau )\partial^\tau\varphi(x)\\
=&\mathcal{L}^{\mu\nu}\partial^\rho\varphi(x)+  {(S^{\mu\nu}_V)^\rho}_\tau\partial^{\tau}\varphi(x),
\end{align}
which is exactly the result we want to prove.

\begin{itemize}
    \item [(b)]
\end{itemize}
\begin{align}
    &{[(S^{\mu\nu}_V),(S^{\rho\sigma}_V)]^\alpha}_\beta\\
    =&{(S^{\mu\nu}_V)^\alpha}_\tau {(S^{\rho\sigma}_V)^\tau}_\beta - {(S^{\rho\sigma}_V)^\alpha}_\tau {(S^{\mu\nu}_V)^\tau}_\beta\\
    =&\frac{\hbar}{i}(g^{\mu\alpha}{\delta^\nu}_\tau -g^{\nu\alpha}{\delta^\mu}_\tau)\frac{\hbar}{i}(g^{\rho\tau}{\delta^\sigma}_\beta -g^{\sigma\tau}{\delta^\rho}_\beta) - \frac{\hbar}{i}(g^{\rho\alpha}{\delta^\sigma}_\tau -g^{\sigma\alpha}{\delta^\rho}_\tau)\frac{\hbar}{i}(g^{\mu\tau}{\delta^\nu}_\beta -g^{\nu\tau}{\delta^\mu}_\beta)\\
    =&\left(\frac{\hbar}{i}\right)^2\bigg[
    {\color{red}g^{\mu\alpha}g^{\rho\nu}{\delta^\sigma}_\beta}
    {\color{black}-g^{\mu\alpha}g^{\sigma\nu}{\delta^\rho}_\beta}
    {\color{blue}-g^{\nu\alpha}g^{\rho\mu}{\delta^\sigma}_\beta}
    {\color{teal}+g^{\nu\alpha}g^{\sigma\mu}{\delta^\rho}_\beta}\\
    &\qquad
    {\color{teal}-g^{\rho\alpha}g^{\mu\sigma}{\delta^\nu}_\beta}
    {\color{black}+g^{\rho\alpha}g^{\nu\sigma}{\delta^\mu}_\beta}
    {\color{blue}+g^{\sigma\alpha}g^{\mu\rho}{\delta^\nu}_\beta}
    {\color{red}-g^{\sigma\alpha}g^{\nu\rho}{\delta^\mu}_\beta}
    \bigg]\\
    =&i\hbar\left(\frac{\hbar}{i}\right) \bigg[
    {\color{blue}g^{\mu\rho}(g^{\nu\alpha}{\delta^\sigma}_\beta -g^{\sigma\alpha}{\delta^\nu}_\beta)}{\color{red}
    -g^{\nu\rho}(g^{\mu\alpha}{\delta^\sigma}_\beta -g^{\sigma\alpha}{\delta^\mu}_\beta)}{\color{teal}
    -g^{\mu\sigma}(g^{\nu\alpha}{\delta^\rho}_\beta -g^{\rho\alpha}{\delta^\nu}_\beta)}
    +g^{\nu\sigma}(g^{\mu\alpha}{\delta^\rho}_\beta -g^{\rho\alpha}{\delta^\mu}_\beta) \bigg]\\
    =&i\hbar\left(g^{\mu\rho}{(S^{\nu\sigma}_V)^\alpha}_\beta - g^{\nu\rho}{(S^{\mu\sigma}_V)^\alpha}_\beta - g^{\mu\sigma}{(S^{\nu\rho}_V)^\alpha}_\beta + g^{\nu\sigma}{(S^{\mu\rho}_V)^\alpha}_\beta \right)\\
    =&i\hbar{\left(g^{\mu\rho}{(S^{\nu\sigma}_V)} - g^{\nu\rho}{(S^{\mu\sigma}_V)} - g^{\mu\sigma}{(S^{\nu\rho}_V)} + g^{\nu\sigma}{(S^{\mu\rho}_V)} \right)^\alpha}_\beta.
\end{align}
This looks exactly the same as
\begin{align}
    &[M^{\mu\nu},M^{\rho\sigma}]\notag\\
    =&i\hbar \left(g^{\nu\sigma} M^{\mu\rho}+ g^{\mu\rho}  M^{\nu\sigma} -g^{\mu\sigma} M^{\nu\rho}- g^{\nu\rho}  M^{\mu\sigma} \right)\tag{2.16}\end{align}
\begin{itemize}
    \item [(c)]
\end{itemize}
    \begin{align}
        \frac{i}{\hbar}{(S^{12}_V)^\mu}_\nu=&(g^{1\mu}{\delta^2}_\nu -g^{2\mu}{\delta^1}_\nu)=\begin{pmatrix}
            0 &0 &0&0\\
            0&0&1&0\\
            0&-1&0&0\\
            0&0&0&0
        \end{pmatrix}\equiv \mathcal{R},\\
        \mathcal{R}^2=&\begin{pmatrix}
            0 &0 &0&0\\
            0&-1&0&0\\
            0&0&-1&0\\
            0&0&0&0
        \end{pmatrix},\\
        \mathcal{R}^3=&\begin{pmatrix}
            0 &0 &0&0\\
            0&0 &-1&0\\
            0&1&0&0\\
            0&0&0&0
        \end{pmatrix}=-\mathcal{R},\\
        \mathcal{R}^4=&\begin{pmatrix}
            0 &0 &0&0\\
            0&1&0&0\\
            0&0&1&0\\
            0&0&0&0
        \end{pmatrix}=-\mathcal{R}^2.
    \end{align}
    Hence, we have 
    \begin{align}
        \Lambda=&\exp{(-i\theta S^{12}_V/\hbar)}=\exp{(-\theta \mathcal{R})}\\
        =&\sum_{n=0}^{\infty}\frac{(-\theta \mathcal{R})^n}{n!}=\sum_{m=0}^{\infty}\frac{(-\theta \mathcal{R})^{2m}}{(2m)!}+\sum_{m=0}^{\infty}\frac{(-\theta \mathcal{R})^{2m+1}}{(2m+1)!}\\
        =&I+\sum_{m=1}^{\infty}\frac{(-1)^m\theta^{2m}\mathcal{R}^{2m}}{(2m)!}+\sum_{m=0}^{\infty}\frac{(-1)\theta^{2m+1}\mathcal{R}^{2m+1}}{(2m+1)!}\\
        =&I+ \sum_{m=1}^{\infty}\frac{(-1)^{m}\theta^{2m}}{(2m)!}\mathcal{R}^2-\sum_{m=0}^{\infty}\frac{(-1)^{m}\theta^{2m+1}}{(2m+1)!}\mathcal{R}\\
        =&I+(\cos{\theta}-1)\mathcal{R}^2-\sin{\theta}\mathcal{R}\\
        =&\begin{pmatrix}
            1 &0 &0&0\\
            0&1&0&0\\
            0&0&1&0\\
            0&0&0&1
        \end{pmatrix}+\begin{pmatrix}
            0 &0 &0&0\\
            0&\cos{\theta}-1&0&0\\
            0&0&\cos{\theta}-1&0\\
            0&0&0&0
        \end{pmatrix}+\begin{pmatrix}
            0 &0 &0&0\\
            0&0 &-\sin{\theta}&0\\
            0&\sin{\theta}&0&0\\
            0&0&0&0
        \end{pmatrix}\\
        =&\begin{pmatrix}
            1 &0 &0&0\\
            0&\cos{\theta}&-\sin{\theta}&0\\
            0&\sin{\theta}&\cos{\theta}&0\\
            0&0&0&1
        \end{pmatrix}.
        \end{align}
    This is exactly the same as the result in Equation~(2.35).    
\begin{itemize}
    \item [(d)]
\end{itemize}
    \begin{align}
        \frac{i}{\hbar}{(S^{30}_V)^\mu}_\nu=&(g^{3\mu}{\delta^0}_\nu -g^{0\mu}{\delta^3}_\nu)=\begin{pmatrix}
            0 &0 &0&1\\
            0&0&0&0\\
            0&0&0&0\\
            1&0&0&0
        \end{pmatrix}\equiv \mathcal{B},\\
        \mathcal{B}^2=&\begin{pmatrix}
            1 &0 &0&0\\
            0&0&0&0\\
            0&0&0&0\\
            0&0&0&1
        \end{pmatrix},\\
        \mathcal{B}^3=&\begin{pmatrix}
            0 &0 &0&1\\
            0&0&0&0\\
            0&0&0&0\\
            1&0&0&0
        \end{pmatrix}=\mathcal{B},\\
        \mathcal{B}^4=&\begin{pmatrix}
            1 &0 &0&0\\
            0&0&0&0\\
            0&0&0&0\\
            0&0&0&1
        \end{pmatrix}=\mathcal{B}^2.
    \end{align}
    Hence, we have 
    \begin{align}
        \Lambda=&\exp{(+i\eta S^{30}_V/\hbar)}=\exp{(\eta \mathcal{B})}\\
        =&\sum_{n=0}^{\infty}\frac{(\eta \mathcal{B})^n}{n!}=\sum_{m=0}^{\infty}\frac{(\eta \mathcal{B})^{2m}}{(2m)!}+\sum_{m=0}^{\infty}\frac{(\eta \mathcal{B})^{2m+1}}{(2m+1)!}\\
        =&I+\sum_{m=1}^{\infty}\frac{\eta^{2m}\mathcal{B}^{2m}}{(2m)!}+\sum_{m=0}^{\infty}\frac{\eta^{2m+1}\mathcal{B}^{2m+1}}{(2m+1)!}\\
        =&I+ \sum_{m=1}^{\infty}\frac{\eta^{2m}}{(2m)!}\mathcal{B}^2+\sum_{m=0}^{\infty}\frac{\eta^{2m+1}}{(2m+1)!}\mathcal{B}\\
        =&I+(\cosh{\eta}-1)\mathcal{B}^2+\sinh{\eta}\mathcal{B}\\
        =&\begin{pmatrix}
            1 &0 &0&0\\
            0&1&0&0\\
            0&0&1&0\\
            0&0&0&1
        \end{pmatrix}+\begin{pmatrix}
            \cosh{\eta}-1 &0 &0&0\\
            0&0&0&0\\
            0&0&0&0\\
            0&0&0&\cosh{\eta}-1
        \end{pmatrix}+\begin{pmatrix}
            0 &0 &0&\sinh{\eta}\\
            0&0&0&0\\
            0&0&0&0\\
            \sinh{\eta}&0&0&0
        \end{pmatrix}\\
        =&\begin{pmatrix}
            \cosh{\eta} &0 &0&\sinh{\eta}\\
            0&1&0&0\\
            0&0&1&0\\
            \sinh{\eta}&0&0&\cosh{\eta}
        \end{pmatrix}.
        \end{align}
    This is exactly the same as the result in Equation~(2.37).\qed
\clearpage
\question{5}{Problem 3.1}

Derive eq.~(3.29) from eqs.~(3.21),~(3.24), and~(3.28).
\begin{align}
    a(\mathbf{k}) & =\int d^3 x e^{-i k x}\left[i \partial_0 \varphi(x)+\omega \varphi(x)\right] \tag{3.21}\\
    \Pi(x)&=\dot{\varphi}(x)=\partial_0\varphi(x)\tag{3.24}\\
    [\varphi(\mathbf{x}, t), \varphi(\mathbf{x}', t)]&=0, \quad [\Pi(\mathbf{x}, t), \Pi(\mathbf{x}', t)]=0, \quad
    [\varphi(\mathbf{x}, t), \Pi(\mathbf{x}', t)]=i \delta^{(3)}(\mathbf{x}-\mathbf{x}')\tag{3.28}
\end{align}
\begin{align}
    [a(\mathbf{k}), a(\mathbf{k}')] & =0, \quad [a^\dagger(\mathbf{k}), a^\dagger(\mathbf{k}')] =0, \quad
    [a(\mathbf{k}), a^\dagger(\mathbf{k}')] = (2\pi)^3 2\omega \delta^{(3)}(\mathbf{k}-\mathbf{k}') \tag{3.29}
\end{align}
\answer{} 
Since $a(\mathbf{x})$ is independent of time, we can set $x^0=y^0$, meaning all time variables are the same. Now, we compute
\begin{align}
    &[a(\mathbf{k}), a(\mathbf{k}')]\\
    =&\int d^3x d^3y e^{-ikx}e^{-ik'y}[i\partial_0\varphi(x)+\omega \varphi(x),i\partial_0\varphi(y)+\omega \varphi(y)]\\
    =&\int d^3x d^3y e^{-ikx}e^{-ik'y}\bigg(
    -[\partial_0\varphi(x),\partial_0\varphi(y)]+i\omega[\partial_0\varphi(x),\varphi(y)]+i\omega[\varphi(x),\partial_0\varphi(y)]+\omega^2[\varphi(x),\varphi(y)]
    \bigg)\\
    =&\int d^3x d^3y e^{-ikx}e^{-ik'y}\bigg(
    i\omega[\partial_0\varphi(x),\varphi(y)]+i\omega[\varphi(x),\partial_0\varphi(y)]
    \bigg)\\
    =&\int d^3x d^3y e^{-ikx}e^{-ik'y}\bigg(
    i\omega(-i\delta^{(3)}(\mathbf{x}-\mathbf{y}))+i\omega(i\delta^{(3)}(\mathbf{x}-\mathbf{y}))
    \bigg)\\
    =&0,
\end{align}
where we have used the commutation relations in Equation~(3.28). Similarly, we can show that 
\begin{align}
    &[a^\dagger(\mathbf{k}), a^\dagger(\mathbf{k}')]\\
    =&\int d^3x d^3y e^{ikx}e^{ik'y}[i\partial_0\varphi(x)-\omega \varphi(x),i\partial_0\varphi(y)-\omega \varphi(y)]\\
    =&\int d^3x d^3y e^{ikx}e^{ik'y}\bigg(
    -[\partial_0\varphi(x),\partial_0\varphi(y)]-i\omega[\partial_0\varphi(x),\varphi(y)]-i\omega[\varphi(x),\partial_0\varphi(y)]+\omega^2[\varphi(x),\varphi(y)]
    \bigg)\\
    =&\int d^3x d^3y e^{ikx}e^{ik'y}\bigg(
    -i\omega[\partial_0\varphi(x),\varphi(y)]-i\omega[\varphi(x),\partial_0\varphi(y)]
    \bigg)\\
    =&\int d^3x d^3y e^{ikx}e^{ik'y}\bigg(
    -i\omega(-i\delta^{(3)}(\mathbf{x}-\mathbf{y}))+i\omega(i\delta^{(3)}(\mathbf{x}-\mathbf{y}))
    \bigg)\\
    =&0.
\end{align}
Now, we compute
\begin{align}
    &[a(\mathbf{k}), a^\dagger(\mathbf{k}')]\\
    =&\int d^3x d^3y e^{-ikx}e^{ik'y}[i\partial_0\varphi(x)+\omega \varphi(x),i\partial_0\varphi(y)-\omega \varphi(y)]\\
    =&\int d^3x d^3y e^{-ikx}e^{ik'y}\bigg(
    -[\partial_0\varphi(x),\partial_0\varphi(y)]-i\omega[\partial_0\varphi(x),\varphi(y)]+i\omega[\varphi(x),\partial_0\varphi(y)]-\omega^2[\varphi(x),\varphi(y)]
    \bigg)\\
    =&\int d^3x d^3y e^{-ikx}e^{ik'y}\bigg(
    -i\omega[\partial_0\varphi(x),\varphi(y)]+i\omega[\varphi(x),\partial_0\varphi(y)]
    \bigg)\\
    =&\int d^3x d^3y e^{-ikx}e^{ik'y}\bigg(
    -i\omega(-i\delta^{(3)}(\mathbf{x}-\mathbf{y}))+i\omega(i\delta^{(3)}(\mathbf{x}-\mathbf{y}))
    \bigg)\\
    =&\int d^3x d^3y e^{-ikx}e^{ik'y}(2\omega \delta^{(3)}(\mathbf{x}-\mathbf{y}))\\
    =&\int d^3x e^{-i(\mathbf{k}-\mathbf{k}')\cdot \mathbf{x}}(2\omega)\\
    =& (2\pi)^3 2\omega \delta^{(3)}(\mathbf{k}-\mathbf{k}').
\end{align}
\qed
\clearpage

\question{6}{Problem 3.5}

Consider a complex (that is, non-hermitian) scalar field $\varphi$ with Lagrangian density 
\begin{align*}
    \mathcal{L}= - \partial^\mu\varphi^\dagger\partial_\mu\varphi - m^2\varphi^\dagger\varphi+\Omega_0.
\end{align*}
\begin{itemize}
    \item [(a)] Show that $\varphi$ obeys the Klein-Gordon equation.
    \item [(b)] Treat $\varphi$ and $\varphi^\dagger$ as independent fields, and find the conjugate momentum for each. Compute the Hamiltonian density in terms of these conjugate momenta and the fields themselves (but not their time derivatives).
    \item [(c)] Write the mode expansion of $\varphi$ as 
    \begin{align*}
        \varphi(x)=\int \widetilde{dk}[a(\mathbf{k})e^{ikx}+b^\dagger(\mathbf{k})e^{-ikx}].
    \end{align*}
    Express $a(\mathbf{k})$ and $b(\mathbf{k})$ in terms of $\varphi$ and $\varphi^\dagger$ and their time derivatives.
    \item [(d)] Assuming canonical commutation relations for the fields and their conjugate momenta, find the commutation relations obeyed by $a(\mathbf{k})$ and $b(\mathbf{k})$ and their Hermitian conjugates.
    \item [(e)] Express the Hamiltonian in terms of $a(\mathbf{k})$ and $b(\mathbf{k})$ and their Hermitian conjugates. What value must $\Omega_0$ have in order for the ground state to have zero energy?
\end{itemize}
\begin{align}
    d\widetilde{k}=\frac{d^3k}{(2\pi)^3 2\omega},\quad \omega=\sqrt{\mathbf{k}^2+m^2} \tag{3.11}
\end{align}
\answer{}
\begin{itemize}
    \item [(a)]
\end{itemize}
\begin{align}
    \frac{\partial \mathcal{L}}{\partial \varphi}=&-m^2\varphi^\dagger,\\
    \frac{\partial \mathcal{L}}{\partial (\partial_\mu\varphi)}=&-\partial^\mu\varphi^\dagger,\\
    \partial_\mu\left(\frac{\partial \mathcal{L}}{\partial (\partial_\mu\varphi)}\right)=&-\partial_\mu\partial^\mu\varphi^\dagger.
\end{align}
Hence, the Euler-Lagrange equation gives that
\begin{align}
    &\frac{\partial \mathcal{L}}{\partial \varphi}-\partial_\mu\left(\frac{\partial \mathcal{L}}{\partial (\partial_\mu\varphi)}\right)=0\\
    \Rightarrow& -m^2\varphi^\dagger+\partial_\mu\partial^\mu\varphi^\dagger=0\\
    \Rightarrow& (\partial_\mu\partial^\mu -m^2)\varphi^\dagger=0.
\end{align}
Similarly, we can show that
\begin{align}
    &\frac{\partial \mathcal{L}}{\partial \varphi^\dagger}=-m^2\varphi,\\
    &\frac{\partial \mathcal{L}}{\partial (\partial_\mu\varphi^\dagger)}=-\partial^\mu\varphi,\\
    &\partial_\mu\left(\frac{\partial \mathcal{L}}{\partial (\partial_\mu\varphi^\dagger)}\right)=-\partial_\mu\partial^\mu\varphi.
\end{align}
Hence, the Euler-Lagrange equation gives that
\begin{align}
    &\frac{\partial \mathcal{L}}{\partial \varphi^\dagger}-\partial_\mu\left(\frac{\partial \mathcal{L}}{\partial (\partial_\mu\varphi^\dagger)}\right)=0\\
    \Rightarrow& -m^2\varphi+\partial_\mu\partial^\mu\varphi=0\\
    \Rightarrow& (\partial_\mu\partial^\mu -m^2)\varphi=0.
\end{align}
Therefore, both $\varphi$ and $\varphi^\dagger$ obey the Klein-Gordon equation.
\begin{itemize}
    \item [(b)]
\end{itemize}
\begin{align}
    \Pi_\varphi=&\frac{\partial \mathcal{L}}{\partial (\partial_0\varphi)}=-\partial^0\varphi^\dagger=-\frac{-\partial}{\partial t}\varphi=+\dot{\varphi}^\dagger,\\
    \Pi_{\varphi^\dagger}=&\frac{\partial \mathcal{L}}{\partial (\partial_0\varphi^\dagger)}=-\partial^0\varphi=-\frac{-\partial}{\partial t}\varphi=+\dot{\varphi}.
\end{align}
The Hamiltonian density is given by
\begin{align}
    \mathcal{H}=&\Pi_\varphi\dot{\varphi}+\Pi_{\varphi^\dagger}\dot{\varphi}^\dagger-\mathcal{L}\\
    =&\dot{\varphi}^\dagger\dot{\varphi}+\dot{\varphi}\dot{\varphi}^\dagger+\partial^\mu\varphi^\dagger\partial_\mu\varphi+m^2\varphi^\dagger\varphi-\Omega_0\\
    =&\dot{\varphi}^\dagger\dot{\varphi}+\dot{\varphi}\dot{\varphi}^\dagger-\dot{\varphi}^\dagger\dot{\varphi}+\nabla\varphi^\dagger\cdot\nabla\varphi+m^2\varphi^\dagger\varphi-\Omega_0\\
    =&\dot{\varphi}\dot{\varphi}^\dagger+\nabla\varphi^\dagger\cdot\nabla\varphi+m^2\varphi^\dagger\varphi-\Omega_0\\
    =&\Pi_\varphi\Pi_{\varphi^\dagger}+\nabla\varphi^\dagger\cdot\nabla\varphi+m^2\varphi^\dagger\varphi-\Omega_0.
\end{align} 
\begin{itemize}
    \item [(c)]
\end{itemize}
\begin{align}
    \varphi(x)=&\int \widetilde{dk}[a(\mathbf{k})e^{ikx}+b^\dagger(\mathbf{k})e^{-ikx}]\\
    \Rightarrow \dot{\varphi}(x)=&\int \widetilde{dk}[i\omega a(\mathbf{k})e^{ikx}-i\omega b^\dagger(\mathbf{k})e^{-ikx}]\\
    \varphi^\dagger(x)=&\int \widetilde{dk}[a^\dagger(\mathbf{k})e^{-ikx}+b(\mathbf{k})e^{ikx}]\\
    \Rightarrow \dot{\varphi}^\dagger(x)=&\int \widetilde{dk}[-i\omega a^\dagger(\mathbf{k})e^{-ikx}+i\omega b(\mathbf{k})e^{ikx}]
\end{align}
Hence, we have
\begin{align}
     &\int d^3x e^{-ikx}\varphi(x)\\
    =&\int d^3x e^{-ikx}\int \widetilde{dk'}[a(\mathbf{k}')e^{ik'x}+b^\dagger(\mathbf{k}')e^{-ik'x}]\\
    =&\int \widetilde{dk'}\int d^3x [a(\mathbf{k}')e^{i(k'-k)x}+b^\dagger(\mathbf{k}')e^{-i(k'+k)x}]\\
    =&\int \widetilde{dk'}\left[a(\mathbf{k}')(2\pi)^3\delta^{(3)}(\mathbf{k}'-\mathbf{k})e^{-i(\omega'-\omega)t}+b^\dagger(\mathbf{k}')(2\pi)^3\delta^{(3)}(\mathbf{k}'+\mathbf{k})e^{i(\omega'+\omega)t}\right] \\
    =&\int \frac{d^3k'}{(2\pi)^3 2\omega'}\left[a(\mathbf{k}')(2\pi)^3\delta^{(3)}(\mathbf{k}'-\mathbf{k})e^{-i(\omega'-\omega)t}+b^\dagger(\mathbf{k}')(2\pi)^3\delta^{(3)}(\mathbf{k}'+\mathbf{k})e^{i(\omega'+\omega)t}\right]\\
    =&\int \frac{d^3k'}{(2\pi)^3 2\omega'}\left[a(\mathbf{k}')(2\pi)^3\delta^{(3)}(\mathbf{k}'-\mathbf{k})e^{-i(\omega'-\omega)t}+b^\dagger(\mathbf{k}')(2\pi)^3\delta^{(3)}(\mathbf{k}'+\mathbf{k})e^{i(\omega'+\omega)t}\right]\\
    =&\frac{1}{2\omega}a(\mathbf{k})e^{-i(\omega-\omega)t}+\frac{1}{2\omega}b^\dagger(-\mathbf{k})e^{i(\omega+\omega)t}\\
    =&\frac{1}{2\omega}a(\mathbf{k})+\frac{1}{2\omega}b^\dagger(-\mathbf{k})e^{i2\omega t}.
\end{align}
Next, we compute
\begin{align}
     &\int d^3x e^{-ikx}\dot{\varphi}(x)\\
    =&\int d^3x e^{-ikx}\int \widetilde{dk'}[-i\omega' a(\mathbf{k}')e^{ik'x}+i\omega' b^\dagger(\mathbf{k}')e^{-ik'x}]\\
    =&\int \widetilde{dk'}\int d^3x [-i\omega' a(\mathbf{k}')e^{i(k'-k)x}+i\omega' b^\dagger(\mathbf{k}')e^{-i(k'+k)x}]\\
    =&\int \widetilde{dk'}\left[-i\omega' a(\mathbf{k}')(2\pi)^3\delta^{(3)}(\mathbf{k}'-\mathbf{k})e^{-i(\omega'-\omega)t}+i\omega' b^\dagger(\mathbf{k}')(2\pi)^3\delta^{(3)}(\mathbf{k}'+\mathbf{k})e^{i(\omega'+\omega)t}\right] \\
    =&\int \frac{d^3k'}{(2\pi)^3 2\omega'}\left[-i\omega' a(\mathbf{k}')(2\pi)^3\delta^{(3)}(\mathbf{k}'-\mathbf{k})e^{-i(\omega'-\omega)t}+i\omega' b^\dagger(\mathbf{k}')(2\pi)^3\delta^{(3)}(\mathbf{k}'+\mathbf{k})e^{i(\omega'+\omega)t}\right]\\
    =&\int \frac{d^3k'}{(2\pi)^3 2\omega'}\left[-i\omega' a(\mathbf{k}')(2\pi)^3\delta^{(3)}(\mathbf{k}'-\mathbf{k})e^{-i(\omega'-\omega)t}+i\omega' b^\dagger(\mathbf{k}')(2\pi)^3\delta^{(3)}(\mathbf{k}'+\mathbf{k})e^{i(\omega'+\omega)t}\right]\\
    =&\frac{-i\omega}{2\omega}a(\mathbf{k})e^{-i(\omega-\omega)t}+\frac{i\omega}{2\omega}b^\dagger(-\mathbf{k})e^{i(\omega+\omega)t}\\
    =&\frac{-i}{2}a(\mathbf{k})+\frac{i}{2}b^\dagger(-\mathbf{k})e^{i2\omega t}.
\end{align}
Therefore, we have
\begin{align}
    \int d^3x e^{-ikx}[i\partial_0 \varphi(x)+\omega \varphi(x)]&=\int d^3x e^{-ikx}[i\dot{\varphi}(x)+\omega \varphi(x)]\\
    &=\left[i\frac{-i}{2}a(\mathbf{k})+i\frac{i}{2}b^\dagger(-\mathbf{k})e^{i2\omega t}\right]+\left[\frac{\omega}{2\omega}a(\mathbf{k})+\frac{\omega}{2\omega}b^\dagger(-\mathbf{k})e^{i2\omega t}\right]\\
    &=a(\mathbf{k}).
\end{align}
Similarly, we can show that 
\begin{align}
    \varphi^\dagger(x)=&\int \widetilde{dk}[a^\dagger(\mathbf{k})e^{-ikx}+b(\mathbf{k})e^{ikx}]\\
    \Rightarrow b(\mathbf{x})=&\int d^3x e^{-ikx}[\omega\varphi^\dagger(x)+i\partial_0\varphi^\dagger(x)],
\end{align}
by exchanging $a(\mathbf{k})$ and $b^\dagger(\mathbf{k})$ with $a^\dagger(\mathbf{k})$ and $b(\mathbf{k})$, respectively.
\begin{itemize}
    \item [(d)]
\end{itemize}
\begin{itemize}
    \item [(e)]
\end{itemize}
