\section*{Problem Set 2 Due 11am, Monday, February 16}

\question{1}{\textbf{Weak decay of pions}}
\begin{itemize}
    \item[(a)]Find the electron energy spectrum $d\Gamma/dE_e$ for the decay $ \pi^{-}\rightarrow \pi^0+ e^{-}+\overline{\nu}_{e}$ in the $\pi^{-}$ rest frame keeping $m_e\neq0$ (take $m_\nu=0$). Assume the hadronic current is dominated by $f_+(0)$ and neglect radiative corrections. Perform the phase-space integration by integrating over the $\pi^0$ and $\overline{\nu}_e$ momenta (i.e. treat $E_e$ as the only observed variable). Give the kinematic endpoints and verify the $m_e \to 0$ limit.
    \item[(b)] Using the electron energy spectrum obtained in part (a), integrate over $E_e$ to extract the leading correction of order $m_e^2/\Delta^2$ to the total decay rate. Write the result in the form
    \begin{align}
        \Gamma(\pi^{-}\rightarrow \pi^0+ e^{-}+\overline{\nu}_{e}) = |V_{ud}|^2 \frac{G_F^2\Delta^5}{30\pi^3}\Bigg( 1- a \frac{\Delta}{m_\pi} -b \frac{m_e^2}{\Delta^2}\Bigg),
    \end{align}
    where $\Delta = m_{\pi^-} - m_{\pi^0}$, and neglecting higher-order terms in $\Delta/m_\pi$ and $m_e^2/\Delta^2$. In the lectures it was shown that $a=3/2$. Determine the coefficient b.
\end{itemize}


\answer{}
\begin{itemize}
    \item [(a)]
\end{itemize}
Let us denote the momenta of $\pi^-$, $\pi^0$, $e^-$, and $\overline{\nu}_e$ as $p$, $p'$, $k$, and $k'$, respectively. The decay amplitude can be written as
\begin{align}
    \mathcal{M} = \langle\pi^0(p') e^-(k) \overline{\nu}_e(k')| \mathcal{H}_W | \pi^-(p) \rangle= -\frac{G_F}{\sqrt{2}} V_{ud} \langle \pi^0(p')| \overline{d}\gamma^\mu u | \pi^-(p) \rangle \langle e^-(k) \overline{\nu}_e(k')| \overline{e}\gamma_\mu(1-\gamma^5)\nu_e | 0 \rangle.
\end{align}
The hadronic matrix element can be parameterized as
\begin{align}
    \langle \pi^0(p')| \overline{d}\gamma^\mu u | \pi^-(p) \rangle = f_+(q^2) (p+p')^\mu + f_-(q^2) (p-p')^\mu\approx f_+(0) (p+p')^\mu=\sqrt{2} (p+p')^\mu,
\end{align}
where $q = p - p'$. Neglecting radiative corrections and using the fact that $f_+(0)$ dominates, we can approximate $f_+(q^2) \approx f_+(0)=\sqrt{2}$. The leptonic matrix element can be evaluated using standard techniques, yielding
\begin{align}
    \langle e^-(k) \overline{\nu}_e(k')| \overline{e}\gamma_\mu(1-\gamma^5)\nu_e | 0 \rangle = \overline{u}(k) \gamma_\mu (1-\gamma^5) v(k').
\end{align}
Now the amplitude can be expressed as
\begin{align}
    &\mathcal{M} =-\frac{G_F}{\sqrt{2}} V_{ud} \sqrt{2} (p+p')^\mu \overline{u}(k) \gamma_\mu (1-\gamma^5) v(k')\\
    =& -G_F V_{ud} (p+p')^\mu \overline{u}(k) \gamma_\mu (1-\gamma^5) v(k').
\end{align}
Then the squared amplitude, summed over final spins, is given by
\begin{align}
    \langle|\mathcal{M}|^2\rangle =& \sum_{\text{spins}} |\mathcal{M}|^2 = G_F^2 |V_{ud}|^2 (p+p')^\mu (p+p')^\nu \overline{u}_{s_1}(k) \gamma_\mu (1-\gamma^5) v_{s_2}(k') \overline{v}_{s_2}(k') \gamma_\nu (1-\gamma^5) u_{s_1}(k)\\
    =& G_F^2 |V_{ud}|^2 (p+p')^\mu (p+p')^\nu \text{Tr}\Big[ (\slashed{k}+m_e) \gamma_\mu (1-\gamma^5) \slashed{k}' \gamma_\nu (1-\gamma^5) \Big]\\
    =&G_F^2 |V_{ud}|^2 (p+p')^\mu (p+p')^\nu k^{\alpha} k'^{\beta} \text{Tr}\Big[(\gamma_\alpha +1m_e) \gamma_\mu (1-\gamma^5) \gamma_\beta \gamma_\nu (1-\gamma^5) \Big].
\end{align}
where we have used the spin sum identities for the electron and neutrino:
\begin{align}
    \sum_{s_1} u_{s_1}(k) \overline{u}_{s_1}(k) = \slashed{k} + m_e, \quad \sum_{s_2} v_{s_2}(k') \overline{v}_{s_2}(k') = \slashed{k}'-m_\nu \approx \slashed{k}'.
\end{align}
We provide the full set of trace identities:
\begin{align}
    &\text{Tr}(\gamma^\mu \gamma^\nu) = 4 g^{\mu\nu}, \quad \text{Tr}(\gamma^\mu \gamma^\nu \gamma^5) = 0, \quad \text{Tr}(\gamma^\mu \gamma^\nu \gamma^\rho \gamma^\sigma) = 4 (g^{\mu\nu} g^{\rho\sigma} - g^{\mu\rho} g^{\nu\sigma} + g^{\mu\sigma} g^{\nu\rho}),\\
    &\text{Tr}(\gamma^\mu \gamma^\nu \gamma^\rho \gamma^\sigma \gamma^5) = 4 i \epsilon^{\mu\nu\rho\sigma}, \quad \text{Tr}(\text{odd number of } \gamma^5) = 0.
\end{align}
Hence, 
\begin{align}
    &\text{Tr}\Big[(\gamma_\alpha +1m_e) \gamma_\mu (1-\gamma^5) \gamma_\beta \gamma_\nu (1-\gamma^5) \Big] \\
    =& \text{Tr}\Big[(\gamma_\alpha +1m_e) \gamma_\mu \gamma_\beta \gamma_\nu (1-\gamma^5) \Big] - \text{Tr}\Big[(\gamma_\alpha +1m_e) \gamma_\mu \gamma_\beta \gamma_\nu \gamma^5 (1-\gamma^5) \Big]
\end{align}










\clearpage
\question{2}{\textbf{Tau decays}}\\
\begin{itemize}
    \item [(a)] Find the decay rate for the two-body decay $\tau^{-}\rightarrow \pi^{-}+\nu_{\tau}$, neglecting neutrino masses and using $\langle 0|\overline{d}\gamma^{\mu}\gamma^5 u|\pi^{-}\rangle = i f_{\pi} p_{\pi}^{\mu}$. Determine the ratio
    \begin{align}
        R_\pi=\frac{\Gamma(\tau^{-}\rightarrow \pi^{-}+\nu_{\tau})}{\Gamma(\tau^{-}\rightarrow e^{-}+\overline{\nu}_{e}+\nu_{\tau})},
    \end{align}
    using the tree-level leptonic rate with $m_e=0$, and compare with the corresponding PDG branching-fraction ratio.
    \item[(b)] Now consider $\tau^{-}\rightarrow \rho^{-}+\nu_{\tau}$ with $\langle 0|\overline{d}\gamma^{\mu}u|\rho^{-}(q,\epsilon)\rangle = f_{\rho} m_{\rho} \epsilon^{\mu}$, and derive the decay rate $\Gamma(\tau^{-}\rightarrow \rho^{-}+\nu_{\tau})$ (neglect the neutrino mass). Form the ratio 
    \begin{align}
        R=\frac{\Gamma(\tau^{-}\rightarrow \rho^{-}+\nu_{\tau})}{\Gamma(\tau^{-}\rightarrow e^{-}+\overline{\nu}_{e}+\nu_{\tau})},
    \end{align}
    and nd compare with the PDG data to extract $f_{\rho}$ (or the ratio $f_{\rho}/f_{\pi}$ ). \\
    Hint: Use the polarization sum
    \begin{align}
        \sum_{\lambda} \epsilon_{\mu}^{(\lambda)}(q)\epsilon_{\nu}^{(\lambda)}{}^{*}(q) = -g_{\mu\nu} + \frac{q_{\mu}q_{\nu}}{m_{\rho}^2}.
    \end{align}
\end{itemize}


