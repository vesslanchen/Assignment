\documentclass[12pt,letterpaper]{article}

\usepackage[letterpaper, left=1.5cm, right=1.5cm, top=1.5cm, bottom=1.5cm]{geometry}
\usepackage[compat=1.0.0]{tikz-feynman}\usepackage{feynmp-auto}
\usepackage{amsmath}
\usepackage{amsthm}
\usepackage{amssymb}
\usepackage{graphicx}
\usepackage{setspace} % 引入這個套件
\usepackage{tensor}
\usepackage{braket}
\usepackage{slashed}
\usepackage{subcaption}
\usepackage{bbold}
\usepackage{hyperref}\hypersetup{colorlinks=true,linkcolor=blue,citecolor=green,urlcolor=red}

\title{University of Minnesota\\
School of Physics and Astronomy\\
\textbf{2026 Spring Physics 8901\\Elementary Particle Physics II}\\ 
Assignment Solution}
\author{Lecture Instructor: Professor Tony Gherghetta \\ \\
        Zong-En Chen\\ chen9613@umn.edu}
\date{\today}  

% 自定義範例環境
\newtheorem{example}{Example}[section]
\newtheorem{solution}{Sol.}[section]
% 自定義習題環境
\newtheorem{exercise}{Ex.}[section]

\newtheorem{theorem}{Theorem}[section]

\newcommand{\question}[2]{\section*{Question #1} #2}
\newcommand{\answer}[1]{\section*{Answer}#1}

\begin{document}
\onehalfspacing
%\maketitle
\begin{titlepage}
    \centering
    \vspace*{\fill} % 從頂端往下推到中間
    {\LARGE University of Minnesota\\
    School of Physics and Astronomy\\[1em]
    \textbf{2026 Spring Physics 8902\\Elementary Particle Physics II}\\ 
    Assignment Solution \par}
    \vspace{2cm}
    {\large Lecture Instructor: Professor Tony Gherghetta \par}
    \vspace{1cm}
    {\large Zong-En Chen\\ chen9613@umn.edu \par}
    \vspace{2cm}
    {\large \today \par}
    \vspace*{\fill} % 從底部往上推到中間
\end{titlepage}

\tikzset{graviton/.style={decorate, decoration={snake, amplitude=.4mm, segment length=1.5mm, pre length=.5mm, post length=.5mm}, double}}

%
\section*{Homework 1 Due to January 29th 11:00 AM}

\question{1}{}
The constant $J$ in the Heisenberg model (slide 6) can be positive or negative. The model can describe either ferromagnetism or anti-ferromagnetism. To which of these to cases positive/ negative $J$ corresponds? Why?
\begin{align}
    \mathcal{H}=-\sum_{\langle i j\rangle} J \mathbf{S}_{i} \cdot \mathbf{S}_{j}
\end{align}
\answer{}
Since \textbf{the Hamiltonian is minimized when the energy is the lowest}, we can analyze the two cases as follows:
\begin{itemize}
    \item For $J>0$, the Hamiltonian becomes $\mathcal{H}=-\sum_{\langle i j\rangle} |J| \mathbf{S}_{i} \cdot \mathbf{S}_{j}$. To minimize the energy, the spins $\mathbf{S}_{i}$ and $\mathbf{S}_{j}$ should align parallel to each other, leading to ferromagnetism.
    \item For $J<0$, the Hamiltonian becomes $\mathcal{H}=-\sum_{\langle i j\rangle} -|J| \mathbf{S}_{i} \cdot \mathbf{S}_{j} = \sum_{\langle i j\rangle} |J| \mathbf{S}_{i} \cdot \mathbf{S}_{j}$. To minimize the energy, the spins $\mathbf{S}_{i}$ and $\mathbf{S}_{j}$ should align anti-parallel to each other, leading to anti-ferromagnetism.
\end{itemize}   
\qed

\clearpage
\question{2}{}
Pirates found a parchment describing location of treasure trove hidden somewhere on the island by their predecessors . The note reads:


\textit{Start from the capsized boat. Go in the direction of the palm tree carefully counting the number of steps. When you reach the palm tree turn exactly right and make exactly the same number of steps. Mark the point you arrived at. Then return to the boat. Go in the direction of the rock counting the number of steps. When you reach the rock turn left. After having made the same number of steps mark the second point. The treasure trove is in the middle of the line connecting two marked points. The problem is that the capsized boat was nowhere in sight, it had disappeared. How can the pirates still find the treasure trove?}
\answer{}
First let's denote the position of the palm tree as point $P(x_1, y_1)$ and the position of the rock as point $R(x_2, y_2)$. We also denote the unknown position of the capsized boat as point $B(x, y)$.

When the pirates walk from the boat to the palm tree, they cover a distance equal to the length of the vector $\overrightarrow{BP} = (x_1 - x, y_1 - y)$. After reaching the palm tree, they turn right and walk the same distance, which corresponds to moving in the direction perpendicular to $\overrightarrow{BP}$. Let's say the right turn results in a new vector $\overrightarrow{PR} = (y_1 - y, -(x_1 - x))$. The coordinates of the first marked point $M_1$ can be expressed as:
\begin{align}
    M_1 = P + \overrightarrow{PR} = (x_1 + (y_1 - y), y_1 - (x_1 - x)) = (x_1 + y_1 - y, y_1 - x_1 + x)
\end{align}
Similarly, when the pirates walk from the boat to the rock, they cover a distance equal to the length of the vector $\overrightarrow{BR} = (x_2 - x, y_2 - y)$. After reaching the rock, they turn left and walk the same distance, which corresponds to moving in the direction perpendicular to $\overrightarrow{BR}$. Let's say the left turn results in a new vector $\overrightarrow{RL} = (-(y_2 - y), x_2 - x)$. The coordinates of the second marked point $M_2$ can be expressed as:
\begin{align}
    M_2 = R + \overrightarrow{RL} = (x_2 - (y_2 - y), y_2 + (x_2 - x)) = (x_2 - y_2 + y, y_2 + x_2 - x) 
\end{align}
The treasure trove is located at the midpoint of the line segment connecting the two marked points $M_1$ and $M_2$. The coordinates of the midpoint $T$ can be calculated as:
\begin{align}
    T = \left( \frac{(x_1 + y_1 - y) + (x_2 - y_2 + y)}{2}, \frac{(y_1 - x_1 + x) + (y_2 + x_2 - x)}{2} \right)
\end{align}
Simplifying the expressions for the coordinates of $T$, we get:
\begin{align}
    T = \left( \frac{x_1 + x_2 + y_1 - y_2}{2}, \frac{y_1 + y_2 - x_1 + x_2}{2} \right)
\end{align}
Notice that the coordinates of the treasure trove $T$ do not depend on the unknown position of the boat $B(x, y)$. Therefore, the pirates can find the treasure trove using only the known positions of the palm tree and the rock, without needing to know the location of the capsized boat.
\qed
\section*{Problem Set 2 due on Due Feb 18 at 11:59pm}

\question{1}{}
Consider 2 objects of mass $m_1$ and $m_2$ with separation $a$, orbiting about a common center of mass. Find the change in the period ($\dot{\tau}/\tau$) due to gravitational radiation. Assume the above result is valid as $a \to 0$, find the time to go from $a=a_0$ to $a=0$.
\answer{}
We first write down the radiation power of the system:
\begin{align}
    P = \frac{dE}{dt} = -\frac{G}{5c^5} \left\langle \dddot{I}_{ij} \dddot{I}_{ij} - \frac{1}{3} (\dddot{I}_{kk})^2 \right\rangle\\
    =-\frac{G}{5} \left\langle \dddot{I}_{ij} \dddot{I}_{ij} - \frac{1}{3} (\dddot{I}_{kk})^2 \right\rangle
\end{align}
where $I_{ij}$ is the quadrupole moment of the system. For a binary system, we can express $I_{ij}$ in terms of the reduced mass $\mu = \frac{m_1 m_2}{m_1 + m_2}$ and the separation vector $\mathbf{r}$ between the two masses:
\begin{align}
    I_{ij} = \mu (r_i r_j - \frac{1}{3} r^2 \delta_{ij})
\end{align}
For a circular orbit, the separation vector can be expressed as $\mathbf{r}(t) = a (\cos(\omega t), \sin(\omega t), 0)$, where $\omega$ is the angular frequency of the orbit. The third time derivative of $I_{ij}$ can be calculated as follows:
\begin{align}
    \dddot{I}_{ij} = \mu \left( \frac{d^3}{dt^3} (r_i r_j) - \frac{1}{3} \frac{d^3}{dt^3} (r^2 \delta_{ij}) \right)
\end{align}
Calculating the third time derivative of $r_i r_j$ and $r^2 \delta_{ij}$, we find:
\begin{align}
    &\frac{d^3}{dt^3} (r_i r_j) = \frac{d^3}{dt^3} 
    \begin{pmatrix}
    a^2 \cos^2(\omega t) & a^2 \cos(\omega t) \sin(\omega t) & 0 \\
    a^2 \cos(\omega t) \sin(\omega t) & a^2 \sin^2(\omega t) & 0 \\
    0 & 0 & 0
    \end{pmatrix} \\
    =& \frac{d^3}{dt^3}  \begin{pmatrix}
    \frac{a^2}{2} (1 + \cos(2\omega t)) & \frac{a^2}{2} \sin(2\omega t) & 0 \\
    \frac{a^2}{2} \sin(2\omega t) & \frac{a^2}{2} (1 - \cos(2\omega t)) & 0 \\
    0 & 0 & 0
    \end{pmatrix} \\
    =&4 a^2 \omega^3   \begin{pmatrix}
    \sin(2\omega t) & -\cos(2\omega t) & 0 \\
    -\cos(2\omega t) & -\sin(2\omega t) & 0 \\
    0 & 0 & 0
    \end{pmatrix}\\
    &\frac{d^3}{dt^3} (r^2 \delta_{ij}) = \frac{d^3}{dt^3} (a^2 \delta_{ij}) = 0
\end{align}
Substituting these results back into the expression for $\dddot{I}_{ij}$, we get:
\begin{align}
    \dddot{I}_{ij} = 4 \mu a^2 \omega^3   \begin{pmatrix}
    \sin(2\omega t) & -\cos(2\omega t) & 0 \\
    -\cos(2\omega t) & -\sin(2\omega t) & 0 \\
    0 & 0 & 0
    \end{pmatrix}
\end{align}
Now we can calculate the power radiated by the system:
\begin{align}
    P = \frac{dE}{dt}&= -\frac{G}{5} \left\langle \dddot{I}_{ij} \dddot{I}_{ij} - \frac{1}{3} (\dddot{I}_{kk})^2 \right\rangle \\
    &= -\frac{G}{5} \left\langle 16 \mu^2 a^4 \omega^6 (2 \sin^2(2\omega t) + 2 \cos^2(2\omega t)) \right\rangle \\
    &= -\frac{32 G}{5} \mu^2 a^4 \omega^6 .
\end{align}
By Kepler's third law, we have $\omega^2 = \frac{G (m_1 + m_2)}{a^3}$, $\mu=\frac{m_1 m_2}{m_1 + m_2}$, which allows us to express the power in terms of the separation $a$:
\begin{align}
    &P = -\frac{32 G^4}{5} \frac{\mu^2 (m_1 + m_2)^3}{a^5}\\
    =& -\frac{32 G^4}{5} \frac{m_1^2 m_2^2 (m_1 + m_2)}{a^5}
\end{align}
The energy of the system is given by the sum of the kinetic and potential energy:
\begin{align}
    E = -\frac{G m_1 m_2}{2a}
\end{align}
The rate of change of the energy is equal to the power radiated:
\begin{align}
    \frac{dE}{dt} = \frac{G m_1 m_2}{2a^2} \frac{da}{dt} = -\frac{32 G^4}{5} \frac{m_1^2 m_2^2 (m_1 + m_2)}{a^5}
\end{align}
Solving for $\frac{da}{dt}$, we find:
\begin{align}
    \frac{da}{dt} = -\frac{64 G^3}{5} \frac{m_1 m_2 (m_1 + m_2)}{a^3}
\end{align}
Now, we can write down the period of the orbit $\tau= \frac{2\pi}{\omega}$:
\begin{align}
    \tau = 2\pi \sqrt{\frac{a^3}{G (m_1 + m_2)}}=\frac{2\pi}{\sqrt{G (m_1 + m_2)}} a^{3/2}
\end{align}
Taking the time derivative of the period, we get:
\begin{align}
    \frac{d\tau}{dt} = \frac{3\pi}{\sqrt{G (m_1 + m_2)}} a^{1/2} \frac{da}{dt} = -\frac{192 \pi G^{5/2}}{5} \frac{m_1 m_2 (m_1 + m_2)^{1/2}}{a^{5/2}}
\end{align}
Finally, we can express the change in the period as:
\begin{align}
    \frac{\dot{\tau}}{\tau} = \frac{d\tau/dt}{\tau} = -\frac{96 G^3}{5} \frac{m_1 m_2}{a^4}
\end{align}
To find the time it takes for the separation to go from $a=a_0$ to $a=0$, we can integrate the expression for $\frac{da}{dt}$:
\begin{align}
    t= \int dt = \int_{a_0}^{0}  \frac{da}{\frac{da}{dt}} = \int_{a_0}^{0} -\frac{5}{64 G^3} \frac{a^3}{m_1 m_2 (m_1 + m_2)} da = \frac{5 a_0^4}{256 G^3 m_1 m_2 (m_1 + m_2)}.
\end{align}
\qed


\clearpage
\question{2}{}
Problem~$7.6$ in Carroll. Two object of mass $M$ have a head on collision at $(0,0,0,0)$.
In the distant past, $t\to -\infty$, the mass mass started at $x\to\pm\infty$ with zero velocity.
\begin{itemize}
    \item [(a)] Using Newtonian theory show that $x(t)= \pm(\frac{9}{8}GMt^2)^{1/3}$.
    \item [(b)] For what separations is the Newtonian approximation reasonable?
    \item [(c)] Calculate $h_{xx}^{TT}$ at $(0,R,0)$.
    \item [(d)] For the same problem, calculate the total energy radiated in the collision.
\end{itemize}

\answer{}
\begin{itemize}
    \item [(a)]
\end{itemize}
Start with energy conservation:
\begin{align}
    2\frac{1}{2} M \dot{x}^2 - \frac{G M^2}{2x} = 0\\
    \implies \dot{x} = \sqrt{\frac{G M}{2x}}
\end{align}
Separating variables and integrating, we get:
\begin{align}
    &\int x^{1/2} dx = \sqrt{\frac{G M}{2}} \int dt\\
    \implies &\frac{2}{3} x^{3/2} = \sqrt{\frac{G M}{2}} t + C, \quad \text{where $C=0$ since $x=0$ at $t=0$}\\
    \implies &x^{3/2} = \frac{3}{2} \sqrt{\frac{G M}{2}} t=\sqrt{\frac{9}{8} G M} t\\
    \implies &x(t) = \pm \left(\frac{9}{8} G M t^2\right)^{1/3}
\end{align}
\begin{itemize}
    \item [(b)]
\end{itemize}
The Newtonian approximation is reasonable when the gravitational field is weak and the velocities are much less than the speed of light. For the speed, we can calculate $\dot{x}$ from the expression we derived in part (a):
\begin{align}
    &\dot{x} = \frac{d}{dt} \left(\frac{9}{8} G M t^2\right)^{1/3} = \frac{2}{3} \left(\frac{9}{8} G M\right)^{1/3} t^{-1/3}<\!\!< c=1\\
    \implies &t >\!\!> \left(\frac{2}{3} \left(\frac{9}{8} G M\right)^{1/3}\right)^3 = \frac{8}{27} G M \\
    \implies &x(t) >\!\!> \left(\frac{9}{8} G M \left(\frac{8}{27} G M\right)^2\right)^{1/3} = \frac{2}{3} G M
\end{align}
For the gravitational field, we can calculate the gravitational potential $\Phi$ at the position of one of the masses:
\begin{align}
    \Phi = -\frac{G M}{2x}
\end{align}
The Newtonian approximation is reasonable when $|\Phi| <\!\!< 1$, which implies:
\begin{align}
    x >\!\!> \frac{G M}{2}
\end{align}
\begin{itemize}
    \item [(c)]
\end{itemize}
By equation~(7.140) in Carroll, the transverse-traceless part of the metric perturbation is given by:
\begin{align}
    h_{ij}^{TT} = \frac{2 G}{R} \frac{d^2}{dt^2} I_{ij}(t_r),
\end{align}
where $I_{ij}$ is the quadrupole moment of the system and $t_r = t - R$ is the retarded time. The quadrupole moment can be calculated as:
\begin{align}
    I_{ij} = \sum_a m_a (x_a^i x_a^j - \frac{1}{3} r_a^2 \delta_{ij}).
\end{align}
Since we want to calculate $h_{xx}^{TT}$, we need to find $I_{xx}$. For the two masses, we have:
\begin{align}
    I_{xx} = M (x^2 - \frac{1}{3} r^2) + M ((-x)^2 - \frac{1}{3} r^2) = \frac{4}{3} M x^2=\frac{4}{3} M \left(\frac{9}{8} G M t^2\right)^{2/3}
\end{align}
Taking the second time derivative, we get:
\begin{align}
    &\frac{d^2}{dt^2} I_{xx} = \frac{4}{3} M \frac{d^2}{dt^2} \left(\frac{9}{8} G M t^2\right)^{2/3} = \frac{4}{3}M \Big(\frac{9}{8} GM\Big)^{2/3} \frac{d^2}{dt^2} t^{4/3}\\
    =& \frac{4}{3}M \Big(\frac{9}{8} GM\Big)^{2/3} \frac{4}{9} t^{-2/3} = \frac{16}{27} M \Big(\frac{9}{8} GM\Big)^{2/3} t^{-2/3} 
\end{align}
Substituting this back into the expression for $h_{xx}^{TT}$, we get:
\begin{align}
    h_{xx}^{TT} = \frac{2 G}{R} \frac{16}{27} M \Big(\frac{9}{8} GM\Big)^{2/3} t^{-2/3} = \frac{32 G M}{27 R} \Big(\frac{9}{8} GM\Big)^{2/3} (t-R)^{-2/3},
\end{align}
where we have replaced $t$ with the retarded time $t_r = t - R$.
\begin{itemize}
    \item [(d)]
\end{itemize}
The total energy radiated in the collision can be calculated using the quadrupole formula for gravitational radiation:
\begin{align}
    E = \frac{G}{5} \int_{-\infty}^{\infty} \left\langle \dddot{I}_{ij} \dddot{I}_{ij} - \frac{1}{3} (\dddot{I}_{kk})^2 \right\rangle dt
\end{align}
Since we have already calculated $I_{xx}$, we can find $\dddot{I}_{xx}$ by taking the third time derivative:
\begin{align}
    &\dddot{I}_{xx} = \frac{d}{dt} \frac{d^2}{dt^2} I_{xx} = \frac{d}{dt} \left(\frac{16}{27} M \Big(\frac{9}{8} GM\Big)^{2/3} t^{-2/3}\right) = -\frac{32}{81} M \Big(\frac{9}{8} GM\Big)^{2/3} t^{-5/3}
\end{align}
Since the quadrupole moment is traceless, we have $\dddot{I}_{kk} = 0$. Therefore, the energy radiated can be expressed as:
\begin{align}
    I_{yy}=I_{zz}=-\frac{1}{2} I_{xx}, \quad \dddot{I}_{yy}=\dddot{I}_{zz}=-\frac{1}{2} \dddot{I}_{xx}
\end{align}
Back to the expression for the energy radiated, we have:
\begin{align}
    E =& \frac{G}{5} \int_{-\infty}^{t_0} \left\langle \dddot{I}_{xx}^2 + 2 \dddot{I}_{yy}^2 - \frac{1}{3} (\dddot{I}_{xx} + 2 \dddot{I}_{yy})^2 \right\rangle dt\\
    =& \frac{G}{5} \int_{-\infty}^{t_0} \left\langle \dddot{I}_{xx}^2 + 2 \left(-\frac{1}{2} \dddot{I}_{xx}\right)^2 - \frac{1}{3} (\dddot{I}_{xx} - \dddot{I}_{xx})^2 \right\rangle dt\\
    =& \frac{G}{5} \int_{-\infty}^{t_0} \left\langle \dddot{I}_{xx}^2 + 2 \cdot \frac{1}{4} \dddot{I}_{xx}^2 - 0 \right\rangle dt\\
    =& \frac{G}{5} \int_{-\infty}^{t_0} \left\langle \frac{3}{2} \dddot{I}_{xx}^2 \right\rangle dt = \frac{3 G}{10} \int_{-\infty}^{t_0}  \left(-\frac{32}{81} M \Big(\frac{9}{8} GM\Big)^{2/3} t^{-5/3}\right)^2 dt\\
    =& \frac{3 G}{10} \frac{1024}{6561} M^2 \Big(\frac{9}{8} GM\Big)^{4/3} \int_{-\infty}^{t_0} t^{-10/3} dt = \frac{3 G}{10} \frac{1024}{6561} M^2 \Big(\frac{9}{8} GM\Big)^{4/3} \cdot \frac{3}{7} t_0^{-7/3}\\
    =&\frac{512}{10935} G M^2 \Big(\frac{9}{8} GM\Big)^{4/3} t_0^{-7/3},
\end{align}
where $t_0$ is the time at which the collision occurs. Since the collision occurs at $t=0$, we can take the limit as $t_0 \to 0$ to find the total energy radiated:
\begin{align}
    E = \lim_{t_0 \to 0} \frac{512}{10935} G M^2 \Big(\frac{9}{8} GM\Big)^{4/3} t_0^{-7/3} = \infty
\end{align}
This result indicates that an infinite amount of energy is radiated in the collision, which is a consequence of the idealized nature of the problem. In reality, the energy radiated would be finite due to various factors such as the \textbf{finite size} of the masses and the presence of other forces that would come into play during the collision.
\qed






\clearpage
\question{3}{}
A ball of mass $m = 100$~g is thrown into the vaccum above the earth (ie. neglect all effects of air resistence), which produces a UNIFORM gravitional field ($g=10^3$~cm s$^{−2}$) with a velocity $v_0=10^3$~cm/s. Normally, this ball would rise to a height of $h=v_0^2/2g=500$~cm. However, the ball will be a source of gravitational radiation and won’t go quite so high. Find $\Delta h$. How does $\Delta h/h$ depend on $v_0$?


Now suppose the ball will fall back and elastically bounce. Left alone, the ball will eventually come to rest. How long will it take.

\answer{}
The power radiated by the ball due to gravitational radiation can be calculated using the quadrupole formula:
\begin{align}
    P = \frac{G}{5} \left\langle \dddot{I}_{ij} \dddot{I}_{ij} - \frac{1}{3} (\dddot{I}_{kk})^2 \right\rangle
\end{align}
For a ball moving in a uniform gravitational field, the quadrupole moment can be expressed as:
\begin{align}
    I_{ij} = m (x_i x_j - \frac{1}{3} r^2 \delta_{ij})
\end{align}
Since the ball is moving vertically, we can express its position as $x(t) = (0, 0, z(t))$, where $z(t)$ is the height of the ball at time $t$. The quadrupole moment can then be simplified to:
\begin{align}
    I_{zz} = m (z^2 - \frac{1}{3} z^2) = \frac{2}{3} m z^2
\end{align}
Since the quadrupole moment is traceless, we have $I_{xx} = I_{yy} = -\frac{1}{2} I_{zz}$. Therefore, the second time derivative of the quadrupole moment can be calculated as:
\begin{align}
    &\dddot{I}_{zz} = \frac{d^3}{dt^3} I_{zz} = \frac{d^3}{dt^3} \left(\frac{2}{3} m z^2\right)\\
    =& \frac{2}{3}m \frac{d^2}{dt^2}(2 z \dot{z}) = \frac{4}{3} m \frac{d}{dt} (z \ddot{z} + \dot{z}^2) \\
    =&\frac{4m}{3} \frac{d}{dt} (z (-g) + \dot{z}^2) = \frac{4m}{3} (-g \dot{z} + 2 \dot{z} \ddot{z}) = \frac{4m}{3} (- g \dot{z} + 2 \dot{z} (-g)) = -4m g \dot{z}=-4mgv
\end{align}
Since $I_{xx} = I_{yy} = -\frac{1}{2} I_{zz}$, we have $\dddot{I}_{xx} = \dddot{I}_{yy} = -\frac{1}{2} \dddot{I}_{zz}$. Substituting these results back into the expression for the power, we get:
\begin{align}
    P =& \frac{G}{5} \left\langle \dddot{I}_{zz}^2 + 2 \dddot{I}_{xx}^2 - \frac{1}{3} (\dddot{I}_{zz} + 2 \dddot{I}_{xx})^2 \right\rangle = \frac{G}{5} \left\langle \dddot{I}_{zz}^2 + 2 \cdot \frac{1}{4} \dddot{I}_{zz}^2 - 0 \right\rangle = \frac{3 G}{10} \left\langle \dddot{I}_{zz}^2 \right\rangle\\
    =& \frac{3 G}{10} \left\langle (-4mgv)^2 \right\rangle = \frac{24 G m^2 g^2}{5}  v^2 
\end{align}
The energy radiated by the ball can be calculated by integrating the power over time, and we also restore the $c^5$ in the denominator:
\begin{align}
    &\Delta E = \int P dt = \int \frac{24 G m^2 g^2}{5 c^5} v^2 dt = \frac{24 G m^2 g^2}{5 c^5} \int v^2 dt\\
    =& \frac{24 G m^2 g^2}{5 c^5} \int v^2 \frac{dv}{-g} = -\frac{24 G m^2 g}{5 c^5} \int v^2 dv = -\frac{24 G m^2 g}{5 c^5} \cdot \frac{v_0^3}{3} = -\frac{8 G m^2 g v_0^3}{5 c^5}
\end{align}
The change in height $\Delta h$ can be calculated by equating the energy radiated to the change in potential energy:
\begin{align}
    \Delta E = m g \Delta h \implies \Delta h = \frac{\Delta E}{m g} = -\frac{8 G m g v_0^3}{5 c^5}
\end{align}
The ratio $\Delta h/h$ can be expressed as:
\begin{align}
    \frac{\Delta h}{h} = \frac{-\frac{8 G m g v_0^3}{5 c^5}}{\frac{v_0^2}{2g}} = -\frac{16 G m g^2 v_0}{5 c^5}
\end{align}
This shows that the ratio $\Delta h/h$ is proportional to the initial velocity $v_0$ of the ball. 

Now, we know the due to the radiation, we have $\Delta E = -\frac{8 G m^2 g v_0^3}{5 c^5}$, which means the ball loses energy at a rate of $\frac{dE}{dt}$: 
\begin{align}
    \frac{dE}{dt} = -\frac{8 G m^2 g v_0^3}{5 c^5} \cdot \frac{1}{t_{\text{total}}}= -\frac{8 G m^2 g v_0^3}{5 c^5} \cdot \frac{1}{2 t_{\text{up}}},
\end{align}
where $t_{\text{total}}$ is the total time for the ball to go up and come back down, and $t_{\text{up}}$ is the time it takes for the ball to reach its maximum height. Since the ball is thrown upwards with an initial velocity of $v_0$, we can calculate $t_{\text{up}}$ using the equation of motion:
\begin{align}
    v = v_0 - g t_{\text{up}} = 0 \implies t_{\text{up}} = \frac{v_0}{g}
\end{align}
Substituting this back into the expression for $\frac{dE}{dt}$,
\begin{align}
    \frac{dE}{dt} = -\frac{8 G m^2 g v_0^3}{5 c^5} \cdot \frac{1}{2 \cdot \frac{v_0}{g}} = -\frac{4 G m^2 g^2 v_0^2}{5 c^5}
\end{align}
Note that the energy of the ball at its maximum height is given by:
\begin{align}
    E = m g h = m g \frac{v_0^2}{2g} = \frac{1}{2} m v_0^2 \implies v_0^2 = \frac{2E}{m},
\end{align}
Hence, we can express $\frac{dE}{dt}$ in terms of the energy $E$:
\begin{align}
    \frac{dE}{dt} = -\frac{4 G m^2 g^2}{5 c^5} \cdot \frac{2E}{m} = -\frac{8 G m g^2}{5 c^5} E
\end{align}
This is a first-order linear differential equation, the solution to which is given by:
\begin{align}
    E(t) = E_0 e^{-\frac{8 G m g^2}{5 c^5} t}
\end{align}
where $E_0$ is the initial energy of the ball at its maximum height. The ball will come to rest when its energy approaches zero, which occurs as $t \to \infty$. 

\qed


\end{document}