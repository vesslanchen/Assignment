\section*{Assignment 14 due on Monday December 15 at 10PM}
\question{1}{}
Solve problem 6 of chapter 6 of Carroll’s book. It is worthwhile to remind yourself of time dilation, the normal Doppler effect, and the relativistic Doppler effect in special relativity as a warmup.\\ \\
\textbf{Carroll problem 6 of chapter 6:} Consider a Kerr black hole with an accretion disk of negligible mass in the equatorial plane. Assume that particles in the disk follow geodesics (that is, ignore any pressure support). Now, suppose the disk contains some iron atoms that are being excited by a source of radiation. When the iron atoms de-excite, they emit radiation with a known frequency $\nu_0$, as measured in their rest frame. Suppose we detect this radiation far from the black hole (we also lie in the equatorial plane). What is the observed frequency of photons emitted from either edge of the disk, and from the center of the disk? Consider cases where the disk and the black hole are rotating in the same and opposite directions. Can we use these measurements to determine the mass and angular momentum of the black hole?
\answer{}
