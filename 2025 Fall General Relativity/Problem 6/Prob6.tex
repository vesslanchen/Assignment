\section*{Assignment 6 due on Monday October 13th at 10PM}
\question{1}{}
Show that
\begin{align}
    g_{\mu \nu, \gamma}=\Gamma_{\mu \nu \gamma}+\Gamma_{\nu \mu \gamma},
\end{align}
where
\begin{align}
    \Gamma_{\mu \nu \lambda} \equiv g_{\mu \sigma} \Gamma_{\nu \lambda}^\sigma,
\end{align}
and that
\begin{align}
    g_{, \gamma}=g g^{\mu \nu} g_{\mu \nu, \gamma}.
\end{align}

\clearpage
\question{2}{}
The metric for the surface of a sphere of radius $a$ is determined by
\begin{align}
    d s^2=a^2\left(d \theta^2+\sin ^2 \theta d \phi^2\right)
\end{align}
\begin{itemize}
    \item [(i)]Calculate the components of the affine connection. It is conventional to denote the components as $\Gamma_{\theta \phi}^\phi$ and similarly for the other ones. In other words, $x^1=\theta$ and $x^2=\phi$.
    \item [(ii)]Referring to figure 3.2 of the textbook, parallel transport a vector along two different paths, starting from a point on the equator $\theta=\pi / 2, \phi=0$, and ending at the north pole. Path A is along a line of fixed longitude $\phi=0$. Path B first goes along the equator to $\phi=\pi / 2$ and then goes north along a line of fixed longitude. Initially the vector you are parallel transporting points in the $-\hat{\theta}$ direction. What is the angle between the two vectors at the north pole? (You must solve the parallel transport equation, just drawing a picture is not sufficient.)
\end{itemize}


\clearpage
\question{3}{}
Starting with the energy-momentum tensor of a massless scalar field $\phi$
\begin{align}
    T_{\mu \nu}=\phi_{, \mu} \phi_{, \nu}-\frac{1}{2} g_{\mu \nu} \phi_{, \sigma} \phi^{, \sigma},
\end{align}
derive the equation of motion for $\phi$ in the presence of a gravitational field.