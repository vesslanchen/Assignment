\section*{Assignment 8 due on Monday November 3 at 10PM}


\question{1}{}
In lecture we showed that $P^0=M$ for the Schwarzschild solution in standard coordinates. Calculate $P^z$ and show that it is zero. Since the metric is isotropic this shows that all components of the $3$-momentum are zero.

\answer{}
\begin{align}
    P^j =&\frac{-1}{16\pi G} \int \Bigg(-\partial_t h_{kk}\delta_{ij}+ \partial_t h_{ij} + \partial_k h_{0k}\delta_{ij} - \partial_i h_{0j} \Bigg) n^i r^2 d\Omega \\
    =&\frac{-1}{16\pi G} \int \Bigg(-\partial_t h_{kk}\delta_{ij}+ \partial_t h_{ij} + \partial_k h_{0k}\delta_{ij} - \partial_i h_{0j} \Bigg) \frac{x^i}{r} r^2 d\Omega \\
    =&\frac{-1}{16\pi G} \int \Bigg(-\partial_t h_{kk}\delta_{ij}+ \partial_t h_{ij} + \partial_k h_{0k}\delta_{ij} - \partial_i h_{0j} \Bigg) x^i r d\Omega \\
    =&\frac{-1}{16\pi G} \int \Bigg(-\partial_t h_{kk}\delta_{ij} x^i + \partial_t h_{ij} x^i + \partial_k h_{0k}\delta_{ij} x^i - \partial_i h_{0j} x^i \Bigg) r d\Omega \\
    =&\frac{-1}{16\pi G} \int \Bigg(-\partial_t h_{kk} x^j + \partial_t h_{ij} x^i + \partial_k h_{0k} x^j - \partial_i h_{0j} x^i \Bigg) r d\Omega \\
    =&\frac{-1}{16\pi G} \int \Bigg(-\partial_t h_{kk} x^j + \partial_t h_{ij} x^i  \Bigg) r d\Omega \quad \text{(since } h_{0j} =h_{0k}= 0 \text{ for Schwarzschild metric)} \\
    =& 0 \quad \text{(since } h_{\mu\nu} \text{ is time-independent for Schwarzschild metric)} \qed 
\end{align}




\clearpage
\question{2}{}
In lecture we were given the components of the affine connection $\Gamma_{\mu \nu}^\lambda$ for the Scharzschild solution in standard coordinates. Using these, calculate one component of the Riemann-Christoffel curvature tensor, namely $R_{r t r}^t$. This is nonvanishing everywhere and goes to zero as $r \rightarrow \infty$. This shows that space is curved even though both the Ricci tensor and curvature scalar vanish.
\answer{}
By the definition of the Riemann-Christoffel curvature tensor, we have
\begin{align}
    R^\lambda{}_{\mu \nu \rho} = \partial_\nu \Gamma^\lambda_{\mu \rho} - \partial_\rho \Gamma^\lambda_{\mu \nu} + \Gamma^\lambda_{\nu \sigma} \Gamma^\sigma_{\mu \rho} - \Gamma^\lambda_{\rho \sigma} \Gamma^\sigma_{\mu \nu}.
\end{align}
Thus, we have
\begin{align}
    R^t{}_{r t r} &= \partial_t \Gamma^t_{r r} - \partial_r \Gamma^t_{r t} + \Gamma^t_{t \sigma} \Gamma^\sigma_{r r} - \Gamma^t_{r \sigma} \Gamma^\sigma_{r t}.
\end{align}
From the lecture notes, we have
\begin{align}
    \Gamma^t_{r t} = \Gamma^t_{t r} = \frac{M}{r^2} \left( 1 - \frac{2M}{r} \right)^{-1}, \quad \Gamma^t_{r r} = 0.
\end{align}
Thus, we have (by Mathematica)
\begin{align}
    R_{r t r}^t &= -\partial_r \Gamma^t_{r t} + \Gamma^t_{t t} \Gamma^t_{r r} + \Gamma^t_{t r} \Gamma^r_{r r} - \Gamma^t_{r t} \Gamma^t_{r t} - \Gamma^t_{r r} \Gamma^r_{r t} \\
     &= \frac{2M}{r^2(r-2M)}
\end{align}
We can see that $R_{r t r}^t$ is nonvanishing everywhere and it goes to zero as $r \rightarrow \infty$. \qed

\clearpage
\question{3}{}
A photon moves in the Schwarzschild metric in the equatorial plane $\theta=\pi / 2$. Using standard coordinates, show that the shape of the orbit is given by the solution to the differential equation

\begin{align}
    \frac{d^2 w}{d \phi^2}+w=3 w^2,
\end{align}

where $w=G M / r$. Assuming that $|w| \ll 1$, solve this equation iteratively to find the deflection angle $\Delta \phi$, and show that it agrees with the answer obtained in class by other means.
\answer{}
For a photon moving in the Schwarzschild metric in the equatorial plane $\theta=\pi / 2$, we have the following equations of motion:
\begin{align}
    \left( \frac{dr}{d\lambda} \right)^2 &= E^2 - \left( 1 - \frac{2GM}{r} \right) \frac{L^2}{r^2}, \\
    \frac{d\phi}{d\lambda} &= \frac{L}{r^2},
\end{align}
where $\lambda$ is an affine parameter along the photon's trajectory, $E$ is the energy per unit mass, and $L$ is the angular momentum per unit mass. Dividing the first equation by the square of the second equation, we have
\begin{align}
    \left( \frac{dr}{d\phi} \right)^2 = \frac{r^4}{L^2} \left( E^2 - \left( 1 - \frac{2GM}{r} \right) \frac{L^2}{r^2} \right).
\end{align}
Rearranging this equation, we have
\begin{align}
    \left( \frac{dr}{d\phi} \right)^2 = \frac{r^4 E^2}{L^2} - r^2 + 2GM r.
\end{align}
Next, we introduce the variable $w = \frac{GM}{r}$. Thus, we have $r = \frac{GM}{w}$ and $\frac{dr}{d\phi} = -\frac{GM}{w^2} \frac{dw}{d\phi}$. Substituting these into the previous equation, we have
\begin{align}
    \left( -\frac{GM}{w^2} \frac{dw}{d\phi} \right)^2 = \frac{\left( \frac{GM}{w} \right)^4 E^2}{L^2} - \left( \frac{GM}{w} \right)^2 + 2GM \left( \frac{GM}{w} \right).
\end{align}
Simplifying this equation, we have
\begin{align}
    \left( \frac{dw}{d\phi} \right)^2 = \frac{E^2 (GM)^2}{L^2} - w^2 + 2 w^3.
\end{align}
Taking the derivative of both sides with respect to $\phi$, we have
\begin{align}
    2 \frac{dw}{d\phi} \frac{d^2 w}{d\phi^2} = -2 w \frac{dw}{d\phi} + 6 w^2 \frac{dw}{d\phi}.
\end{align}
Dividing both sides by $2 \frac{dw}{d\phi}$, we have
\begin{align}
    \frac{d^2 w}{d\phi^2} = -w + 3 w^2.
\end{align}
Rearranging this equation, we have
\begin{align}
    \frac{d^2 w}{d\phi^2} + w = 3 w^2.
\end{align}
To solve this equation iteratively, we first solve the homogeneous equation:
\begin{align}
    \frac{d^2 w_0}{d\phi^2} + w_0 = 0.
\end{align}
The general solution to this equation is
\begin{align}
    w_0(\phi) = A \cos\phi + B \sin\phi,
\end{align}
where $A$ and $B$ are constants determined by initial conditions. By considering a photon coming from infinity, we set $A =\frac{GM}{b}$ and $B = 0$, where $b$ is the impact parameter. In other words,
\begin{align}
    w_0(0) = \frac{GM}{b}, \quad \frac{dw_0}{d\phi} \bigg|_{\phi=0} = 0.
\end{align} Thus, we have
\begin{align}
    w_0(\phi) = \frac{GM}{b} \cos\phi.
\end{align}
Next, we substitute $w_0$ into the right-hand side of the original equation to find the first-order correction $w_1$:
\begin{align}
    \frac{d^2 w_1}{d\phi^2} + w_1 = 3 w_0^2 = 3 \left( \frac{GM}{b} \cos\phi \right)^2 = 3 \left( \frac{GM}{b} \right)^2 \cos^2\phi.
\end{align}
Using the identity $\cos^2\phi = \frac{1 + \cos 2\phi}{2}$, we have
\begin{align}
    \frac{d^2 w_1}{d\phi^2} + w_1 = \frac{3}{2} \left( \frac{GM}{b} \right)^2 (1 + \cos 2\phi).
\end{align}
By considering a photon coming from infinity, we have 
\begin{align}
    w_1(0) = 0, \quad \frac{dw_1}{d\phi} \bigg|_{\phi=0} = 0.
\end{align}
Thus, we have (by mathematica)
\begin{align}
    w_1(\phi) &= \frac{2 G^2 M^2 \sin ^2\left(\frac{\phi }{2}\right) (\cos (\phi )+2)}{b^2}\\
    &= \frac{G^2 M^2}{2 b^2} (3-2 \cos\phi - \cos 2\phi).
\end{align}
Hence the approximate solution up to first order is
\begin{align}
    w(\phi) = w_0(\phi) + w_1(\phi) = \frac{GM}{b} \cos\phi + \frac{G^2 M^2}{2 b^2} (3-2 \cos\phi - \cos 2\phi).
\end{align}
To find the deflection angle $\Delta \phi$, we set $w(\phi) = 0$ and solve for $\phi$ when the photon is far away from the mass. Thus, we have
\begin{align}
    0 = \frac{GM}{b} \cos\phi + \frac{G^2 M^2}{2 b^2} (3-2 \cos\phi - \cos 2\phi).
\end{align}
Expanding $\cos 2\phi = 2 \cos^2\phi - 1$, we have
\begin{align}
    0 =& \frac{GM}{b} \cos\phi + \frac{G^2 M^2}{2 b^2} (4 - 4 \cos\phi + 2 \cos^2\phi)\\
    =& \frac{G^2 M^2}{b^2} \cos^2\phi + \left( \frac{GM}{b} - \frac{2 G^2 M^2}{b^2} \right) \cos\phi + \frac{2 G^2 M^2}{b^2}\\
    \approx& \frac{G^2 M^2}{b^2} \cos^2\phi + \frac{GM}{b} \cos\phi + \frac{2 G^2 M^2}{b^2}.
\end{align}
Solving this quadratic equation for $\cos\phi$, we have
\begin{align}
    \cos\phi = \frac{-\frac{GM}{b} \pm \sqrt{\left( \frac{GM}{b} \right)^2 - 8 \left(\frac{G M}{b}\right)^4}}{2 \frac{G^2 M^2}{b^2}}\approx -2\frac{GM}{b}
\end{align}
Thus, we have
\begin{align}
    \phi \approx \frac{\pi}{2} + 2\frac{GM}{b}.
\end{align}
Since the photon comes from infinity and goes back to infinity, the total deflection angle is
\begin{align}
    \Delta \phi = 2 \left( \phi - \frac{\pi}{2} \right) = \frac{4GM}{b}.
\end{align}
This agrees with the answer obtained in class by other means. \qed


\clearpage
\question{4}{}
The deflection of light by a spherical static body whose physical radius is smaller than its Schwarzschild radius should produce comet-like orbits suffering substantial deflection before returning to infinite radius if the distance of closest approach $r_0$ becomes comparable to the Schwarzschild radius. Using the differential equation in problem 1 show that there is a critical orbit $w_c(\phi)$, corresponding to a critical radius $r_c$. Explain what happens when $r_0>r_c, r_0=r_c$, and $r_0<r_c$. Draw some orbits for illustration. This phenomenon was observed by the "Event Horizon Telescope".
\answer{}
From problem 3, we have the differential equation
\begin{align}
    \frac{d^2 w}{d\phi^2} + w = 3 w^2,
\end{align}
where $w = \frac{GM}{r}$. To find the critical orbit $w_c(\phi)$, we look for a circular orbit where $w$ is constant. Thus, we set $\frac{d^2 w}{d\phi^2} = 0$.
Thus, we have
\begin{align}
    w_c = 3 w_c^2.
\end{align}
Solving this equation, we have
\begin{align}
    w_c = 0 \quad \text{or} \quad w_c = \frac{{1}}{{3}}.
\end{align}
The solution $w_c = 0$ corresponds to an orbit at infinite radius, which is not physically interesting. The solution $w_c = \frac{1}{3}$ corresponds to a critical radius
\begin{align}
    r_c = 3 GM.
\end{align}
When the distance of closest approach $r_0$ is greater than the critical radius $r_c$ ($r_0 > r_c$), the photon will be deflected but will eventually escape to infinity, resulting in a hyperbolic-like orbit. When $r_0$ is equal to the critical radius $r_c$ ($r_0 = r_c$), the photon will spiral around the mass at the critical radius, resulting in a circular orbit. When $r_0$ is less than the critical radius $r_c$ ($r_0 < r_c$), the photon will be captured by the mass and will spiral inward, eventually falling into the black hole. This results in a trajectory that does not return to infinity. 

\begin{figure}
    \centering
    \includegraphics[width=0.7\textwidth]{Problem 8/r<r_c.pdf}
    \caption{Orbit for $r_0 < r_c$, photon spirals inward and falls into the black hole.}
\end{figure}
\begin{figure}
    \centering
    \includegraphics[width=0.7\textwidth]{Problem 8/r=r_c.pdf}
    \caption{Orbit for $r_0 = r_c$, photon spirals around the mass at the critical radius.}
\end{figure}
\begin{figure}
    \centering
    \includegraphics[width=0.7\textwidth]{Problem 8/r>r_c.pdf}
    \caption{Orbit for $r_0 > r_c$, photon is deflected but escapes to infinity.}
\end{figure}
\qed