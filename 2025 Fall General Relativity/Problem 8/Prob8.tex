\section*{Assignment 8 due on Monday November 3 at 10PM}


\question{1}{}
In lecture we showed that $P^0=M$ for the Schwarzschild solution in standard coordinates. Calculate $P^z$ and show that it is zero. Since the metric is isotropic this shows that all components of the $3$-momentum are zero.

\answer{}


\clearpage
\question{2}{}
In lecture we were given the components of the affine connection $\Gamma_{\mu \nu}^\lambda$ for the Scharzschild solution in standard coordinates. Using these, calculate one component of the Riemann-Christoffel curvature tensor, namely $R_{r t r}^t$. This is nonvanishing everywhere and goes to zero as $r \rightarrow \infty$. This shows that space is curved even though both the Ricci tensor and curvature scalar vanish.
\answer{}

\clearpage
\question{3}{}
A photon moves in the Schwarzschild metric in the equatorial plane $\theta=\pi / 2$. Using standard coordinates, show that the shape of the orbit is given by the solution to the differential equation

\begin{align}
    \frac{d^2 w}{d \phi^2}+w=3 w^2,
\end{align}

where $w=G M / r$. Assuming that $|w| \ll 1$, solve this equation iteratively to find the deflection angle $\Delta \phi$, and show that it agrees with the answer obtained in class by other means.

\clearpage
\question{4}{}
The deflection of light by a spherical static body whose physical radius is smaller than its Schwarzschild radius should produce comet-like orbits suffering substantial deflection before returning to infinite radius if the distance of closest approach $r_0$ becomes comparable to the Schwarzschild radius. Using the differential equation in problem 1 show that there is a critical orbit $w_c(\phi)$, corresponding to a critical radius $r_c$. Explain what happens when $r_0>r_c, r_0=r_c$, and $r_0<r_c$. Draw some orbits for illustration. This phenomenon was observed by the "Event Horizon Telescope".
\answer{}