\documentclass[12pt,letterpaper]{article}

\usepackage[letterpaper, left=1.5cm, right=1.5cm, top=1.5cm, bottom=1.5cm]{geometry}
\usepackage[compat=1.0.0]{tikz-feynman}
\usepackage{amsmath}
\usepackage{amsthm}
\usepackage{amssymb}
\usepackage{graphicx}
\usepackage{setspace} % 引入這個套件
\usepackage{tensor}
\usepackage{braket}
\usepackage{hyperref}\hypersetup{colorlinks=true,linkcolor=black,citecolor=green,urlcolor=blue}

\title{University of Minnesota\\
School of Physics and Astronomy\\
\textbf{2025 Fall Physics 8501 \\ General Relativity I}\\ 
Assignment Solution}
\author{Lecture Instructor: Professor Joseph Kapusta \\ \\
        Zong-En Chen\\ chen9613@umn.edu}
\date{\today}  

% 自定義範例環境
\newtheorem{example}{Example}[section]
\newtheorem{solution}{Sol.}[section]
% 自定義習題環境
\newtheorem{exercise}{Ex.}[section]

\newtheorem{theorem}{Theorem}[section]

\newcommand{\question}[2]{\section*{Question #1}}
\newcommand{\answer}[1]{\section*{Answer}#1}

\begin{document}
\onehalfspacing
%\maketitle
\begin{titlepage}
    \centering
    \vspace*{\fill} % 從頂端往下推到中間
    {\LARGE University of Minnesota\\
    School of Physics and Astronomy\\[1em]
    \textbf{2025 Fall Physics 8501\\ General Relativity I}\\ 
    Assignment Solution \par}
    \vspace{2cm}
    {\large Lecture Instructor: Professor Joseph Kapusta \par}
    \vspace{1cm}
    {\large Zong-En Chen\\ chen9613@umn.edu \par}
    \vspace{2cm}
    {\large \today \par}
    \vspace*{\fill} % 從底部往上推到中間
\end{titlepage}

%
\section*{Assignment 1 due on Wednesday September 10th at 5PM}

\question{1}{In lecture we found that ${\Lambda^0}_0 = \gamma$ and ${\Lambda^i}_0=\gamma v_i$ for boosts. The other components should be of the form ${\Lambda^i}_j=a(v)\delta_{ij}+b(v)v_iv_j$ and ${\Lambda^0}_j=c(v)v_j$, where a, b, and c can be functions of the speed v. Determine these functions from the constraint $\eta_{\alpha\beta}{\Lambda^\alpha}_\gamma{\Lambda^\beta}_\delta=\eta_{\gamma\delta}$.}

\answer{}
We can use tensor notation to reduce the work. From the constraint
\begin{align}
    \eta_{\alpha\beta}{\Lambda^\alpha}_\gamma{\Lambda^\beta}_\delta=\eta_{\gamma\delta},
\end{align}
and the relation $\gamma=\frac{1}{\sqrt{1-v^2}} $ (or $\gamma^2-\gamma^2v^2=\gamma^2(1-v^2)=1$), we can consider different $\gamma,\delta$, meaning that
\begin{itemize}
    \item $(\gamma,\delta)=(0,0)$: 
    \begin{align}
    -({\Lambda^0}_0)^2+\sum_{i=1}^3({\Lambda^i}_0)^2=-\gamma^2+\gamma^2\sum_{i=1}^3(v_i)^2=-\gamma^2+\gamma^2 v^2  =-1  =\eta_{00}
    \end{align}
    \item $(\gamma,\delta)=(0,i)$:
    \begin{align}
         &-{\Lambda^0}_0{\Lambda^0}_i+\sum_{k=1}^3{\Lambda^k}_0{\Lambda^k}_i\\
        =&-\gamma c v_i +\sum_{k=1}^3 \gamma v_k (a\delta_{ki}+bv_kv_i)\\
        =&-\gamma c v_i + \gamma a v_i +\gamma b v^2 v_i = \gamma v _i (-c+a+bv^2)=\eta_{0i}=0
    \end{align}
    Then we have $c=a+bv^2$.
    \item $(\gamma,\delta)=(i,0)$
    \begin{align}
        &-{\Lambda^0}_i{\Lambda^0}_0+\sum_{k=1}^3{\Lambda^k}_i{\Lambda^k}_0\\
       =&\gamma v _i (-c+a+bv^2)
    \end{align}
    \item $(\gamma,\delta)=(i,i)$
    \begin{align}
         &-{\Lambda^0}_i{\Lambda^0}_i+\sum_{k=1}^3{\Lambda^k}_i{\Lambda^k}_i\\
        =&-c^2v_i^2 + \sum_{k=1}^3 (a\delta_{ki}+bv_kv_i)^2\\
        =&-c^2v_i^2 + \sum_{k=1}^3(a^2\delta_{ki}+2ab\delta_{ki}v_kv_i+b^2v_i^2v_k^2)\\
        =&-c^2v_i^2 +a^2 +2abv_i^2+b^2v_i^2v^2=\eta_{ii}=1
    \end{align}
    \item $(\gamma,\delta)=(i,j), i\neq j$
    \begin{align}
        &-{\Lambda^0}_i{\Lambda^0}_j+\sum_{k=1}^3{\Lambda^k}_i{\Lambda^k}_j\\
        =&-c^2 v_iv_j+\sum_{k=1}^3 (a\delta_{ki}+bv_kv_i) (a\delta_{kj}+bv_kv_j)\\
        =&-c^2 v_iv_j+\sum_{k=1}^3 (a^2 \delta_{ki}\delta_{kj}+ab(\delta_{ki}v_kv_j+\delta_{kj}v_kv_i)+b^2v_k^2v_iv_j)\\
        =&-c^2 v_iv_j +2 ab v_iv_j +b^2 v^2 v_i v_j =\eta_{ij}=0
    \end{align}
    Then we have $c^2=b^2v^2+2ab$
\end{itemize}
Combining the above information, we can have 
\begin{align}
    &c = a+bv^2 \\
    &c^2 = a^2+b^2v^4+2abv^2 = b^2v^2+2ab.  \label{Eq:c^2}
\end{align}
Also, we have 
\begin{align}
    &-c^2v_i^2 +a^2 +2abv_i^2+b^2v_i^2v^2=1\\
    \rightarrow 1 &= -(b^2v^2+2ab)v_i^2+a^2 +2abv_i^2+b^2v_i^2v^2\\
    &=a^2        \label{Eq:a^2=1}
\end{align}
Hence, from Eq.~\ref{Eq:c^2} and Eq.~\ref{Eq:a^2=1}, we have 
\begin{align}
    &c^2 = 1 + b^2v^4+2abv^2=b^2v^2+2ab\\
    \rightarrow&a=\pm1=\frac{1+b^2v^4-b^2v^2}{2b(1-v^2)}=\frac{1+b^2v^2(v^2-1)}{2b(1-v^2)}=\frac{\gamma^2-b^2v^2}{2b}\\
    \rightarrow&b^2v^2\pm2b-\gamma^2=0
\end{align}
For $a=+1$, we have 
\begin{align}
    b = \frac{-1\pm\sqrt{1+\gamma^2v^2}}{v^2}=\frac{-1\pm\gamma}{v^2},
\end{align}
and for $a=-1$, we have 
\begin{align}
    b = \frac{1\pm\gamma}{v^2}.
\end{align}
Hence, we choose $a=1$ and $b=\frac{-1+\gamma}{v^2}$ by convention and have 
\begin{align}
    c &= a+bv^2 = 1 + (-1+\gamma)=+\gamma\\
\end{align}
Finally, we derive 
\begin{align}
    {\Lambda^i}_j=&a(v)\delta_{ij}+b(v)v_iv_j= \delta_{ij}+\frac{\gamma-1}{v^2}v_iv_j\\
    {\Lambda^0}_j=&c(v)v_j=\gamma v_j,
\end{align}
which are mentioned in the Week 1 lecture.\qed

\newpage
\question{2}{ A rod of length $L$ lies at rest along the x-axis in frame $\mathcal{O}$. An observer located along a perpendicular axis in frame $\mathcal{O}'$ sees frame $\mathcal{O}$ moving with speed v in the positive $x$ direction. The observer in frame $\mathcal{O}'$, which may be an eye or a camera, looks in the direction of the rod. When the center of the rod is at the distance of closest approach $D$ to the observer, lights at the ends of the rod send out flashes of light. When $v \ll c$, the opening angle $\theta$ between the light flashes would be seen as $\tan{\theta/2} = L/2D$. Show that for any value of $v < c$, the opening angle is the same, meaning that the observer does not see length contraction. Drawing a picture will help.}
\answer{}
In the $\mathcal{O}$ frame, we set the condition for the right-hand side of the rod:
\begin{align*}
    &x^{\mu}_{emit}=(0,L/2,D,0),x^{\mu}_{receive}=(t,vt,0,0),\\
    &\Delta x^{\mu}=x^{\mu}_{receive}-x^{\mu}_{emit}=(t,vt-L/2,-D,0)\\
    &\Delta x^\mu\Delta x_\mu=0=t^2-(L/2-vt)^2-D^2.
\end{align*}
Also, with $dt=t, dx=vt-L/2, dy=-D$, we have 
\begin{align*}
    \Delta t'&=\gamma(t+v(vt-L/2))\\
    \Delta x'&=\gamma(v+(vt-L/2))\\
    \Delta y'&=-D
\end{align*}
We can substitute the solution $t$ to solve the $dx'$ and $dt'$, and define the opening angle 
\begin{align}
    \tan{\theta'/2}=\frac{\Delta x'}{\Delta y'}.
\end{align}
Note that this is the calculation for right-hand side of the rod. Hence, it should be denoted as $\theta_R'$. Now we can apply the procedure to the left-hand side of the rod. After plugging all information into {\tt Mathematica} and using the equation, we have 
\begin{align}
    \tan{(\theta_R'+\theta_L')}=\frac{\tan\theta_R'+\tan\theta_L'}{1-\tan\theta_R'\tan\theta_L'}=\frac{4DL}{4D^2-L^2}=\frac{2\frac{L}{2D}}{1-\left(\frac{L}{2D}\right)^2}=\tan{\theta}.
\end{align}
Hence, we prove that the opening angle keeps the same value. \qed
%\section*{HW2 Due to October 7 11:59 PM}

\question{1}{Problem 5.1}\\
Work out the LSZ reduction formula for the complex scalar field that was introduced in problem~$3.5$. Note that we must specify the type ($a$ or $b$) of each incoming and outgoing particle.
\answer{}
We start with the mode expansion of the complex scalar field:
\begin{align}
    \varphi(x)=\int \frac{d^3k}{(2\pi)^3 2\omega}[a(\mathbf{k})e^{ikx}+b^\dagger(\mathbf{k})e^{-ikx}]\\
    \varphi^\dagger(x)=\int \frac{d^3k}{(2\pi)^3 2\omega}[b(\mathbf{k})e^{ikx}+a^\dagger(\mathbf{k})e^{-ikx}]
\end{align}
\begin{align}
    a(\mathbf{k})=&\int d^3 x e^{-i k x}\left[i \partial_0 \varphi(x)+\omega \varphi(x)\right],\\
    b(\mathbf{k})=&\int d^3x e^{-ikx}[\omega\varphi^\dagger(x)+i\partial_0\varphi^\dagger(x)].
\end{align}
First, we define the $|i\rangle$ and $|f\rangle$ states as
\begin{align}
    |i\rangle=& \lim_{t\to -\infty} a_1^\dagger(t)a_2^\dagger(t)\cdots b_1^\dagger(t)b_2^\dagger(t)\cdots |0\rangle,\\
    |f\rangle=& \lim_{t\to +\infty} a_{1'}^\dagger(t)a_{2'}^\dagger(t)\cdots b_{1'}^\dagger(t)b_{2'}^\dagger(t)\cdots |0\rangle.
\end{align}
And $a_i$ and $b_i$ are given by 
\begin{align}
    a_i^\dagger=&\int d^3k f_i(\mathbf{k})a^\dagger(\mathbf{k})\\
    b_i^\dagger=&\int d^3k g_i(\mathbf{k})b^\dagger(\mathbf{k}),
\end{align}
where
\begin{align}
    f_i(\mathbf{k}),g_i(\mathbf{k}) \propto& \exp(-{(\mathbf{k}-\mathbf{k}_i)^2}/{4\sigma^2}).
\end{align}
Now we can compute the difference between $a_1^\dagger(+\infty)$ and $a_1^\dagger(-\infty)$:
\begin{align}
    a_1^\dagger(+\infty)-a_1^\dagger(-\infty)=&\int_{-\infty}^{+\infty} dt \partial_0 a_1^\dagger(t)\\
    =&\int_{-\infty}^{+\infty} dt \int d^3k f_1(\mathbf{k})\int d^3x e^{ikx}[\omega\varphi(x)-i\partial_0\varphi(x)]\\
    =& -i\int d^3k f_1(\mathbf{k})\int d^4x e^{ikx}(-\partial_\mu\partial^\mu +m^2)\varphi(x),
\end{align}
where I quote the equation in the textbook. Similarly, we can get
\begin{align}
    b_1^\dagger(+\infty)-b_1^\dagger(-\infty)=& -i\int d^3k g_1(\mathbf{k})\int d^4x e^{ikx}(-\partial_\mu\partial^\mu +m^2)\varphi^\dagger(x),\\
    a_{1'}(+\infty)-a_{1'}(-\infty)=& i\int d^3k f_{1'}^*(\mathbf{k})\int d^4x e^{-ikx}(-\partial_\mu\partial^\mu +m^2)\varphi(x),\\
    b_{1'}(+\infty)-b_{1'}(-\infty)=& i\int d^3k g_{1'}^*(\mathbf{k})\int d^4x e^{-ikx}(-\partial_\mu\partial^\mu +m^2)\varphi^\dagger(x).
\end{align}
Now we can express the S-matrix element $\langle f|i\rangle$ as
\begin{align}
    \langle f|i\rangle=&\langle 0|\mathcal{T} b_{1'}(+\infty)b_{2'}(+\infty)\cdots a_{1'}(+\infty)a_{2'}(+\infty)\cdots a_1^\dagger(-\infty)a_2^\dagger(-\infty)\cdots b_1^\dagger(-\infty)b_2^\dagger(-\infty)\cdots |0\rangle\\
    =&\langle 0|\mathcal{T} [b_{1'}(-\infty)+i\int d^3k g_{1'}^*(\mathbf{k})\int d^4x e^{-ikx}(-\partial_\mu\partial^\mu +m^2)\varphi^\dagger(x)]\cdots\notag\\
    &\cdots [a_{1'}(-\infty)+i\int d^3k f_{1'}^*(\mathbf{k})\int d^4x e^{-ikx}(-\partial_\mu\partial^\mu +m^2)\varphi(x)]\cdots\notag\\
    &\cdots [a_1^\dagger(+\infty)+i\int d^3k f_1(\mathbf{k})\int d^4x e^{ikx}(-\partial_\mu\partial^\mu +m^2)\varphi(x)]\cdots\notag\\
    &\cdots [b_1^\dagger(+\infty)+i\int d^3k g_1(\mathbf{k})\int d^4x e^{ikx}(-\partial_\mu\partial^\mu +m^2)\varphi^\dagger(x)]\cdots |0\rangle\\
    =&\langle 0|\mathcal{T} [i\int d^3k g_{1'}^*(\mathbf{k})\int d^4x e^{-ikx}(-\partial_\mu\partial^\mu +m^2)\varphi^\dagger(x)]\cdots\notag\\
    &\cdots [i\int d^3k f_{1'}^*(\mathbf{k})\int d^4x e^{-ikx}(-\partial_\mu\partial^\mu +m^2)\varphi(x)]\cdots\notag\\
    &\cdots [i\int d^3k f_1(\mathbf{k})\int d^4x e^{ikx}(-\partial_\mu\partial^\mu +m^2)\varphi(x)]\cdots\notag\\
    &\cdots [i\int d^3k g_1(\mathbf{k})\int d^4x e^{ikx}(-\partial_\mu\partial^\mu +m^2)\varphi^\dagger(x)]\cdots |0\rangle\\
    =&(i)^{n+n'+m+m'}\langle 0|\mathcal{T} [\prod_{j'}^{n'}  \int d^4x e^{-ik_{j'}x}(-\partial_\mu\partial^\mu +m^2)\varphi^\dagger(x)]
    [\prod_{l'}^{n'}  \int d^4x e^{-ik_{l'}x}(-\partial_\mu\partial^\mu +m^2)\varphi(x)]\\
    & [\prod_{l}^{m}\int d^4x e^{ik_{l}x}(-\partial_\mu\partial^\mu +m^2)\varphi(x)] [\prod_{j}^{n}\int d^4x e^{ik_{j}x}(-\partial_\mu\partial^\mu +m^2)\varphi^\dagger(x)] |0\rangle,
\end{align}
where we have used the fact that $a_i|0\rangle=b_i|0\rangle=0$ and $\langle 0|a_i^\dagger=\langle 0|b_i^\dagger=0$. Here $n$ and $m$ are the number of incoming $a$ and $b$ particles, while $n'$ and $m'$ are the number of outgoing $a$ and $b$ particles, respectively. We also impose the $\sigma\to0$ limit, so that $f_i(\mathbf{k})$ and $g_i(\mathbf{k})$ become delta functions. Finally, we can rewrite the S-matrix element as
\begin{align}
    \langle f|i\rangle=&(i)^{n+n'+m+m'}\int d^4x_1 e^{-ik_1 x_1}\cdots \int d^4x_n e^{-ik_n x_n}\int d^4x_{1'} e^{ik_{1'} x_{1'}}\cdots \int d^4x_{n'} e^{ik_{n'} x_{n'}}\notag\\
    &\int d^4y_1 e^{-ip_1 y_1}\cdots \int d^4y_m e^{-ip_m y_m}\int d^4y_{1'} e^{ip_{1'} y_{1'}}\cdots \int d^4y_{m'} e^{ip_{m'} y_{m'}}\notag\\
    &(-\partial_\mu\partial^\mu_{x_1}+m^2)\cdots(-\partial_\mu\partial^\mu_{x_n}+m^2)(-\partial_\mu\partial^\mu_{x_{1'}}+m^2)\cdots(-\partial_\mu\partial^\mu_{x_{n'}}+m^2)\notag\\
    &(-\partial_\mu\partial^\mu_{y_1}+m^2)\cdots(-\partial_\mu\partial^\mu_{y_m}+m^2)(-\partial_\mu\partial^\mu_{y_{1'}}+m^2)\cdots(-\partial_\mu\partial^\mu_{y_{m'}}+m^2)\notag\\
    &\langle 0|\mathcal{T} \varphi^\dagger(y_{1'})\cdots \varphi^\dagger(y_{m'})\varphi(x_{1'})\cdots \varphi(x_{n'})\varphi(x_1)\cdots \varphi(x_n)\varphi^\dagger(y_1)\cdots \varphi^\dagger(y_m)|0\rangle.
\end{align}
This is the LSZ reduction formula for the complex scalar field. 
\qed

\clearpage
\question{2}{Problem 6.1}
\begin{itemize}
    \item [(a)] Find an explicit formula for $\mathcal{D}q$ in eq.~(6.9). Your formula should be of the form $\mathcal{D}q = C \prod_{j=1}^N dq_j$, where $C$ is a constant that you should compute.
    \item [(b)] For the case of a free particle, $V(Q)=0$, evaluate the path integral of eq.~(6.9) explicitly. Hint: integrate over $q_1$, then $q_2$, etc, and look for a pattern. Express you final answer in terms of $q',t',q'',t''$ and $m$. Restore $\hbar$ by dimensional analysis.
    \item [(c)] Compute the $\langle q'',t''|q',t' \rangle =\langle q'' |e^{-iH(t''-t')}|q'\rangle$ by inserting a complete set of momentum eigenstates, and performing the integral over the momentum. Compare your result in part (b).
\end{itemize}
\begin{align}
    \langle q'',t''| q',t'\rangle &=\int \prod_{k=1}^N dq_k\prod_{j=0}^N \frac{dp_j}{2\pi} e^{ip_j (q_{j+1}-q_j)}e^{-iH(p_j,\overline{q}_j)\delta t},  \tag{6.7}\\
    \langle q'',t''| q',t'\rangle &=\int \mathcal{D}q \exp{\Bigg[ i\int_{t'}^{t''}dt L(\dot{q}(t),q(t))  \Bigg]}.  \tag{6.9}
\end{align}
\answer{}
\begin{itemize}
    \item [(a)] First, from eq.~(6.7), we can see that
\end{itemize}
\begin{align}
    \langle q'',t''| q',t'\rangle =&\int \prod_{k=1}^N dq_k\prod_{j=0}^N \frac{dp_j}{2\pi} e^{ip_j (q_{j+1}-q_j)}e^{-iH(p_j,\overline{q}_j)\delta t},\quad \text{assuming }H(p,q)=\frac{1}{2m}p^2+V(q)\\
    =&\int \prod_{k=1}^N dq_k\prod_{j=0}^N \frac{dp_j}{2\pi} e^{ip_j (q_{j+1}-q_j)}e^{-i(\frac{1}{2m}p_j^2+V(\overline{q}_j))\delta t}\\
    =&\int \prod_{k=1}^N dq_k\prod_{j=0}^N \frac{dp_j}{2\pi} e^{ip_j\delta t \dot{q_j}}e^{-i(\frac{1}{2m}p_j^2+V(\overline{q}_j))\delta t},\quad \text{where }\dot{q_j}=\frac{q_{j+1}-q_j}{\delta t}\\
    =&\int \prod_{k=1}^N dq_k\prod_{j=0}^N \frac{dp_j}{2\pi} e^{-i\delta t(\frac{1}{2m}p_j^2 - p_j\dot{q_j}+V(\overline{q}_j))}\\
    =&\int \prod_{k=1}^N dq_k\prod_{j=0}^N \frac{dp_j}{2\pi} e^{-i\delta t(\frac{1}{2m}(p_j-m\dot{q_j})^2 - \frac{1}{2}m\dot{q_j}^2+V(\overline{q}_j))}\\
    =&\int \prod_{k=1}^N dq_k\prod_{j=0}^N \frac{dp_j}{2\pi} e^{-i\delta t\frac{1}{2m}(p_j-m\dot{q_j})^2}e^{i\delta t(\frac{1}{2}m\dot{q_j}^2 - V(\overline{q}_j))}\\
    =&\int \prod_{k=1}^N dq_k\prod_{j=0}^N \frac{dp_j}{2\pi} e^{-i\delta t\frac{1}{2m}(p_j-m\dot{q_j})^2}e^{i\delta t L(\dot{q_j},\overline{q}_j)},\quad \text{where }L(\dot{q},q)=\frac{1}{2}m\dot{q}^2 - V(q)\\
    =&\int \prod_{k=1}^N dq_k \Bigg[\prod_{j=0}^N \int \frac{dp_j}{2\pi} e^{-i\delta t\frac{1}{2m}(p_j-m\dot{q_j})^2}\Bigg]e^{i \sum_{j=0}^N \delta t L(\dot{q_j},\overline{q}_j)},
\end{align}
where we have used the definition of $\dot{q_j}$ and $L(\dot{q},q)$. Now we can compute the integral over $p_j$:
\begin{align}
    \int \frac{dp_j}{2\pi} e^{-i\delta t\frac{1}{2m}(p_j-m\dot{q_j})^2}=&\int \frac{dp_j}{2\pi} e^{-i\frac{\delta t}{2m}p_j^2} \quad \text{(by shifting }p_j\to p_j+m\dot{q_j}\text{)}\\
    =&\frac{1}{2\pi} \sqrt{\frac{2m\pi}{i\delta t}} \quad \text{(by Gaussian integral)}\\
    =&\sqrt{\frac{m}{2\pi i \delta t}}.
\end{align}
Thus, we have
\begin{align}
    \langle q'',t''| q',t'\rangle =&\int \prod_{k=1}^N dq_k \Bigg[\prod_{j=0}^N \sqrt{\frac{m}{2\pi i \delta t}}\Bigg]e^{i \sum_{j=0}^N \delta t L(\dot{q_j},\overline{q}_j)}\\
    =&\int \prod_{k=1}^N dq_k \Bigg(\frac{m}{2\pi i \delta t}\Bigg)^{\frac{N+1}{2}}e^{i \sum_{j=0}^N\delta t L(\dot{q_j},\overline{q}_j)}\\
    =&\int \prod_{k=1}^N dq_k \Bigg(\frac{m}{2\pi i \delta t}\Bigg)^{\frac{N+1}{2}}e^{i \int_{t'}^{t''} dt L(\dot{q}(t),q(t))} \quad \text{(by definition of Riemann integral)}.
\end{align}
Therefore, we can identify
\begin{align}
    \mathcal{D}q=&\Bigg(\frac{m}{2\pi i \delta t}\Bigg)^{\frac{N+1}{2}} \prod_{j=1}^N dq_j.
\end{align}
This is the explicit formula for $\mathcal{D}q$ in eq.~(6.9).
\begin{itemize}
    \item [(b)] Now if we consider the case of a free particle, i.e. $V(Q)=0$, then we have
\end{itemize}
\begin{align}
    L(\dot{q},q)=&\frac{1}{2}m\dot{q}^2,\\
    \langle q'',t''| q',t'\rangle =&\int \prod_{k=1}^N dq_k \Bigg(\frac{m}{2\pi i \delta t}\Bigg)^{\frac{N+1}{2}}e^{i \int_{t'}^{t''} dt \frac{1}{2}m\dot{q}^2}\\
    =&\int \prod_{k=1}^N dq_k \Bigg(\frac{m}{2\pi i \delta t}\Bigg)^{\frac{N+1}{2}}e^{i \sum_{j=0}^N \delta t \frac{1}{2}m\dot{q_j}^2}.\\
\end{align}
The terms in the exponent are given by:
\begin{align}
    \sum_{j=0}^N \delta t \frac{1}{2}m\dot{q_j}^2=\sum_{j=0}^N \delta t \frac{1}{2}m\Big(\frac{q_{j+1}-q_j}{\delta t}\Big)^2
    =\sum_{j=0}^N \frac{m}{2\delta t}(q_{j+1}^2 - 2q_{j+1}q_j + q_j^2).
\end{align}
Thus, we focus on the integral and compute it step by step:
\begin{align}
    &\int dq_N\cdots \int dq_2\exp{\Big( \sum_{j=2}^N \frac{im}{2\delta t}(q_{j+1}^2 - 2q_{j+1}q_j + q_j^2) \Big)}  \int dq_1 \exp\Big[\frac{im}{2\delta t}\Big((q_2^2 - 2q_2 q_1 + q_1^2)+(q_1^2 - 2q_1 q_0 + q_0^2)\Big)\Big]\\
    =&\int dq_N\cdots \int dq_2\exp{\Big( \sum_{j=2}^N i\lambda(q_{j+1}-q_j)^2 \Big)} \notag\\
     &\int dq_1 \exp\Big[i\lambda\Big( (q_2-q_1)^2+(q_1-q_0)^2 \Big)\Big],\quad \text{where }\lambda=\frac{m}{2\delta t}
\end{align}
Before performing the integral over $q_1$, we consider the following integral:
\begin{align}
    \int_{-\infty}^{+\infty} dx e^{i\alpha(x-\beta)^2}=\sqrt{\frac{i\pi}{\alpha}}
\end{align}
Then we also consider the more complicated integral:
\begin{align}
    \int_{-\infty}^{+\infty} dx e^{i\alpha(x-c_1)^2+i\beta(x-c_2)^2}=&\int_{-\infty}^{+\infty} dx e^{i(\alpha+\beta)x^2 - 2i(\alpha c_1 + \beta c_2)x + i(\alpha c_1^2 + \beta c_2^2)}\\
    =&e^{i\frac{\alpha\beta}{\alpha+\beta}(c_1-c_2)^2}\int_{-\infty}^{+\infty} dx e^{i(\alpha+\beta)(x-\frac{\alpha c_1 + \beta c_2}{\alpha+\beta})^2}\\
    =&e^{i\frac{\alpha\beta}{\alpha+\beta}(c_1-c_2)^2}\sqrt{\frac{i\pi}{\alpha+\beta}}=e^{i\frac{1}{\frac{1}{\alpha}+\frac{1}{\beta}}(c_1-c_2)^2}\sqrt{\frac{i\pi}{\alpha+\beta}},
\end{align}
where I quoted the result from \texttt{Mathematica}. Now we can perform the integral over $q_1$:
\begin{align}
    &\int dq_1 \exp\Big[i\lambda\Big( (q_2-q_1)^2+(q_1-q_0)^2 \Big)\Big]\\
    =&\int dq_1 \exp\Big[i\lambda(q_1-q_2)^2 + i\lambda(q_1-q_0)^2 \Big]\\
    =&e^{i\frac{\lambda^2}{2\lambda}(q_2-q_0)^2}\sqrt{\frac{i\pi}{2\lambda}}\\
    =&e^{i\frac{\lambda}{2}(q_2-q_0)^2}\sqrt{\frac{i\pi}{2\lambda}}.
\end{align}
Thus, we have
\begin{align}
    &\int dq_N\cdots \int dq_2\exp{\Big( \sum_{j=2}^N i\lambda(q_{j+1}-q_j)^2 \Big)} \int dq_1 \exp\Big[i\lambda\Big( (q_2-q_1)^2+(q_1-q_0)^2 \Big)\Big]\\
    =&\sqrt{\frac{i\pi}{2\lambda}}\int dq_N\cdots \int dq_2\exp{\Big( \sum_{j=2}^N i\lambda(q_{j+1}-q_j)^2 + i\frac{\lambda}{2}(q_2-q_0)^2 \Big)}\\
    =&\sqrt{\frac{i\pi}{2\lambda}}\int dq_N\cdots \int dq_3\exp{\Big( \sum_{j=3}^N i\lambda(q_{j+1}-q_j)^2 \Big)} \int dq_2 \exp{\Big(i\lambda(q_3-q_2)^2 + i\frac{\lambda}{2}(q_2-q_0)^2 \Big)}\\
    =&\sqrt{\frac{i\pi}{2\lambda}}\sqrt{\frac{i\pi}{\frac{3}{2}\lambda}}\int dq_N\cdots \int dq_3\exp{\Big( \sum_{j=3}^N i\lambda(q_{j+1}-q_j)^2 + i\frac{\lambda}{3}(q_3-q_0)^2 \Big)}\\
    =&\sqrt{\left(\frac{i\pi}{\lambda}\right)^2}\frac{1}{\sqrt{3}}\int dq_N\cdots \int dq_4\exp{\Big( \sum_{j=4}^N i\lambda(q_{j+1}-q_j)^2 \Big)} \int dq_3 \exp{\Big(i\lambda(q_4-q_3)^2 + i\frac{\lambda}{3}(q_3-q_0)^2 \Big)}\\
    =&\cdots\\
    =&\sqrt{\left(\frac{i\pi}{\lambda}\right)^{N-1}}\frac{1}{\sqrt{N}}\int dq_N \exp{\Big(i\lambda(q_{N+1}-q_N)^2 + i\frac{\lambda}{N}(q_N-q_0)^2 \Big)}\\
    =&\sqrt{\left(\frac{i\pi}{\lambda}\right)^{N}}\frac{1}{\sqrt{N+1}}e^{i\frac{\lambda}{N+1}(q_{N+1}-q_0)^2}.
\end{align}
Combine with the prefactor, we have
\begin{align}
    \langle q'',t''| q',t'\rangle =&\Bigg(\frac{m}{2\pi i \delta t}\Bigg)^{\frac{N+1}{2}}\sqrt{\left(\frac{i\pi}{\lambda}\right)^{N}}\frac{1}{\sqrt{N+1}}e^{i\frac{\lambda}{N+1}(q_{N+1}-q_0)^2}\\
    =&\Bigg(\frac{m}{2\pi i \delta t}\Bigg)^{\frac{N+1}{2}}\sqrt{\left(\frac{i\pi}{\frac{m}{2\delta t}}\right)^{N}}\frac{1}{\sqrt{N+1}}e^{i\frac{\frac{m}{2\delta t}}{N+1}(q''-q')^2}\\
    =&\Bigg(\frac{m}{2\pi i \delta t}\Bigg)^{\frac{N+1}{2}}\left(\frac{2i\pi \delta t}{m}\right)^{\frac{N}{2}}\frac{1}{\sqrt{N+1}}e^{i\frac{m}{2(N+1)\delta t}(q''-q')^2}\\
    =&\sqrt{\frac{m}{2\pi i (N+1)\delta t}}e^{i\frac{m}{2(N+1)\delta t}(q''-q')^2}\\
    =&\sqrt{\frac{m}{2\pi i (t''-t')}}e^{\frac{im}{2(t''-t')}(q''-q')^2}, \quad \text{where }t''-t'=(N+1)\delta t.
\end{align}
Then restore $\hbar$ by dimensional analysis, we have
\begin{align}
    \langle q'',t''| q',t'\rangle =&\sqrt{\frac{m}{2\pi i \hbar (t''-t')}}e^{\frac{im}{2\hbar (t''-t')}(q''-q')^2}.
\end{align}
\begin{itemize}
    \item [(c)] We can also compute $\langle q'',t''| q',t' \rangle$ by inserting a complete set of momentum eigenstates:
\end{itemize}
\begin{align}
    \langle q'',t''| q',t' \rangle =&\langle q''|e^{-iH(t''-t')}|q'\rangle\\
    =&\int dp \langle q''|e^{-iH(t''-t')}|p\rangle \langle p|q'\rangle\\
    =&\int \frac{dp}{2\pi} e^{-i\frac{p^2}{2m}(t''-t')}e^{ip(q''-q')},
\end{align}
where we have used $H=\frac{p^2}{2m}$ and $\langle p|q'\rangle=\frac{1}{\sqrt{2\pi}}e^{-ipq'}$. Now we can perform the integral over $p$:
\begin{align}
    \int_{-\infty}^{+\infty} dp e^{-i\frac{p^2}{2m}(t''-t')}e^{ip(q''-q')}=&\int_{-\infty}^{+\infty} dp e^{-i\frac{t''-t'}{2m}\Big(p^2 - \frac{2m}{t''-t'}(q''-q')p\Big)}\\
    =&\int_{-\infty}^{+\infty} dp e^{-i\frac{t''-t'}{2m}\Big(p - \frac{m}{t''-t'}(q''-q')\Big)^2 + i\frac{m}{2(t''-t')}(q''-q')^2}\\
    =&e^{i\frac{m}{2(t''-t')}(q''-q')^2}\int_{-\infty}^{+\infty} dp e^{-i\frac{t''-t'}{2m}\Big(p - \frac{m}{t''-t'}(q''-q')\Big)^2}\\
    =&e^{i\frac{m}{2(t''-t')}(q''-q')^2}\sqrt{\frac{2m\pi}{i(t''-t')}}.
\end{align}
Thus, we have
\begin{align}
    \langle q'',t''| q',t' \rangle =&\int \frac{dp}{2\pi} e^{-i\frac{p^2}{2m}(t''-t')}e^{ip(q''-q')}\\
    =&\frac{1}{2\pi}e^{i\frac{m}{2(t''-t')}(q''-q')^2}\sqrt{\frac{2m\pi}{i(t''-t')}}\\
    =&\sqrt{\frac{m}{2\pi i (t''-t')}}e^{\frac{im}{2(t''-t')}(q''-q')^2}.
\end{align}
This is exactly the same as the result we obtained in part (b). \qed
\clearpage
\question{3}{Problem 7.3}
\begin{itemize}
    \item [(a)]Use the Heisenberg equations of motion, $\dot{A}=i[H,A]$, to find explicit expressions for $\dot{Q}$ and $\dot{P}$. Solve these to get the Heisenberg-picture operators $Q(t)$ and $P(t)$ in terms of the Schr\"odinger-picture operators $Q$ and $P$.
    \item [(b)] Write the Schr\"odinger-picture operators $Q$ and $P$ in terms of the creation and annihilation operators $a$ and $a^\dagger$, where $H=\hbar\omega (a^\dagger a +\frac{1}{2})$. Then, using your result from part (a), write the Heisenberg-picture operator $Q(t)$ and $P(t)$ in terms of $a$ and $a^\dagger$.
    \item [(c)] Using your result from part (b), and $a|0\rangle=\langle0|a^\dagger=0$,verify eqs.~(7.16) and (7.17).
\end{itemize}
\answer{}
\begin{itemize}
    \item [(a)] First, we can compute $\dot{Q}$ and $\dot{P}$ using the Heisenberg equations of motion:
\end{itemize}
\begin{align}
    \dot{Q}=&i[H,Q]=i\Big[\frac{P^2}{2m}+\frac{1}{2}m\omega^2 Q^2,Q\Big]=i\frac{1}{2m}[P^2,Q]=\frac{P}{m},\\
    \dot{P}=&i[H,P]=i\Big[\frac{P^2}{2m}+\frac{1}{2}m\omega^2 Q^2,P\Big]=i\frac{1}{2}m\omega^2[Q^2,P]=-m\omega^2 Q.
\end{align}
These are the equations of motion for a harmonic oscillator. Now we can solve these equations to get $Q(t)$ and $P(t)$:
\begin{align}
    \ddot{Q}(t)=&\frac{\dot{P}}{m}=-\omega^2 Q(t),\\
    Q(t)=&Q\cos{\omega t}+\frac{P}{m\omega}\sin{\omega t},\\
    P(t)=&m\dot{Q}(t)=-m\omega Q\sin{\omega t}+P\cos{\omega t}.
\end{align}
Note that we have used the initial conditions $Q(0)=Q$ and $P(0)=P$ to determine the integration constants.
\begin{itemize}
    \item [(b)] Next, we can write the Schr\"odinger-picture operators $Q$ and $P$ in terms of the creation and annihilation operators $a$ and $a^\dagger$:
\end{itemize}
\begin{align}
    Q=&\sqrt{\frac{\hbar}{2m\omega}}(a+a^\dagger),\\
    P=&-i\sqrt{\frac{m\omega\hbar}{2}}(a-a^\dagger).
\end{align}
Then, using the result from part (a), we can write the Heisenberg-picture operators $Q(t)$ and $P(t)$ in terms of $a$ and $a^\dagger$:
\begin{align}
    Q(t)=& \sqrt{\frac{\hbar}{2m\omega}}(a+a^\dagger)\cos{\omega t} + \frac{-i\sqrt{\frac{m\omega\hbar}{2}}(a-a^\dagger)}{m\omega}\sin{\omega t}\\
    =& \sqrt{\frac{\hbar}{2m\omega}}(a+a^\dagger)\cos{\omega t} - i\sqrt{\frac{\hbar}{2m\omega}}(a-a^\dagger)\sin{\omega t}\\
    =& \sqrt{\frac{\hbar}{2m\omega}}\Big[a(\cos{\omega t}-i\sin{\omega t}) + a^\dagger(\cos{\omega t}+i\sin{\omega t})\Big]\\
    =& \sqrt{\frac{\hbar}{2m\omega}}\Big[ae^{-i\omega t} + a^\dagger e^{i\omega t}\Big],\\
    P(t)=& -m\omega \sqrt{\frac{\hbar}{2m\omega}}(a+a^\dagger)\sin{\omega t} - i\sqrt{\frac{m\omega\hbar}{2}}(a-a^\dagger)\cos{\omega t}\\
    =& -\sqrt{\frac{m\omega\hbar}{2}}(a+a^\dagger)\sin{\omega t} - i\sqrt{\frac{m\omega\hbar}{2}}(a-a^\dagger)\cos{\omega t}\\
    =& -\sqrt{\frac{m\omega\hbar}{2}}\Big[a(\sin{\omega t}+i\cos{\omega t}) + a^\dagger(\sin{\omega t}-i\cos{\omega t})\Big]\\
    =& -i \sqrt{\frac{m\omega\hbar}{2}}\Big[ a (\cos{\omega t}-i\sin{\omega t}) - a^\dagger(\cos{\omega t}+i\sin{\omega t})\Big]\\
    =& -i \sqrt{\frac{m\omega\hbar}{2}}\Big[ a e^{-i\omega t} - a^\dagger e^{i\omega t}\Big].
\end{align}
\begin{itemize}
    \item [(c)]Recall eqs.~(7.14),~(7.16) and (7.17):
\end{itemize}
\begin{align}
    G(t-t')=& \frac{i}{2\omega}\exp\Big(-i\omega |t-t'|\Big)\notag\\
    =& \frac{i}{2\omega} \Big(\theta(t-t')e^{-i\omega(t-t')}+\theta(t'-t)e^{-i\omega(t'-t)}\Big), \tag{7.14}\\
    \langle 0| \mathrm{T} Q\left(t_1\right) Q\left(t_2\right)|0\rangle & =\left.\frac{1}{i} \frac{\delta}{\delta f\left(t_1\right)} \frac{1}{i} \frac{\delta}{\delta f\left(t_2\right)}\langle 0 \mid 0\rangle_f\right|_{f=0}\notag \\
    & =\left.\frac{1}{i} \frac{\delta}{\delta f\left(t_1\right)}\left[\int_{-\infty}^{+\infty} d t^{\prime} G\left(t_2-t^{\prime}\right) f\left(t^{\prime}\right)\right]\langle 0 \mid 0\rangle_f\right|_{f=0}\notag \\
    & =\left.\left[\frac{1}{i} G\left(t_2-t_1\right)+\left(\text { term with } f^{\prime} \mathrm{s}\right)\right]\langle 0 \mid 0\rangle_f\right|_{f=0}\notag \\
    & =\frac{1}{i} G\left(t_2-t_1\right), \tag{7.16}\\    
    \langle 0| \mathrm{T} Q\left(t_1\right) Q\left(t_2\right) Q\left(t_3\right) Q\left(t_4\right)|0\rangle= & \frac{1}{i^2}\left[G\left(t_1-t_2\right) G\left(t_3-t_4\right)\right. \notag\\
    & +G\left(t_1-t_3\right) G\left(t_2-t_4\right) \notag\\
    & \left.+G\left(t_1-t_4\right) G\left(t_2-t_3\right)\right] .\tag{7.17}
\end{align}
Using the result from part (b), we can compute $\langle 0| \mathrm{T} Q\left(t_1\right) Q\left(t_2\right)|0\rangle$:
\begin{align}
    \langle 0| \mathrm{T} Q\left(t_1\right) Q\left(t_2\right)|0\rangle =& \frac{\hbar}{2m\omega}\langle 0| \mathrm{T} \Big[ a e^{-i\omega t_1} + a^\dagger e^{i\omega t_1} \Big] \Big[ a e^{-i\omega t_2} + a^\dagger e^{i\omega t_2} \Big] |0\rangle\\
    =& \frac{\hbar}{2m\omega}\langle 0| \mathrm{T} \Big[ aa e^{-i\omega (t_1+t_2)} + aa^\dagger e^{-i\omega t_1}e^{i\omega t_2} + a^\dagger a e^{i\omega t_1}e^{-i\omega t_2} + a^\dagger a^\dagger e^{i\omega (t_1+t_2)} \Big] |0\rangle\\
    =& \frac{\hbar}{2m\omega}\langle 0| \mathrm{T} \Big[ aa^\dagger e^{-i\omega t_1}e^{i\omega t_2}  \Big] |0\rangle\\
    =& \frac{\hbar}{2m\omega}\langle 0| \mathrm{T} \Big[ (1+a^\dagger a) e^{-i\omega t_1}e^{i\omega t_2}  \Big] |0\rangle\\
    =& \frac{\hbar}{2m\omega}\langle 0| \mathrm{T} e^{-i\omega t_1}e^{i\omega t_2} |0\rangle\\
    =& \frac{\hbar}{2m\omega}\Big[\theta(t_1-t_2)e^{-i\omega (t_1-t_2)} + \theta(t_2-t_1)e^{-i\omega (t_2-t_1)}\Big]\\
    =& \frac{1}{2\omega}\Big[\theta(t_1-t_2)e^{-i\omega (t_1-t_2)} + \theta(t_2-t_1)e^{-i\omega (t_2-t_1)}\Big] ,\quad \text{by setting } \hbar=m=1,\\
    =& \frac{1}{2\omega} e^{-i\omega |t_1-t_2|}\\
    =& \frac{1}{i} G(t_2-t_1)
\end{align}
where we have used $a|0\rangle=\langle0|a^\dagger=0$ and the definition of $G(t)$ in eq.~(7.14). This verifies eq.~(7.16). Next, we can compute $\langle 0| \mathrm{T} Q\left(t_1\right) Q\left(t_2\right) Q\left(t_3\right) Q\left(t_4\right)|0\rangle$:

\begin{align}
    &\langle 0| \mathrm{T} Q\left(t_1\right) Q\left(t_2\right) Q\left(t_3\right) Q\left(t_4\right)|0\rangle \notag\\
    =& \frac{\hbar^2}{4m^2\omega^2}\langle 0| \mathrm{T} \Big[ a e^{-i\omega t_1} + a^\dagger e^{i\omega t_1} \Big] \Big[ a e^{-i\omega t_2} + a^\dagger e^{i\omega t_2} \Big] \notag\\
    &\Big[ a e^{-i\omega t_3} + a^\dagger e^{i\omega t_3} \Big] \Big[ a e^{-i\omega t_4} + a^\dagger e^{i\omega t_4} \Big] |0\rangle\\
    =& \frac{\hbar^2}{4m^2\omega^2}\langle 0| \mathrm{T} \Big[ {\color{red}aaaa} e^{-i\omega (t_1+t_2+t_3+t_4)} + {\color{red}aa a a^\dagger} e^{-i\omega (t_1+t_2+t_3)}e^{i\omega t_4} + {\color{red}aa a^\dagger a} e^{-i\omega (t_1+t_2+t_4)}e^{i\omega t_3} \notag\\
    &+ {\color{blue}aa a^\dagger a^\dagger} e^{-i\omega (t_1+t_2)}e^{i\omega (t_3+t_4)} + {\color{red}aa^\dagger aa} e^{-i\omega (t_1+t_3+t_4)}e^{i\omega t_2} + {\color{blue}aa^\dagger aa^\dagger} e^{-i\omega (t_1+t_3)}e^{i\omega (t_2+t_4)} \notag\\
    &+ {\color{red}aa^\dagger a^\dagger a} e^{-i\omega (t_1+t_4)}e^{i\omega (t_2+t_3)} + {\color{red}aa^\dagger a^\dagger a^\dagger} e^{-i\omega t_1}e^{i\omega (t_2+t_3+t_4)} + {\color{red}a^\dagger aaa} e^{-i\omega (t_2+t_3+t_4)}e^{i\omega t_1} \notag\\
    &+ {\color{red}a^\dagger aa a^\dagger} e^{-i\omega (t_2+t_3)}e^{i\omega (t_1+t_4)} + {\color{red}a^\dagger aa^\dagger a} e^{-i\omega (t_2+t_4)}e^{i\omega (t_1+t_3)} + {\color{red}a^\dagger aa^\dagger a^\dagger} e^{-i\omega t_2}e^{i\omega (t_1+t_3+t_4)} \notag\\
    &+ {\color{red}a^\dagger a^\dagger aa} e^{-i\omega (t_3+t_4)}e^{i\omega (t_1+t_2)} + {\color{red}a^\dagger a^\dagger aa^\dagger} e^{-i\omega t_3}e^{i\omega (t_1+t_2+t_4)} + {\color{red}a^\dagger a^\dagger a^\dagger a} e^{-i\omega t_4}e^{i\omega (t_1+t_2+t_3)} \notag\\
    &+ {\color{red}a^\dagger a^\dagger a^\dagger a^\dagger} e^{i\omega (t_1+t_2+t_3+t_4)} \Big] |0\rangle,\quad\text{ red terms vanish but blue terms can survive,}\\
    =& \frac{\hbar^2}{4m^2\omega^2}\langle 0| \mathrm{T} \Big[ {\color{blue}aa a^\dagger a^\dagger} e^{-i\omega (t_1+t_2)}e^{i\omega (t_3+t_4)} + {\color{blue}aa^\dagger aa^\dagger} e^{-i\omega (t_1+t_3)}e^{i\omega (t_2+t_4)} \Big] |0\rangle
\end{align}
\begin{align}
    &\langle 0|aa a^\dagger a^\dagger |0\rangle=\langle 0|a(1+a^\dagger a)a^\dagger |0\rangle=\langle 0|aa^\dagger |0\rangle +\langle 0|aa^\dagger aa^\dagger |0\rangle \\
    =&\langle 0|(1+a^\dagger a) |0\rangle+\langle 0|(1+a^\dagger a) (1+a^\dagger a)|0\rangle=2\\
    &\langle 0|aa^\dagger aa^\dagger |0\rangle=\langle 0|(1+a^\dagger a) (1+a^\dagger a)|0\rangle=1,
\end{align}
Hence, we have
\begin{align}
    &\langle 0| \mathrm{T} Q\left(t_1\right) Q\left(t_2\right) Q\left(t_3\right) Q\left(t_4\right)|0\rangle \notag\\
    =& \frac{\hbar^2}{4m^2\omega^2}\langle 0| \mathrm{T} \Big[ 2 e^{-i\omega (t_1+t_2)}e^{i\omega (t_3+t_4)} +  e^{-i\omega (t_1+t_3)}e^{i\omega (t_2+t_4)}  \Big] |0\rangle\\
    =& \frac{\hbar^2}{4m^2\omega^2}\Big[ \langle 0| \mathrm{T} e^{-i\omega (t_1+t_2)}e^{i\omega (t_3+t_4)} |0\rangle+ \langle 0| \mathrm{T} e^{-i\omega (t_1+t_2)}e^{i\omega (t_3+t_4)} |0\rangle + \langle 0| \mathrm{T} e^{-i\omega (t_1+t_3)}e^{i\omega (t_2+t_4)} |0\rangle  \Big]\\
    =& \frac{\hbar^2}{4m^2\omega^2}\Big[ \langle 0| \mathrm{T} e^{-i\omega (t_1-t_3)}e^{-i\omega (t_2-t_4)} |0\rangle+ \langle 0| \mathrm{T} e^{-i\omega (t_1-t_4)}e^{-i\omega (t_2-t_3)} |0\rangle + \langle 0| \mathrm{T} e^{-i\omega (t_1-t_2)}e^{-i\omega (t_3-t_4)} |0\rangle  \Big]\\
    =& \frac{1}{4\omega^2}\Big[ \Big(\theta(t_1-t_3)e^{-i\omega (t_1-t_3)} + \theta(t_3-t_1)e^{-i\omega (t_3-t_1)}\Big)\Big(\theta(t_2-t_4)e^{-i\omega (t_2-t_4)} + \theta(t_4-t_2)e^{-i\omega (t_4-t_2)}\Big) \notag\\
    & + \Big(\theta(t_1-t_4)e^{-i\omega (t_1-t_4)} + \theta(t_4-t_1)e^{-i\omega (t_4-t_1)}\Big) \Big(\theta(t_2-t_3)e^{-i\omega (t_2-t_3)} + \theta(t_3-t_2)e^{-i\omega (t_3-t_2)}\Big) \notag\\
    & + \Big(\theta(t_1-t_2)e^{-i\omega (t_1-t_2)} + \theta(t_2-t_1)e^{-i\omega (t_2-t_1)}\Big)\Big(\theta(t_3-t_4)e^{-i\omega (t_3-t_4)} + \theta(t_4-t_3)e^{-i\omega (t_4-t_3)}\Big)\Big],\\
    &\quad \text{by setting } \hbar=m=1,\notag\\
    =& \frac{1}{i^2}\Big[ G(t_1-t_3) G(t_2-t_4) + G(t_1-t_4) G(t_2-t_3) + G(t_1-t_2) G(t_3-t_4) \Big]
\end{align}
where we have used the definition of $G(t)$ in eq.~(7.14). This verifies eq.~(7.17). \qed

\clearpage
\question{4}{Problem 7.4}\\
Consider a harmonic oscillator in its ground state at $t=-\infty$. It is then subjected to an external force $f(t)$. Compute the probability $|\langle0|0\rangle_f|^2$ that the oscillator is still in its ground state at $t=+\infty$. Write your answer as a manifestly real expression, and in terms of the Fourier transform $\tilde{f}(E)=\int^{+\infty}_{-\infty}e^{iEt}f(t)$. Your answer should not involve any other unevaluated integrals.
\answer{}
Recall eqs.~(7.10),~(7.11)and~(7.14):
\begin{align}
    \langle 0|0\rangle_f=& \exp{\Big[\frac{i}{2} \int_{-\infty}^{\infty} \frac{dE}{2\pi}\frac{\widetilde{f}(E)\widetilde{f}(-E)}{-E^2+\omega^2-i\epsilon}\Big]}, \tag{7.10}\\
    =&\exp\Big[\frac{i}{2}\int_{-\infty}^{+\infty} dt  dt' f(t) G(t-t') f(t')\Big] \tag{7.11}\\
    =&\exp\Big[\frac{i}{2}\int_{-\infty}^{+\infty} dt  dt' f(t) \frac{i}{2\omega}e^{-i\omega|t-t'|} f(t')\Big], \quad \text{by eq.~(7.14)}\notag
\end{align}
\begin{align}
    |\langle0|0\rangle_f|^2=& \langle 0|0\rangle_f \langle 0|0\rangle_f^*\\
    =&\exp\Big[\frac{i}{2}\int_{-\infty}^{+\infty} dt  dt' f(t) \frac{i}{2\omega}e^{-i\omega|t-t'|} f(t')\Big] \exp\Big[-\frac{i}{2}\int_{-\infty}^{+\infty} dt  dt' f(t) \frac{-i}{2\omega}e^{i\omega|t-t'|} f(t')\Big]\\
    =&\exp\Big[-\frac{1}{2}\int_{-\infty}^{+\infty} dt  dt' f(t) \frac{1}{2\omega}e^{-i\omega|t-t'|} f(t') -\frac{1}{2}\int_{-\infty}^{+\infty} dt  dt' f(t) \frac{1}{2\omega}e^{i\omega|t-t'|} f(t')\Big]\\
    =&\exp\Big[-\frac{1}{2}\int_{-\infty}^{+\infty} dt  dt' f(t) \frac{1}{2\omega}\Big(e^{-i\omega|t-t'|}+e^{i\omega|t-t'|}\Big) f(t')\Big]\\
    =&\exp\Big[-\frac{1}{2}\int_{-\infty}^{+\infty} dt  dt' f(t) \frac{1}{\omega}\cos{\omega|t-t'|} f(t')\Big]\\
    =&\exp\Big[-\frac{1}{2}\int_{-\infty}^{+\infty} dt  dt' f(t) \frac{1}{2\omega}(e^{-i\omega(t-t')}+e^{i\omega(t-t')}) f(t')\Big]
\end{align}
By the definition of Fourier transform, we have
\begin{align}
    \tilde{f}(E)=&\int_{-\infty}^{+\infty} e^{iE t} f(t) dt.
\end{align}
Thus, we have
\begin{align}
    &\int_{-\infty}^{+\infty} dt  dt' f(t) \frac{1}{2\omega}(e^{-i\omega(t-t')}+e^{i\omega(t-t')}) f(t')\\
    =&\int_{-\infty}^{+\infty} dt  dt' f(t) \frac{1}{2\omega}e^{-i\omega(t-t')} f(t') + \int_{-\infty}^{+\infty} dt  dt' f(t) \frac{1}{2\omega}e^{i\omega(t-t')} f(t')\\
    =&\frac{1}{2\omega}\int_{-\infty}^{+\infty} dt  dt' f(t) e^{-i\omega t} e^{i\omega t'} f(t') + \frac{1}{2\omega}\int_{-\infty}^{+\infty} dt  dt' f(t) e^{i\omega t} e^{-i\omega t'} f(t')\\
    =&\frac{1}{2\omega}\Big(\int_{-\infty}^{+\infty} dt f(t) e^{-i\omega t}\Big)\Big(\int_{-\infty}^{+\infty} dt' f(t') e^{i\omega t'}\Big) + \frac{1}{2\omega}\Big(\int_{-\infty}^{+\infty} dt f(t) e^{i\omega t}\Big)\Big(\int_{-\infty}^{+\infty} dt' f(t') e^{-i\omega t'}\Big)\\
    =&\frac{1}{2\omega}\tilde{f}(-\omega)\tilde{f}(\omega) + \frac{1}{2\omega}\tilde{f}(\omega)\tilde{f}(-\omega)\\
    =&\frac{1}{\omega}\tilde{f}(\omega)\tilde{f}(-\omega) 
\end{align}
Therefore, we have
\begin{align}
    |\langle0|0\rangle_f|^2=&\exp\Big[-\frac{1}{2}\int_{-\infty}^{+\infty} dt  dt' f(t) \frac{1}{2\omega}(e^{-i\omega(t-t')}+e^{i\omega(t-t')}) f(t')\Big]\\
    =&\exp\Big[-\frac{1}{2}\cdot \frac{1}{\omega}\tilde{f}(\omega)\tilde{f}(-\omega)\Big]\\
    =&\exp\Big[-\frac{1}{2\omega}|\tilde{f}(\omega)|^2\Big],
\end{align}
where we have used the fact that $\tilde{f}(-\omega)$ is the complex conjugate of $\tilde{f}(\omega)$. This is a manifestly real expression and does not involve any other unevaluated integrals. \qed
%\section*{Assignment 3 due on Monday September 22th at 5PM}
\question{1}{}
Show explicitly that the $4$-vector current density for a collection of point charges satisfies $\partial_\mu J^\mu=0$
\answer{}
In the class, we defined the $4$-vector current density for a collection of point charges as
\begin{align}
    J^0(t,\mathbf{x})&=\rho(t,\mathbf{x})=\sum_a q_a \delta^{(3)}(\mathbf{x}-\mathbf{x}_a(t))\\
    \mathbf{J}(t,\mathbf{x})&=\sum_a q_a \mathbf{v}_a(t)\delta^{(3)}(\mathbf{x}-\mathbf{x}_a(t)), \quad \mathbf{v}_a(t)=\frac{d\mathbf{x}_a(t)}{dt}.
\end{align}
Then we have 
\begin{align}
    \partial_\mu J^\mu&=\frac{\partial \rho}{\partial t}+\nabla\cdot \mathbf{J}\\
    &=\sum_a q_a \left[\frac{\partial}{\partial t}\delta^{(3)}(\mathbf{x}-\mathbf{x}_a(t))+\nabla\cdot(\mathbf{v}_a(t)\delta^{(3)}(\mathbf{x}-\mathbf{x}_a(t)))\right]\\
    &=\sum_a q_a \left[-\mathbf{v}_a(t)\cdot\nabla \delta^{(3)}(\mathbf{x}-\mathbf{x}_a(t))+\nabla\cdot(\mathbf{v}_a(t)\delta^{(3)}(\mathbf{x}-\mathbf{x}_a(t)))\right]\\
    &=\sum_a q_a \left[-\mathbf{v}_a(t)\cdot\nabla \delta^{(3)}(\mathbf{x}-\mathbf{x}_a(t))+\mathbf{v}_a(t)\cdot\nabla \delta^{(3)}(\mathbf{x}-\mathbf{x}_a(t))\right]\\
    &=0.
\end{align}
\clearpage
\question{2}{}
Prove that the electromagnetic energy density squared minus the square of the Poynting vector is a Lorentz invariant for an electromagnetic field by expressing this quantity in terms of tensors. You might consider using the dual field strength tensor defined by $\widetilde{F}^{\mu\nu}=\frac{1}{2}\varepsilon^{\mu\nu\alpha\beta}F_{\alpha\beta}$. 
\answer{}
In the EM class, we defined the electromagnetic energy density and the Poynting vector as
\begin{align}
    u&=\frac{1}{2}(\mathbf{E}^2+\mathbf{B}^2)\\
    \mathbf{S}&=\mathbf{E}\times \mathbf{B}.
\end{align}
Also, we have the following relations
\begin{align}
    F^{\mu\nu}=\begin{pmatrix}
    0 & E_x & E_y & E_z \\
    -E_x & 0 & B_z & -B_y \\
    -E_y & -B_z & 0 & B_x \\
    -E_z & B_y & -B_x & 0
    \end{pmatrix}
\end{align}
Besides, the EM field energy momentum strength tensor is given by (from Weinberg's GR book)
\begin{align}
    T_{EM}^{\mu\nu}&=F^{\mu\alpha}{F^\nu}_\alpha-\frac{1}{4}\eta^{\mu\nu}F_{\alpha\beta}F^{\alpha\beta},\\
    u = T_{EM}^{00}&=\frac{1}{2}(\mathbf{E}^2+\mathbf{B}^2), \quad S^i = T_{EM}^{i0} = (\mathbf{E}\times \mathbf{B})^i,
\end{align}
where $u$ is the EM energy density and $\mathbf{S}$ is the Poynting vector. Hence, we can define a $4$-vector $U^\mu$ as
\begin{align}
    U^\mu = (u, \mathbf{S}) = (T_{EM}^{00}, T_{EM}^{i0})=T^{\mu0}_{EM}.
\end{align}
Then we have
\begin{align}
    T^{00}_{EM}&=F^{0\alpha}{F^0}_\alpha-\frac{1}{4}\eta^{00}F_{\alpha\beta}F^{\alpha\beta}=F^{0i}{F^0}_i+\frac{1}{4}F_{\alpha\beta}F^{\alpha\beta}\\
    &=\mathbf{E}^2-\frac{1}{2}(\mathbf{E}^2-\mathbf{B}^2)\\
    &=\frac{1}{2}(\mathbf{E}^2+\mathbf{B}^2)=u\\
\end{align}
and
\begin{align}
    T^{i0}_{EM}&=F^{i\alpha}{F^0}_\alpha-\frac{1}{4}\eta^{i0}F_{\alpha\beta}F^{\alpha\beta}=F^{ij}{F^0}_j\\
    &=\epsilon_{ijk}B_k E_j\\
    &=(\mathbf{E}\times \mathbf{B})^i=S^i.
\end{align}
Therefore, we have 
\begin{align}
    U^\mu U_\mu&=-\eta_{\mu\nu}U^\mu U^\nu\\
    &=-\eta_{\mu\nu}T^{\mu0}_{EM}T^{\nu0}_{EM}\\
    &=T^{00}_{EM}T^{00}_{EM}-T^{i0}_{EM}T^{i0}_{EM}=u^2-\mathbf{S}^2.
\end{align}
We can claim that $U^\mu U_\mu$ is a Lorentz invariant since it is the contraction of two tensors. Hence, we conclude that $u^2-\mathbf{S}^2$ is a Lorentz invariant.

\textbf{Remark:} We can also prove this by using the dual field strength tensor $\widetilde{F}^{\mu\nu}=\frac{1}{2}\varepsilon^{\mu\nu\alpha\beta}F_{\alpha\beta}$. First, we have
\begin{align}
    \widetilde{F}^{0i}&=\frac{1}{2}\varepsilon^{0ijk}F_{jk}=\frac{1}{2}\epsilon^{ijk}F_{jk}=\frac{1}{2}\epsilon^{ijk}\epsilon_{jkl}B_l=B^i,\\
    \widetilde{F}^{ij}&=\frac{1}{2}\varepsilon^{ij0k}F_{0k}=\frac{1}{2}(\varepsilon^{ij0k}-\varepsilon^{ji0k})F_{0k}=\varepsilon^{ij0k}E_k=\epsilon^{ijk}E_k.
\end{align}
Then we can calculate the following two Lorentz invariants:
\begin{align}
    F_{\mu\nu}F^{\mu\nu}&=2(\mathbf{B}^2-\mathbf{E}^2),\\
    \widetilde{F}_{\mu\nu}F^{\mu\nu}&=-4\mathbf{E}\cdot \mathbf{B}.
\end{align}
Now we can calculate
\begin{align}
    (F_{\mu\nu}F^{\mu\nu})^2+(\widetilde{F}_{\mu\nu}F^{\mu\nu})^2&=4(\mathbf{B}^2-\mathbf{E}^2)^2+16(\mathbf{E}\cdot \mathbf{B})^2\\
    &=4[(\mathbf{B}^2+\mathbf{E}^2)^2-4\mathbf{E}^2\mathbf{B}^2+4(\mathbf{E}\cdot \mathbf{B})^2]\\
    &=4[(\mathbf{B}^2+\mathbf{E}^2)^2-4(\mathbf{E}\times \mathbf{B})^2]\\
    &=4(u^2-\mathbf{S}^2).
\end{align}
This quantity is Lorentz invariant since all indices are contracted. Hence, we conclude that $u^2-\mathbf{S}^2$ is a Lorentz invariant.
\clearpage
\question{3}{}
Calculate the scalar ${T^\alpha}_\alpha$ associated with the electromagnetic stress tensor.
\answer{}
Consider the energy momentum tensor with the EM field:
\begin{align}
    T^{\alpha\beta}_{total}&=T^{\alpha\beta}+T^{\alpha\beta}_{EM}= \sum_n p_n^\alpha(t) \frac{dx_n^\beta}{dt}\delta^{(3)}(\mathbf{x}-\mathbf{x}_n(t))+F^{\alpha\mu}{F^\beta}_\mu-\frac{1}{4}\eta^{\alpha\beta}F_{\mu\nu}F^{\mu\nu}\\
    &=\sum_n\frac{p_n^\alpha p_n^\beta}{E_n}\delta^3(\mathbf{x}-\mathbf{x}_n(t)) +F^{\alpha\mu}{F^\beta}_\mu-\frac{1}{4}\eta^{\alpha\beta}F_{\mu\nu}F^{\mu\nu},
\end{align}
We have
\begin{align}
    {T^\alpha}_\alpha &= \eta_{\alpha\beta}T^{\alpha\beta}_{total}\\
    &=\sum_n \frac{p_n^\alpha p_{n\alpha}}{E_n}\delta^3(\mathbf{x}-\mathbf{x}_n(t)) +F^{\alpha\mu}{F_{\alpha\mu}}-\frac{1}{4}\eta_{\alpha\beta}\eta^{\alpha\beta} F_{\mu\nu}F^{\mu\nu}\\
    &=\sum_n \frac{m_n^2}{E_n}\delta^3(\mathbf{x}-\mathbf{x}_n(t)) +0,\quad \text{by } \eta^{\alpha\beta}\eta_{\alpha\beta}=4\\
    &=\sum_n \frac{m_n^2}{E_n}\delta^3(\mathbf{x}-\mathbf{x}_n(t)) \\
    &=\sum_n \frac{m_n^2}{E_n}\delta^3(\mathbf{x}-\mathbf{x}_n(t)) .
\end{align}

\textbf{Remark:} In class, we have shown that \begin{align}
    \frac{\delta^3( \mathbf{x}-\mathbf{x}_n(t))}{E_n}
\end{align}
is a Lorentz invariant. Hence, we can conclude that ${T^\alpha}_\alpha$ is a Lorentz invariant since $m_n$ is also a Lorentz invariant. In other words, ${T^\alpha}_\alpha$ is a Lorentz scalar.

\textbf{Remark 2 (after deadline):} Actually, I don't know whetehr should I only consider the EM part or the total energy momentum tensor. If I only consider the EM part, then we have $0$. \textit{Seth} and I discussed this question and we think that the question might be ambiguous. So I just write both of them here.
%\question{1}{\textbf{Gell-Mann Okubo for the baryon octet}}
\\
The generators \(t_{a}(a=1, \ldots, 8)\) of \(\mathrm{SU}(3)\) are normalized as \(t_{a}=\frac{\lambda_{a}}{2}\) with \(\operatorname{Tr}\left(t_{a} t_{b}\right)=\frac{1}{2} \delta_{a b}\), where \(\lambda_{a}\) are the Gell-Mann matrices. They satisfy \(\left[t_{a}, t_{b}\right]=i f_{a b c} t_{c}\) and \(\left\{t_{a}, t_{b}\right\}= \frac{1}{3} \delta_{a b} \mathbf{1}+d_{a b c} t_{c}\), where \(f_{a b c}\) are totally antisymmetric and \(d_{a b c}\) are totally symmetric structure constants.


Let \(B\) and \(\bar{B}\) be the baryon octet \(3 \times 3\) traceless matrices, expanded in the generator basis as \(B=B^{i} t_{i}\) and \(\bar{B}=\bar{B}^{i} t_{i}\) where \(B^{i}, \bar{B}^{i}\) are the adjoint components. Define the two bilinear combinations \(O_{A} \equiv[\bar{B}, B]=\bar{B} B-B \bar{B}\) and \(O_{S} \equiv\{\bar{B}, B\}-\frac{2}{3} \mathbf{1} \operatorname{Tr}(\bar{B} B)\).
\begin{itemize}
    \item [(a)]Show that both \(O_{A}\) and \(O_{S}\) are traceless and therefore transform in the adjoint (octet) representation.
    \item [(b)] Expand \(O_{A}\) and \(O_{S}\) in components using the generator basis and show that \(O_{A}=i\left(\bar{B}^{i} B^{j}\right) f_{i j k} t_{k}\) and \(O_{S}=\left(\bar{B}^{i} B^{j}\right) d_{i j k} t_{k}\), so that \(\left(O_{A}\right)^{k}=i f_{i j k} \bar{B}^{i} B^{j}\) and \(\left(O_{S}\right)^{k}= d_{i j k} \bar{B}^{i} B^{j}\).
    \item [(c)]Introduce a flavor-breaking spurion \(H_{8}=H_{8}^{i} t_{i}\), with real components \(H_{8}^{i}\). Construct the two independent \(\mathrm{SU}(3)\)-invariant mass terms:
    \begin{align}
        S_{f}=\left(O_{A}\right)_{b}^{a}\left(H_{8}\right)_{a}^{b}, \quad S_{d}=\left(O_{S}\right)_{b}^{a}\left(H_{8}\right)_{a}^{b}
    \end{align}
    Assuming \(H_{8}\) points in the 8 -direction (i.e. \(H_{8}^{i} \propto \delta_{i 8}\) ), argue that \(S_{f}\) and \(S_{d}\) correspond to the \(f\)-type and \(d\)-type symmetry breaking terms in the baryon mass operator, respectively.
    \item [(d)] Given that an adjoint operator \(O_{8}\) acts on an octet state \(B\) as \(O_{8}(B)=\left[O_{8}, B\right]\), show that the invariant scalars in (c) are equivalent to the matrix elements \(S_{f} \propto \langle\bar{B}| t_{8}|B\rangle \equiv \operatorname{Tr}\left(\bar{B}\left[t_{8}, B\right]\right)\) and \(S_{d} \propto\langle\bar{B}| d_{8 i j} t_{i} t_{j}|B\rangle \equiv \operatorname{Tr}\left(\bar{B}\left[d_{8 i j}\left[t_{i},\left[t_{j}, B\right]\right]\right)\right.\).
     \item [(e)] Hence, argue that for each entry \(B_{i j}\) of the baryon octet matrix, \(S_{f} \propto Y\) and \(S_{d} \propto I(I+1)-Y^{2} / 4\) where \(I, Y\) are the isospin and hypercharge of the baryon \(B\), respectively, thereby reproducing the Gell-Mann-Okubo mass formula for the baryon octet.
\end{itemize}
(Hint: Verify, entrywise, that \(\left[\frac{2}{\sqrt{3}} t_{8}, B\right]=Y B\) and the normalized operator \(\frac{2}{\sqrt{3}} d_{8 i j}\left[t_{i},\left[t_{j}, B\right]\right]+ \frac{1}{3}\left[t_{i},\left[t_{i}, B\right]\right]=\left(I(I+1)-Y^{2} / 4\right) B\) acts diagonally on each baryon field. The \(\left[t_{i},\left[t_{i}, B\right]\right]\) term is the adjoint Casimir ( \(S U(3)\) singlet) which just shifts all octet components uniformly so that the \(\Lambda\) eigenvalue becomes 0 . It can be absorbed into the overall singlet part of the GMO formula. On the diagonal remember \(B_{11}, B_{22}\) and \(B_{33}\) mix \(\Sigma^{0}\) and \(\Lambda\), so \(\left.B_{\text {diag }}=\Sigma^{0} \operatorname{diag}(1,-1,0) / \sqrt{2}+\Lambda \operatorname{diag}(1,1,-2) / \sqrt{6}\right)\).
\answer{}
\begin{itemize}
    \item [(a)]
\end{itemize}
To show that both \(O_{A}\) and \(O_{S}\) are traceless, we first understand their definitions:
\begin{align}
O_{A} &= [\bar{B}, B] = \bar{B} B - B \bar{B}=[\bar{B}, B]\\
      &= [\bar{B}^{i} t_{i}, B^{j} t_{j}] = \bar{B}^{i} B^{j} [t_{i}, t_{j}] = i \bar{B}^{i} B^{j} f_{ijk} t_{k}
\end{align}
Taking the trace of \(O_{A}\):
\begin{align}
\operatorname{Tr}(O_{A}) &= \operatorname{Tr}(iBar{B}^{i} B^{j} f_{ijk} t_{k}) = i \bar{B}^{i} B^{j} f_{ijk} \operatorname{Tr}(t_{k}) = 0.
\end{align}
Similarly, for \(O_{S}\):
\begin{align}
O_{S} &= \{\bar{B}, B\} - \frac{2}{3} \mathbf{1} \operatorname{Tr}(\bar{B} B) = \bar{B} B + B \bar{B} - \frac{2}{3} \mathbf{1} \operatorname{Tr}(\bar{B} B)\\
      &= (\bar{B}^{i} t_{i})(B^{j} t_{j}) + (B^{j} t_{j})(\bar{B}^{i} t_{i}) - \frac{2}{3} \mathbf{1} \operatorname{Tr}(\bar{B}^{i} t_{i} B^{j} t_{j})\\
      &= \bar{B}^{i} B^{j}\{t_{i}, t_{j}\} - \frac{2}{3} \mathbf{1} \operatorname{Tr}(\bar{B}^{i} B^{j} t_{i} t_{j})\\
      &= \bar{B}^{i} B^{j}\left(\frac{1}{3}\delta_{ij}\mathbf{1}+d_{ijk}t_{k}\right) - \frac{2}{3}\mathbf{1}\left(\frac{1}{2}\bar{B}^{i} B^{j}\delta_{ij}\right)\\
      &= \bar{B}^{i} B^{j} d_{ijk} t_{k}
\end{align}


\clearpage
\question{2}{\textbf{$\rho$-$\omega$ mixing}}
\\
The vector mesons $\rho(770)$ and $\omega(782)$ are very close in mass. For this reason the effects of isospin violation are somewhat enhanced in these mesons and can be parametrized in terms of $\rho$-$\omega$ mixing. Namely, the physical $\rho^{0}$ and $\omega$ mesons can be viewed as orthogonal mixed states of a pure isospin triplet and isospin singlet:
\[
\rho^{0} = \cos\theta \frac{(u\bar{u} - d\bar{d})}{\sqrt{2}} + \sin\theta \frac{(u\bar{u} + d\bar{d})}{\sqrt{2}},
\]
\[
\omega = -\sin\theta \frac{(u\bar{u} - d\bar{d})}{\sqrt{2}} + \cos\theta \frac{(u\bar{u} + d\bar{d})}{\sqrt{2}},
\]
where $\theta$ is a (small) mixing angle.

\begin{enumerate}
\item[(a)] Determine $\theta$ (up to a sign) using experimental data on the decay $\omega \rightarrow \pi^{+} \pi^{-}$. Estimate the error in the value of the mixing angle.
\item[(b)] Using the value of $\theta$ predict the decay rates $\Gamma(\rho^{0} \rightarrow e^{+} e^{-})$ and $\Gamma(\omega \rightarrow e^{+} e^{-})$, assuming the amplitude for a quark pair annihilation into an $e^{+} e^{-}$ pair is proportional to the electric charge $Q$ of the quark.
\item[(c)] Assume that the transition amplitude between different spin states of a $q\bar{q}$ quark pair with emission of a photon: $(q\bar{q}) \rightarrow (q\bar{q}) + \gamma$ is proportional to the quark electric charge $Q$. Use the value of the $\rho$-$\omega$ mixing angle $\theta$ to determine the ratios of the decay rates:
\begin{enumerate}
\item[(i)] $\Gamma(\rho^{0} \rightarrow \pi^{0} \gamma) / \Gamma(\omega^{0} \rightarrow \pi^{0} \gamma)$,
\item[(ii)] $\Gamma(\rho^{0} \rightarrow \eta \gamma) / \Gamma(\omega^{0} \rightarrow \eta \gamma)$.
\end{enumerate}
Compare with the PDG experimental data. How does the inclusion of $\rho$-$\omega$ mixing improve the agreement with the data?
\end{enumerate}

\answer{}
\clearpage

\question{3}{\textbf{Baryon magnetic moments}}
\\
The octet of spin-$\frac{1}{2}$ baryons has magnetic moments $\mu$. The operator that describes the magnetic moment is an $\mathrm{SU}(3)_{f}$ octet operator which is proportional to the quark charge $Q$. The charge
\[
Q = t_{3} + \frac{1}{\sqrt{3}} t_{8}
\]
is traceless ($\operatorname{Tr} Q = 0$) and can be promoted to a purely $\mathrm{SU}(3)_{f}$ octet spurion $\mathbf{8}_{Q}$ (with no singlet piece, as in contrast to the GMO mass formula). Hence, when determining the baryon magnetic moment
\[
\mu(B) = \langle \bar{B} | \mu | B \rangle \propto \mathbf{8}_{\bar{B}} \times \mathbf{8}_{Q} \times \mathbf{8}_{B}
\]
there are two independent octet structures (the $f$- and $d$-type couplings, as for the baryon mass), given by
\[
\mu(B) = c_{f} \operatorname{Tr}(B^{\dagger} [Q, B]) + c_{d} \operatorname{Tr}(B^{\dagger} \{Q, B\}) = \alpha_{+} \operatorname{Tr}(B B^{\dagger} Q) + \alpha_{-} \operatorname{Tr}(B^{\dagger} B Q),
\]
where $\alpha_{+} \equiv c_{d} + c_{f}$, $\alpha_{-} \equiv c_{d} - c_{f}$ are arbitrary constants and
\[
B = \begin{pmatrix}
\frac{\Sigma^{0}_{u}}{\sqrt{2}} + \frac{\Lambda}{\sqrt{6}} & \Sigma^{+} & p \\
\Sigma^{-} & -\frac{\Sigma^{0}_{u}}{\sqrt{2}} + \frac{\Lambda}{\sqrt{6}} & n \\
-\Xi^{-} & \Xi^{0} & -\frac{2\Lambda}{\sqrt{6}}
\end{pmatrix}, \quad
Q = \begin{pmatrix}
\frac{2}{3} & 0 & 0 \\
0 & -\frac{1}{3} & 0 \\
0 & 0 & -\frac{1}{3}
\end{pmatrix}.
\]

Determine all the spin-$\frac{1}{2}$ baryon magnetic moments in terms of $\mu(p)$ and $\mu(n)$ (by eliminating $c_{f,d}$ or $\alpha_{\pm}$) and compare with the PDG experimental values. These predictions were first worked out by Coleman and Glashow in 1961. Note that imposing the full $\mathrm{SU}(6)$ spin-flavor symmetry further predicts $\mu(p) / \mu(n) = -\frac{3}{2}$, which you can ignore in this problem.

\answer{}
\section*{Assignment 5 due on Monday October 6th at 5PM}
\question{1}{}
In lecture we studied the time dilation of a slowly moving object in a weak stationary gravitational field. We found that the frequency difference of identical clocks located at points $\mathbf{x}_1$ and $\mathbf{x}_2$ and at rest in the gravitational field is


\begin{align}
\frac{\nu_2-\nu_1}{\nu_0}=\frac{\Delta \nu}{\nu_0}=\phi_2-\phi_1,    
\end{align}
where $\phi$ is the gravitational potential. Generalize this formula to the case when they are moving with velocities $\mathbf{v}_1$ and $\mathbf{v}_2$ which are small compared to the speed of light. This results in contributions from both gravity and special relativity.



\clearpage
\question{2}{}
Prove that $V_{\mu ; \nu} \equiv \partial V_\mu / \partial x^\nu-\Gamma_{\mu \nu}^\lambda V_\lambda$ transforms as a second rank tensor, similar to how we proved it for $V^\mu{ }_{; \nu}$ in class, assuming that $V$ transforms as a 4 -vector.


\clearpage
\question{3}{}
Show that
\begin{align}
A_{\mu \nu ; \lambda}+A_{\lambda \mu ; \nu}+A_{\nu \lambda ; \mu}=A_{\mu \nu, \lambda}+A_{\lambda \mu, \nu}+A_{\nu \lambda, \mu},    
\end{align}
when $A_{\mu \nu}$ is an anti-symmetric tensor.
\end{document}