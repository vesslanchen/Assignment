\section*{Assignment 12 due on Monday December 1 at 10PM}

\question{1}{}
Verify the Reissner-Nordstrom solution as presented in class and in the textbook. That is, show in detail that both Maxwell’s equations
\begin{align}
    \frac{\partial}{\partial x^\nu} \left( \sqrt{g} F^{\mu \nu} \right) = 0,\\
    F_{\mu\nu,\lambda} + F_{\nu\lambda,\mu} + F_{\lambda\mu,\nu} = 0,
\end{align}
and Einstein’s equations
\begin{align}
    R_{\mu\nu} - \frac{1}{2} R g_{\mu\nu} = -8 \pi G T_{\mu\nu},
\end{align}
are satisfied with the metric:
\begin{align}
    d\tau^2 = \left( 1 - \frac{2GM}{r} + \frac{GQ^2}{r^2} \right) dt^2 - \left( 1 - \frac{2GM}{r} + \frac{GQ^2}{r^2} \right)^{-1} dr^2 - r^2d\Omega^2,
\end{align}
and with a radial electric field $E_r=\frac{Q}{r^2}$. Note that this is in Gaussian
units. Weinberg uses Heaviside-Lorentz units.
\answer{}
First, we write down the field strength tensor $F_{\mu\nu}$ for a radial electric field $E_r=\frac{Q}{r^2}$:
\begin{align}
    F_{\mu\nu} = \begin{pmatrix}
    0 & -\frac{Q}{r^2} & 0 & 0 \\
    \frac{Q}{r^2} & 0 & 0 & 0 \\
    0 & 0 & 0 & 0 \\
    0 & 0 & 0 & 0
    \end{pmatrix}.
\end{align}
with $g^{tt}=\left(1-\frac{2GM}{r}+\frac{GQ^2}{r^2}\right)^{-1}$ and $g^{rr}=-\left(1-\frac{2GM}{r}+\frac{GQ^2}{r^2}\right)$.
Raising the indices, we have
\begin{align}
    F^{\mu\nu} = g^{\mu\alpha} g^{\nu\beta} F_{\alpha\beta} = \begin{pmatrix}
    0 & \frac{Q}{r^2} & 0 & 0 \\
    -\frac{Q}{r^2}& 0 & 0 & 0 \\
    0 & 0 & 0 & 0 \\
    0 & 0 & 0 & 0
    \end{pmatrix}.
\end{align}
Next, we calculate $\sqrt{g}$:
\begin{align}
    \sqrt{g} = r^2 \sin\theta.
\end{align}
Now, we can verify Maxwell's equations. For $\mu=0$,
\begin{align}
    \frac{\partial}{\partial x^\nu} \left( \sqrt{g} F^{0 \nu} \right) = \frac{\partial}{\partial r} \left( r^2 \sin\theta \cdot \frac{Q}{r^2} \right) = 0.
\end{align}
For $\mu=1$,
\begin{align}
    \frac{\partial}{\partial x^\nu} \left( \sqrt{g} F^{1 \nu} \right) = \frac{\partial}{\partial t} \left( r^2 \sin\theta \cdot -\frac{Q}{r^2} \right) = 0.
\end{align}
For $\mu=2$ and $\mu=3$, the equations are trivially satisfied since $F^{2\nu} = F^{3\nu} = 0$. Thus, $\frac{\partial}{\partial x^\nu} \left( \sqrt{g} F^{\mu \nu} \right) = 0$ holds for all $\mu$. Now we check the second Maxwell equation:
\begin{align}
    F_{\mu\nu,\lambda} + F_{\nu\lambda,\mu} + F_{\lambda\mu,\nu} = 0.
\end{align}
Since $F_{\mu\nu}$ only has non-zero components for $\mu=0$ and $\nu=1$ (or vice versa), we only need to check the case when $(\mu,\nu,\lambda) = (0,1,r)$:
\begin{align}
    F_{01,r} + F_{1r,0} + F_{r0,1} = \frac{\partial}{\partial r} \left( -\frac{Q}{r^2} \right) + 0 + \frac{\partial}{\partial r} \left( \frac{Q}{r^2} \right) = 0.
\end{align}
All other combinations yield zero trivially. Thus, the second Maxwell equation is also satisfied. For Einstein's equations, please check the attached file for the detailed calculations of the Ricci tensor $R_{\mu\nu}$, Ricci scalar $R$, and energy-momentum tensor $T_{\mu\nu}$. After computing these quantities, we find that Einstein's equations are satisfied with the given metric and electric field configuration. \qed





\clearpage
\question{2}{}
Read pages~259-261 of the textbook by Carroll and then solve exercise~1 on page~272. You can use any information provided in class. As Carroll writes in his book, this is an amazing exact solution.
\\
\textbf{Carroll: }Show that the coupled Einstein-Maxwell equations can be simultaneously solved by the metric~(6.62) and the electrostatic potential~(6.67) if H (i) obeys Laplace's equation,
\begin{align}
    \nabla^2 H = 0.
\end{align}
\begin{align}
    ds=-H^{-2} dt^2 + H^2 (dx^2 + dy^2 + dz^2),
\end{align}
where where $H = H(\vec{x})$. The electrostatic potential is given by
\begin{align}
    A_\mu = \left( \frac{1}{\sqrt{G}}\frac{1}{H}-1, 0, 0, 0 \right).
\end{align}
\answer{}
To verify that the coupled Einstein-Maxwell equations are satisfied by the given metric and electrostatic potential, we start by writing down the metric:
\begin{align}
    ds^2 = -H^{-2} dt^2 + H^2 (dx^2 + dy^2 + dz^2),
\end{align}
where $H = H(\vec{x})$. The electrostatic potential is given by
\begin{align}
    A_\mu = \left( \frac{1}{\sqrt{G}}\frac{1}{H}-1, 0, 0, 0 \right).
\end{align}
See the attached file for the full solution.
\begin{align}
    R_{\mu\nu} - \frac{1}{2} R g_{\mu\nu} + 8 \pi GT_{\mu\nu}=
    \left(
\begin{array}{cccc}
 \frac{2 \left(H^{(0,0,2)}(x,y,z)+H^{(0,2,0)}(x,y,z)+H^{(2,0,0)}(x,y,z)\right)}{H(x,y,z)^5} & 0 & 0 & 0 \\
 0 & 0 & 0 & 0 \\
 0 & 0 & 0 & 0 \\
 0 & 0 & 0 & 0 \\
\end{array}
\right)
\end{align}
Thus, if $H$ satisfies Laplace's equation $\nabla^2 H = 0$, then the coupled Einstein-Maxwell equations are satisfied. I also verified Maxwell's equations in the attached file.
\qed.