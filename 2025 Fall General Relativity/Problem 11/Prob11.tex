\section*{Assignment 11 due on Monday November 24 at 10PM}
\question{1}{}
Calculate numerically the Schwarzschild radius $R_s$, the characteristic collapse time $t_\text{collapse}$, the characteristic radial redshift time $2R_s/c$, the
characteristic radial flux time $R_s/2c$, and the characteristic total luminosity time $3\sqrt{3}R_s/2c$ in the appropriate units (seconds, days, years, kilometers,
light-years, etc) using the dust cloud model for the following initial conditions:
\begin{itemize}
    \item [(i)]An object of three solar masses with an initial radius of 1 AU.
    \item [(ii)]An object of 108 solar masses with an initial radius of 100 light-years.
\end{itemize}
\answer{}
We have
\begin{align}
    R_s = \frac{2GM}{c^2},\\
    t_\text{collapse} = \frac{\pi}{2} \sqrt{\frac{R^3}{2GM}},\\
    t_\text{redshift} = \frac{2R_s}{c},\\
    t_\text{flux} = \frac{R_s}{2c},\\
    t_\text{luminosity} = \frac{3\sqrt{3}R_s}{2c}.
\end{align}
\begin{itemize}
    \item [(i)] For an object of three solar masses with an initial radius of 1 AU:
\end{itemize}
Using $M = 3M_\odot = 3 \times 1.989 \times 10^{30} \text{ kg}$, $R = 1 \text{ AU} = 1.496 \times 10^{11} \text{ m}$, $G = 6.67430 \times 10^{-11} \text{ m}^3 \text{ kg}^{-1} \text{ s}^{-2}$, and $c = 3 \times 10^8 \text{ m/s}$, we find (see \textit{mathematica} code for calculation):
\begin{align}
    R_s \approx& 8844.42 \text{ m}= 8.84442 \text{ km},\\
    t_\text{collapse} \approx& 3.22152\times 10^6 \text{ sec}=37.2862 \text{ days},\\
    t_\text{redshift} \approx& 5.89628\times10^{-5} \text{ seconds}= 58.9628 \text{ microseconds},\\
    t_\text{flux} \approx& 1.47407 \times 10^{-5} \text{ seconds}= 14.7407 \text{ microseconds},\\
    t_\text{luminosity} \approx& 7.65949 \times 10^{-5} \text{ seconds}= 76.5949 \text{ microseconds}.
\end{align}
\begin{itemize}
    \item [(ii)] For an object of $10^8$ solar masses with an initial radius of 100 light-years:
\end{itemize}
Using $M = 10^8 M_\odot = 10^8 \times 1.989 \times 10^{30} \text{ kg}$, $R = 100 \text{ light-years} = 9.461 \times 10^{17} \text{ m}$, $G = 6.67430 \times 10^{-11} \text{ m}^3 \text{ kg}^{-1} \text{ s}^{-2}$, and $c = 3 \times 10^8 \text{ m/s}$, we find (see \textit{mathematica} code for calculation):
\begin{align}
    R_s \approx& 2.94814\times 10^{11} \text{ m}=1.971 \text{ AU},\\
    t_\text{collapse} \approx& 8.87422\times 10^{12} \text{ sec}=281400 \text{ years},\\
    t_\text{redshift} \approx& 1965.43 \text{ seconds}=32.7572 \text{ minutes},\\
    t_\text{flux} \approx& 491.357 \text{ seconds}=8.18928 \text{ minutes},\\
    t_\text{luminosity} \approx& 2553.16 \text{ seconds}=42.5527 \text{ minutes}.
\end{align}
\qed



\clearpage
\question{2}{}
In the 1960’s it was shown that the light received from a collapsing, luminous cloud of dust is dominated by photons deposited near the unstable orbit as the surface of the cloud crosses the radius $r = 3/2 R_s$. Calculate the redshift $z$ for photons emitted \textit{radially} near this surface to at least $2$ significant digits. (A more elaborate calculation shows that most photons are emitted with nonzero angular momentum. These photons escape after orbiting the dust cloud many times, resulting in a redshift $z = 2$.)
\answer{}
The redshift $z$ for photons emitted radially from a radius $r$ in the Schwarzschild metric is given by
\begin{align}
    1 + z = \frac{1}{\sqrt{1 - \frac{R_s}{r}}}.
\end{align}
Hence, for $r = \frac{3}{2} R_s$, we have
\begin{align}
    1 + z = \frac{1}{\sqrt{1 - \frac{R_s}{\frac{3}{2} R_s}}} = \frac{1}{\sqrt{1 - \frac{2}{3}}} = \frac{1}{\sqrt{\frac{1}{3}}} = \sqrt{3}.
\end{align}
Thus, the redshift is
\begin{align}
    z = \sqrt{3} - 1 \approx 0.732.
\end{align}
But this is the redshift effect only due to the gravitational field. We can also consider the Doppler effect due to the motion of the collapsing surface. The total redshift considering both effects is given by
\begin{align}
    1 + z_{total} = (1 + z_\text{gravitational})(1 + z_\text{Doppler}).
\end{align}
The Doppler redshift for a radially infalling object is given by
\begin{align}
    1 + z_\text{Doppler} = \sqrt{\frac{1 + v/c}{1 - v/c}},
\end{align}
where $v$ is the infall velocity at radius $r$. The infall velocity can be found using energy conservation in the Schwarzschild metric:
\begin{align}
    v = c \sqrt{\frac{R_s}{r}}=\sqrt{\frac{R_s}{r}}.
\end{align}
For $r = \frac{3}{2} R_s$, we have
\begin{align}
    v =  \sqrt{\frac{R_s}{\frac{3}{2} R_s}} = \sqrt{\frac{2}{3}}.
\end{align}
Thus, the Doppler redshift is
\begin{align}
    1 + z_\text{Doppler} = \sqrt{\frac{1 + \sqrt{\frac{2}{3}}}{1 - \sqrt{\frac{2}{3}}}}.
\end{align}
Combining both effects, we have
\begin{align}
    1 + z_{total} = \sqrt{3} \cdot \sqrt{\frac{1 + \sqrt{\frac{2}{3}}}{1 - \sqrt{\frac{2}{3}}}}.
\end{align}
Calculating this numerically, we find
\begin{align}
    z_{total} \approx 4.45.
\end{align}
\qed