\section*{Assignment 3 due on Monday September 22th at 5PM}
\question{1}{}
Show explicitly that the $4$-vector current density for a collection of point charges satisfies $\partial_\mu J^\mu=0$
\answer{}
In the class, we defined the $4$-vector current density for a collection of point charges as
\begin{align}
    J^0(t,\mathbf{x})&=\rho(t,\mathbf{x})=\sum_a q_a \delta^{(3)}(\mathbf{x}-\mathbf{x}_a(t))\\
    \mathbf{J}(t,\mathbf{x})&=\sum_a q_a \mathbf{v}_a(t)\delta^{(3)}(\mathbf{x}-\mathbf{x}_a(t)), \quad \mathbf{v}_a(t)=\frac{d\mathbf{x}_a(t)}{dt}.
\end{align}
Then we have 
\begin{align}
    \partial_\mu J^\mu&=\frac{\partial \rho}{\partial t}+\nabla\cdot \mathbf{J}\\
    &=\sum_a q_a \left[\frac{\partial}{\partial t}\delta^{(3)}(\mathbf{x}-\mathbf{x}_a(t))+\nabla\cdot(\mathbf{v}_a(t)\delta^{(3)}(\mathbf{x}-\mathbf{x}_a(t)))\right]\\
    &=\sum_a q_a \left[-\mathbf{v}_a(t)\cdot\nabla \delta^{(3)}(\mathbf{x}-\mathbf{x}_a(t))+\nabla\cdot(\mathbf{v}_a(t)\delta^{(3)}(\mathbf{x}-\mathbf{x}_a(t)))\right]\\
    &=\sum_a q_a \left[-\mathbf{v}_a(t)\cdot\nabla \delta^{(3)}(\mathbf{x}-\mathbf{x}_a(t))+\mathbf{v}_a(t)\cdot\nabla \delta^{(3)}(\mathbf{x}-\mathbf{x}_a(t))\right]\\
    &=0.
\end{align}
\clearpage
\question{2}{}
Prove that the electromagnetic energy density squared minus the square of the Poynting vector is a Lorentz invariant for an electromagnetic field by expressing this quantity in terms of tensors. You might consider using the dual field strength tensor defined by $\widetilde{F}^{\mu\nu}=\frac{1}{2}\varepsilon^{\mu\nu\alpha\beta}F_{\alpha\beta}$.
\answer{}
In the EM class, we defined the electromagnetic energy density and the Poynting vector as
\begin{align}
    u&=\frac{1}{2}(\mathbf{E}^2+\mathbf{B}^2)\\
    \mathbf{S}&=\mathbf{E}\times \mathbf{B}.
\end{align}
Also, we have the following relations
\begin{align}
    F^{\mu\nu}=\begin{pmatrix}
    0 & E_x & E_y & E_z \\
    -E_x & 0 & B_z & -B_y \\
    -E_y & -B_z & 0 & B_x \\
    -E_z & B_y & -B_x & 0
    \end{pmatrix}
\end{align}
Besides, the EM field energy momentum strength tensor is given by
\begin{align}
    T_{EM}^{\mu\nu}&=F^{\mu\alpha}{F^\nu}_\alpha-\frac{1}{4}\eta^{\mu\nu}F_{\alpha\beta}F^{\alpha\beta},\\
    u = T_{EM}^{00}&=\frac{1}{2}(\mathbf{E}^2+\mathbf{B}^2), \quad S^i = T_{EM}^{i0} = (\mathbf{E}\times \mathbf{B})^i.
\end{align}
\clearpage
\question{3}{}
Calculate the scalar ${T^\alpha}_\alpha$ associated with the electromagnetic stress tensor.
\answer{}