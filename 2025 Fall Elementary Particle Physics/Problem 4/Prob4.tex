\section*{Problem Set 4 due 9:30 AM, Monday, October 27th}


\question{1}{\textbf{Gell-Mann Okubo for the baryon octet}}
\\
The generators \(t_{a}(a=1, \ldots, 8)\) of \(\mathrm{SU}(3)\) are normalized as \(t_{a}=\frac{\lambda_{a}}{2}\) with \(\operatorname{Tr}\left(t_{a} t_{b}\right)=\frac{1}{2} \delta_{a b}\), where \(\lambda_{a}\) are the Gell-Mann matrices. They satisfy \(\left[t_{a}, t_{b}\right]=i f_{a b c} t_{c}\) and \(\left\{t_{a}, t_{b}\right\}= \frac{1}{3} \delta_{a b} \mathbf{1}+d_{a b c} t_{c}\), where \(f_{a b c}\) are totally antisymmetric and \(d_{a b c}\) are totally symmetric structure constants.


Let \(B\) and \(\bar{B}\) be the baryon octet \(3 \times 3\) traceless matrices, expanded in the generator basis as \(B=B^{i} t_{i}\) and \(\bar{B}=\bar{B}^{i} t_{i}\) where \(B^{i}, \bar{B}^{i}\) are the adjoint components. Define the two bilinear combinations \(O_{A} \equiv[\bar{B}, B]=\bar{B} B-B \bar{B}\) and \(O_{S} \equiv\{\bar{B}, B\}-\frac{2}{3} \mathbf{1} \operatorname{Tr}(\bar{B} B)\).
\begin{itemize}
    \item [(a)]Show that both \(O_{A}\) and \(O_{S}\) are traceless and therefore transform in the adjoint (octet) representation.
    \item [(b)] Expand \(O_{A}\) and \(O_{S}\) in components using the generator basis and show that \(O_{A}=i\left(\bar{B}^{i} B^{j}\right) f_{i j k} t_{k}\) and \(O_{S}=\left(\bar{B}^{i} B^{j}\right) d_{i j k} t_{k}\), so that \(\left(O_{A}\right)^{k}=i f_{i j k} \bar{B}^{i} B^{j}\) and \(\left(O_{S}\right)^{k}= d_{i j k} \bar{B}^{i} B^{j}\).
    \item [(c)]Introduce a flavor-breaking spurion \(H_{8}=H_{8}^{i} t_{i}\), with real components \(H_{8}^{i}\). Construct the two independent \(\mathrm{SU}(3)\)-invariant mass terms:
    \begin{align}
        S_{f}=\left(O_{A}\right)_{b}^{a}\left(H_{8}\right)_{a}^{b}, \quad S_{d}=\left(O_{S}\right)_{b}^{a}\left(H_{8}\right)_{a}^{b}
    \end{align}
    Assuming \(H_{8}\) points in the 8 -direction (i.e. \(H_{8}^{i} \propto \delta_{i 8}\) ), argue that \(S_{f}\) and \(S_{d}\) correspond to the \(f\)-type and \(d\)-type symmetry breaking terms in the baryon mass operator, respectively.
    \item [(d)] Given that an adjoint operator \(O_{8}\) acts on an octet state \(B\) as \(O_{8}(B)=\left[O_{8}, B\right]\), show that the invariant scalars in (c) are equivalent to the matrix elements \(S_{f} \propto \langle\bar{B}| t_{8}|B\rangle \equiv \operatorname{Tr}\left(\bar{B}\left[t_{8}, B\right]\right)\) and \(S_{d} \propto\langle\bar{B}| d_{8 i j} t_{i} t_{j}|B\rangle \equiv \operatorname{Tr}\left(\bar{B}\left[d_{8 i j}\left[t_{i},\left[t_{j}, B\right]\right]\right)\right.\).
     \item [(e)] Hence, argue that for each entry \(B_{i j}\) of the baryon octet matrix, \(S_{f} \propto Y\) and \(S_{d} \propto I(I+1)-Y^{2} / 4\) where \(I, Y\) are the isospin and hypercharge of the baryon \(B\), respectively, thereby reproducing the Gell-Mann-Okubo mass formula for the baryon octet.
\end{itemize}
(Hint: Verify, entrywise, that \(\left[\frac{2}{\sqrt{3}} t_{8}, B\right]=Y B\) and the normalized operator \(\frac{2}{\sqrt{3}} d_{8 i j}\left[t_{i},\left[t_{j}, B\right]\right]+ \frac{1}{3}\left[t_{i},\left[t_{i}, B\right]\right]=\left(I(I+1)-Y^{2} / 4\right) B\) acts diagonally on each baryon field. The \(\left[t_{i},\left[t_{i}, B\right]\right]\) term is the adjoint Casimir ( \(S U(3)\) singlet) which just shifts all octet components uniformly so that the \(\Lambda\) eigenvalue becomes 0 . It can be absorbed into the overall singlet part of the GMO formula. On the diagonal remember \(B_{11}, B_{22}\) and \(B_{33}\) mix \(\Sigma^{0}\) and \(\Lambda\), so \(\left.B_{\text {diag }}=\Sigma^{0} \operatorname{diag}(1,-1,0) / \sqrt{2}+\Lambda \operatorname{diag}(1,1,-2) / \sqrt{6}\right)\).
\answer{}
\begin{itemize}
    \item [(a)]
\end{itemize}
To show that both \(O_{A}\) and \(O_{S}\) are traceless, we first understand their definitions:
\begin{align}
O_{A} &= [\bar{B}, B] = \bar{B} B - B \bar{B}=[\bar{B}, B]\\
      &= [\bar{B}^{i} t_{i}, B^{j} t_{j}] = \bar{B}^{i} B^{j} [t_{i}, t_{j}] = i \bar{B}^{i} B^{j} f_{ijk} t_{k}
\end{align}
Taking the trace of \(O_{A}\):
\begin{align}
\operatorname{Tr}(O_{A}) &= \operatorname{Tr}(iBar{B}^{i} B^{j} f_{ijk} t_{k}) = i \bar{B}^{i} B^{j} f_{ijk} \operatorname{Tr}(t_{k}) = 0.
\end{align}
Similarly, for \(O_{S}\):
\begin{align}
O_{S} &= \{\bar{B}, B\} - \frac{2}{3} \mathbf{1} \operatorname{Tr}(\bar{B} B) = \bar{B} B + B \bar{B} - \frac{2}{3} \mathbf{1} \operatorname{Tr}(\bar{B} B)\\
      &= (\bar{B}^{i} t_{i})(B^{j} t_{j}) + (B^{j} t_{j})(\bar{B}^{i} t_{i}) - \frac{2}{3} \mathbf{1} \operatorname{Tr}(\bar{B}^{i} t_{i} B^{j} t_{j})\\
      &= \bar{B}^{i} B^{j}\{t_{i}, t_{j}\} - \frac{2}{3} \mathbf{1} \operatorname{Tr}(\bar{B}^{i} B^{j} t_{i} t_{j})\\
      &= \bar{B}^{i} B^{j}\left(\frac{1}{3}\delta_{ij}\mathbf{1}+d_{ijk}t_{k}\right) - \frac{2}{3}\mathbf{1}\left(\frac{1}{2}\bar{B}^{i} B^{j}\delta_{ij}\right)\\
      &= \bar{B}^{i} B^{j} d_{ijk} t_{k}
\end{align}
Taking the trace of \(O_{S}\):
\begin{align}
    \operatorname{Tr}(O_{S}) &= \operatorname{Tr}(\bar{B}^{i} B^{j} d_{ijk} t_{k}) = \bar{B}^{i} B^{j} d_{ijk} \operatorname{Tr}(t_{k}) = 0.
\end{align}
Thus, both \(O_{A}\) and \(O_{S}\) are traceless and transform in the adjoint (octet) representation.
\begin{itemize}
    \item [(b)]
\end{itemize}
Expanding \(O_{A}\) and \(O_{S}\) in components using the generator basis, we have already derived:
\begin{align}
O_{A} &= i \bar{B}^{i} B^{j} f_{ijk} t_{k}\\
O_{S} &= \bar{B}^{i} B^{j} d_{ijk} t_{k}
\end{align}
Thus, the components are:
\begin{align}
(O_{A})^{k} &= i f_{ijk} \bar{B}^{i} B^{j}\\
(O_{S})^{k} &= d_{ijk} \bar{B}^{i} B^{j}
\end{align}
\begin{itemize}
    \item [(c)]
\end{itemize}
Introducing a flavor-breaking spurion \(H_{8} = H_{8}^{i} t_{i}\), with real components \(H_{8}^{i}\), we can construct the two independent \(\mathrm{SU}(3)\)-invariant mass terms:
\begin{align}
    S_{f} &= (O_{A})_{b}^{a}(H_{8})_{a}^{b} = \operatorname{Tr}(O_{A} H_{8})\\
    S_{d} &= (O_{S})_{b}^{a}(H_{8})_{a}^{b} = \operatorname{Tr}(O_{S} H_{8})
\end{align}
Assuming \(H_{8}\) points in the 8-direction (i.e., \(H_{8}^{i} \propto \delta_{i8}\)), we can write:
\begin{align}
    H_{8} &= H_{8}^{i} t_{i}= H_{8}^{8} t_{8}
\end{align}
Substituting this into the expressions for \(S_{f}\) and \(S_{d}\):
\begin{align}
    S_{f} &= \operatorname{Tr}(O_{A} H_{8}) = \operatorname{Tr}(if_{ijk} \bar{B}^{i} B^{j} t_{k} H_{8}^{l} t_{l}) = i H_{8}^{8} f_{ij8} \bar{B}^{i} B^{j} \operatorname{Tr}(t_{k} t_{8})= \frac{i}{2} H_{8}^{8} f_{ij8} \bar{B}^{i} B^{j}\\
    S_{d} &= \operatorname{Tr}(O_{S} H_{8}) = \operatorname{Tr}(d_{ijk} \bar{B}^{i} B^{j} t_{k} H_{8}^{l} t_{l}) = H_{8}^{8} d_{ij8} \bar{B}^{i} B^{j} \operatorname{Tr}(t_{k} t_{8})= \frac{1}{2} H_{8}^{8} d_{ij8} \bar{B}^{i} B^{j}
\end{align}
Since \(O_{A}\) and \(O_{S}\) are octet operators, \(S_{f}\) and \(S_{d}\) correspond to the \(f\)-type and \(d\)-type symmetry breaking terms in the baryon mass operator, respectively.
\begin{itemize}
    \item [(d)]
\end{itemize}
\begin{align}
    &\langle\bar{B}| t_{8}|B\rangle \equiv\operatorname{Tr}(\bar{B} [t_{8}, B])\\
    =& \operatorname{Tr}(\bar{B} (t_{8} B - B t_{8})) = \operatorname{Tr}(\bar{B} t_{8} B) - \operatorname{Tr}(\bar{B} B t_{8})\\
    =& \operatorname{Tr}(B\bar{B} t_{8} ) - \operatorname{Tr}(\bar{B} B t_{8})\\
    =& \operatorname{Tr}((\bar{B} B - B \bar{B}) t_{8}) = \operatorname{Tr}(O_{A} t_{8}) \propto S_{f}
\end{align}
This is beccause $S_f=\operatorname{Tr}(O_{A} H_{8})=\operatorname{Tr}(O_{A} H_{8}^8t_8)=H_8^8\operatorname{Tr}(O_{A}t_8)$. Similarly, for \(S_{d}\):
\begin{align}
    &\langle\bar{B}| d_{8 i j} t_{i} t_{j}|B\rangle \equiv \operatorname{Tr}(\bar{B} d_{8 i j}[t_{i},[t_{j}, B]])\\
    =& \operatorname{Tr}(\bar{B}^\alpha  B^\beta t_\alpha d_{8 i j} [t_{i},[t_{j}, t_\beta]])\\
    =& \bar{B}^\alpha B^\beta d_{8 i j} \operatorname{Tr}(t_\alpha  [t_{i},[t_{j}, t_\beta]])\\
    =& \bar{B}^\alpha B^\beta d_{8 i j} \operatorname{Tr}(t_\alpha [t_i,if_{j \beta \gamma} t_\gamma])\\
    =& i \bar{B}^\alpha B^\beta d_{8 i j} f_{j \beta \gamma} \operatorname{Tr}(t_\alpha [t_i, t_\gamma])\\
    =& i \bar{B}^\alpha B^\beta d_{8 i j} f_{j \beta \gamma} if_{i \alpha \delta} \operatorname{Tr}(t_\alpha t_\delta)\\
    =& - \frac{1}{2} \bar{B}^\alpha B^\beta d_{8 i j} f_{j \beta \gamma} f_{i \alpha \gamma}\\
    =&- \frac{1}{2} \bar{B}^a B^b d_{8 i j} f_{j b c} f_{i a c}
\end{align}
By \texttt{Mathematica}, we have 
\begin{align}
    d_{8 i j} f_{j b c} f_{i a c} = 3/2 d_{8 a b}
\end{align}
Thus,
\begin{align}
    \langle\bar{B}| d_{8 i j} t_{i} t_{j}|B\rangle = - \frac{3}{4} \bar{B}^a B^b d_{8 a b} = - \frac{3}{2} \operatorname{Tr}(O_{S} t_{8}) \propto S_{d}
\end{align}
This is because $S_d=\operatorname{Tr}(O_{S} H_{8})=\operatorname{Tr}(O_{S} H_{8}^8t_8)=H_8^8\operatorname{Tr}(O_{S}t_8)$.
\begin{itemize}
    \item [(e)]
\end{itemize}
First, we write down the explicit form of $B$:
\begin{align}
    B = \begin{pmatrix}
    \frac{\Sigma^{0}}{\sqrt{2}} + \frac{\Lambda^0}{\sqrt{6}} & \Sigma^{+} & p \\
    \Sigma^{-} & -\frac{\Sigma^{0}}{\sqrt{2}} + \frac{\Lambda^0}{\sqrt{6}} & n \\
    -\Xi^{-} & \Xi^{0} & -\frac{2\Lambda^0}{\sqrt{6}}
    \end{pmatrix}
\end{align}
Next, we write down the hypercharge \(Y\) and isospin \(I\) and $I(I+1)-Y^2/4$ values for each baryon:
\begin{align}
    &p: Y=1, I=1/2, I(I+1)-Y^2/4=3/4-1/4=1/2\\
    &n: Y=1, I=1/2, I(I+1)-Y^2/4=3/4-1/4=1/2\\
    &\Sigma^{+}: Y=0, I=1, I(I+1)-Y^2/4=2-0=2\\
    &\Sigma^{0}: Y=0, I=1, I(I+1)-Y^2/4=2-0=2\\
    &\Sigma^{-}: Y=0, I=1, I(I+1)-Y^2/4=2-0=2\\
    &\Xi^{0}: Y=-1, I=1/2, I(I+1)-Y^2/4=3/4-1/4=1/2\\
    &\Xi^{-}: Y=-1, I=1/2, I(I+1)-Y^2/4=3/4-1/4=1/2\\
    &\Lambda^{0}: Y=0, I=0, I(I+1)-Y^2/4=0-0=0
\end{align}
Now, we compute \(\left[\frac{2}{\sqrt{3}} t_{8}, B\right]\), and it can be easily show that (see my \textit{Mathematica} notebook for details):
\begin{align}
    [\frac{2}{\sqrt{3}} t_{8}, B] = 
    \begin{pmatrix}
    0 & 0 & p \\
    0 & 0 & n \\
    -(-\Xi^{-}) & -\Xi^{0} & 0
    \end{pmatrix}= Y B
\end{align}
where \(Y\) is the hypercharge of the baryon \(B\). Next, we compute the normalized operator \(\frac{2}{\sqrt{3}} d_{8 i j}\left[t_{i},\left[t_{j}, B\right]\right]+ \frac{1}{3}\left[t_{i},\left[t_{i}, B\right]\right]\):
\begin{align}
    &\frac{2}{\sqrt{3}} d_{8 i j}\left[t_{i},\left[t_{j}, B\right]\right]+ \frac{1}{3}\left[t_{i},\left[t_{i}, B\right]\right] \\
    =& \begin{pmatrix}
    \sqrt{2}\Sigma^{0} & 2\Sigma^{+} & \frac{1}{2}p \\
    2\Sigma^{-} & -\sqrt{2}\Sigma^{0} & \frac{1}{2}n \\
    \frac{1}{2}(-\Xi^{-}) & \frac{1}{2}\Xi^{0} & 0
    \end{pmatrix}=\begin{pmatrix}
    2\frac{\Sigma^{0}}{\sqrt{2}} & 2\Sigma^{+} & \frac{1}{2}p \\
    2\Sigma^{-} & -2\frac{\Sigma^{0}}{\sqrt{2}} & \frac{1}{2}n \\
    \frac{1}{2}(-\Xi^{-}) & \frac{1}{2}\Xi^{0} & 0
    \end{pmatrix} = \left(I(I+1)-\frac{Y^{2}}{4}\right) B
\end{align}
where \(I(I+1)-\frac{Y^{2}}{4}\) is the value for each baryon \(B\). Thus, we have shown that for each entry \(B_{ij}\) of the baryon octet matrix, \(S_{f} \propto Y\) and \(S_{d} \propto I(I+1)-Y^{2} / 4\), thereby reproducing the Gell-Mann-Okubo mass formula for the baryon octet, meaning 
\begin{align}
    M_B(Y,I) = M_{0} + M_A Y + M_S \left(I(I+1)-\frac{Y^{2}}{4}\right),
\end{align}
where \(M_{0}, M_{A}, M_{S}\) are constants.
\qed

\clearpage
\question{2}{\textbf{$\rho$-$\omega$ mixing}}
\\
The vector mesons $\rho(770)$ and $\omega(782)$ are very close in mass. For this reason the effects of isospin violation are somewhat enhanced in these mesons and can be parametrized in terms of $\rho$-$\omega$ mixing. Namely, the physical $\rho^{0}$ and $\omega$ mesons can be viewed as orthogonal mixed states of a pure isospin triplet and isospin singlet:
\[
\rho^{0} = \cos\theta \frac{(u\bar{u} - d\bar{d})}{\sqrt{2}} + \sin\theta \frac{(u\bar{u} + d\bar{d})}{\sqrt{2}},
\]
\[
\omega = -\sin\theta \frac{(u\bar{u} - d\bar{d})}{\sqrt{2}} + \cos\theta \frac{(u\bar{u} + d\bar{d})}{\sqrt{2}},
\]
where $\theta$ is a (small) mixing angle.

\begin{enumerate}
\item[(a)] Determine $\theta$ (up to a sign) using experimental data on the decay $\omega \rightarrow \pi^{+} \pi^{-}$. Estimate the error in the value of the mixing angle.
\item[(b)] Using the value of $\theta$ predict the decay rates $\Gamma(\rho^{0} \rightarrow e^{+} e^{-})$ and $\Gamma(\omega \rightarrow e^{+} e^{-})$, assuming the amplitude for a quark pair annihilation into an $e^{+} e^{-}$ pair is proportional to the electric charge $Q$ of the quark.
\item[(c)] Assume that the transition amplitude between different spin states of a $q\bar{q}$ quark pair with emission of a photon: $(q\bar{q}) \rightarrow (q\bar{q}) + \gamma$ is proportional to the quark electric charge $Q$. Use the value of the $\rho$-$\omega$ mixing angle $\theta$ to determine the ratios of the decay rates:
\begin{enumerate}
\item[(i)] $\Gamma(\rho^{0} \rightarrow \pi^{0} \gamma) / \Gamma(\omega^{0} \rightarrow \pi^{0} \gamma)$,
\item[(ii)] $\Gamma(\rho^{0} \rightarrow \eta \gamma) / \Gamma(\omega^{0} \rightarrow \eta \gamma)$.
\end{enumerate}
Compare with the PDG experimental data. How does the inclusion of $\rho$-$\omega$ mixing improve the agreement with the data?
\end{enumerate}

\answer{}
Before we start, we declare the notation for physical states and pure isospin states as follows:
\begin{align}
    |\rho^0\rangle &= \cos\theta |\rho^0_I\rangle + \sin\theta |\omega_I\rangle,\\
    |\omega\rangle &= -\sin\theta |\rho^0_I\rangle + \cos\theta |\omega_I\rangle,\\
    |\rho^0_I\rangle &= \frac{1}{\sqrt{2}} (|u\bar{u}\rangle - |d\bar{d}\rangle)=|I=1, I_3=0\rangle,\\
    |\omega_I\rangle &= \frac{1}{\sqrt{2}} (|u\bar{u}\rangle + |d\bar{d}\rangle)=|I=0, I_3=0\rangle.
\end{align}
\begin{itemize}
    \item [(a)]
\end{itemize}
The decay amplitude for $\omega \rightarrow \pi^+ \pi^-$ and $\rho^0 \rightarrow \pi^+ \pi^-$ can be written as:
\begin{align}
    \mathcal{M}(\omega \rightarrow \pi^+ \pi^-) &= \langle \pi^+ \pi^- | H | \omega \rangle\\
    &= \langle \pi^+ \pi^- | H | -\sin\theta |\rho^0_I\rangle + \cos\theta |\omega_I\rangle \\  
    &= -\sin\theta \langle \pi^+ \pi^- | H | \rho^0_I \rangle + \cos\theta \langle \pi^+ \pi^- | H | \omega_I \rangle\\
    \mathcal{M}(\rho^0 \rightarrow \pi^+ \pi^-) &= \langle \pi^+ \pi^- | H | \rho^0 \rangle\\
    &= \langle \pi^+ \pi^- | H | \cos\theta |\rho^0_I\rangle + \sin\theta |\omega_I\rangle \\
    &= \cos\theta \langle \pi^+ \pi^- | H | \rho^0_I \rangle + \sin\theta \langle \pi^+ \pi^- | H | \omega_I \rangle.
\end{align}
Actually, since both of \(\rho^0\) and \(\omega\) are vector mesons with \(J^{PC}=1^{--}\) and \(\pi^+\pi^-\) is a pseudoscalar meson pair with \(J^{PC}=0^{-+}\), the decay must proceed via a P-wave to conserve angular momentum and parity due to strong interaction. Thus the spatial wave function of \(\pi^+\pi^-\) must be antisymmetric under exchange of the two pions. Since pions are bosons, the total wave function must be symmetric under exchange of the two pions. Therefore, the isospin wave function of \(\pi^+\pi^-\) must also be antisymmetric under exchange of the two pions. This means that the \(\pi^+\pi^-\) state can only be in an isospin \(I=1\) state, since the \(I=0\) and \(I=2\) states are symmetric under exchange of the two pions. Thus, by Clebsch-Gordan decomposition, we only have: 
\begin{align}
    |\pi^+ \pi^- \rangle_{\text{anti-sym}} &=\frac{1}{\sqrt{2}}\Big( |\pi^+\rangle| \pi^- \rangle- |\pi^-\rangle| \pi^+ \rangle\Big)\\
    &=  |I=1, I_3=0\rangle.
\end{align}
Using this, we can evaluate the matrix elements:
\begin{align}
    \langle \pi^+ \pi^- | H | \rho^0_I \rangle &\propto \langle I=1, I_3=0 | H | I=1, I_3=0 \rangle = A,\\
    \langle \pi^+ \pi^- | H | \omega_I \rangle &\propto \langle I=1, I_3=0 | H | I=0, I_3=0 \rangle = 0,
\end{align}
where \(A\) is the amplitude for the isospin-conserving decay. Thus, the decay amplitudes become:
\begin{align}
    \mathcal{M}(\omega \rightarrow \pi^+ \pi^-) &= -\sin\theta A,\\
    \mathcal{M}(\rho^0 \rightarrow \pi^+ \pi^-) &= \cos\theta A.
\end{align}
The decay rates are proportional to the square of the amplitude:
\begin{align}
    \Gamma(\omega \rightarrow \pi^+ \pi^-) &\propto |\mathcal{M}(\omega \rightarrow \pi^+ \pi^-)|^2 = \sin^2\theta |A|^2,\\
    \Gamma(\rho^0 \rightarrow \pi^+ \pi^-) &\propto |\mathcal{M}(\rho^0 \rightarrow \pi^+ \pi^-)|^2 = \cos^2\theta |A|^2.
\end{align}
Note that we ignore the phase space factors since \(m_\omega \approx m_{\rho^0}\). Taking the ratio of the decay rates, we have:
\begin{align}
    \frac{\Gamma(\omega \rightarrow \pi^+ \pi^-)}{\Gamma(\rho^0 \rightarrow \pi^+ \pi^-)} &= \frac{\sin^2\theta}{\cos^2\theta} = \tan^2\theta.  
\end{align}
Using the experimental values from PDG:
\begin{align}
    \Gamma(\omega \rightarrow \pi^+ \pi^-) &\approx 0.133 \text{ MeV},\\
    \Gamma(\rho^0 \rightarrow \pi^+ \pi^-) &\approx 147.1 \text{ MeV},
\end{align}
we can solve for \(\theta\):
\begin{align}
    \tan^2\theta &= \frac{0.133}{147.1},\\
    \theta &\approx  0.0301 \text{ radians} \approx 1.72^\circ.
\end{align}
The error in the value of the mixing angle can be estimated by propagating the uncertainties in the decay rates. However, since the uncertainties in the decay rates are relatively small compared to their values, the error in \(\theta\) will also be small. For simplicity, we can estimate the error as:
\begin{align}
    \Delta \theta &\approx \frac{1}{2} \frac{\Delta \Gamma(\omega \rightarrow \pi^+ \pi^-)}{\Gamma(\rho^0 \rightarrow \pi^+ \pi^-)} \frac{1}{\tan\theta}.
\end{align}
Using the uncertainties from PDG, $\Delta \Gamma(\omega \rightarrow \pi^+ \pi^-)$ is about $0.01$ MeV, we find:
\begin{align}
    \Delta \theta &\approx \frac{1}{2} \frac{0.01}{147.1} \frac{1}{\tan(0.0301)} \approx 0.00113041 \text{ radians} \approx 0.065^\circ.
\end{align}
\begin{itemize}
    \item [(b)]
\end{itemize}
\begin{figure}[!h]
    \centering
    \begin{tikzpicture}
    \begin{feynman}
      \vertex (p1) at (-2,  1) {\(q\)};
      \vertex (p2) at (-2, -1) {\(\bar{q}\)};

      % 中間交換點
      \vertex (c)  at (-1,  0) ;
      \vertex (d)  at (1, 0);

      % 右邊出射粒子
      \vertex (p1p) at ( 2,  1) {\(e^+\)};
      \vertex (p2p) at ( 2, -1) {\(e^-\)};
      % 畫圖
      \diagram {
        (p1) -- [fermion] (c) -- [fermion] (p2),
        (p1p) -- [fermion] (d) -- [fermion] (p2p),
        (c) -- [boson, edge label=\(\gamma\)] (d), % 可改成 gluon, Z, h ...
      };
  \end{feynman}
\end{tikzpicture}
    \caption{The Feynman diagram for $q\bar{q}\to e^+e^-$ .}
    \label{HW4-plt:qq-to-ee}
\end{figure}
We can draw the Feynman diagram in Figure~\ref{HW4-plt:qq-to-ee} for the decay of a vector meson into an electron-positron pair via a virtual photon. The amplitude for the decay can be written as:

\begin{align}
    \mathcal{M}(V \rightarrow e^+ e^-) &\propto \langle e^+ e^- | H | V \rangle,
\end{align}
where \(V\) represents either \(\rho^0\) or \(\omega\). The quark content of the vector mesons is:
\begin{align}
    |\rho^0_I\rangle &= \frac{1}{\sqrt{2}} (|u\bar{u}\rangle - |d\bar{d}\rangle),\\
    |\omega_I\rangle &= \frac{1}{\sqrt{2}} (|u\bar{u}\rangle + |d\bar{d}\rangle).
\end{align}
The amplitude for the decay can be expressed in terms of the quark charges:
\begin{align}
    \mathcal{M}(\rho^0_I \rightarrow e^+ e^-) &\propto \frac{1}{\sqrt{2}} (Q_u - Q_d) = \frac{1}{\sqrt{2}} \left(\frac{2}{3} - \left(-\frac{1}{3}\right)\right) = \frac{1}{\sqrt{2}} \cdot 1 = \frac{1}{\sqrt{2}},\\
    \mathcal{M}(\omega_I \rightarrow e^+ e^-) &\propto \frac{1}{\sqrt{2}} (Q_u + Q_d) = \frac{1}{\sqrt{2}} \left(\frac{2}{3} + \left(-\frac{1}{3}\right)\right) = \frac{1}{\sqrt{2}} \cdot \frac{1}{3} = \frac{1}{3\sqrt{2}}.
\end{align}
Using the mixing relations, we can write the amplitudes for the physical states:
\begin{align}
    \mathcal{M}(\rho^0 \rightarrow e^+ e^-) &= \cos\theta \mathcal{M}(\rho^0_I \rightarrow e^+ e^-) + \sin\theta \mathcal{M}(\omega_I \rightarrow e^+ e^-)\\
    &= \cos\theta \cdot \frac{1}{\sqrt{2}} + \sin\theta \cdot \frac{1}{3\sqrt{2}},\\
    \mathcal{M}(\omega \rightarrow e^+ e^-) &= -\sin\theta \mathcal{M}(\rho^0_I \rightarrow e^+ e^-) + \cos\theta \mathcal{M}(\omega_I \rightarrow e^+ e^-)\\
    &= -\sin\theta \cdot \frac{1}{\sqrt{2}} + \cos\theta \cdot \frac{1}{3\sqrt{2}}.
\end{align}
The decay rates are proportional to the square of the amplitudes:
\begin{align}
    \Gamma(\rho^0 \rightarrow e^+ e^-) &\propto \left|\cos\theta \cdot \frac{1}{\sqrt{2}} + \sin\theta \cdot \frac{1}{3\sqrt{2}}\right|^2,\\
    \Gamma(\omega \rightarrow e^+ e^-) &\propto \left|-\sin\theta \cdot \frac{1}{\sqrt{2}} + \cos\theta \cdot \frac{1}{3\sqrt{2}}\right|^2.
\end{align}
Substituting the value of \(\theta \approx 0.0301\) radians, we can calculate the decay rates:
\begin{align}
    \Gamma(\rho^0 \rightarrow e^+ e^-) &\propto \left|\cos(0.0301) \cdot \frac{1}{\sqrt{2}} + \sin(0.0301) \cdot \frac{1}{3\sqrt{2}}\right|^2 \approx 0.5009,\\ 
    \Gamma(\omega \rightarrow e^+ e^-) &\propto \left|-\sin(0.0301) \cdot \frac{1}{\sqrt{2}} + \cos(0.0301) \cdot \frac{1}{3\sqrt{2}}\right|^2 \approx 0.046.
\end{align}
Taking the ratio of the decay rates, we have:
\begin{align}
    \frac{\Gamma(\rho^0 \rightarrow e^+ e^-)}{\Gamma(\omega \rightarrow e^+ e^-)} &\approx \frac{0.5009}{0.046} \approx 11.09.
\end{align}
Using the experimental values from PDG: 
\begin{align}
    \frac{\Gamma(\rho^0 \rightarrow e^+ e^-)}{\Gamma(\omega \rightarrow e^+ e^-)} &\approx \frac{6.99 \text{ keV}}{0.64 \text{ keV}} \approx 10.86,
\end{align}
we see that our prediction is in good agreement with the experimental data.
\begin{itemize}
    \item [(c)]
\end{itemize}
The decay amplitude for the transition \((q\bar{q}) \rightarrow (q\bar{q}) + \gamma\) can be written as:
\begin{align}
    \mathcal{M}(V \rightarrow P + \gamma) &\propto \langle P \gamma | H | V \rangle,
\end{align}
where \(V\) represents either \(\rho^0\) or \(\omega\), and \(P\) represents either \(\pi^0\) or \(\eta\). The quark content of the pseudoscalar mesons is: 
\begin{align}
    |\rho^0_I\rangle &= \frac{1}{\sqrt{2}} (|u\bar{u}\rangle - |d\bar{d}\rangle),\\
    |\omega_I\rangle &= \frac{1}{\sqrt{2}} (|u\bar{u}\rangle + |d\bar{d}\rangle),\\
    |\pi^0\rangle &= \frac{1}{\sqrt{2}} (|u\bar{u}\rangle - |d\bar{d}\rangle),\\
    |\eta\rangle &= \frac{1}{\sqrt{6}} (|u\bar{u}\rangle + |d\bar{d}\rangle - 2|s\bar{s}\rangle).
\end{align}
The amplitude for the decay can be expressed in terms of the quark charges:
\begin{align}
    \mathcal{M}(\rho^0_I \rightarrow \pi^0 + \gamma) &\propto Q_u \langle u \bar{u}|u \bar{u}\rangle + Q_d \langle d \bar{d}|d \bar{d}\rangle =  \frac{2}{3} + \left(-\frac{1}{3}\right) = \frac{1}{3},\\
    \mathcal{M}(\omega_I \rightarrow \pi^0 + \gamma) &\propto Q_u \langle u \bar{u}|u \bar{u}\rangle - Q_d \langle d \bar{d}|d \bar{d}\rangle =  \frac{2}{3} - \left(-\frac{1}{3}\right) = 1,\\
    \mathcal{M}(\rho^0_I \rightarrow \eta + \gamma) &\propto Q_u \langle u \bar{u}|u \bar{u}\rangle - Q_d \langle d \bar{d}|d \bar{d}\rangle =  \frac{2}{3} - \left(-\frac{1}{3}\right) = 1,\\
    \mathcal{M}(\omega_I \rightarrow \eta + \gamma) &\propto Q_u \langle u \bar{u}|u \bar{u}\rangle + Q_d \langle d \bar{d}|d \bar{d}\rangle =  \frac{2}{3} + \left(-\frac{1}{3}\right) = \frac{1}{3}.
\end{align}
Using the mixing relations, we can write the amplitudes for the physical states:
\begin{align}
    \mathcal{M}(\rho^0 \rightarrow \pi^0 + \gamma) &= \cos\theta \mathcal{M}(\rho^0_I \rightarrow \pi^0 + \gamma) + \sin\theta \mathcal{M}(\omega_I \rightarrow \pi^0 + \gamma)\\
    &= \cos\theta \cdot \frac{1}{3} + \sin\theta \cdot 1,\\
    \mathcal{M}(\omega \rightarrow \pi^0 + \gamma) &= -\sin\theta \mathcal{M(\rho^0_I \rightarrow \pi^0 + \gamma) + \cos\theta \mathcal{M}(\omega_I \rightarrow \pi^0 + \gamma)}\\
    &= -\sin\theta \cdot \frac{1}{3} + \cos\theta \cdot 1,\\
    \mathcal{M}(\rho^0 \rightarrow \eta + \gamma) &= \cos\theta \mathcal{M(\rho^0_I \rightarrow \eta + \gamma) + \sin\theta \mathcal{M}(\omega_I \rightarrow \eta + \gamma)}\\
    &= \cos\theta \cdot 1 + \sin\theta \cdot \frac{1}{3},\\
    \mathcal{M}(\omega \rightarrow \eta + \gamma) &= -\sin\theta \mathcal{M(\rho^0_I \rightarrow \eta + \gamma) + \cos\theta \mathcal{M}(\omega_I \rightarrow \eta + \gamma)}\\
    &= -\sin\theta \cdot 1 + \cos\theta \cdot \frac{1}{3}.
\end{align}
The decay rates are proportional to the square of the amplitudes:
\begin{align}
    \Gamma(\rho^0 \rightarrow \pi^0 + \gamma) &\propto \left|\cos\theta \cdot \frac{1}{3} + \sin\theta \cdot 1\right|^2,\\
    \Gamma(\omega \rightarrow \pi^0 + \gamma) &\propto \left|-\sin\theta \cdot \frac{1}{3} + \cos\theta \cdot 1\right|^2,\\
    \Gamma(\rho^0 \rightarrow \eta + \gamma) &\propto \left|\cos\theta \cdot 1 + \sin\theta \cdot \frac{1}{3}\right|^2,\\
    \Gamma(\omega \rightarrow \eta + \gamma) &\propto \left|-\sin\theta \cdot 1 + \cos\theta \cdot \frac{1}{3}\right|^2.
\end{align}
Hence, the ratios of the decay rates are:
\begin{align}
    \frac{\Gamma(\rho^0 \rightarrow \pi^0 + \gamma)}{\Gamma(\omega \rightarrow \pi^0 + \gamma)} &\approx \Bigg(\frac{\cos\theta+3\sin\theta}{3\cos\theta-\sin\theta}\Bigg)^2\\
    \frac{\Gamma(\rho^0 \rightarrow \eta + \gamma)}{\Gamma(\omega \rightarrow \eta + \gamma)} &\approx \Bigg(\frac{3\cos\theta+\sin\theta}{\cos\theta-3\sin\theta}\Bigg)^2.
\end{align}
Hence, we can have 
\begin{align}
    \frac{\Gamma(\rho^0 \rightarrow \pi^0 + \gamma)}{\Gamma(\omega \rightarrow \pi^0 + \gamma)}\Bigg|_{\theta=0} &= \Bigg(\frac{1+0}{3-0}\Bigg)^2 = \frac{1}{9} \approx 0.111,\\
    \frac{\Gamma(\rho^0 \rightarrow \pi^0 + \gamma)}{\Gamma(\omega \rightarrow \pi^0 + \gamma)}\Bigg|_{\theta=0.0301} &\approx 0.135,\\
    \frac{\Gamma(\rho^0 \rightarrow \eta + \gamma)}{\Gamma(\omega \rightarrow \eta + \gamma)}\Bigg|_{\theta=0} &= \Bigg(\frac{3+0}{1-0}\Bigg)^2 = 9,\\
    \frac{\Gamma(\rho^0 \rightarrow \eta + \gamma)}{\Gamma(\omega \rightarrow \eta + \gamma)}\Bigg|_{\theta=0.0301} &\approx 11.1.
\end{align}
Compare with the experimental data from PDG:
\begin{align}
    \frac{\Gamma(\rho^0 \rightarrow \pi^0 + \gamma)}{\Gamma(\omega \rightarrow \pi^0 + \gamma)} &\approx \frac{69.3 \text{ keV}}{723 \text{ keV}} \approx 0.096,\\
    \frac{\Gamma(\rho^0 \rightarrow \eta + \gamma)}{\Gamma(\omega \rightarrow \eta + \gamma)} &\approx \frac{44.22 \text{ keV}}{3.9 \text{ keV}} \approx 11.32.
\end{align}
 \textbf{Hence, we see that the inclusion of \(\rho\)-\(\omega\) mixing improves the agreement with the experimental data for the \(\rho^0 \rightarrow \eta + \gamma\) decay, while it worsens the agreement for the \(\rho^0 \rightarrow \pi^0 + \gamma\) decay.}\qed

\clearpage
\question{3}{\textbf{Baryon magnetic moments}}
\\
The octet of spin-$\frac{1}{2}$ baryons has magnetic moments $\mu$. The operator that describes the magnetic moment is an $\mathrm{SU}(3)_{f}$ octet operator which is proportional to the quark charge $Q$. The charge
\[
Q = t_{3} + \frac{1}{\sqrt{3}} t_{8}
\]
is traceless ($\operatorname{Tr} Q = 0$) and can be promoted to a purely $\mathrm{SU}(3)_{f}$ octet spurion $\mathbf{8}_{Q}$ (with no singlet piece, as in contrast to the GMO mass formula). Hence, when determining the baryon magnetic moment
\[
\mu(B) = \langle \bar{B} | \mu | B \rangle \propto \mathbf{8}_{\bar{B}} \times \mathbf{8}_{Q} \times \mathbf{8}_{B}
\]
there are two independent octet structures (the $f$- and $d$-type couplings, as for the baryon mass), given by
\[
\mu(B) = c_{f} \operatorname{Tr}(B^{\dagger} [Q, B]) + c_{d} \operatorname{Tr}(B^{\dagger} \{Q, B\}) = \alpha_{+} \operatorname{Tr}(B B^{\dagger} Q) + \alpha_{-} \operatorname{Tr}(B^{\dagger} B Q),
\]
where $\alpha_{+} \equiv c_{d} + c_{f}$, $\alpha_{-} \equiv c_{d} - c_{f}$ are arbitrary constants and
\[
B = \begin{pmatrix}
\frac{\Sigma^{0}_{u}}{\sqrt{2}} + \frac{\Lambda}{\sqrt{6}} & \Sigma^{+} & p \\
\Sigma^{-} & -\frac{\Sigma^{0}_{u}}{\sqrt{2}} + \frac{\Lambda}{\sqrt{6}} & n \\
-\Xi^{-} & \Xi^{0} & -\frac{2\Lambda}{\sqrt{6}}
\end{pmatrix}, \quad
Q = \begin{pmatrix}
\frac{2}{3} & 0 & 0 \\
0 & -\frac{1}{3} & 0 \\
0 & 0 & -\frac{1}{3}
\end{pmatrix}.
\]

Determine all the spin-$\frac{1}{2}$ baryon magnetic moments in terms of $\mu(p)$ and $\mu(n)$ (by eliminating $c_{f,d}$ or $\alpha_{\pm}$) and compare with the PDG experimental values. These predictions were first worked out by Coleman and Glashow in 1961. Note that imposing the full $\mathrm{SU}(6)$ spin-flavor symmetry further predicts $\mu(p) / \mu(n) = -\frac{3}{2}$, which you can ignore in this problem.

\answer{}
See my \textit{Mathematica} notebook for detailed calculations. I also impose the normalized condition to simplify the result. The final results are summarized in the table below:
\begin{itemize}
    \item $p$: $c_f+\frac{1}{3}c_d$
    \item $n$: $-\frac{2}{3}c_d$
    \item $\Lambda$: $-\frac{1}{3}c_d$
    \item $\Sigma^{+}$: $c_f+\frac{1}{3}c_d$
    \item $\Sigma^{0}$: $\frac{1}{3}c_d$
    \item $\Sigma^{-}$: $-c_f+\frac{1}{3}c_d$
    \item $\Xi^{0}$: $-\frac{2}{3}c_d$
    \item $\Xi^{-}$: $-c_f+\frac{1}{3}c_d$
\end{itemize}
\begin{center}
    \begin{tabular}{|c|c|c|}
    \hline
    Baryon & Predicted Magnetic Moment ($\mu_N$) & Experimental Magnetic Moment ($\mu_N$) \\
    \hline
    $p$ & $\mu(p)$ & $2.793$ \\
    $n$ & $\mu(n)$ & $-1.913$ \\
    $\Lambda$ & $\frac{1}{2} \mu(n)$ & $-0.613$ \\
    $\Sigma^{+}$ & $\mu(p)$ & $2.458$ \\
    $\Sigma^{0}$ & $-\frac{1}{2} \mu(n)$ & $\approx0$\\
    $\Sigma^{-}$ & $-(\mu(p) + \mu(n))$ & $-1.160$ \\
    $\Xi^{0}$ & $\mu(n)$ & $-1.250$ \\
    $\Xi^{-}$ & $-(\mu(p) + \mu(n))$ & $-0.651$ \\
    \hline
    \end{tabular}
\end{center}
\textbf{The predictions are in good agreement with the experimental values, demonstrating the effectiveness of the $\mathrm{SU}(3)_f$ symmetry approach in describing baryon magnetic moments except for $\Sigma^0$ where the experimental value is not well matched.}
\qed