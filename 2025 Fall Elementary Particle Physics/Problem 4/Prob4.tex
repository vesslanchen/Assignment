\question{1}{\textbf{Gell-Mann Okubo for the baryon octet}}
\\
The generators \(t_{a}(a=1, \ldots, 8)\) of \(\mathrm{SU}(3)\) are normalized as \(t_{a}=\frac{\lambda_{a}}{2}\) with \(\operatorname{Tr}\left(t_{a} t_{b}\right)=\frac{1}{2} \delta_{a b}\), where \(\lambda_{a}\) are the Gell-Mann matrices. They satisfy \(\left[t_{a}, t_{b}\right]=i f_{a b c} t_{c}\) and \(\left\{t_{a}, t_{b}\right\}= \frac{1}{3} \delta_{a b} \mathbf{1}+d_{a b c} t_{c}\), where \(f_{a b c}\) are totally antisymmetric and \(d_{a b c}\) are totally symmetric structure constants.


Let \(B\) and \(\bar{B}\) be the baryon octet \(3 \times 3\) traceless matrices, expanded in the generator basis as \(B=B^{i} t_{i}\) and \(\bar{B}=\bar{B}^{i} t_{i}\) where \(B^{i}, \bar{B}^{i}\) are the adjoint components. Define the two bilinear combinations \(O_{A} \equiv[\bar{B}, B]=\bar{B} B-B \bar{B}\) and \(O_{S} \equiv\{\bar{B}, B\}-\frac{2}{3} \mathbf{1} \operatorname{Tr}(\bar{B} B)\).
\begin{itemize}
    \item [(a)]Show that both \(O_{A}\) and \(O_{S}\) are traceless and therefore transform in the adjoint (octet) representation.
    \item [(b)] Expand \(O_{A}\) and \(O_{S}\) in components using the generator basis and show that \(O_{A}=i\left(\bar{B}^{i} B^{j}\right) f_{i j k} t_{k}\) and \(O_{S}=\left(\bar{B}^{i} B^{j}\right) d_{i j k} t_{k}\), so that \(\left(O_{A}\right)^{k}=i f_{i j k} \bar{B}^{i} B^{j}\) and \(\left(O_{S}\right)^{k}= d_{i j k} \bar{B}^{i} B^{j}\).
    \item [(c)]Introduce a flavor-breaking spurion \(H_{8}=H_{8}^{i} t_{i}\), with real components \(H_{8}^{i}\). Construct the two independent \(\mathrm{SU}(3)\)-invariant mass terms:
    \begin{align}
        S_{f}=\left(O_{A}\right)_{b}^{a}\left(H_{8}\right)_{a}^{b}, \quad S_{d}=\left(O_{S}\right)_{b}^{a}\left(H_{8}\right)_{a}^{b}
    \end{align}
    Assuming \(H_{8}\) points in the 8 -direction (i.e. \(H_{8}^{i} \propto \delta_{i 8}\) ), argue that \(S_{f}\) and \(S_{d}\) correspond to the \(f\)-type and \(d\)-type symmetry breaking terms in the baryon mass operator, respectively.
    \item [(d)] Given that an adjoint operator \(O_{8}\) acts on an octet state \(B\) as \(O_{8}(B)=\left[O_{8}, B\right]\), show that the invariant scalars in (c) are equivalent to the matrix elements \(S_{f} \propto \langle\bar{B}| t_{8}|B\rangle \equiv \operatorname{Tr}\left(\bar{B}\left[t_{8}, B\right]\right)\) and \(S_{d} \propto\langle\bar{B}| d_{8 i j} t_{i} t_{j}|B\rangle \equiv \operatorname{Tr}\left(\bar{B}\left[d_{8 i j}\left[t_{i},\left[t_{j}, B\right]\right]\right)\right.\).
     \item [(e)] Hence, argue that for each entry \(B_{i j}\) of the baryon octet matrix, \(S_{f} \propto Y\) and \(S_{d} \propto I(I+1)-Y^{2} / 4\) where \(I, Y\) are the isospin and hypercharge of the baryon \(B\), respectively, thereby reproducing the Gell-Mann-Okubo mass formula for the baryon octet.
\end{itemize}
(Hint: Verify, entrywise, that \(\left[\frac{2}{\sqrt{3}} t_{8}, B\right]=Y B\) and the normalized operator \(\frac{2}{\sqrt{3}} d_{8 i j}\left[t_{i},\left[t_{j}, B\right]\right]+ \frac{1}{3}\left[t_{i},\left[t_{i}, B\right]\right]=\left(I(I+1)-Y^{2} / 4\right) B\) acts diagonally on each baryon field. The \(\left[t_{i},\left[t_{i}, B\right]\right]\) term is the adjoint Casimir ( \(S U(3)\) singlet) which just shifts all octet components uniformly so that the \(\Lambda\) eigenvalue becomes 0 . It can be absorbed into the overall singlet part of the GMO formula. On the diagonal remember \(B_{11}, B_{22}\) and \(B_{33}\) mix \(\Sigma^{0}\) and \(\Lambda\), so \(\left.B_{\text {diag }}=\Sigma^{0} \operatorname{diag}(1,-1,0) / \sqrt{2}+\Lambda \operatorname{diag}(1,1,-2) / \sqrt{6}\right)\).
\answer{}
\begin{itemize}
    \item [(a)]
\end{itemize}
To show that both \(O_{A}\) and \(O_{S}\) are traceless, we first understand their definitions:
\begin{align}
O_{A} &= [\bar{B}, B] = \bar{B} B - B \bar{B}=[\bar{B}, B]\\
      &= [\bar{B}^{i} t_{i}, B^{j} t_{j}] = \bar{B}^{i} B^{j} [t_{i}, t_{j}] = i \bar{B}^{i} B^{j} f_{ijk} t_{k}
\end{align}
Taking the trace of \(O_{A}\):
\begin{align}
\operatorname{Tr}(O_{A}) &= \operatorname{Tr}(iBar{B}^{i} B^{j} f_{ijk} t_{k}) = i \bar{B}^{i} B^{j} f_{ijk} \operatorname{Tr}(t_{k}) = 0.
\end{align}
Similarly, for \(O_{S}\):
\begin{align}
O_{S} &= \{\bar{B}, B\} - \frac{2}{3} \mathbf{1} \operatorname{Tr}(\bar{B} B) = \bar{B} B + B \bar{B} - \frac{2}{3} \mathbf{1} \operatorname{Tr}(\bar{B} B)\\
      &= (\bar{B}^{i} t_{i})(B^{j} t_{j}) + (B^{j} t_{j})(\bar{B}^{i} t_{i}) - \frac{2}{3} \mathbf{1} \operatorname{Tr}(\bar{B}^{i} t_{i} B^{j} t_{j})\\
      &= \bar{B}^{i} B^{j}\{t_{i}, t_{j}\} - \frac{2}{3} \mathbf{1} \operatorname{Tr}(\bar{B}^{i} B^{j} t_{i} t_{j})\\
      &= \bar{B}^{i} B^{j}\left(\frac{1}{3}\delta_{ij}\mathbf{1}+d_{ijk}t_{k}\right) - \frac{2}{3}\mathbf{1}\left(\frac{1}{2}\bar{B}^{i} B^{j}\delta_{ij}\right)\\
      &= \bar{B}^{i} B^{j} d_{ijk} t_{k}
\end{align}


\clearpage
\question{2}{\textbf{$\rho$-$\omega$ mixing}}
\\
The vector mesons $\rho(770)$ and $\omega(782)$ are very close in mass. For this reason the effects of isospin violation are somewhat enhanced in these mesons and can be parametrized in terms of $\rho$-$\omega$ mixing. Namely, the physical $\rho^{0}$ and $\omega$ mesons can be viewed as orthogonal mixed states of a pure isospin triplet and isospin singlet:
\[
\rho^{0} = \cos\theta \frac{(u\bar{u} - d\bar{d})}{\sqrt{2}} + \sin\theta \frac{(u\bar{u} + d\bar{d})}{\sqrt{2}},
\]
\[
\omega = -\sin\theta \frac{(u\bar{u} - d\bar{d})}{\sqrt{2}} + \cos\theta \frac{(u\bar{u} + d\bar{d})}{\sqrt{2}},
\]
where $\theta$ is a (small) mixing angle.

\begin{enumerate}
\item[(a)] Determine $\theta$ (up to a sign) using experimental data on the decay $\omega \rightarrow \pi^{+} \pi^{-}$. Estimate the error in the value of the mixing angle.
\item[(b)] Using the value of $\theta$ predict the decay rates $\Gamma(\rho^{0} \rightarrow e^{+} e^{-})$ and $\Gamma(\omega \rightarrow e^{+} e^{-})$, assuming the amplitude for a quark pair annihilation into an $e^{+} e^{-}$ pair is proportional to the electric charge $Q$ of the quark.
\item[(c)] Assume that the transition amplitude between different spin states of a $q\bar{q}$ quark pair with emission of a photon: $(q\bar{q}) \rightarrow (q\bar{q}) + \gamma$ is proportional to the quark electric charge $Q$. Use the value of the $\rho$-$\omega$ mixing angle $\theta$ to determine the ratios of the decay rates:
\begin{enumerate}
\item[(i)] $\Gamma(\rho^{0} \rightarrow \pi^{0} \gamma) / \Gamma(\omega^{0} \rightarrow \pi^{0} \gamma)$,
\item[(ii)] $\Gamma(\rho^{0} \rightarrow \eta \gamma) / \Gamma(\omega^{0} \rightarrow \eta \gamma)$.
\end{enumerate}
Compare with the PDG experimental data. How does the inclusion of $\rho$-$\omega$ mixing improve the agreement with the data?
\end{enumerate}

\answer{}
\clearpage

\question{3}{\textbf{Baryon magnetic moments}}
\\
The octet of spin-$\frac{1}{2}$ baryons has magnetic moments $\mu$. The operator that describes the magnetic moment is an $\mathrm{SU}(3)_{f}$ octet operator which is proportional to the quark charge $Q$. The charge
\[
Q = t_{3} + \frac{1}{\sqrt{3}} t_{8}
\]
is traceless ($\operatorname{Tr} Q = 0$) and can be promoted to a purely $\mathrm{SU}(3)_{f}$ octet spurion $\mathbf{8}_{Q}$ (with no singlet piece, as in contrast to the GMO mass formula). Hence, when determining the baryon magnetic moment
\[
\mu(B) = \langle \bar{B} | \mu | B \rangle \propto \mathbf{8}_{\bar{B}} \times \mathbf{8}_{Q} \times \mathbf{8}_{B}
\]
there are two independent octet structures (the $f$- and $d$-type couplings, as for the baryon mass), given by
\[
\mu(B) = c_{f} \operatorname{Tr}(B^{\dagger} [Q, B]) + c_{d} \operatorname{Tr}(B^{\dagger} \{Q, B\}) = \alpha_{+} \operatorname{Tr}(B B^{\dagger} Q) + \alpha_{-} \operatorname{Tr}(B^{\dagger} B Q),
\]
where $\alpha_{+} \equiv c_{d} + c_{f}$, $\alpha_{-} \equiv c_{d} - c_{f}$ are arbitrary constants and
\[
B = \begin{pmatrix}
\frac{\Sigma^{0}_{u}}{\sqrt{2}} + \frac{\Lambda}{\sqrt{6}} & \Sigma^{+} & p \\
\Sigma^{-} & -\frac{\Sigma^{0}_{u}}{\sqrt{2}} + \frac{\Lambda}{\sqrt{6}} & n \\
-\Xi^{-} & \Xi^{0} & -\frac{2\Lambda}{\sqrt{6}}
\end{pmatrix}, \quad
Q = \begin{pmatrix}
\frac{2}{3} & 0 & 0 \\
0 & -\frac{1}{3} & 0 \\
0 & 0 & -\frac{1}{3}
\end{pmatrix}.
\]

Determine all the spin-$\frac{1}{2}$ baryon magnetic moments in terms of $\mu(p)$ and $\mu(n)$ (by eliminating $c_{f,d}$ or $\alpha_{\pm}$) and compare with the PDG experimental values. These predictions were first worked out by Coleman and Glashow in 1961. Note that imposing the full $\mathrm{SU}(6)$ spin-flavor symmetry further predicts $\mu(p) / \mu(n) = -\frac{3}{2}$, which you can ignore in this problem.

\answer{}