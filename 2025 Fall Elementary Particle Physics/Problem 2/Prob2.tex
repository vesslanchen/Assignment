\section*{Problem Set 2 due 9:30 AM, Monday, September 29th}

\question{1}{\textbf{The $\tau-\theta$ Puzzle}}

In the $1950$’s, two particles $\tau$, $\theta$ were discovered with the same mass and lifetime that decayed differently. At the time, physicists believed that parity was conserved in all interactions.
\begin{itemize}
    \item [(a)] Consider the decay $\theta \to \pi +\pi^0$. Assuming parity invariance and zero for the spin of $\theta$, find the parity of $\theta$.
    \item [(b)] Now consider the decay process $\tau\to\pi^+\pi^+\pi^-$. (This is an old symbol for the $K$ meson.) Let $l$ be the orbital angular momentum of $\pi^+\pi^+$ and $l'$ the orbital angular momentum of $\pi^-$ relative to the center-of-mass of $\pi^+\pi^+$. Assuming parity invariance and the spin of $\tau$ equal to zero, find its parity.
    \item [(c)] What resolved the $\tau-\theta$ puzzle?
\end{itemize}

\clearpage
\question{2}{}
List all applicable conservation laws that are or would be violated in the following decays: 
\begin{enumerate}
    \item $\rho^0\to\pi^0\pi^0$
    \item $\rho\to\gamma\gamma$
    \item $K^+\to\pi^+\pi^0$
    \item $\pi^0\to5\gamma$
\end{enumerate}
(Look up the corresponding parities from the Particle Data Group at http://pdg.lbl.gov.)

\clearpage
\question{3}{}
List all states ($J^{PC}$) with total spin $J = 0, 1, 2$ and $P, C$ parities that cannot be realized as a fermion-antifermion system (i.e., as $e^+e^-$ or quark-antiquark). (Hypothetical particles with such combinations of quantum numbers are called exotic, and are being sought for in experiments, so far unsuccessfully.)

\clearpage
\question{4}{}
State which of the following combinations can or cannot exist in a state of isospin $I=1$, and give the reasons: 
\begin{enumerate}
    \item $\pi^0\pi^0$
    \item $\pi^+\pi^-$
    \item $\Sigma^0\pi^0$
    \item $\Lambda\pi^0$
\end{enumerate}

