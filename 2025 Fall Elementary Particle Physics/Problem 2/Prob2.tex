\section*{Problem Set 2 due 9:30 AM, Monday, September 29th}

\question{1}{\textbf{The $\tau-\theta$ Puzzle}}

In the $1950$’s, two particles $\tau$, $\theta$ were discovered with the same mass and lifetime that decayed differently. At the time, physicists believed that parity was conserved in all interactions.
\begin{itemize}
    \item [(a)] Consider the decay $\theta \to \pi^+\pi^0$. Assuming parity invariance and zero for the spin of $\theta$, find the parity of $\theta$.
    \item [(b)] Now consider the decay process $\tau\to\pi^+\pi^+\pi^-$. (This is an old symbol for the $K$ meson.) Let $l$ be the orbital angular momentum of $\pi^+\pi^+$ and $l'$ the orbital angular momentum of $\pi^-$ relative to the center-of-mass of $\pi^+\pi^+$. Assuming parity invariance and the spin of $\tau$ equal to zero, find its parity.
    \item [(c)] What resolved the $\tau-\theta$ puzzle?
\end{itemize}
\answer{}
\begin{itemize}
    \item [(a)] 
\end{itemize}
First, we note that the intrinsic parity of a pion is $-1$. The parity of a two-particle system is given by
\begin{align}
    P_\theta = P_1 P_2 (-1)^l,
\end{align}
where $P_1$ and $P_2$ are the intrinsic parities of the two particles, and $l$ is their relative orbital angular momentum. Since the $\theta$ and pions have spin $0$, the system of two pions must have total angular momentum $j=s+l=0$, in order to satisfy the conservation of total angular momentum. This means that $l$ must be 0, too. Therefore, we have $P_\theta = 1$. It is even parity.
\begin{itemize}
    \item [(b)]
\end{itemize}
The parity of a three-particle system is given by
\begin{align}
    P_\tau = P_1 P_2 P_3 (-1)^{l+l'},
\end{align}
where $P_1$, $P_2$, and $P_3$ are the intrinsic parities of the three particles, and $l$ and $l'$ are their relative orbital angular momenta. Since the $\tau$ and pions have spin $0$, the system of three pions must have total angular momentum $j=s+l+l'=0$, in order to satisfy the conservation of total angular momentum. This means that $l+l'$ must be 0, too. Therefore, we have $P_\tau = -1$. It is odd parity.
\begin{itemize}
    \item [(c)]
\end{itemize}
Since the $\tau$ and $\theta$ have the same mass and lifetime, they are actually the same particle, now known as the $K$ meson. The resolution of the $\tau-\theta$ puzzle was the discovery that parity is not conserved in weak interactions, which is how the $K$ meson decays. \qed
\clearpage
\question{2}{}
List all applicable conservation laws that are or would be violated in the following decays: 
\begin{enumerate}
    \item $\rho^0\to\pi^0\pi^0$
    \item $\rho\to\gamma\gamma$
    \item $K^+\to\pi^+\pi^0$
    \item $\pi^0\to5\gamma$
\end{enumerate}
(Look up the corresponding parities from the Particle Data Group at http://pdg.lbl.gov.)
\answer{}
\begin{enumerate}
    \item The decay $\rho^0\to\pi^0\pi^0$ violates Bose-Einstein statistics. The $\rho^0$ has quantum numbers $J^{PC}=1^{--}$, while the two $\pi^0$ system has quantum numbers $J^{PC}=0^{++}, 2^{++}, 4^{++}, \ldots$. Therefore, the decay is forbidden.
\end{enumerate}
\clearpage
\question{3}{}
List all states ($J^{PC}$) with total spin $J = 0, 1, 2$ and $P, C$ parities that cannot be realized as a fermion-antifermion system (i.e., as $e^+e^-$ or quark-antiquark). (Hypothetical particles with such combinations of quantum numbers are called exotic, and are being sought for in experiments, so far unsuccessfully.)
\answer{}
\clearpage
\question{4}{}
State which of the following combinations can or cannot exist in a state of isospin $I=1$, and give the reasons: 
\begin{enumerate}
    \item $\pi^0\pi^0$
    \item $\pi^+\pi^-$
    \item $\Sigma^0\pi^0$
    \item $\Lambda\pi^0$
\end{enumerate}
\answer{}
