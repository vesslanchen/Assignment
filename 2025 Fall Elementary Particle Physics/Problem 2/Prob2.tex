\section*{Problem Set 2 due 9:30 AM, Monday, September 29th}

\question{1}{\textbf{The $\tau-\theta$ Puzzle}}

In the $1950$’s, two particles $\tau$, $\theta$ were discovered with the same mass and lifetime that decayed differently. At the time, physicists believed that parity was conserved in all interactions.
\begin{itemize}
    \item [(a)] Consider the decay $\theta \to \pi^+\pi^0$. Assuming parity invariance and zero for the spin of $\theta$, find the parity of $\theta$.
    \item [(b)] Now consider the decay process $\tau\to\pi^+\pi^+\pi^-$. (This is an old symbol for the $K$ meson.) Let $l$ be the orbital angular momentum of $\pi^+\pi^+$ and $l'$ the orbital angular momentum of $\pi^-$ relative to the center-of-mass of $\pi^+\pi^+$. Assuming parity invariance and the spin of $\tau$ equal to zero, find its parity.
    \item [(c)] What resolved the $\tau-\theta$ puzzle?
\end{itemize}
\answer{}
\begin{itemize}
    \item [(a)] 
\end{itemize}
First, we note that the intrinsic parity of a pion is $-1$. The parity of a two-particle system is given by
\begin{align}
    P_\theta = P_1 P_2 (-1)^l,
\end{align}
where $P_1$ and $P_2$ are the intrinsic parities of the two particles, and $l$ is their relative orbital angular momentum. Since the $\theta$ and pions have spin $0$, the system of two pions must have total angular momentum $j=s+l=0$, in order to satisfy the conservation of total angular momentum. This means that $l$ must be 0, too. Therefore, we have $P_\theta = 1$. It is even parity.
\begin{itemize}
    \item [(b)]
\end{itemize}
The parity of a three-particle system is given by
\begin{align}
    P_\tau = P_1 P_2 P_3 (-1)^{l+l'},
\end{align}
where $P_1$, $P_2$, and $P_3$ are the intrinsic parities of the three particles, and $l$ and $l'$ are their relative orbital angular momenta. Since the $\tau$ and pions have spin $0$, the system of three pions must have total angular momentum $j=s+l+l'=0$, in order to satisfy the conservation of total angular momentum. This means that $l+l'$ must be 0, too. Therefore, we have $P_\tau = -1$. It is odd parity.
\begin{itemize}
    \item [(c)]
\end{itemize}
Since the $\tau$ and $\theta$ have the same mass and lifetime, they are actually the same particle, now known as the $K$ meson. The resolution of the $\tau-\theta$ puzzle was the discovery that parity is not conserved in weak interactions, which is how the $K$ meson decays. \qed
\clearpage
\question{2}{}
List all applicable conservation laws that are or would be violated in the following decays: 
\begin{enumerate}
    \item $\rho^0\to\pi^0\pi^0$
    \item $\rho\to\gamma\gamma$
    \item $K^+\to\pi^+\pi^0$
    \item $\pi^0\to5\gamma$
\end{enumerate}
(Look up the corresponding parities from the Particle Data Group at http://pdg.lbl.gov.)
\answer{}
Before we analyze each decay, we list the conservation laws that we will check for each decay:
\begin{itemize}
    \item Conservation of electric charge
    \item Conservation of angular momentum (total angular momentum $J$).
    \item Conservation of isospin
    \item Conservation of parity
    \item Conservation of C-parity
    \item Conservation of G-parity
\end{itemize}
\begin{enumerate}
    \item The $\rho^0$ has quantum numbers $I^G(J^{PC})=1^+(1^{--})$, the $\pi^0$ has quantum numbers $I^G(J^{PC})=1^-(0^{-+})$. 
    \begin{itemize}
        \item Electric charge: The $\rho^0$ has charge $0$, and the two $\pi^0$'s have charge $0+0=0$. Electric charge is conserved.
        \item Angular momentum: The $\rho^0$ has spin $1$, and the two $\pi^0$'s have spin $0\otimes0=0$. To conserve total angular momentum, the two-pion system must have orbital angular momentum $l=1$. Total angular momentum is conserved.
        \item Isospin: The $\rho^0$ has isospin $I=1$, and the two $\pi^0$'s can form isospin $I=1\otimes1=0, 1, 2$. Therefore, the decay can proceed through the $I=1$ channel. Isospin is conserved.
        \item Parity: The parity of the two-pion system is given by
        \begin{align}
            P_\rho = P_\pi P_\pi (-1)^l = (-1)(-1)(-1)^l = (-1)^l.
        \end{align}
        Since the $\rho^0$ has spin $1$ and the pions have spin $0$, the two-pion system must have orbital angular momentum $l=1$ to conserve total angular momentum. Therefore, the parity of the two-pion system is $P=-1$, which massches the parity of the $\rho^0$. Parity is conserved.
        \item C-parity: The C-parity of the two-pion system is given by
        \begin{align}
            C_\rho = C_\pi C_\pi (-1)^l = (+1)(+1)(-1)^l = (-1)^l.
        \end{align}
        Since the $\rho^0$ has spin $1$ and the pions have spin $0$, the two-pion system must have orbital angular momentum $l=1$ to conserve total angular momentum. Therefore, the C-parity of the two-pion system is $C=-1$, which matches the C-parity of the $\rho^0$. C-parity is conserved.
        \item G-parity: The G-parity of the two-pion system is given by
        \begin{align}
            G_\rho = G_\pi G_\pi  = (-1)(-1) = 1.
        \end{align}
        Hence, the G parity is $G=+1$, which matches the G-parity of the $\rho^0$. G-parity is conserved.
    \end{itemize}
    All conservation laws are satisfied.\\
    \textbf{Remark:} If we check the decay mode of $\rho\to\pi\pi$, we find that the branch ratio is close to $100\%$. This is consistent with our analysis that the decay can occur.
    \item The $\rho$ has quantum numbers $I^G(J^{PC})=1^+(1^{--})$, the photon has quantum numbers $I^G(J^{PC})=0^-(1^{--})$. \\
    \textbf{Remark:} Actually, PDG show that the photon has isospin $I=0,1$. Here, I choose $I=0$ because the photon do not involve in strong interactions. In other words, the photon is a singlet under the strong interaction. I think $I=1$ case is for the weak interaction, but I am not sure. 
    \begin{itemize}
        \item Electric charge: The $\rho$ has charge $0$, and the two photons have charge $0+0=0$. Electric charge is conserved.
        \item Angular momentum: The $\rho$ has spin $1$, and the two photons have spin $s=1\otimes1=0, 1, 2$. To conserve total angular momentum, the two-photon system must have orbital angular momentum $l=0 (\text{when }s=1), l=1 (\text{when }s=0,1,2), l=2(\text{when }s=1)$. Hence, the total angular momentum might be conserved. However, by the \textbf{Landau-Yang theorem}, a massive spin-1 particle cannot decay into two photons. Therefore, the decay cannot occur. \textbf{Angular momentum is not conserved}.
        \item Isospin: The $\rho$ has isospin $I=1$, and the two photons can couple to isospin $I=0$. Therefore, the decay cannot proceed through any isospin channel. \textbf{Isospin is not conserved}.
        \item Parity: The parity of the two-photon system is given by
        \begin{align}
            P_\rho = P_\gamma P_\gamma (-1)^l = (-1)(-1)(-1)^l = (-1)^l.
        \end{align}
        By checking the possible values of $l$ above, we know the $l$ should be $1$ to conserve the parity.
        \item C-parity: The C-parity of the two-photon system is given by
        \begin{align}
            C_\gamma C_\gamma = (-1)(-1) = 1\neq -1 = C_\rho.
        \end{align}
        Hence, the C parity is $C=+1$, which does not match the C-parity of the $\rho$. \textbf{C-parity is not conserved}.
    \end{itemize}
    The angular momentum, isospin, and C-parity are not conserved.\\
    \textbf{Remark:} If we check the decay mode of $\rho\to\gamma\gamma$, we find that the branch ratio is $0\%$.
    \item The $K^+$ has quantum numbers $I(J^P)=\frac{1}{2}(0^-)$, the $\pi^+$ has quantum numbers $I(J^P)=1(0^-)$, and the $\pi^0$ has quantum numbers $I(J^P)=1(0^-)$.
    \begin{itemize}
        \item Electric charge: The $K^+$ has charge $+1$, and the two pions have charge $1+0=+1$. Electric charge is conserved.
        \item Angular momentum: The $K^+$ has spin $0$, and the two pions have spin $0\otimes0=0$. To conserve total angular momentum, the two-pion system must have orbital angular momentum $l=0$. Total angular momentum is conserved.
        \item Isospin: The $K^+$ has isospin $I=\frac{1}{2}$, and the two pions can form isospin $I=0, 1, 2$. \textbf{Therefore, the decay cannot proceed through any isospin channel}.
        \item Parity: The parity of the two-pion system is given by
        \begin{align}
            -1=P_K = P_\pi P_\pi (-1)^l = (-1)(-1)(-1)^l = (-1)^l.
        \end{align}
        Since the $K^+$ has spin $0$ and the pions have spin $0$, the two-pion system must have orbital angular momentum $l=0$ to conserve total angular momentum. Therefore, the parity of the two-pion system is $P=+1$, which does not match the parity of the $K^+$. \textbf{Parity is not conserved}.
        \item C-parity: Not applicable, since the particles are not neutral.
    \end{itemize}
    The isospin and parity are not conserved.\\
    \textbf{Remark:} If we check the decay mode of $K^+\to\pi^+\pi^0$, we find that the branch ratio is $21.13\%$. This is a \textbf{weak decay}, in which isospin and parity are not conserved.
    \item The $\pi^0$ has quantum numbers $I^G(J^{PC})=1^-(0^{-+})$, the photon has quantum numbers $I^G(J^{PC})=0^-(1^{--})$.
    \begin{itemize}
        \item Electric charge: The $\pi^0$ has charge $0$, and the five photons have charge $0+0+0+0+0=0$. Electric charge is conserved.
        \item Angular momentum: The $\pi^0$ has spin $0$, and the five photons can have total spin $1,2,3,4,5$. To conserve total angular momentum, the five-photon system must have orbital angular momentum $l=1,2,3,4,5$ to form the correct combinations. Hence, the total angular momentum might be conserved.
        \item Isospin: The $\pi^0$ has isospin $I=1$, and the five photons can form isospin $I=0$. \textbf{Therefore, the decay cannot proceed through any isospin channel}.
        \item Parity: The parity of the five-photon system is given by
        \begin{align}
            -1=P_\pi = P_\gamma P_\gamma P_\gamma P_\gamma P_\gamma (-1)^l = -1\times(-1)^l.
        \end{align}
        By checking the possible values of $l$ above, we know the $l$ should be $2,4$ to conserve the parity.
        \item C-parity: The C-parity of the five-photon system is given by
        \begin{align}
            +1=C_\pi \neq C_\gamma C_\gamma C_\gamma C_\gamma C_\gamma = (-1)^5 = -1,
        \end{align}
        Hence, the C parity is $C=-1$, which does not match the C-parity of the $\pi^0$. \textbf{C-parity is not conserved}.
    \end{itemize}
    The isospin and C-parity are not conserved.\\
    \textbf{Remark:} If we check the decay mode of $\pi^0\to5\gamma$, we find that the branch ratio is $0\%$. The dominant decay mode is $\pi^0\to2\gamma$, which has a branch ratio of $98.823\%$. This is consistent with our analysis that the decay cannot occur.
\end{enumerate}\qed
\clearpage
\question{3}{}
List all states $(J^{PC})$ with total spin $J = 0, 1, 2$ and $P, C$ parities that cannot be realized as a fermion-antifermion system (i.e., as $e^+e^-$ or quark-antiquark). (Hypothetical particles with such combinations of quantum numbers are called exotic, and are being sought for in experiments, so far unsuccessfully.)
\answer{}
First we note that a fermion-antifermion system has the following properties:
\begin{itemize}
    \item The intrinsic parity of a fermion is $+1$, and the intrinsic parity of an antifermion is $-1$. Therefore, the intrinsic parity of a fermion-antifermion system is $-1$. Hence the parity of a fermion-antifermion system is given by
    \begin{align}
        P = P_f P_{\bar{f}} (-1)^l = -(-1)^l = (-1)^{l+1},
    \end{align}
    \item The C-parity of a fermion-antifermion system is given by
    \begin{align}
        C =  (-1)^{l+s},
    \end{align}
    where $s$ is the total spin of the fermion-antifermion system, which can be $0$ or $1$.
\end{itemize}
Based on the above properties, we can list all possible states $(J^{PC})$ with total spin $J = 0, 1, 2$ for a fermion-antifermion system:
\begin{itemize}
    \item For $J=0$:
    \begin{itemize}
        \item When $l=0$, $s=0$: $P=(-1)^{0+1}=-1$, $C=(-1)^{0+0}=+1$, so $J^{PC}=0^{-+}$.
        \item When $l=1$, $s=1$: $P=(-1)^{1+1}=+1$, $C=(-1)^{1+1}=+1$, so $J^{PC}=0^{++}$.
    \end{itemize}
    \item For $J=1$:
    \begin{itemize}
        \item When $l=0$, $s=1$: $P=(-1)^{0+1}=-1$, $C=(-1)^{0+1}=-1$, so $J^{PC}=1^{--}$.
        \item When $l=1$, $s=0$: $P=(-1)^{1+1}=+1$, $C=(-1)^{1+0}=-1$, so $J^{PC}=1^{+-}$.
        \item When $l=1$, $s=1$: $P=(-1)^{1+1}=+1$, $C=(-1)^{1+1}=+1$, so $J^{PC}=1^{++}$.
        \item When $l=2$, $s=1$: $P=(-1)^{2+1}=-1$, $C=(-1)^{2+1}=-1$, so $J^{PC}=1^{--}$.
    \end{itemize}
    \item For $J=2$:
    \begin{itemize}
        \item When $l=1$, $s=1$: $P=(-1)^{1+1}=+1$, $C=(-1)^{1+1}=+1$, so $J^{PC}=2^{++}$.
        \item When $l=2$, $s=0$: $P=(-1)^{2+1}=-1$, $C=(-1)^{2+0}=+1$, so $J^{PC}=2^{-+}$.
        \item When $l=2$, $s=1$: $P=(-1)^{2+1}=-1$, $C=(-1)^{2+1}=-1$, so $J^{PC}=2^{--}$.
        \item When $l=3$, $s=1$: $P=(-1)^{3+1}=+1$, $C=(-1)^{3+1}=+1$, so $J^{PC}=2^{++}$.
    \end{itemize} 
    Therefore, the states $(J^{PC})$ with total spin $J = 0, 1, 2$ and $P, C$ parities that cannot be realized as a fermion-antifermion system are:
    \begin{itemize}
        \item $0^{+-}$, $0^{--}$
        \item $1^{-+}$
        \item $2^{+-}$
    \end{itemize}
\end{itemize}\qed
\clearpage
\question{4}{}
State which of the following combinations can or cannot exist in a state of isospin $I=1$, and give the reasons: 
\begin{enumerate}
    \item $\pi^0\pi^0$
    \item $\pi^+\pi^-$
    \item $\pi^+\pi^+$
    \item $\Sigma^0\pi^0$
    \item $\Lambda\pi^0$
\end{enumerate}
\answer{}
First, we note the isospin quantum numbers of the particles involved:
\begin{itemize}
    \item The $\pi^0$ has isospin $I=1$, $I_3=0$.
    \item The $\pi^+$ has isospin $I=1$, $I_3=+1$.
    \item The $\pi^-$ has isospin $I=1$, $I_3=-1$.
    \item The $\Sigma^0$ has isospin $I=1$, $I_3=0$.
    \item The $\Lambda$ has isospin $I=0$, $I_3=0$.
\end{itemize}
Now we analyze each combination:
\begin{enumerate}
    \item $\pi^0\pi^0$: The two $\pi^0$'s can form isospin $I=0, 1, 2$. Therefore, the combination can exist in a state of isospin $I=1$. But since the two pions are identical bosons, their total wavefunction must be symmetric under exchange. When the isospin state is $I=1$ (which is antisymmetric), the spatial part must be antisymmetric (odd orbital angular momentum) to make the total wavefunction symmetric. Hence, the combination can exist in a state of isospin $I=1$ \textbf{with odd orbital angular momentum} $L=1,3,5$ and so on.\\
    \textbf{Remark:} I also check the C-G coefficients, and find that the state $|I=1, I_3=0\rangle$ is given by
    \begin{align}
        |1,0\rangle = \frac{1}{\sqrt{2}}(|\pi^+\pi^-\rangle - |\pi^-\pi^+\rangle), \\
        |\pi^0\pi^0\rangle = \sqrt{\frac{2}{3}}|2,0\rangle - \sqrt{\frac{1}{3}}|0,0\rangle.
    \end{align}
     The $|1,0\rangle $ state does not contain the $|\pi^0\pi^0\rangle$ component. This means that the two $\pi^0$'s cannot form isospin $I=1$ state. This is consistent with our analysis above that the two $\pi^0$'s can exist in a state of isospin $I=1$. In my opinion, I think we cannot consider orbital angular momentum when we analyze the isospin state. Therefore, I think the two $\pi^0$'s cannot form isospin $I=1$ state. But I am not sure about this point.
    \item $\pi^+\pi^-$: The $\pi^+$ and $\pi^-$ can form isospin $I=0, 1, 2$. Therefore, the combination can exist in a state of isospin $I=1$. Since the two pions are not identical particles, there is no symmetry requirement on their total wavefunction. Hence, the combination can exist in a state of isospin $I=1$.\\
    \textbf{Remark:} I also check the C-G coefficients, and find that the state $|I=1, I_3=0\rangle$ is given by
    \begin{align}
        |1,0\rangle = \frac{1}{\sqrt{2}}(|\pi^+\pi^-\rangle - |\pi^-\pi^+\rangle).
    \end{align}
    The $|1,0\rangle $ state contains the $|\pi^+\pi^-\rangle$ component. This means that the $\pi^+$ and $\pi^-$ can form isospin $I=1$ state. This is consistent with our analysis above.
    \item $\pi^+\pi^+$: The two $\pi^+$'s can only form isospin $I=2$ since $I_3=+2$. Therefore, the combination cannot exist in a state of isospin $I=1$.\\
    \textbf{Remark:} I also check the C-G coefficients, and find that the state $|I=2, I_3=2\rangle$ is given by
    \begin{align}
        |2,2\rangle = |\pi^+\pi^+\rangle.
    \end{align}
    The $|2,2\rangle $ state contains the $|\pi^+\pi^+\rangle$ component. This means that the two $\pi^+$'s can only form isospin $I=2$ state. This is consistent with our analysis above.
    \item $\Sigma^0\pi^0$: The $\Sigma^0$ has isospin $I=1$, and the $\pi^0$ has isospin $I=1$. For the same realized reason and the C-G coefficients in part (1), this cannot exist in a state of isospin $I=1$.\\
    \textbf{Remark:} I also check the C-G coefficients, and find that the state $|I=1, I_3=0\rangle$ is given by
    \begin{align}
        |1,0\rangle = \frac{1}{\sqrt{2}}(|\Sigma^+\pi^-\rangle - |\Sigma^-\pi^+\rangle), \\
        |\Sigma^0\pi^0\rangle = \sqrt{\frac{2}{3}}|2,0\rangle - \sqrt{\frac{1}{3}}|0,0\rangle.
    \end{align}
    The $|1,0\rangle $ state does not contain the $|\Sigma^0\pi^0\rangle$ component. This means that the $\Sigma^0$ and $\pi^0$ cannot form isospin $I=1$ state. This is consistent with our analysis above.
    \item $\Lambda\pi^0$: The $\Lambda$ has isospin $I=0$, and the $\pi^0$ has isospin $I=1$. The combination can only form isospin $I=1$. Therefore, the combination can exist in a state of isospin $I=1$.\\
    \textbf{Remark:} I also check the C-G coefficients, and find that the state $|I=1, I_3=0\rangle$ is given by
    \begin{align}
        |1,0\rangle = |\Lambda\pi^0\rangle.
    \end{align}
    The $|1,0\rangle $ state contains the $|\Lambda\pi^0\rangle$ component. This means that the $\Lambda$ and $\pi^0$ can form isospin $I=1$ state. This is consistent with our analysis above.\qed
\end{enumerate}