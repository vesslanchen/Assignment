\section*{Problem Set 3 due 9:30 AM, Monday, October 13th}


\question{1}{}\textbf{p-d reactions}\\
Consider the reactions
\begin{align}
p+d \rightarrow \pi^{+}+{ }^3 \mathrm{H}, \quad p+d \rightarrow \pi^0+{ }^3 \mathrm{He} .    
\end{align}
Since the deuteron is in a ${ }^3 S_1$ state, it must be an isospin singlet. Therefore, the initial state $p+d$ is a pure $I=\frac{1}{2}$ state. Given that ${ }^3 \mathrm{H}$ and ${ }^3 \mathrm{He}$ form an isodoublet, write down the isospin decomposition of the final states, and from this, the ratio of the two cross sections.
\answer{}
First, we can use $I_3$ to decide the isospin for ${}^3 \mathrm{H}$ and ${}^3 \mathrm{He}$. See the initial state $p+d$ has $I_3=+\frac{1}{2}$, so the final state must also have $I_3=+\frac{1}{2}$. Since $\pi^+$ has $I_3=+1$ and $\pi^0$ has $I_3=0$, we can conclude that ${}^3 \mathrm{H}$ has $I_3=-\frac{1}{2}$ and ${}^3 \mathrm{He}$ has $I_3=+\frac{1}{2}$. Therefore, ${}^3 \mathrm{H}$ and ${}^3 \mathrm{He}$ form an isodoublet with $I=\frac{1}{2}$.
Now we can write down the isospin decomposition of the final states. For the first reaction, we have
\begin{align}
|\pi^+ + {}^3 \mathrm{H}\rangle = |\pi^+ \rangle \otimes |{}^3 \mathrm{H}\rangle = |1,1\rangle \otimes |\tfrac{1}{2}, -\tfrac{1}{2}\rangle.
\end{align}
Using the Clebsch-Gordan coefficients, we can decompose this into total isospin
\begin{align}
|1,1\rangle \otimes |\tfrac{1}{2}, -\tfrac{1}{2}\rangle = \tfrac{1}{\sqrt{3}} |\tfrac{3}{2}, \tfrac{1}{2}\rangle + \sqrt{\tfrac{2}{3}} |\tfrac{1}{2}, \tfrac{1}{2}\rangle.
\end{align}
For the second reaction, we have
\begin{align}
|\pi^0 + {}^3 \mathrm{He}\rangle = |\pi^0 \rangle \otimes |{}^3 \mathrm{He}\rangle = |1,0\rangle \otimes |\tfrac{1}{2}, \tfrac{1}{2}\rangle.
\end{align}
Using the Clebsch-Gordan coefficients, we can decompose this into total isospin
\begin{align}
|1,0\rangle \otimes |\tfrac{1}{2}, \tfrac{1}{2}\rangle = \sqrt{\tfrac{2}{3}} |\tfrac{3}{2}, \tfrac{1}{2}\rangle - \tfrac{1}{\sqrt{3}} |\tfrac{1}{2}, \tfrac{1}{2}\rangle.
\end{align}
Now, since the initial state $p+d$ is a pure $I=\frac{1}{2}$ state, only the $I=\frac{1}{2}$ component of the final states will contribute to the cross sections. Therefore, we can write the amplitudes for the two reactions as
\begin{align}
\mathcal{A}(p+d \rightarrow \pi^+ + {}^3 \mathrm{H}) &\propto \sqrt{\tfrac{2}{3}}, \\
\mathcal{A}(p+d \rightarrow \pi^0 + {}^3 \mathrm{He}) &\propto -\tfrac{1}{\sqrt{3}}.
\end{align}
The cross sections are proportional to the square of the amplitudes, so we have
\begin{align}
\sigma(p+d \rightarrow \pi^+ + {}^3 \mathrm{H}) &\propto \left|\sqrt{\tfrac{2}{3}}\right|^2 = \tfrac{2}{3}, \\
\sigma(p+d \rightarrow \pi^0 + {}^3 \mathrm{He}) &\propto \left|-\tfrac{1}{\sqrt{3}}\right|^2 = \tfrac{1}{3}.
\end{align}
Finally, the ratio of the two cross sections is
\begin{align}
\frac{\sigma(p+d \rightarrow \pi^+ + {}^3 \mathrm{H})}{\sigma(p+d \rightarrow \pi^0 + {}^3 \mathrm{He})} = \frac{\tfrac{2}{3}}{\tfrac{1}{3}} = 2.
\end{align}
\qed

\clearpage
\question{2}{}
\textbf{Particle production by strong interactions}\\
Explain why the processes $\pi^{-}+p \rightarrow \pi^{+}+\Sigma^{-}, \quad \pi^{-}+p \rightarrow K^0+n, \quad \pi^{-}+p \rightarrow \Sigma^{+}+K^{-}$ cannot be observed.
\answer{}
Before we analyze the processes, let's summarize the quantum numbers of the particles involved:
\begin{itemize}
    \item $\pi^-$: $I=1$, $I_3=-1$, $S=0$, $B=0$
    \item $\pi^+$: $I=1$, $I_3=+1$, $S=0$, $B=0$
    \item $p$: $I=\tfrac{1}{2}$, $I_3=+\tfrac{1}{2}$, $S=0$, $B=1$
    \item $n$: $I=\tfrac{1}{2}$, $I_3=-\tfrac{1}{2}$, $S=0$, $B=1$
    \item $\Sigma^-$: $I=1$, $I_3=-1$, $S=-1$, $B=1$
    \item $\Sigma^+$: $I=1$, $I_3=+1$, $S=-1$, $B=1$
    \item $K^0$: $I=\tfrac{1}{2}$, $I_3=+\tfrac{1}{2}$, $S=+1$, $B=0$
    \item $K^-$: $I=\tfrac{1}{2}$, $I_3=-\tfrac{1}{2}$, $S=-1$, $B=0$
\end{itemize}
Now, let's analyze each process:
\begin{itemize}
    \item [(a)] $\pi^{-}+p \rightarrow \pi^{+}+\Sigma^{-}$:
    \begin{itemize}
        \item Initial state: $I_3 = -1 + \tfrac{1}{2} = -\tfrac{1}{2}$, $S = 0 + 0 = 0$, $B = 0 + 1 = 1$
        \item Final state: $I_3 = +1 - 1 = 0$, $S = 0 - 1 = -1$, $B = 0 + 1 = 1$
    \end{itemize}
    The strangeness $S$ changes from 0 to -1, which is not allowed in strong interactions. The isospin $I_3$ also changes from $-\tfrac{1}{2}$ to $0$. Therefore, this process cannot be observed.
    \item [(b)] $\pi^{-}+p \rightarrow K^0+n$:
    \begin{itemize}
        \item Initial state: $I_3 = -1 + \tfrac{1}{2} = -\tfrac{1}{2}$, $S = 0 + 0 = 0$, $B = 0 + 1 = 1$
        \item Final state: $I_3 = +\tfrac{1}{2} - \tfrac{1}{2} = 0$, $S = +1 + 0 = +1$, $B = 0 + 1 = 1$
    \end{itemize}
    The strangeness $S$ changes from 0 to +1, which is not allowed in strong interactions. The isospin $I_3$ also changes from $-\tfrac{1}{2}$ to $0$. Therefore, this process cannot be observed.
    \item [(c)] $\pi^{-}+p \rightarrow \Sigma^{+}+K^{-}$:
    \begin{itemize}
        \item Initial state: $I_3 = -1 + \tfrac{1}{2} = -\tfrac{1}{2}$, $S = 0 + 0 = 0$, $B = 0 + 1 = 1$
        \item Final state: $I_3 = +1 - \tfrac{1}{2} = +\tfrac{1}{2}$, $S = -1 - 1 = -2$, $B = 1 + 0 = 1$
    \end{itemize}
    The strangeness $S$ changes from 0 to -2, which is not allowed in strong interactions. The isospin $I_3$ also changes from $-\tfrac{1}{2}$ to $+\tfrac{1}{2}$. Therefore, this process cannot be observed.
\end{itemize}\qed

\clearpage
\question{3}{}
\textbf{SU(2) invariants and pseudoreal representations}
\begin{itemize}
    \item [(a)] Show that $\delta^a{ }_b$ and $\epsilon_{a b}$ are invariant tensors under $\mathrm{SU}(2)$ transformations.
    \item [(b)] The nucleon doublet $N^a=\binom{p}{n}, a=1,2$ transforms as the fundamental 2 of $\mathrm{SU}(2)$, while its conjugate $\bar{N}_a \equiv\left(N^a\right)^{\dagger}=(\bar{p}, \bar{n})$ transforms as $\overline{\mathbf{2}}$. Use $\delta^a{ }_b$ to form an $\mathrm{SU}(2)$ invariant with $N, \bar{N}$ and write it explicitly in terms of the proton and neutron fields.
    \item [(c)] Define $\tilde{N}^b=\epsilon^{b c} \bar{N}_c^T$ which maps the $\overline{\mathbf{2}}$ representation (lower index) into $\mathbf{2}$ (upper index). Construct an $\mathrm{SU}(2)$ invariant with $N, \tilde{N}$ using $\epsilon_{a b}$, and write it in terms of the components. Verify that the result is identical to part (b), demonstrating that the $\mathbf{2}$ and $\overline{\mathbf{2}}$ representations are equivalent (or pseudoreal) in $\mathrm{SU}(2)$ and that any invariant constructed with $\delta^a{ }_b$ can be rewritten using $\epsilon_{a b}$.
    \item [(d)] Consider $\mathrm{SU}(3)$, with the quark triplet $q^a(a=1,2,3)$ transforming as $\mathbf{3}$ and its conjugate $\overline{q}_a \equiv\left(q^a\right)^{\dagger}$ transforming as $\overline{\mathbf{3}}$. Discuss why a similar mapping using the $\mathrm{SU}(3)$ invariant $\epsilon_{a b c}$ does not make $\mathbf{3}$ and $\overline{\mathbf{3}}$ equivalent. Write down the possible $\mathrm{SU}(3)$ invariants involving $q, \overline{q}$.
\end{itemize}
\answer{}
\begin{itemize}
    \item [(a)]
\end{itemize}
\begin{align}
    \delta^a{ }_b &\rightarrow =\delta'^{a}{}_{b}= U^a{ }_c \delta^c{ }_d (U^\dagger)^d{ }_b = U^a{ }_c (U^\dagger)^c{ }_b =\mathbf{1}^{a}{}_b= \delta^a{ }_b, \\
    \epsilon_{a b} &\rightarrow \epsilon'_{a b} = (U^\dagger)^c{ }_a (U^\dagger)^d{}_b \epsilon_{c d} = \det(U^\dagger) \epsilon_{a b} = \epsilon_{a b}.
\end{align}
\begin{itemize}
    \item [(b)]
\end{itemize}
Using $\delta^a{ }_b$, we can form the invariant
\begin{align}
    \bar{N}_a N^a = \delta^a{ }_b \bar{N}_a N^b = \bar{p} p + \bar{n} n.
\end{align}
We can verify that this is indeed invariant under $\mathrm{SU}(2)$ transformations:
\begin{align}
    \bar{N}_a N^a &\rightarrow \bar{N}'_a N'^a = \bar{N}_b (U^\dagger)^b{ }_a U^a{ }_c N^c = \bar{N}_b \delta^b{ }_c N^c = \bar{N}_a N^a.
\end{align}
\begin{itemize}
    \item [(c)]
\end{itemize}
Using $\epsilon_{a b}$, we can form the invariant
\begin{align}
    \epsilon_{a b} N^a \tilde{N}^b = \epsilon_{a b} N^a \epsilon^{b c} \bar{N}_c^T = \delta_a{ }^c N^a \bar{N}_c^T = N^a \bar{N}_a^T = \bar{N}_a N^a = \bar{p} p + \bar{n} n.    
\end{align}
We can verify that this is indeed invariant under $\mathrm{SU}(2)$ transformations:
\begin{align}
    \epsilon_{a b} N^a \tilde{N}^b &\rightarrow \epsilon_{a b} N'^a \tilde{N}'^b =\epsilon_{a b} U^a{ }_c N^c U^b{ }_d \tilde{N}^d = \det(U) \epsilon_{c d} N^c \tilde{N}^d = \epsilon_{c d} N^c \tilde{N}^d.
\end{align}
This demonstrates that the $\mathbf{2}$ and $\overline{\mathbf{2}}$ representations are equivalent (or pseudoreal) in $\mathrm{SU}(2)$ and that any invariant constructed with $\delta^a{ }_b$ can be rewritten using $\epsilon_{a b}$.
\begin{itemize}
    \item [(d)]
\end{itemize}
The possible $\mathrm{SU}(3)$ invariants involving $q$ and $\overline{q}$ are:
\begin{align}
    \overline{q}_a q^a, \quad \epsilon_{a b c} q^a q^b q^c, \quad \epsilon^{a b c} \overline{q}_a \overline{q}_b \overline{q}_c.
\end{align}
We can start with the $q^aq^b$,
\begin{align}
    q^a q^b = \frac{1}{2}(q^a q^b + q^b q^a) + \frac{1}{2}(q^a q^b - q^b q^a) = S^{ab}+A^{ab},
\end{align}
where $S^{ab}$ is symmetric and $A^{ab}$ is antisymmetric. Moreover, for the antisymmetric part, we can use $\epsilon_{abc}$ to lower an index and get
\begin{align}
     \theta_c =\epsilon_{abc} A^{ab} = \epsilon_{abc} \frac{1}{2}(q^a q^b - q^b q^a) = \epsilon_{abc} q^a q^b.
\end{align}
Now we can see that $\theta_c$ transforms as $\overline{\mathbf{3}}$. In order to see this, we can apply an $\mathrm{SU}(3)$ transformation:
\begin{align}
    \theta'_c &= \epsilon_{abc} q'^a q'^b = \epsilon_{abc} U^a{ }_{a'} U^b{ }_{b'} q^{a'}  q^{b'} = \epsilon_{abc'} \delta^{c'}{ }_c U^a{ }_{a'} U^b{ }_{b'} q^{a'}  q^{b'} \\
    &= \epsilon_{abc'} U^{c'}{}_{k}(U^\dagger)^{k}{}_{c}  U^a{ }_{a'} U^b{ }_{b'} q^{a'}  q^{b'}= \det(U) (U^\dagger)^{k}{}_{c} \epsilon_{a'b'k}  q^{a'}  q^{b'} \\
    &= (U^\dagger)^{k}{}_{c} \epsilon_{a'b'k} q^{a'}  q^{b'} = (U^\dagger)^{k}{}_{c} \theta_k=(U^\dagger)^{c'}{}_{c} \theta_{c'}.
\end{align}
Therefore, $q^a q^b$ can be decomposed into a symmetric part transforming as $\mathbf{6}$ and an antisymmetric part transforming as $\overline{\mathbf{3}}$. This shows that there is no way to map $\overline{\mathbf{3}}$ back to $\mathbf{3}$ using $\epsilon_{abc}$, unlike the case in $\mathrm{SU}(2)$ where we could use $\epsilon_{ab}$ to map between $\mathbf{2}$ and $\overline{\mathbf{2}}$. Hence, the representations $\mathbf{3}$ and $\overline{\mathbf{3}}$ are not equivalent in $\mathrm{SU}(3)$.\qed

\clearpage
\question{4}{}
\textbf{Applications of U-spin}

\begin{itemize}
    \item [(a)] Show that $U_{ \pm}=t_6 \pm i t_7$ and $U_3=\left(\sqrt{3} t_8-t_3\right) / 2$ satisfy the $\mathrm{SU}(2)$ algebra
    $$
    \left[U_3, U_{ \pm}\right]= \pm U_{ \pm}, \quad\left[U_{+}, U_{-}\right]=2 U_3 .
    $$
    \item [(b)]Show that the charge operator $Q=t_3+t_8 / \sqrt{3}$ is a U-scalar i.e. it has U-spin $U=0$ or $\left[Q, U_i\right]=0$ for $i= \pm, 3$. Write the electromagnetic current operator in terms of quark fields.
    \item [(c)] Show that for the meson octet, the ( $U_3=0$ ) component of the U-triplet is $\pi_U^0=\left(-\pi^0+\sqrt{3} \eta\right) / 2$, and the U-singlet is $\eta_U^0=\left(\sqrt{3} \pi^0+\eta\right) / 2$. Since $\pi_U^0$ is a U -spin vector component it cannot couple to the electromagnetic current. Show that for the $2 \gamma$ decay mode, $\left\langle\pi^0 \mid 2 \gamma\right\rangle=\sqrt{3}\langle\eta \mid 2 \gamma\rangle$. How does this U-spin prediction compare with the experimental decay widths?
\end{itemize}
\answer{}
\begin{itemize}
    \item [(a)]
\end{itemize}
Here we recap the Gell-Mann matrices $t_3, t_6, t_7, t_8$:
\begin{align}
    t_3 &= \frac{1}{2}\begin{pmatrix}
    1 & 0 & 0 \\
    0 & -1 & 0 \\
    0 & 0 & 0
    \end{pmatrix}, \quad t_6 = \frac{1}{2}\begin{pmatrix}
    0 & 0 & 0 \\
    0 & 0 & 1 \\
    0 & 1 & 0
    \end{pmatrix}, \\
    t_7 &= \frac{1}{2}\begin{pmatrix}
    0 & 0 & 0 \\
    0 & 0 & -i \\
    0 & i & 0
    \end{pmatrix}, \quad t_8 = \frac{1}{2\sqrt{3}}\begin{pmatrix}
    1 & 0 & 0 \\
    0 & 1 & 0 \\
    0 & 0 & -2
    \end{pmatrix}.
\end{align}
Hence, we have 
\begin{align}
    U_+ &= t_6 + i t_7 = \begin{pmatrix}
    0 & 0 & 0 \\
    0 & 0 & 1 \\
    0 & 0 & 0
    \end{pmatrix}, \quad U_- = t_6 - i t_7 = \begin{pmatrix}
    0 & 0 & 0 \\
    0 & 0 & 0 \\
    0 & 1 & 0
    \end{pmatrix}, \\
    U_3 &= \frac{\sqrt{3} t_8 - t_3}{2} =\begin{pmatrix}
    0 & 0 & 0 \\
    0 & \frac{1}{2} & 0 \\
    0 & 0 & -\frac{1}{2}
    \end{pmatrix}.
\end{align}
Now we can verify the $\mathrm{SU}(2)$ algebra:
\begin{align}
    [U_3, U_+] &= U_3 U_+ - U_+ U_3 = \begin{pmatrix}
    0 & 0 & 0 \\
    0 & \frac{1}{2} & 0 \\
    0 & 0 & -\frac{1}{2}
    \end{pmatrix} \begin{pmatrix}
    0 & 0 & 0 \\
    0 & 0 & 1 \\
    0 & 0 & 0
    \end{pmatrix} - \begin{pmatrix}
    0 & 0 & 0 \\
    0 & 0 & 1 \\
    0 & 0 & 0
    \end{pmatrix} \begin{pmatrix}
    0 & 0 & 0 \\
    0 & \frac{1}{2} & 0 \\
    0 & 0 & -\frac{1}{2}
    \end{pmatrix} \\
    &= \begin{pmatrix}
    0 & 0 & 0 \\
    0 & 0 & \frac{1}{2} \\
    0 & 0 & 0
    \end{pmatrix} - \begin{pmatrix}
    0 & 0 & 0 \\
    0 & 0 & -\frac{1}{2} \\
    0 & 0 & 0
    \end{pmatrix} = \begin{pmatrix}
    0 & 0 & 0 \\
    0 & 0 & 1 \\
    0 & 0 & 0
    \end{pmatrix} = U_+, \\
    [U_3, U_-] &= U_3 U_- - U_- U_3 = \begin{pmatrix}
    0 & 0 & 0 \\
    0 & \frac{1}{2} & 0 \\
    0 & 0 & -\frac{1}{2}
    \end{pmatrix} \begin{pmatrix}
    0 & 0 & 0 \\
    0 & 0 & 0 \\
    0 & 1 & 0
    \end{pmatrix} - \begin{pmatrix}
    0 & 0 & 0 \\
    0 & 0 & 0 \\
    0 & 1 & 0
    \end{pmatrix} \begin{pmatrix}
    0 & 0 & 0 \\
    0 & \frac{1}{2} & 0 \\
    0 & 0 & -\frac{1}{2}
    \end{pmatrix} \\
    &= \begin{pmatrix}
    0 & 0 & 0 \\
    0 & 0 & 0 \\
    0 & -\frac{1}{2} & 0
    \end{pmatrix} - \begin{pmatrix}
    0 & 0 & 0 \\
    0 & 0 & 0 \\
    0 & \frac{1}{2} & 0
    \end{pmatrix} = \begin{pmatrix}
    0 & 0 & 0 \\
    0 & 0 & 0 \\
    0 & -1 & 0
    \end{pmatrix} = -U_-, \\
    [U_+, U_-] &= U_+ U_- - U_- U_+ = \begin{pmatrix}
    0 & 0 & 0 \\
    0 & 0 & 1 \\
    0 & 0 & 0
    \end{pmatrix} \begin{pmatrix}
    0 & 0 & 0 \\
    0 & 0 & 0 \\
    0 & 1 & 0
    \end{pmatrix} - \begin{pmatrix}
    0 & 0 & 0 \\
    0 & 0 & 0 \\
    0 & 1 & 0
    \end{pmatrix} \begin{pmatrix}
    0 & 0 & 0 \\
    0 & 0 & 1 \\
    0 & 0 & 0
    \end{pmatrix} \\
    &= \begin{pmatrix}
    0 & 0 & 0 \\
    0 & 1 & 0 \\
    0 & 0 & 0
    \end{pmatrix} - \begin{pmatrix}
    0 & 0 & 0 \\
    0 & 0 & 0 \\
    0 & 0 & 1
    \end{pmatrix} = \begin{pmatrix}
    0 & 0 & 0 \\
    0 & 1 & 0 \\
    0 & 0 & -1
    \end{pmatrix} = 2 U_3.
\end{align}
\begin{itemize}
    \item [(b)]
\end{itemize}
First, we write down the charge operator:
\begin{align}
    Q=t_3 + \frac{t_8}{\sqrt{3}} = \begin{pmatrix}
    \frac{2}{3} & 0 & 0 \\
    0 & -\frac{1}{3} & 0 \\
    0 & 0 & -\frac{1}{3}
    \end{pmatrix}.
\end{align}
Now we can verify that $Q$ is a U-scalar:
\begin{align}
    [Q, U_+] &= Q U_+ - U_+ Q = \begin{pmatrix}
    \frac{2}{3} & 0 & 0 \\
    0 & -\frac{1}{3} & 0 \\
    0 & 0 & -\frac{1}{3}
    \end{pmatrix} \begin{pmatrix}
    0 & 0 & 0 \\
    0 & 0 & 1 \\
    0 & 0 & 0
    \end{pmatrix} - \begin{pmatrix}
    0 & 0 & 0 \\
    0 & 0 & 1 \\
    0 & 0 & 0
    \end{pmatrix} \begin{pmatrix}
    \frac{2}{3} & 0 & 0 \\
    0 & -\frac{1}{3} & 0 \\
    0 & 0 & -\frac{1}{3}
    \end{pmatrix} \\
    &= \begin{pmatrix}
    0 & 0 & 0 \\
    0 & 0 & -\frac{1}{3} \\
    0 & 0 & 0
    \end{pmatrix} - \begin{pmatrix}
    0 & 0 & 0 \\
    0 & 0 & -\frac{1}{3} \\
    0 & 0 & 0
    \end{pmatrix} = 0, \\
    [Q, U_-] &= Q U_- - U_- Q = \begin{pmatrix}
    \frac{2}{3} & 0 & 0 \\
    0 & -\frac{1}{3} & 0 \\
    0 & 0 & -\frac{1}{3}
    \end{pmatrix} \begin{pmatrix}
    0 & 0 & 0 \\
    0 & 0 & 0 \\
    0 & 1 & 0
    \end{pmatrix} - \begin{pmatrix}
    0 & 0 & 0 \\
    0 & 0 & 0 \\
    0 & 1 & 0
    \end{pmatrix} \begin{pmatrix}
    \frac{2}{3} & 0 & 0 \\
    0 & -\frac{1}{3} & 0 \\
    0 & 0 & -\frac{1}{3}
    \end{pmatrix} \\
    &= \begin{pmatrix}
    0 & 0 & 0 \\
    0 & 0 & 0 \\
    0 & -\frac{1}{3} & 0
    \end{pmatrix} - \begin{pmatrix}
    0 & 0 & 0 \\
    0 & 0 & 0 \\
    0 & -\frac{1}{3} & 0
    \end{pmatrix} = 0, \\
    [Q, U_3] &= Q U_3 - U_3 Q = \begin{pmatrix}
    \frac{2}{3} & 0 & 0 \\
    0 & -\frac{1}{3} & 0 \\
    0 & 0 & -\frac{1}{3}
    \end{pmatrix} \begin{pmatrix}
    0 & 0 & 0 \\
    0 & \frac{1}{2} & 0 \\
    0 & 0 & -\frac{1}{2}
    \end{pmatrix} - \begin{pmatrix}
    0 & 0 & 0 \\
    0 & \frac{1}{2} & 0 \\
    0 & 0 & -\frac{1}{2}
    \end{pmatrix} \begin{pmatrix}
    \frac{2}{3} & 0 & 0 \\
    0 & -\frac{1}{3} & 0 \\
    0 & 0 & -\frac{1}{3}
    \end{pmatrix} \\
    &= \begin{pmatrix}
    0 & 0 & 0 \\
    0 & -\frac{1}{6} & 0 \\
    0 & 0 & \frac{1}{6}
    \end{pmatrix} - \begin{pmatrix}
    0 & 0 & 0 \\
    0 & -\frac{1}{6} & 0 \\
    0 & 0 & \frac{1}{6}
    \end{pmatrix} = 0.
\end{align}
Thus, we have shown that $[Q, U_i] = 0$ for $i = \pm, 3$, confirming that $Q$ is a U-scalar.
In QFT, the electormagnetic current operator in terms of fermion fields is given by
\begin{align}
    J_\mu^{\text{em}} = \sum_f Q_f \overline{\psi}_f \gamma_\mu \psi_f,
\end{align}
where the sum runs over all fermion flavors $f$, $Q_f$ is the electric charge of the fermion in units of the elementary charge, $\psi_f$ is the fermion field, and $\gamma_\mu$ are the gamma matrices. For the quark fields $u, d, s$, the electromagnetic current operator can be explicitly written as
\begin{align}
    J_\mu^{\text{em}} = \frac{2}{3} \overline{u} \gamma_\mu u - \frac{1}{3} \overline{d} \gamma_\mu d - \frac{1}{3} \overline{s} \gamma_\mu s.
\end{align}
\begin{itemize}
    \item [(c)]
\end{itemize}
First, we can express $\pi^0$ and $\eta$ in terms of quark content:
\begin{align}
    \pi^0 &= \frac{1}{\sqrt{2}}(u\overline{u} - d\overline{d}), \\
    \eta &= \frac{1}{\sqrt{6}}(u\overline{u} + d\overline{d} - 2s\overline{s}).
\end{align}
Now, we can construct the U-triplet and U-singlet components:
\begin{align}
    \pi_U^0 &= \frac{-\pi^0 + \sqrt{3} \eta}{2} = \frac{-\frac{1}{\sqrt{2}}(u\overline{u} - d\overline{d}) + \sqrt{3} \frac{1}{\sqrt{6}}(u\overline{u} + d\overline{d} - 2s\overline{s})}{2} \\
    &= \frac{-\frac{1}{\sqrt{2}}(u\overline{u} - d\overline{d}) + \frac{1}{\sqrt{2}}(u\overline{u} + d\overline{d} - 2s\overline{s})}{2} = \frac{2d\overline{d}-2s\overline{s}}{2\sqrt{2}} = \frac{d\overline{d}-s\overline{s}}{\sqrt{2}}, \\
    \eta_U^0 &= \frac{\sqrt{3} \pi^0 + \eta}{2} = \frac{\sqrt{3} \frac{1}{\sqrt{2}}(u\overline{u} - d\overline{d}) + \frac{1}{\sqrt{6}}(u\overline{u} + d\overline{d} - 2s\overline{s})}{2} \\
    &= \frac{1}{2\sqrt{6}}(3u\overline{u} - 3d\overline{d} + u\overline{u} + d\overline{d} - 2s\overline{s}) = \frac{4u\overline{u}-2d\overline{d}-2s\overline{s}}{2\sqrt{6}} = \frac{2u\overline{u}-d\overline{d}-s\overline{s}}{\sqrt{6}}\\
    &=\frac{2(u\overline{u}+d\overline{d}+s\overline{s})}{\sqrt{6}}-\frac{3d\overline{d}+3s\overline{s}}{\sqrt{6}}=\frac{2(u\overline{u}+d\overline{d}+s\overline{s})}{\sqrt{6}}.
\end{align}
Since $\pi_U^0$ is a U-spin vector component, it cannot couple to the electromagnetic current. Therefore, we have
\begin{align}
    \langle \pi_U^0 | 2\gamma \rangle = 0 \implies \left\langle \frac{-\pi^0 + \sqrt{3} \eta}{2} | 2\gamma \right\rangle = 0 &\implies -\frac{1}{2} \langle \pi^0 | 2\gamma \rangle + \frac{\sqrt{3}}{2} \langle \eta | 2\gamma \rangle = 0.\\
    &\implies \langle \pi^0 | 2\gamma \rangle = \sqrt{3} \langle \eta | 2\gamma \rangle.
\end{align}
The decay width $\Gamma$ is proportional to the square of the amplitude, so we have
\begin{align}
    \Gamma(\pi^0 \rightarrow 2\gamma) \propto |\langle \pi^0 | 2\gamma \rangle|^2, \quad \Gamma(\eta \rightarrow 2\gamma) \propto |\langle \eta | 2\gamma \rangle|^2.
\end{align}
Using the relation we derived, we find
\begin{align}
    \frac{\Gamma(\pi^0 \rightarrow 2\gamma)}{\Gamma(\eta \rightarrow 2\gamma)} \propto \frac{|\langle \pi^0 | 2\gamma \rangle|^2}{|\langle \eta | 2\gamma \rangle|^2} = 3.
\end{align}
Here I ignore the mass difference between $\pi^0$ and $\eta$ for simplicity. Experimentally, the decay widths are approximately:
\begin{align}
    \Gamma(\pi^0 \rightarrow 2\gamma) &\approx 7.8 \text{ eV}, \\
    \Gamma(\eta \rightarrow 2\gamma) &\approx 0.51 \text{ keV} = 510 \text{ eV}.
\end{align}
Thus, the experimental ratio is
\begin{align}
    \frac{\Gamma(\pi^0 \rightarrow 2\gamma)}{\Gamma(\eta \rightarrow 2\gamma)} \approx \frac{7.8 \text{ eV}}{510 \text{ eV}} \approx 0.0153,
\end{align}
which is significantly different from the U-spin prediction of 3. \qed\\
\textbf{Remark:} I check the decay widths formula for $\pi^0$ (I quote eq.~(30.14) from Schwarz's QFT book):
\begin{align}
    \Gamma(\pi^0 \rightarrow 2\gamma) = \frac{\alpha^2 m_{\pi}^3}{64 \pi^3 f_{\pi}^2} \approx 7.73 \text{ eV}.
\end{align}
It show that $\Gamma$ is proportional to $m^3$, so the mass difference between $\pi^0$ and $\eta$ cannot be ignored. Including the mass difference, we have
\begin{align}
    \frac{\Gamma(\pi^0 \rightarrow 2\gamma)}{\Gamma(\eta \rightarrow 2\gamma)} = 3 \left(\frac{m_{\pi}}{m_{\eta}}\right)^3 =3\left(\frac{139.6\text{ MeV}}{547.862\text{ MeV}}\right)^3\approx 0.049,
\end{align}
which is still significantly different from the experimental value of approximately 0.0153. 