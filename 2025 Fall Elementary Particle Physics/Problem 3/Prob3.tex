\section*{Problem Set 3 due 9:30 AM, Monday, October 13th}


\question{1}{}\textbf{p-d reactions}

Consider the reactions
\begin{align}
p+d \rightarrow \pi^{+}+{ }^3 \mathrm{H}, \quad p+d \rightarrow \pi^0+{ }^3 \mathrm{He} .    
\end{align}
Since the deuteron is in a ${ }^3 S_1$ state, it must be an isospin singlet. Therefore, the initial state $p+d$ is a pure $I=\frac{1}{2}$ state. Given that ${ }^3 \mathrm{H}$ and ${ }^3 \mathrm{He}$ form an isodoublet, write down the isospin decomposition of the final states, and from this, the ratio of the two cross sections.

\clearpage
\question{2}{}
\textbf{Particle production by strong interactions}

Explain why the processes $\pi^{-}+p \rightarrow \pi^{+}+\Sigma^{-}, \quad \pi^{-}+p \rightarrow K^0+n, \quad \pi^{-}+p \rightarrow \Sigma^{+}+K^{-}$ cannot be observed.


\clearpage
\question{3}{}
\textbf{SU(2) invariants and pseudoreal representations}
\begin{itemize}
    \item [(a)] Show that $\delta^a{ }_b$ and $\epsilon_{a b}$ are invariant tensors under $\mathrm{SU}(2)$ transformations.
    \item [(b)] The nucleon doublet $N^a=\binom{p}{n}, a=1,2$ transforms as the fundamental 2 of $\mathrm{SU}(2)$, while its conjugate $\bar{N}_a \equiv\left(N^a\right)^{\dagger}=(\bar{p}, \bar{n})$ transforms as $\overline{\mathbf{2}}$. Use $\delta^a{ }_b$ to form an $\mathrm{SU}(2)$ invariant with $N, \bar{N}$ and write it explicitly in terms of the proton and neutron fields.
    \item [(c)] Define $\tilde{N}^b=\epsilon^{b c} \bar{N}_c^T$ which maps the $\overline{\mathbf{2}}$ representation (lower index) into $\mathbf{2}$ (upper index). Construct an $\mathrm{SU}(2)$ invariant with $N, \tilde{N}$ using $\epsilon_{a b}$, and write it in terms of the components. Verify that the result is identical to part (b), demonstrating that the $\mathbf{2}$ and $\overline{\mathbf{2}}$ representations are equivalent (or pseudoreal) in $\mathrm{SU}(2)$ and that any invariant constructed with $\delta^a{ }_b$ can be rewritten using $\epsilon_{a b}$.
    \item [(d)] Consider $\mathrm{SU}(3)$, with the quark triplet $q^a(a=1,2,3)$ transforming as $\mathbf{3}$ and its conjugate $\bar{q}_a \equiv\left(q^a\right)^{\dagger}$ transforming as $\overline{\mathbf{3}}$. Discuss why a similar mapping using the $\mathrm{SU}(3)$ invariant $\epsilon_{a b c}$ does not make $\mathbf{3}$ and $\overline{\mathbf{3}}$ equivalent. Write down the possible $\mathrm{SU}(3)$ invariants involving $q, \bar{q}$.
\end{itemize}


\clearpage
\question{4}{}
\textbf{Applications of U-spin}

\begin{itemize}
    \item [(a)] Show that $U_{ \pm}=t_6 \pm i t_7$ and $U_3=\left(\sqrt{3} t_8-t_3\right) / 2$ satisfy the $\mathrm{SU}(2)$ algebra
    $$
    \left[U_3, U_{ \pm}\right]= \pm U_{ \pm}, \quad\left[U_{+}, U_{-}\right]=2 U_3 .
    $$
    \item [(b)]Show that the charge operator $Q=t_3+t_8 / \sqrt{3}$ is a U-scalar i.e. it has U-spin $U=0$ or $\left[Q, U_i\right]=0$ for $i= \pm 3$. Write the electromagnetic current operator in terms of quark fields.
    \item [(c)] Show that for the meson octet, the ( $U_3=0$ ) component of the U-triplet is $\pi_U^0=\left(-\pi^0+\sqrt{3} \eta\right) / 2$, and the U-singlet is $\eta_U^0=\left(\sqrt{3} \pi^0+\eta\right) / 2$. Since $\pi_U^0$ is a U -spin vector component it cannot couple to the electromagnetic current. Show that for the $2 \gamma$ decay mode, $\left\langle\pi^0 \mid 2 \gamma\right\rangle=\sqrt{3}\langle\eta \mid 2 \gamma\rangle$. How does this U-spin prediction compare with the experimental decay widths?
\end{itemize}
