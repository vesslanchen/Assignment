\section*{Problem Set 5 due 9:30 AM, Monday, November 10}

\question{1}{}
Use the nonrelativistic hydrogen atom result to determine the energy levels for two particles of equal mass m interacting via an attractive potential $-\alpha/r$ , where $\alpha$ is the coupling constant. The $2^3S-1^3S$ separation is $\sim600$~MeV for charmonium
and $\sim5$~eV for positronium. Justify this factor $10^8$ in energy scale in terms of the
constituent masses and couplings in the two cases. For charmonium assume a color-
Coulomb potential $V(r)=-C_F \frac{\alpha_s}{r}$ with $C_F=4/3$. Estimate the strong force coupling
$\alpha_S$, needed to reproduce the observed splitting.
\answer{}

\clearpage

\question{2}{\textbf{Quarkonia}}\\
\begin{itemize}
    \item [(a)] Meson such as the $phi(s\bar{s})$, $J/\psi(c\bar{c})$ and $\Upsilon(b\bar{b})$ are comparatively narrow hadronic resonances, even though they are strongly interacting. Explain, using qualitative arguments, why these quarkonium state do \textit{not} readily decay into lighter flavor mesons such as $Q\bar{q}$ and $\bar{Q}q$ (where $Q=c,b$ and $q=u,d,s$). Discuss how the mass of each resonance with respect to the lowest threshold (e.g. $K\bar{K},D\bar{D},B\bar{B}$) controls whether strong decays are allowed.\\
    \textbf{Hint}: Model a $Q\bar{Q}$ ground state with the Cornell potential and include a kinetic term via the uncertainties principle $p\sim1/r$ to obtain: 
    \begin{align}
        E(r)=\frac{1}{2\mu r^2} - \frac{A}{r}+kr,
    \end{align}
    where $\mu=m_Q/2$ and $A=4\alpha_S/3$. Minimize $E(r)$ with respect to $r$ to obtain an estimate for the ground state size $r_*$. Use your $r_*$ to estimate each piece in $E(r)$ and show how this changes in going from $s\bar{s}$ to $c\bar{c}$ to $b\bar{b}$. 
    When estimating the bound state energies, use $\alpha_S\sim1(0.4)$ for the $s\bar{s}(c\bar{c},b\bar{b})$ systems, and the "constituent" quark masses, $m_s\sim450$~MeV, $m_c\sim1.5$~GeV, and $m_b\sim4.8$~GeV, which include the gluon self-energy corrections.
    \item [(b)] The vector quarkonia $\phi(s\bar{s})$, $J/\psi(c\bar{c})$ and $\Upsilon(b\bar{b})$ follow an approximate factor-of-three scaling in mass from one flavor to the next. Before the discovery of the top quark in 1995, this scaling was used to estimate the "toponium" mass. Determine this mass and estimate whether this fictional top-antitop ($t\bar{t}$) pair would have had time to form a bound state before decaying? Compare with the real top quark which has a mass $m_t\approx173$~GeV.
\end{itemize}
\answer{}

\clearpage
\question{3}{\textbf{Regge trajectories}}\\
In the string model of hadrons a meson (quark-antiquark pair) can be thought of as a string of length $2r_0$ with string tension $k$ and whose ends rotate at a speed $v=c$.

\begin{itemize}
    \item [(a)]Show that the total mass of the rotating string is $M=\pi kr_0$ and the orbital angular momentum is $J=\frac{1}{2}\pi k r_0^2$. Hence, obtain the relation between $J$ and $M^2$.
    \item [(b)] Draw a Chew-Frautschi plot (i.e. $J$ vs. $M^2$) of the following mesons $\rho(770)$, $f_2(1270)$, $\rho_3(1690)$, $f_4(2050)$, $\rho_5(2350)$ and $f_6(2510)$ and baryons $\Delta(1232)$, $\Delta(1950)$, $\Delta(2420)$, $\Delta(2950)$. From your plot, determine the experimental value of the string tension $k$. Compare the meson and baryon values and discuss whether a universal string tension describes both within errors.
\end{itemize}
\answer{}