
\section*{Assignment 1 due on Wednesday September 10th at 5PM}

\question{1}{In lecture we found that ${\Lambda^0}_0 = \gamma$ and ${\Lambda^i}_0=\gamma v_i$ for boosts. The other components should be of the form ${\Lambda^i}_j=a(v)\delta_{ij}+b(v)v_iv_j$ and ${\Lambda^0}_j=c(v)v_j$, where a, b, and c can be functions of the speed v. Determine these functions from the constraint $\eta_{\alpha\beta}{\Lambda^\alpha}_\gamma{\Lambda^\beta}_\delta=\eta_{\gamma\delta}$.}

\answer{}
We can use tensor notation to reduce the work. From the constraint
\begin{align}
    \eta_{\alpha\beta}{\Lambda^\alpha}_\gamma{\Lambda^\beta}_\delta=\eta_{\gamma\delta},
\end{align}
and the relation $\gamma=\frac{1}{\sqrt{1-v^2}} $ (or $\gamma^2-\gamma^2v^2=\gamma^2(1-v^2)=1$), we can consider different $\gamma,\delta$, meaning that
\begin{itemize}
    \item $(\gamma,\delta)=(0,0)$: 
    \begin{align}
    -({\Lambda^0}_0)^2+\sum_{i=1}^3({\Lambda^i}_0)^2=-\gamma^2+\gamma^2\sum_{i=1}^3(v_i)^2=-\gamma^2+\gamma^2 v^2  =-1  =\eta_{00}
    \end{align}
    \item $(\gamma,\delta)=(0,i)$:
    \begin{align}
         &-{\Lambda^0}_0{\Lambda^0}_i+\sum_{k=1}^3{\Lambda^k}_0{\Lambda^k}_i\\
        =&-\gamma c v_i +\sum_{k=1}^3 \gamma v_k (a\delta_{ki}+bv_kv_i)\\
        =&-\gamma c v_i + \gamma a v_i +\gamma b v^2 v_i = \gamma v _i (-c+a+bv^2)=\eta_{0i}=0
    \end{align}
    Then we have $c=a+bv^2$.
    \item $(\gamma,\delta)=(i,0)$
    \begin{align}
        &-{\Lambda^0}_i{\Lambda^0}_0+\sum_{k=1}^3{\Lambda^k}_i{\Lambda^k}_0\\
       =&\gamma v _i (-c+a+bv^2)
    \end{align}
    \item $(\gamma,\delta)=(i,i)$
    \begin{align}
         &-{\Lambda^0}_i{\Lambda^0}_i+\sum_{k=1}^3{\Lambda^k}_i{\Lambda^k}_i\\
        =&-c^2v_i^2 + \sum_{k=1}^3 (a\delta_{ki}+bv_kv_i)^2\\
        =&-c^2v_i^2 + \sum_{k=1}^3(a^2\delta_{ki}+2ab\delta_{ki}v_kv_i+b^2v_i^2v_k^2)\\
        =&-c^2v_i^2 +a^2 +2abv_i^2+b^2v_i^2v^2=\eta_{ii}=1
    \end{align}
    \item $(\gamma,\delta)=(i,j), i\neq j$
    \begin{align}
        &-{\Lambda^0}_i{\Lambda^0}_j+\sum_{k=1}^3{\Lambda^k}_i{\Lambda^k}_j\\
        =&-c^2 v_iv_j+\sum_{k=1}^3 (a\delta_{ki}+bv_kv_i) (a\delta_{kj}+bv_kv_j)\\
        =&-c^2 v_iv_j+\sum_{k=1}^3 (a^2 \delta_{ki}\delta_{kj}+ab(\delta_{ki}v_kv_j+\delta_{kj}v_kv_i)+b^2v_k^2v_iv_j)\\
        =&-c^2 v_iv_j +2 ab v_iv_j +b^2 v^2 v_i v_j =\eta_{ij}=0
    \end{align}
    Then we have $c^2=b^2v^2+2ab$
\end{itemize}
Combining the above information, we can have 
\begin{align}
    &c = a+bv^2 \\
    &c^2 = a^2+b^2v^4+2abv^2 = b^2v^2+2ab.  \label{Eq:c^2}
\end{align}
Also, we have 
\begin{align}
    &-c^2v_i^2 +a^2 +2abv_i^2+b^2v_i^2v^2=1\\
    \rightarrow 1 &= -(b^2v^2+2ab)v_i^2+a^2 +2abv_i^2+b^2v_i^2v^2\\
    &=a^2        \label{Eq:a^2=1}
\end{align}
Hence, from Eq.~\ref{Eq:c^2} and Eq.~\ref{Eq:a^2=1}, we have 
\begin{align}
    &c^2 = 1 + b^2v^4+2abv^2=b^2v^2+2ab\\
    \rightarrow&a=\pm1=\frac{1+b^2v^4-b^2v^2}{2b(1-v^2)}=\frac{1+b^2v^2(v^2-1)}{2b(1-v^2)}=\frac{\gamma^2-b^2v^2}{2b}\\
    \rightarrow&b^2v^2\pm2b-\gamma^2=0
\end{align}
For $a=+1$, we have 
\begin{align}
    b = \frac{-1\pm\sqrt{1+\gamma^2v^2}}{v^2}=\frac{-1\pm\gamma}{v^2},
\end{align}
and for $a=-1$, we have 
\begin{align}
    b = \frac{1\pm\gamma}{v^2}.
\end{align}
Hence, we choose $a=1$ and $b=\frac{-1+\gamma}{v^2}$ by convention and have 
\begin{align}
    c &= a+bv^2 = 1 + (-1+\gamma)=+\gamma\\
\end{align}
Finally, we derive 
\begin{align}
    {\Lambda^i}_j=&a(v)\delta_{ij}+b(v)v_iv_j= \delta_{ij}+\frac{\gamma-1}{v^2}v_iv_j\\
    {\Lambda^0}_j=&c(v)v_j=\gamma v_j,
\end{align}
which are mentioned in the Week 1 lecture.\qed

\newpage
\question{2}{ A rod of length $L$ lies at rest along the x-axis in frame $\mathcal{O}$. An observer located along a perpendicular axis in frame $\mathcal{O}'$ sees frame $\mathcal{O}$ moving with speed v in the positive $x$ direction. The observer in frame $\mathcal{O}'$, which may be an eye or a camera, looks in the direction of the rod. When the center of the rod is at the distance of closest approach $D$ to the observer, lights at the ends of the rod send out flashes of light. When $v \ll c$, the opening angle $\theta$ between the light flashes would be seen as $\tan{\theta/2} = L/2D$. Show that for any value of $v < c$, the opening angle is the same, meaning that the observer does not see length contraction. Drawing a picture will help.}
\answer{}
In the $\mathcal{O}$ frame, we set the condition for the right-hand side of the rod:
\begin{align*}
    &x^{\mu}_{emit}=(0,L/2,D,0),x^{\mu}_{receive}=(t,vt,0,0),\\
    &\Delta x^{\mu}=x^{\mu}_{receive}-x^{\mu}_{emit}=(t,vt-L/2,-D,0)\\
    &\Delta x^\mu\Delta x_\mu=0=t^2-(L/2-vt)^2-D^2.
\end{align*}
Also, with $dt=t, dx=vt-L/2, dy=-D$, we have 
\begin{align*}
    \Delta t'&=\gamma(t+v(vt-L/2))\\
    \Delta x'&=\gamma(v+(vt-L/2))\\
    \Delta y'&=-D
\end{align*}
We can substitute the solution $t$ to solve the $dx'$ and $dt'$, and define the opening angle 
\begin{align}
    \tan{\theta'/2}=\frac{\Delta x'}{\Delta y'}.
\end{align}
Note that this is the calculation for right-hand side of the rod. Hence, it should be denoted as $\theta_R'$. Now we can apply the procedure to the left-hand side of the rod. After plugging all information into {\tt Mathematica} and using the equation, we have 
\begin{align}
    \tan{(\theta_R'+\theta_L')}=\frac{\tan\theta_R'+\tan\theta_L'}{1-\tan\theta_R'\tan\theta_L'}=\frac{4DL}{4D^2-L^2}=\frac{2\frac{L}{2D}}{1-\left(\frac{L}{2D}\right)^2}=\tan{\theta}.
\end{align}
Hence, we prove that the opening angle keeps the same value. \qed