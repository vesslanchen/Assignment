
\section*{Problem Set 1 due on Due Feb 9 at 11:59pm}

\question{1}{}
\begin{itemize}
    \item [(a)]Consider the outer surface of the ergosphere for a Kerr black hole. Construct the normal vector to this surface and show that it is a $2$-way surface.
    \item [(b)]Using the condition $n_\alpha n^\alpha = 0$ for a null surface, derive the equation equation for the critical $1$-way surface for a time-independent axial symmetric surface. That is start with some $u(r, \theta)$ to derive the equation defining this surface. What is its physical interpretation.
\end{itemize}

\answer{}
\begin{itemize}
    \item [(a)]
\end{itemize}
Starting with the metric for a Kerr black hole in Boyer-Lindquist coordinates,
\begin{align}
    ds^2 = -\left(1-\frac{2Mr}{\Sigma}\right)dt^2 - \frac{4Mar\sin^2\theta}{\Sigma}dtd\phi + \frac{\Sigma}{\Delta}dr^2 + \Sigma d\theta^2 + \left(r^2 + a^2 + \frac{2Ma^2r\sin^2\theta}{\Sigma}\right)\sin^2\theta d\phi^2,
\end{align}
where $\Sigma = r^2 + a^2\cos^2\theta$ and $\Delta = r^2 - 2Mr + a^2$. The outer surface of the ergosphere is defined by the condition $g_{tt} = 0$, which gives 
\begin{align}
    1-\frac{2Mr}{\Sigma} = 0 \implies r^2 - 2Mr + a^2\cos^2\theta = 0.
\end{align}
We can define a surface function $f(r, \theta) = r^2 - 2Mr + a^2\cos^2\theta=0$. The normal vector to this surface is given by the gradient of $f$:
\begin{align}
    n_\mu = \partial_\mu f = \left(\frac{\partial f}{\partial t}, \frac{\partial f}{\partial r}, \frac{\partial f}{\partial \theta}, \frac{\partial f}{\partial \phi}\right) = (0, 2r - 2M, -2a^2\cos\theta\sin\theta, 0).
\end{align}
To show that this is a $2$-way surface, we need to compute the norm of the normal vector:
\begin{align}
    n_\mu n^\mu = g^{\mu\nu} n_\mu n_\nu.
\end{align}
We try to write out the matrix form of $g_{\mu\nu}$:
\begin{align}
    g_{\mu\nu} = \begin{pmatrix}
    -\left(1-\frac{2Mr}{\Sigma}\right) & 0 & 0 & -\frac{2Mar\sin^2\theta}{\Sigma} \\
    0 & \frac{\Sigma}{\Delta} & 0 & 0 \\
    0 & 0 & \Sigma & 0 \\
    -\frac{2Mar\sin^2\theta}{\Sigma} & 0 & 0 & \left(r^2 + a^2 + \frac{2Ma^2r\sin^2\theta}{\Sigma}\right)\sin^2\theta
    \end{pmatrix}.
\end{align}
Since we are only interested in the $rr$ and $\theta\theta$ components for the norm calculation, we can focus on those:
\begin{align}
    n_\mu n^\mu = g^{rr} n_r n_r + g^{\theta\theta} n_\theta n_\theta = g^{rr} (2r - 2M)^2 + g^{\theta\theta} (-2a^2\cos\theta\sin\theta)^2.
\end{align}
Calculating $g^{rr}$ and $g^{\theta\theta}$ from the inverse metric, we find:
\begin{align}
    g^{rr} = \frac{\Delta}{\Sigma}, \quad g^{\theta\theta} = \frac{1}{\Sigma}.
\end{align}
Substituting these back into the norm expression:
\begin{align}
    &n_\mu n^\mu = \frac{\Delta}{\Sigma} (2r - 2M)^2 + \frac{1}{\Sigma} ( -2a^2\cos\theta\sin\theta)^2\\
    =&\frac{4}{\Sigma} \left[\Delta (r - M)^2 + a^4\cos^2\theta\sin^2\theta\right]\\
    =&\frac{4}{\Sigma} \left[(r^2 - 2Mr + a^2)(r - M)^2 + a^4\cos^2\theta\sin^2\theta\right].
\end{align}
We have $r^2-2Mr = -a^2\cos^2\theta$, and $(r-M)^2=r^2 - 2Mr + M^2= -a^2\cos^2\theta + M^2$ on the ergosphere, so substituting this in:
\begin{align}
    n_\mu n^\mu = \frac{4}{\Sigma} \left[a^2\sin^2\theta ( r-M )^2 + a^4\cos^2\theta\sin^2\theta\right] = \frac{4a^2\sin^2\theta}{\Sigma} \left[ (r-M)^2 + a^2\cos^2\theta \right].
\end{align} 
Since for a Kerr black hole $a \neq 0$ and $\sin\theta \neq 0$ except at the poles, we have $n_\mu n^\mu > 0$ almost everywhere on the ergosphere, indicating that the normal vector is spacelike. Therefore, the ergosphere is a $2$-way surface.

\begin{itemize}
    \item [(b)]
\end{itemize}
For a time-independent axially symmetric surface, we can define the surface function as $u(r, \theta) = 0$. The normal vector to this surface is given by:
\begin{align}
    n_\mu = \partial_\mu u = \left(0, \frac{\partial u}{\partial r}, \frac{\partial u}{\partial \theta}, 0\right).
\end{align}
Using the condition for a null surface, $n_\mu n^\mu = 0$, we have:
\begin{align}
    n_\mu n^\mu = g^{rr} \left(\frac{\partial u}{\partial r}\right)^2 + g^{\theta\theta} \left(\frac{\partial u}{\partial \theta}\right)^2 = 0.    
\end{align}
If we plug the $g^{rr}=\frac{\Delta}{\Sigma}$ and $g^{\theta\theta}=\frac{1}{\Sigma}$ from the Kerr metric, we get:
\begin{align}
    \frac{\Delta}{\Sigma} \left(\frac{\partial u}{\partial r}\right)^2 + \frac{1}{\Sigma} \left(\frac{\partial u}{\partial \theta}\right)^2 = 0\\
    \implies \Delta \left(\frac{\partial u}{\partial r}\right)^2 + \left(\frac{\partial u}{\partial \theta}\right)^2 = 0.
\end{align}
If we want this equation to hold, we can have following condition:
\begin{enumerate}
    \item $\Delta = 0$ and $\frac{\partial u}{\partial \theta} = 0$. This corresponds to the event horizon of the Kerr black hole, which is a $1$-way surface.
    \item $\frac{\partial u}{\partial r} = 0$ and $\frac{\partial u}{\partial \theta} = 0$. This would imply that $u$ is constant, which is not a valid surface definition.
\end{enumerate}
Thus, the critical $1$-way surface is defined by $\Delta = 0$ and $\frac{\partial u}{\partial \theta} = 0$. That is, $u(r,\theta)=u(r)$ only, and the surface is located at the event horizon of the Kerr black hole. The condition $\Delta = 0$ corresponds to the event horizon, which is a null surface that allows one-way passage of information.
\qed




\clearpage
\question{2}{}
Show that 
\begin{align}
    (L_X T)^{a_1 a_2 \dots a_r}{}_{b_1 b_2 \dots b_s} = (\nabla_X T)^{a_1 a_2 \dots a_r}_{b_1 b_2 \dots b_s}-\sum_i T^{a_1 \dots j \dots a_r}_{b_1 b_2 \dots b_s} X^{a_i}_{;j} + \sum_i T^{a_1 a_2 \dots a_r}_{b_1 \dots j \dots b_s} X^j_{;b_i}
\end{align}

\answer{}
In the course, we have already know the definition of the Lie derivative of a tensor field $T$ along a vector field $X$:
\begin{align}
    (L_X T)^{a_1 a_2 \dots a_r}{}_{b_1 b_2 \dots b_s} = X^c \partial_c T^{a_1 a_2 \dots a_r}{}_{b_1 b_2 \dots b_s} - \sum_i T^{a_1 \dots j \dots a_r}_{b_1 b_2 \dots b_s} X^{a_i}_{,j} + \sum_i T^{a_1 a_2 \dots a_r}_{b_1 \dots j \dots b_s} X^j_{,b_i}.
\end{align}
On the other hand, the covariant derivative of $T$ along $X$ is given by:
\begin{align}
    (\nabla_X T)^{a_1 a_2 \dots a_r}{}_{b_1 b_2 \dots b_s} = X^c \partial_c T^{a_1 a_2 \dots a_r}{}_{b_1 b_2 \dots b_s} + \sum_i \Gamma^{a_i}_{cj} T^{a_1 \dots j \dots a_r}_{b_1 b_2 \dots b_s} X^c - \sum_i \Gamma^j_{cb_i} T^{a_1 a_2 \dots a_r}_{b_1 \dots j \dots b_s} X^c.
\end{align}
Subtracting the two expressions, we get:
\begin{align}
    (L_X T)^{a_1 a_2 \dots a_r}{}_{b_1 b_2 \dots b_s} - (\nabla_X T)^{a_1 a_2 \dots a_r}{}_{b_1 b_2 \dots b_s} =& - \sum_i T^{a_1 \dots j \dots a_r}_{b_1 b_2 \dots b_s} X^{a_i}_{,j} + \sum_i T^{a_1 a_2 \dots a_r}_{b_1 \dots j \dots b_s} X^j_{,b_i} \\
    &- \sum_i \Gamma^{a_i}_{cj} T^{a_1 \dots j \dots a_r}_{b_1 b_2 \dots b_s} X^c + \sum_i \Gamma^j_{cb_i} T^{a_1 a_2 \dots a_r  }_{b_1 \dots j \dots b_s} X^c\\
    =& - \sum_i T^{a_1 \dots j \dots a_r}_{b_1 b_2 \dots b_s} X^{a_i}_{;j} + \sum_i T^{a_1 a_2 \dots a_r}_{b_1 \dots j \dots b_s} X^j_{;b_i}.
\end{align}
We have used the definition of the covariant derivative of a vector field, $X^a_{;b} = X^a_{,b} + \Gamma^a_{cb} X^c$, to rewrite the terms involving the Christoffel symbols. Thus, we have shown that:
\begin{align}
    (L_X T)^{a_1 a_2 \dots a_r}{}_{b_1 b_2 \dots b_s} = (\nabla_X T)^{a_1 a_2 \dots a_r}_{b_1 b_2 \dots b_s}-\sum_i T^{a_1 \dots j \dots a_r}_{b_1 b_2 \dots b_s} X^{a_i}_{;j} + \sum_i T^{a_1 a_2 \dots a_r}_{b_1 \dots j \dots b_s} X^j_{;b_i}.
\end{align}
\qed



\clearpage
\question{3}{}
Consider an arbitrary unit vector, $X$ transported along a latitude line on the surface of the sphere. Use $ds^2 =  d\theta^2 +  \sin^2 \theta d\phi^2$, and recall that there are only 2 distinct $\Gamma$’s.
\begin{itemize}
    \item [(a)] Show the behavior of the angle between that vector and the tangent vector, $T$ , which generates the isometry (latitude line) as that vector is parallel transported along the latitude line.
    \item[(b)] Do the same assuming the vector is Lie transported. That is instead of $\nabla_T X =0$, assume that $L_T X =0$.
\end{itemize}
\answer{}
Before we start, let's write down the non-zero Christoffel symbols for the metric on the surface of the sphere:
\begin{align}
    \Gamma^\theta_{\phi\phi} = -\sin\theta\cos\theta, \quad \Gamma^\phi_{\theta\phi} = \Gamma^\phi_{\phi\theta} = \cot\theta.
\end{align}
\begin{itemize}
    \item [(a)]
\end{itemize}
The metric on the surface of the sphere is given by $ds^2 = d\theta^2 + \sin^2\theta d\phi^2$. The tangent vector to the latitude line is $T = \frac{\partial}{\partial \phi}$, which generates the isometry. The unit vector $X$ can be expressed in terms of the coordinate basis as $X = X^\theta \frac{\partial}{\partial \theta} + X^\phi \frac{\partial}{\partial \phi}$. The angle $\alpha$ between $X$ and $T$ is given by:
\begin{align}
    \cos\alpha = \frac{g(X, T)}{\sqrt{g(X, X) g(T, T)}}.
\end{align}
Since $T = \frac{\partial}{\partial \phi}$, we have $g(T, T) = g_{\phi\phi} = \sin^2\theta$. The inner product $g(X, T)$ is given by:
\begin{align}
    g(X, T) = g_{\phi\phi} X^\phi = \sin^2\theta X^\phi.
\end{align}
The norm of $X$ is given by:
\begin{align}
    g(X, X) = g_{\theta\theta} (X^\theta)^2 + g_{\phi\phi} (X^\phi)^2 = (X^\theta)^2 + \sin^2\theta (X^\phi)^2.
\end{align}
For convenience, we set the length of $X$ to be 1,Thus, the cosine of the angle is:
\begin{align}
    &\cos\alpha = \frac{\sin^2\theta X^\phi}{\sqrt{((X^\theta)^2 + \sin^2\theta (X^\phi)^2) \sin^2\theta}} = \frac{\sin\theta X^\phi}{\sqrt{(X^\theta)^2 + \sin^2\theta (X^\phi)^2}}\\
    =& \frac{\sin\theta X^\phi}{\sqrt{1}} = \sin\theta X^\phi.
\end{align} 
To find how the angle changes as $X$ is parallel transported along the latitude line, we need to solve the parallel transport equation $\nabla_T X = 0$. This gives us:
\begin{align}
    \nabla_T X^\theta = T^\phi \partial_\phi X^\theta + \Gamma^\theta_{\phi\phi} T^\phi X^\phi = 0,\\
    \nabla_T X^\phi = T^\phi \partial_\phi X^\phi + \Gamma^\phi_{\phi\theta} T^\phi X^\theta + \Gamma^\phi_{\phi\phi} T^\phi X^\phi = 0.
\end{align}
Substituting $T^\phi = 1$ and the Christoffel symbols, we get:
\begin{align}
    \partial_\phi X^\theta - \sin\theta\cos\theta X^\phi = 0,\\
    \partial_\phi X^\phi + \cot\theta X^\theta = 0.
\end{align}
Now we can try to solve these coupled differential equations. From the first equation, we can express $X^\phi$ in terms of $X^\theta$:
\begin{align}
    X^\phi = \frac{1}{\sin\theta\cos\theta} \partial_\phi X^\theta.
\end{align}
Substituting this into the second equation gives us a second-order differential equation for $X^\theta$:
\begin{align}
    \partial_\phi \left(\frac{1}{\sin\theta\cos\theta} \partial_\phi X^\theta\right) + \cot\theta X^\theta = 0\\
    \implies \frac{1}{\sin\theta\cos\theta} \partial_\phi^2 X^\theta + \cot\theta X^\theta = 0\\
    \implies \partial_\phi^2 X^\theta + \cot\theta \sin\theta\cos\theta X^\theta = 0\\
    \implies \partial_\phi^2 X^\theta + \cos^2\theta X^\theta = 0.
\end{align}
The general solution to this equation is:
\begin{align}
    X^\theta = A \cos((\cos\theta) \phi ) + B \sin((\cos\theta) \phi ) = C \sin((\cos\theta) \phi + \delta),
\end{align}
where $A$, $B$, $C$, and $\delta$ are constants determined by the initial conditions. Substituting this back into the expression for $X^\phi$, we get:
\begin{align}
    X^\phi =& \frac{1}{\sin\theta\cos\theta} \partial_\phi X^\theta = \frac{C }{\sin\theta} \cos((\cos\theta) \phi + \delta).
\end{align}
Now we can compute the angle $\alpha$ as a function of $\phi$:
\begin{align}
    \cos\alpha(\phi) = \sin\theta X^\phi = C \cos((\cos\theta) \phi + \delta).
\end{align}
Thus, the angle between the vector $X$ and the tangent vector $T$ oscillates as $X$ is parallel transported along the latitude line, with a frequency determined by $\cos\theta$.
\begin{itemize}
    \item [(b)]
\end{itemize}
Now we assume that the vector $X$ is Lie transported along the latitude line, which means that $L_T X = 0$. The Lie derivative of $X$ along $T$ is given by:
\begin{align}
    (L_T X)^\theta = T^\phi \partial_\phi X^\theta - X^\phi \partial_\phi T^\theta = \partial_\phi X^\theta,\\
    (L_T X)^\phi = T^\phi \partial_\phi X^\phi - X^\theta \partial_\phi T^\phi = \partial_\phi X^\phi.
\end{align}
Setting $L_T X = 0$ gives us:
\begin{align}
    \partial_\phi X^\theta = 0,\\
    \partial_\phi X^\phi = 0.
\end{align}
This means that both $X^\theta$ and $X^\phi$ are constants along the latitude line. Therefore, the angle $\alpha$ between $X$ and $T$ is also constant:
\begin{align}
    \cos\alpha = \sin\theta X^\phi = \text{constant}.
\end{align}
Thus, when the vector $X$ is Lie transported along the latitude line, the angle between $X$ and the tangent vector $T$ remains constant, in contrast to the oscillatory behavior observed in the case of parallel transport. 
\qed