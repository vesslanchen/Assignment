\documentclass[12pt,letterpaper]{article}

\usepackage[letterpaper, left=1.5cm, right=1.5cm, top=1.5cm, bottom=1.5cm]{geometry}
\usepackage[compat=1.0.0]{tikz-feynman}
\usepackage{amsmath}
\usepackage{amsthm}
\usepackage{amssymb}
\usepackage{graphicx}
\usepackage{setspace} % 引入這個套件
\usepackage{tensor}
\usepackage{braket}
\usepackage{hyperref}\hypersetup{colorlinks=true,linkcolor=black,citecolor=green,urlcolor=blue}

\title{University of Minnesota\\
School of Physics and Astronomy\\
\textbf{2026 Spring Physics 8502 \\ General Relativity II}\\ 
Assignment Solution}
\author{Lecture Instructor: Professor Keith Olive \\ \\
        Zong-En Chen\\ chen9613@umn.edu}
\date{\today}  

% 自定義範例環境
\newtheorem{example}{Example}[section]
\newtheorem{solution}{Sol.}[section]
% 自定義習題環境
\newtheorem{exercise}{Ex.}[section]

\newtheorem{theorem}{Theorem}[section]

\newcommand{\question}[2]{\section*{Question #1}}
\newcommand{\answer}[1]{\section*{Answer}#1}

\begin{document}
\onehalfspacing
%\maketitle
\begin{titlepage}
    \centering
    \vspace*{\fill} % 從頂端往下推到中間
    {\LARGE University of Minnesota\\
    School of Physics and Astronomy\\[1em]
    \textbf{2026 Spring Physics 8502 \\ General Relativity II}\\ 
    Assignment Solution \par}
    \vspace{2cm}
    {\large Lecture Instructor: Professor Keith Olive \par}
    \vspace{1cm}
    {\large Zong-En Chen\\ chen9613@umn.edu \par}
    \vspace{2cm}
    {\large \today \par}
    \vspace*{\fill} % 從底部往上推到中間
\end{titlepage}


\section*{Homework 1 Due to January 29th 11:00 AM}

\question{1}{}
The constant $J$ in the Heisenberg model (slide 6) can be positive or negative. The model can describe either ferromagnetism or anti-ferromagnetism. To which of these to cases positive/ negative $J$ corresponds? Why?
\begin{align}
    \mathcal{H}=-\sum_{\langle i j\rangle} J \mathbf{S}_{i} \cdot \mathbf{S}_{j}
\end{align}
\answer{}
Since \textbf{the Hamiltonian is minimized when the energy is the lowest}, we can analyze the two cases as follows:
\begin{itemize}
    \item For $J>0$, the Hamiltonian becomes $\mathcal{H}=-\sum_{\langle i j\rangle} |J| \mathbf{S}_{i} \cdot \mathbf{S}_{j}$. To minimize the energy, the spins $\mathbf{S}_{i}$ and $\mathbf{S}_{j}$ should align parallel to each other, leading to ferromagnetism.
    \item For $J<0$, the Hamiltonian becomes $\mathcal{H}=-\sum_{\langle i j\rangle} -|J| \mathbf{S}_{i} \cdot \mathbf{S}_{j} = \sum_{\langle i j\rangle} |J| \mathbf{S}_{i} \cdot \mathbf{S}_{j}$. To minimize the energy, the spins $\mathbf{S}_{i}$ and $\mathbf{S}_{j}$ should align anti-parallel to each other, leading to anti-ferromagnetism.
\end{itemize}   
\qed

\clearpage
\question{2}{}
Pirates found a parchment describing location of treasure trove hidden somewhere on the island by their predecessors . The note reads:


\textit{Start from the capsized boat. Go in the direction of the palm tree carefully counting the number of steps. When you reach the palm tree turn exactly right and make exactly the same number of steps. Mark the point you arrived at. Then return to the boat. Go in the direction of the rock counting the number of steps. When you reach the rock turn left. After having made the same number of steps mark the second point. The treasure trove is in the middle of the line connecting two marked points. The problem is that the capsized boat was nowhere in sight, it had disappeared. How can the pirates still find the treasure trove?}
\answer{}
First let's denote the position of the palm tree as point $P(x_1, y_1)$ and the position of the rock as point $R(x_2, y_2)$. We also denote the unknown position of the capsized boat as point $B(x, y)$.

When the pirates walk from the boat to the palm tree, they cover a distance equal to the length of the vector $\overrightarrow{BP} = (x_1 - x, y_1 - y)$. After reaching the palm tree, they turn right and walk the same distance, which corresponds to moving in the direction perpendicular to $\overrightarrow{BP}$. Let's say the right turn results in a new vector $\overrightarrow{PR} = (y_1 - y, -(x_1 - x))$. The coordinates of the first marked point $M_1$ can be expressed as:
\begin{align}
    M_1 = P + \overrightarrow{PR} = (x_1 + (y_1 - y), y_1 - (x_1 - x)) = (x_1 + y_1 - y, y_1 - x_1 + x)
\end{align}
Similarly, when the pirates walk from the boat to the rock, they cover a distance equal to the length of the vector $\overrightarrow{BR} = (x_2 - x, y_2 - y)$. After reaching the rock, they turn left and walk the same distance, which corresponds to moving in the direction perpendicular to $\overrightarrow{BR}$. Let's say the left turn results in a new vector $\overrightarrow{RL} = (-(y_2 - y), x_2 - x)$. The coordinates of the second marked point $M_2$ can be expressed as:
\begin{align}
    M_2 = R + \overrightarrow{RL} = (x_2 - (y_2 - y), y_2 + (x_2 - x)) = (x_2 - y_2 + y, y_2 + x_2 - x) 
\end{align}
The treasure trove is located at the midpoint of the line segment connecting the two marked points $M_1$ and $M_2$. The coordinates of the midpoint $T$ can be calculated as:
\begin{align}
    T = \left( \frac{(x_1 + y_1 - y) + (x_2 - y_2 + y)}{2}, \frac{(y_1 - x_1 + x) + (y_2 + x_2 - x)}{2} \right)
\end{align}
Simplifying the expressions for the coordinates of $T$, we get:
\begin{align}
    T = \left( \frac{x_1 + x_2 + y_1 - y_2}{2}, \frac{y_1 + y_2 - x_1 + x_2}{2} \right)
\end{align}
Notice that the coordinates of the treasure trove $T$ do not depend on the unknown position of the boat $B(x, y)$. Therefore, the pirates can find the treasure trove using only the known positions of the palm tree and the rock, without needing to know the location of the capsized boat.
\qed

\end{document}